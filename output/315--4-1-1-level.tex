\subsection{First-order languages}
The alphabet of the predicate logic consists of
\begin{itemize}
\item
variables;
\item
function symbols;
\item
predicate symbols;
\item
logical connectives $\wedge, \vee, \neg, \to, \forall, \exists$;
\item
auxiliary symbols $($ and $)$;
\item
equality symbol $=$.
\end{itemize}

Informally speaking (and as indicated in the introduction), a variable is an object,
a function is an operation with objects whose result is also an object,
and a predicate is a statement about one or several object (in other words, an operation with objects whose result is a truth value).

Compared to the propositional logic, we have two new logical connectives:
the \emph{universal quantifier} $\forall$ and the \emph{existential quantifier} $\exists$.

The equality symbol is not always included in the alphabet.
Accordingly, there are two slightly different versions of predicate logic: logic with equality and logic without equality.

Before stating the rules according to which the alphabet symbols can be combined one has to fix a \emph{signature}.
This is a list of function symbols and predicate symbols together with the number of arguments for each of them.
