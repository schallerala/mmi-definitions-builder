\documentclass[preview, multi=page, margin=5mm, class=report]{standalone}
\usepackage[utf8]{inputenc}

\usepackage{amsmath,amssymb,amsthm}
\usepackage{graphicx,color}
\usepackage{hyperref,url}
\graphicspath{{Fig/}}

\usepackage{mathtools}
\usepackage{bussproofs}
\usepackage{stackengine}
\def\ruleoffset{1pt}
\newcommand\specialvdash[2]{\mathrel{\ensurestackMath{
  \mkern2mu\rule[-\dp\strutbox]{.4pt}{\baselineskip}\stackon[\ruleoffset]{
    \stackunder[\dimexpr\ruleoffset-.5\ht\strutbox+.5\dp\strutbox]{
      \rule[\dimexpr.5\ht\strutbox-.5\dp\strutbox]{2.5ex}{.4pt}}{
        \scriptstyle #1}}{\scriptstyle#2}\mkern2mu}}
}

\usepackage[table]{xcolor}

\renewcommand\thesection{\arabic{section}}
\renewcommand\thefigure{\arabic{figure}}
\renewcommand\theequation{\arabic{equation}}

\newtheorem{dfn}{Definition}[section]
\newtheorem{thm}[dfn]{Theorem}
\newtheorem{lem}[dfn]{Lemma}
\newtheorem{cor}[dfn]{Corollary}


\theoremstyle{definition}
\newtheorem{exl}[dfn]{Example}
\newtheorem{rem}[dfn]{Remark}
\newtheorem{exc}{Exercise}[section]

\def\R{\mathbb{R}}
\def\N{\mathbb{N}}
\def\Z{\mathbb{Z}}
\def\C{\mathbb{C}}
\def\cP{\mathcal{P}}
\def\cV{\mathcal{V}}
\def\cF{\mathcal{F}}
\def\Th{\mathrm{Th}}

\renewcommand{\emptyset}{\varnothing}
\renewcommand{\phi}{\varphi}
\renewcommand{\epsilon}{\varepsilon}
\def\gcd{\operatorname{gcd}}

\def\Prop{\mathrm{PROP}}
\begin{document}
\setcounter{section}{1}
\setcounter{subsection}{5}
\setcounter{dfn}{14}

Consider the signature without functions and with nullary predicates only:
\begin{equation}
\label{eqn:NullSign}
\cP:\ P_1(), P_2(), \ldots
\end{equation}
Formulas in this signature contain no terms, because a term must occur as an argument of a predicate, but nullary predicates have no arguments.
We can introduce variables in the formulas only with quantifiers by writing something like $\forall x \exists y P \wedge Q$,
but in any structure this formula evaluates in the same way as $P \wedge Q$.
Thus the formulas in signature \eqref{eqn:NullSign} look like propositional formulas with variable symbols $P_i$.

How does $P \wedge Q$ actually evaluate?
By definition, a first-order structure $(U,I)$ assigns to each nullary predicate $P$ a truth value $I(P)$.
Then the truth value of $P \wedge Q$ is $I(P) \wedge I(Q)$ (and the universe $U$ has no significance).
Similarly for every other formula: an evaluation with respect to interpretation $I$ is the same as evaluation of a propositional formula
with $I$ viewed as valuation $v$.

Thus signature \eqref{eqn:NullSign} realizes the propositional logic as a special case of the predicate logic.
One can state this as follows.


\end{document}
