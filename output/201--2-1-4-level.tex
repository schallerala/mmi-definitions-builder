\subsection{Incidence and adjacency}
An edge is said to be \emph{incident} with a vertex if it contains this vertex.
In other words, the edge $\{v, w\}$ is incident with the vertices $v$ and $w$.

The \emph{degree} of a vertex $v$, denoted $\deg v$, is the number of edges incident with $v$.
A vertex of degree zero, that is without incident edges, is called an \emph{isolated} vertex.

In the complete graph $K_n$ all vertices have degree $n-1$.
In the cycle $C_n$, all vertices have degree $2$.
