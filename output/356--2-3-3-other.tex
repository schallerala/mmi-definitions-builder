
A minor of a graph $G$ is any graph obtained by repeated vertex deletions, edge deletions and edge contractions.
See Figure \ref{fig:K5Contraction} for an example of an edge contraction.
If an edge contraction results in a multiple edge, then we replace a multiple edge by a simple edge.
If it results in a loop, then we remove a loop.

\begin{figure}[ht]
\begin{center}
\input{Fig/K5Contraction.pdf_t}
\end{center}
\caption{An example of edge contraction.}
\label{fig:K5Contraction}
\end{figure}

Note that a minor of $G$ is not necessarily isomorphic to a subgraph of $G$.
For example, the cycle $C_3$ is a minor of $C_4$ but not its subgraph.

Similarly to the Kuratowski theorem, one direction of the Wagner theorem is easy to prove, but not the other.


