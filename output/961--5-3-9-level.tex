\documentclass[preview, multi=page, margin=5mm, class=report]{standalone}
\usepackage[utf8]{inputenc}

\usepackage{amsmath,amssymb,amsthm}
\usepackage{graphicx,color}
\usepackage{hyperref,url}
\graphicspath{{Fig/}}

\usepackage{mathtools}
\usepackage{bussproofs}
\usepackage{stackengine}
\def\ruleoffset{1pt}
\newcommand\specialvdash[2]{\mathrel{\ensurestackMath{
  \mkern2mu\rule[-\dp\strutbox]{.4pt}{\baselineskip}\stackon[\ruleoffset]{
    \stackunder[\dimexpr\ruleoffset-.5\ht\strutbox+.5\dp\strutbox]{
      \rule[\dimexpr.5\ht\strutbox-.5\dp\strutbox]{2.5ex}{.4pt}}{
        \scriptstyle #1}}{\scriptstyle#2}\mkern2mu}}
}

\usepackage[table]{xcolor}

\renewcommand\thesection{\arabic{section}}
\renewcommand\thefigure{\arabic{figure}}
\renewcommand\theequation{\arabic{equation}}

\newtheorem{dfn}{Definition}[section]
\newtheorem{thm}[dfn]{Theorem}
\newtheorem{lem}[dfn]{Lemma}
\newtheorem{cor}[dfn]{Corollary}


\theoremstyle{definition}
\newtheorem{exl}[dfn]{Example}
\newtheorem{rem}[dfn]{Remark}
\newtheorem{exc}{Exercise}[section]

\def\R{\mathbb{R}}
\def\N{\mathbb{N}}
\def\Z{\mathbb{Z}}
\def\C{\mathbb{C}}
\def\cP{\mathcal{P}}
\def\cV{\mathcal{V}}
\def\cF{\mathcal{F}}
\def\Th{\mathrm{Th}}

\renewcommand{\emptyset}{\varnothing}
\renewcommand{\phi}{\varphi}
\renewcommand{\epsilon}{\varepsilon}
\def\gcd{\operatorname{gcd}}

\def\Prop{\mathrm{PROP}}
\begin{document}
\setcounter{section}{4}
\setcounter{subsection}{0}
\setcounter{dfn}{22}

\subsection{More about partitions}
\begin{itemize}
\item 
An infinite (but fast convergent) series that computes $p_n$ was found by Ramanujan and Hardy and later improved by Rademacher.
A consequences of the latter is the asymptotics for the number of partitions:
\[
p_n \sim \frac{1}{4n\sqrt{3}} e^{\pi\sqrt{2n/3}}.
\]
\item
Ramanujan observed and later proved that
\begin{gather*}
p_{5n+4} \text{ is divisible by } 5,\\
p_{7n+5} \text{ is divisible by } 7,\\
p_{11n+6} \text{ is divisible by } 11.
\end{gather*}
\item
Erd\"os and Lehner proved that a ``random'' partition of $n$ has $\frac{2\pi}{\sqrt{6}} \sqrt{n} \log n$ summands.
\end{itemize}

Both Ramanujan and Erd\"os were extraordinary figures.
For the biography of Ramanujan see, for example,
\url{http://www-history.mcs.st-andrews.ac.uk/Biographies/Ramanujan.html}.

Further reading about partitions: \cite{AE04}.


\newpage


\end{document}