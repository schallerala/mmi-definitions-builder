\documentclass[preview, multi=page, margin=5mm, class=report]{standalone}
\usepackage[utf8]{inputenc}

\usepackage{amsmath,amssymb,amsthm}
\usepackage{graphicx,color}
\usepackage{hyperref,url}
\graphicspath{{Fig/}}

\usepackage{mathtools}
\usepackage{bussproofs}
\usepackage{stackengine}
\def\ruleoffset{1pt}
\newcommand\specialvdash[2]{\mathrel{\ensurestackMath{
  \mkern2mu\rule[-\dp\strutbox]{.4pt}{\baselineskip}\stackon[\ruleoffset]{
    \stackunder[\dimexpr\ruleoffset-.5\ht\strutbox+.5\dp\strutbox]{
      \rule[\dimexpr.5\ht\strutbox-.5\dp\strutbox]{2.5ex}{.4pt}}{
        \scriptstyle #1}}{\scriptstyle#2}\mkern2mu}}
}

\usepackage[table]{xcolor}

\renewcommand\thesection{\arabic{section}}
\renewcommand\thefigure{\arabic{figure}}
\renewcommand\theequation{\arabic{equation}}

\newtheorem{dfn}{Definition}[section]
\newtheorem{thm}[dfn]{Theorem}
\newtheorem{lem}[dfn]{Lemma}
\newtheorem{cor}[dfn]{Corollary}


\theoremstyle{definition}
\newtheorem{exl}[dfn]{Example}
\newtheorem{rem}[dfn]{Remark}
\newtheorem{exc}{Exercise}[section]

\def\R{\mathbb{R}}
\def\N{\mathbb{N}}
\def\Z{\mathbb{Z}}
\def\C{\mathbb{C}}
\def\cP{\mathcal{P}}
\def\cV{\mathcal{V}}
\def\cF{\mathcal{F}}
\def\Th{\mathrm{Th}}

\renewcommand{\emptyset}{\varnothing}
\renewcommand{\phi}{\varphi}
\renewcommand{\epsilon}{\varepsilon}
\def\gcd{\operatorname{gcd}}

\def\Prop{\mathrm{PROP}}
\begin{document}
\setcounter{section}{4}
\setcounter{subsection}{1}
\setcounter{dfn}{0}

Consider the following problem:
\begin{quote}
\emph{How many different words of length $n=k+l+m$ can be written with $k$ letters $a$, $l$ letters $b$, and $m$ letters $c$?}
\end{quote}

We have $n$ places for the letters.
First choose $k$ places where to put the letters $a$.
This can be done in $\binom{n}{k}$ different ways.
Then from $n-k$ remaining places choose $l$ places where to put the letters $b$.
This can be done in $\binom{n-k}{l}$ different ways.
Thus by the (general) product rule the number of different words is
\[
\binom{n}{k} \binom{n-k}{l} = \frac{n!}{k!(n-k)!} \frac{(n-k)!}{l!(n-k-l)!} = \frac{n!}{k!l!m!}.
\]

We might have started by choosing $l$ places for the letters $b$, and then, say, choose $m$ places for the letters $c$.
Then we would compute the product
\[
\binom{n}{l} \binom{n-l}{m}
\]
which is the same.

The following notation is used:
\[
\frac{n!}{k!l!m!} =: \binom{n}{k, l, m}.
\]

Let us now consider a more general problem and solve it in a different way.


\end{document}
