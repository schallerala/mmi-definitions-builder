\subsection{More about partitions}
\begin{itemize}
\item 
An infinite (but fast convergent) series that computes $p_n$ was found by Ramanujan and Hardy and later improved by Rademacher.
A consequences of the latter is the asymptotics for the number of partitions:
\[
p_n \sim \frac{1}{4n\sqrt{3}} e^{\pi\sqrt{2n/3}}.
\]
\item
Ramanujan observed and later proved that
\begin{gather*}
p_{5n+4} \text{ is divisible by } 5,\\
p_{7n+5} \text{ is divisible by } 7,\\
p_{11n+6} \text{ is divisible by } 11.
\end{gather*}
\item
Erd\"os and Lehner proved that a ``random'' partition of $n$ has $\frac{2\pi}{\sqrt{6}} \sqrt{n} \log n$ summands.
\end{itemize}

Both Ramanujan and Erd\"os were extraordinary figures.
For the biography of Ramanujan see, for example,
\url{http://www-history.mcs.st-andrews.ac.uk/Biographies/Ramanujan.html}.

Further reading about partitions: \cite{AE04}.


\newpage
