\documentclass[preview, multi=page, margin=5mm, class=report]{standalone}
\usepackage[utf8]{inputenc}

\usepackage{amsmath,amssymb,amsthm}
\usepackage{graphicx,color}
\usepackage{hyperref,url}
\graphicspath{{Fig/}}

\usepackage{mathtools}
\usepackage{bussproofs}
\usepackage{stackengine}
\def\ruleoffset{1pt}
\newcommand\specialvdash[2]{\mathrel{\ensurestackMath{
  \mkern2mu\rule[-\dp\strutbox]{.4pt}{\baselineskip}\stackon[\ruleoffset]{
    \stackunder[\dimexpr\ruleoffset-.5\ht\strutbox+.5\dp\strutbox]{
      \rule[\dimexpr.5\ht\strutbox-.5\dp\strutbox]{2.5ex}{.4pt}}{
        \scriptstyle #1}}{\scriptstyle#2}\mkern2mu}}
}

\usepackage[table]{xcolor}

\renewcommand\thesection{\arabic{section}}
\renewcommand\thefigure{\arabic{figure}}
\renewcommand\theequation{\arabic{equation}}

\newtheorem{dfn}{Definition}[section]
\newtheorem{thm}[dfn]{Theorem}
\newtheorem{lem}[dfn]{Lemma}
\newtheorem{cor}[dfn]{Corollary}


\theoremstyle{definition}
\newtheorem{exl}[dfn]{Example}
\newtheorem{rem}[dfn]{Remark}
\newtheorem{exc}{Exercise}[section]

\def\R{\mathbb{R}}
\def\N{\mathbb{N}}
\def\Z{\mathbb{Z}}
\def\C{\mathbb{C}}
\def\cP{\mathcal{P}}
\def\cV{\mathcal{V}}
\def\cF{\mathcal{F}}
\def\Th{\mathrm{Th}}

\renewcommand{\emptyset}{\varnothing}
\renewcommand{\phi}{\varphi}
\renewcommand{\epsilon}{\varepsilon}
\def\gcd{\operatorname{gcd}}

\def\Prop{\mathrm{PROP}}
\begin{document}
\setcounter{section}{2}
\setcounter{subsection}{2}
\setcounter{dfn}{4}

% Note that among the formulas $A_i$, $B_j$ some can be non-closed, so it is necessary to define variable assignments for their free variables.
% (We are not taking the universal closure of non-closed formulas in this case.)

Axioms of the Gentzen system are, as before, sequents $\Gamma \vdash \Delta$
with some formula occurring on both sides: $A \in \Gamma \cap \Delta$.
Axioms are valid (because no structure can at the same time satisfy and falsify $A$).
Inference rules for the logical connectives $\wedge, \vee, \to, \neg$ are the same.
There are four new rules involving the quantifiers.

\begin{align*}
&(\forall\text{ \rm left}):\
\AxiomC{$A[t/x], \forall x A, \Gamma \vdash \Delta$}
\UnaryInfC{$\forall x A, \Gamma \vdash \Delta$}
\DisplayProof
&&(\forall\text{ \rm right}):\
\AxiomC{$\Gamma \vdash A[y/x], \Delta$}
\UnaryInfC{$\Gamma \vdash \forall x A, \Delta$}
\DisplayProof
\\
&(\exists\text{ \rm left}):\
\AxiomC{$A[y/x], \Gamma \vdash \Delta$}
\UnaryInfC{$\exists x A, \Gamma \vdash \Delta$}
\DisplayProof
&&(\exists\text{ \rm right}):\
\AxiomC{$\Gamma \vdash A[t/x], \exists x A, \Delta$}
\UnaryInfC{$\Gamma \vdash \exists x A, \Delta$}
\DisplayProof
\end{align*}

Here $t$ is any term free for $x$ in $A$, and $y$ is any variable free for $x$ in $A$ and not occurring freely in the conclusion
(that is, in the sequent $\Gamma \vdash \forall x A, \Delta$ for the right $\forall$ rule and in the sequent $\exists x A, \Gamma \vdash \Delta$
for the left $\exists$ rule).
In particular, one can substitute $x$ for itself if $x$ does not occur freely in $\Gamma$ and~$\Delta$.

As before, a deduction tree is a proof tree if all of its leaves are axioms.
A sequent is called provable if there is a proof tree with this sequent as a root.


\end{document}