\begin{proof}
By Theorem \ref{thm:BooleNF} every Boolean function can be represented by a formula using $\neg, \wedge, \vee$ only.
Replace every occurrence of $\vee$ using a De Morgan law and the double negation:
\[
A \vee B \simeq \neg(\neg A \wedge \neg B).
\]
This gives an equivalent formula with connectives $\neg$ and $\wedge$ only.
One removes conjunctions in a similar way with the help of the equivalence
\[
A \wedge B \simeq \neg (\neg A \vee \neg B).
\]
\end{proof}

For example,
\begin{align*}
(\neg p \wedge q) \vee (p \wedge \neg q) &\simeq \neg(\neg(\neg p \wedge q) \wedge \neg (p \wedge \neg q))\\
&\simeq \neg(p \vee \neg q) \vee \neg(\neg p \vee q)
\end{align*}

One can achieve an absolute minimalism by introducing a logical connective $\uparrow$ with the truth table
\begin{center}
\begin{tabular}{|c|c||c|c|c|}
\hline
$A$ & $B$ & $A \uparrow B$\\\hline
$0$ & $0$ & $1$\\\hline
$0$ & $1$ & $1$\\\hline
$1$ & $0$ & $1$\\\hline
$1$ & $1$ & $0$\\\hline
\end{tabular}
\end{center}
(also called $NAND$ for obvious reasons).
