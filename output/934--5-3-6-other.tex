\documentclass[preview, multi=page, margin=5mm, class=report]{standalone}
\usepackage[utf8]{inputenc}

\usepackage{amsmath,amssymb,amsthm}
\usepackage{graphicx,color}
\usepackage{hyperref,url}
\graphicspath{{Fig/}}

\usepackage{mathtools}
\usepackage{bussproofs}
\usepackage{stackengine}
\def\ruleoffset{1pt}
\newcommand\specialvdash[2]{\mathrel{\ensurestackMath{
  \mkern2mu\rule[-\dp\strutbox]{.4pt}{\baselineskip}\stackon[\ruleoffset]{
    \stackunder[\dimexpr\ruleoffset-.5\ht\strutbox+.5\dp\strutbox]{
      \rule[\dimexpr.5\ht\strutbox-.5\dp\strutbox]{2.5ex}{.4pt}}{
        \scriptstyle #1}}{\scriptstyle#2}\mkern2mu}}
}

\usepackage[table]{xcolor}

\renewcommand\thesection{\arabic{section}}
\renewcommand\thefigure{\arabic{figure}}
\renewcommand\theequation{\arabic{equation}}

\newtheorem{dfn}{Definition}[section]
\newtheorem{thm}[dfn]{Theorem}
\newtheorem{lem}[dfn]{Lemma}
\newtheorem{cor}[dfn]{Corollary}


\theoremstyle{definition}
\newtheorem{exl}[dfn]{Example}
\newtheorem{rem}[dfn]{Remark}
\newtheorem{exc}{Exercise}[section]

\def\R{\mathbb{R}}
\def\N{\mathbb{N}}
\def\Z{\mathbb{Z}}
\def\C{\mathbb{C}}
\def\cP{\mathcal{P}}
\def\cV{\mathcal{V}}
\def\cF{\mathcal{F}}
\def\Th{\mathrm{Th}}

\renewcommand{\emptyset}{\varnothing}
\renewcommand{\phi}{\varphi}
\renewcommand{\epsilon}{\varepsilon}
\def\gcd{\operatorname{gcd}}

\def\Prop{\mathrm{PROP}}
\begin{document}
\setcounter{section}{3}
\setcounter{subsection}{6}
\setcounter{dfn}{15}

\begin{proof}
Conjugation defines a self-bijection of the set of all partitions of $n$.
Diagrams with $k$ rows are conjugate to diagrams with $k$ columns,
thus the number of diagrams of the first kind is equal to the number of diagrams of the second kind.
On the other hand, diagrams with $k$ rows correspond to partitions into $k$ parts,
and diagrams with $k$ columns correspond to partitions whose largest part is equal to $k$.
Thus we have as many partitions of the first kind as partitions of the second kind.

For partitions into at most $k$ parts and partitions with all parts $\le k$ the argument is similar.
\end{proof}


Let us present another elegant statement about partitions.
First, call a partition $\lambda$ \emph{self-conjugate} if it is conjugate to itself:
the Ferrers diagram of $\lambda$ is symmetric with respect to the northwest-southeast diagonal.
Figure \ref{fig:FerrersSelfConj} shows all self-conjugate partitions of $12$.

\begin{figure}[ht]
\begin{center}
\includegraphics[width=.8\textwidth]{ConjPart12}
\end{center}
\caption{Self-conjugate partitions of $12$.}
\label{fig:FerrersSelfConj}
\end{figure}



\end{document}