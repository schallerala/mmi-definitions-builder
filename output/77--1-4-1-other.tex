\begin{proof}
Make the balls of the same color distinguishable (by writing on them numbers, for example).
Then all $n$ balls can be arranged in $n!$ different ways.
Now apply the quotient rule.
There is a forgetful map from the set $X$ of arrangements with balls of the same color distinguishable
to the set $Y$ of arrangements where balls of the same color are undistinguishable.
What is the multiplicity of this map, that is the cardinalities of the preimages $|f^{-1}(y)|$?
The balls of the $i$-th color can be permuted amongst themselves in $k_i!$ different ways.
Permutations within every color can be performed independently, which gives us $k_1! \cdots k_m!$ elements in $Y$
all of which correspond to the same element in $X$.
Thus we have $|f^{-1}(y)| = k_1! \cdots k_m!$, and hence
\[
|Y| = \frac{|X|}{k_1! \cdots k_m!} = \frac{n!}{k_1! \cdots k_m!}.
\]
\end{proof}


