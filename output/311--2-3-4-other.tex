\documentclass[preview, multi=page, margin=5mm, class=report]{standalone}
\usepackage[utf8]{inputenc}

\usepackage{amsmath,amssymb,amsthm}
\usepackage{graphicx,color}
\usepackage{hyperref,url}
\graphicspath{{Fig/}}

\usepackage{mathtools}
\usepackage{bussproofs}
\usepackage{stackengine}
\def\ruleoffset{1pt}
\newcommand\specialvdash[2]{\mathrel{\ensurestackMath{
  \mkern2mu\rule[-\dp\strutbox]{.4pt}{\baselineskip}\stackon[\ruleoffset]{
    \stackunder[\dimexpr\ruleoffset-.5\ht\strutbox+.5\dp\strutbox]{
      \rule[\dimexpr.5\ht\strutbox-.5\dp\strutbox]{2.5ex}{.4pt}}{
        \scriptstyle #1}}{\scriptstyle#2}\mkern2mu}}
}

\usepackage[table]{xcolor}

\renewcommand\thesection{\arabic{section}}
\renewcommand\thefigure{\arabic{figure}}
\renewcommand\theequation{\arabic{equation}}

\newtheorem{dfn}{Definition}[section]
\newtheorem{thm}[dfn]{Theorem}
\newtheorem{lem}[dfn]{Lemma}
\newtheorem{cor}[dfn]{Corollary}


\theoremstyle{definition}
\newtheorem{exl}[dfn]{Example}
\newtheorem{rem}[dfn]{Remark}
\newtheorem{exc}{Exercise}[section]

\def\R{\mathbb{R}}
\def\N{\mathbb{N}}
\def\Z{\mathbb{Z}}
\def\C{\mathbb{C}}
\def\cP{\mathcal{P}}
\def\cV{\mathcal{V}}
\def\cF{\mathcal{F}}
\def\Th{\mathrm{Th}}

\renewcommand{\emptyset}{\varnothing}
\renewcommand{\phi}{\varphi}
\renewcommand{\epsilon}{\varepsilon}
\def\gcd{\operatorname{gcd}}

\def\Prop{\mathrm{PROP}}
\begin{document}
\setcounter{section}{3}
\setcounter{subsection}{4}
\setcounter{dfn}{11}

\subsection{Duality for embedded graphs}
Let $G$ be a connected plane graph.
Define a new plane graph $G^*$ as follows.
Inside every face $f$ of $G$ choose a point $f^*$.
For every edge $e$ of $G$ draw an arc $e^*$ crossing the edge $e$ and joining the points $f_1^*$ and $f_2^*$ inside the faces incident with $e$.
(If $e$ is incident to one face only, then the arc $e^*$ is a loop.)

It is possible to draw all arcs $e^*$ so that they do not intersect each other.
(Mark a point in the interior of every edge;
inside every face $f$, join the point $f^*$ to the points marked on the incident edges in a non-self-intersecting way.)

See Figure \ref{fig:DualGraph} for an example.

\begin{figure}[ht]
\begin{center}
\includegraphics[width=.6\textwidth]{DualGraph.pdf}
\end{center}
\caption{A graph and its dual.}
\label{fig:DualGraph}
\end{figure}

Different planar embeddings of the same graph may have different duals.
For example, consider the duals of the graphs on Figure \ref{fig:TwoEmbeddings}.

Let us describe those graphs whose duals have no loops and multiple edges.


\end{document}