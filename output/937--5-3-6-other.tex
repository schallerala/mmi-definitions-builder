\begin{proof}
In a self-conjugate partition join the dots as shown on Figure \ref{fig:OddDist}.
That is, take the union of the first row with the first column,
then the union of what remained of the second row with what remained of the second column, etc.
Each of these sets has an odd number of elements due to the symmetry of the diagram.
Besides, every set has less elements than the previous one.
This transforms a self-conjugate partition into a partition with distinct odd parts.

\begin{figure}[ht]
\begin{center}
\includegraphics[width=.8\textwidth]{OddDist}
\end{center}
\caption{From self-conjugate partitions to partitions into distinct odd parts: $12 = 11 + 1 = 9 + 3 = 7 + 5$.}
\label{fig:OddDist}
\end{figure}

The above transformation has an inverse.
Take any partition with distinct odd parts and ``bend'' each part in the middle.
Packing these hooks one inside another in the decreasing order produces the diagram of a self-conjugate partition.
Thus we have a bijection between the set of self-conjugate partitions and the set of partitions into distinct odd parts.
\end{proof}




