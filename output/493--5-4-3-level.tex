\subsection{Bracket-variable expressions}
Assume that we have to multiply three variables $x$, $y$, and $z$.
Here by ``multiplication'' we mean any binary operation.
If this operation is commutative and associative, then neither the order of variables nor the order of operations is important:
\[
(xy)z = x(yz) = x(zy).
\]
If the operation is associative but not commutative (as multiplication of matrices for example),
then the order of variables matters, but the order of operations does not:
\[
(xy)z = x(yz) \ne x(zy).
\]
Finally, if the operation is neither commutative nor associative (as, for example, $xy := x^y$),
then we must take care both of the order of variables and the order of operations:
\[
(xy)z \ne x(yz).
\]
In how many ways can one multiply a given sequence of variables without changing their order?
For two variables there is only one way, for three variables two: $(xy)z$ and $x(yz)$,
below are all expressions with four variables:
\[
((x_1x_2)x_3)x_4, \quad (x_1(x_2x_3))x_4, \quad (x_1x_2)(x_3x_4), \quad x_1((x_2x_3)x_4), \quad x_1(x_2(x_3x_4)).
\]
