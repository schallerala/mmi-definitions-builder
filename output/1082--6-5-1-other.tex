\documentclass[preview, multi=page, margin=5mm, class=report]{standalone}
\usepackage[utf8]{inputenc}

\usepackage{amsmath,amssymb,amsthm}
\usepackage{graphicx,color}
\usepackage{hyperref,url}
\graphicspath{{Fig/}}

\usepackage{mathtools}
\usepackage{bussproofs}
\usepackage{stackengine}
\def\ruleoffset{1pt}
\newcommand\specialvdash[2]{\mathrel{\ensurestackMath{
  \mkern2mu\rule[-\dp\strutbox]{.4pt}{\baselineskip}\stackon[\ruleoffset]{
    \stackunder[\dimexpr\ruleoffset-.5\ht\strutbox+.5\dp\strutbox]{
      \rule[\dimexpr.5\ht\strutbox-.5\dp\strutbox]{2.5ex}{.4pt}}{
        \scriptstyle #1}}{\scriptstyle#2}\mkern2mu}}
}

\usepackage[table]{xcolor}

\renewcommand\thesection{\arabic{section}}
\renewcommand\thefigure{\arabic{figure}}
\renewcommand\theequation{\arabic{equation}}

\newtheorem{dfn}{Definition}[section]
\newtheorem{thm}[dfn]{Theorem}
\newtheorem{lem}[dfn]{Lemma}
\newtheorem{cor}[dfn]{Corollary}


\theoremstyle{definition}
\newtheorem{exl}[dfn]{Example}
\newtheorem{rem}[dfn]{Remark}
\newtheorem{exc}{Exercise}[section]

\def\R{\mathbb{R}}
\def\N{\mathbb{N}}
\def\Z{\mathbb{Z}}
\def\C{\mathbb{C}}
\def\cP{\mathcal{P}}
\def\cV{\mathcal{V}}
\def\cF{\mathcal{F}}
\def\Th{\mathrm{Th}}

\renewcommand{\emptyset}{\varnothing}
\renewcommand{\phi}{\varphi}
\renewcommand{\epsilon}{\varepsilon}
\def\gcd{\operatorname{gcd}}

\def\Prop{\mathrm{PROP}}
\begin{document}
\setcounter{section}{5}
\setcounter{subsection}{1}
\setcounter{dfn}{3}


In order to describe formally what it means that a word can be derived from the start symbol, let us fix some notations and terminology.
If $A \to \beta$ is any production in $P$, and $\alpha, \gamma \in (V \cup T)^*$,
then we write $\alpha A \gamma \xRightarrow[G]{} \alpha \beta \gamma$
and say that $\alpha A \gamma$ \emph{directly derives} $\alpha\beta\gamma$ in grammar~$G$.
If $\alpha_1, \ldots, \alpha_n \in (V \cup T)^*$ are such that
\[
\alpha_1 \xRightarrow[G]{} \alpha_2, \quad \alpha_2 \xRightarrow[G]{} \alpha_3, \quad \ldots, \quad  \alpha_{n-1} \xRightarrow[G]{} \alpha_n,
\]
then we write $\alpha_1 \xRightarrow[G]{*} \alpha_n$ and say that $\alpha_1$ \emph{derives} $\alpha_n$ in $G$.
When it is clear which grammar we use, then we omit $G$ and write $\xRightarrow[]{}$ and $\xRightarrow[]{*}$, respectively.

Now we can describe the language generated by $G$ as
\[
L(G) = \{w \in T^* \mid S \xRightarrow[G]{*} w\}.
\]
A language is called a \emph{context-free language} (CFL) if it is generated by some context-free grammar.


\end{document}