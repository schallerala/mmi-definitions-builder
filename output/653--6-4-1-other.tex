\begin{proof}
The reflexivity and the symmetry are obvious.
To prove the transitivity, let $u \sim_L v$, $v \sim_L w$, and assume that $u \not\sim_L w$.
Then there is a distinguishing extension $x$ for $u$ and $w$.
Without loss of generality, $ux \in L$, $wx \notin L$.
Then if $vx \in L$, we have $v \not\sim_L w$, and if $vx \notin L$, we have $u \not\sim_L v$.
\end{proof}

An equivalence relation splits the set $\Sigma^*$ of all words into \emph{equivalence classes}:
\begin{equation}
\label{eqn:SigmaEqClasses}
\Sigma^* = S_0 \cup S_1 \cup S_2 \cup \cdots
\end{equation}
where $u \sim_L v$ if and only if $u$ and $v$ belong to the same class.
As we noticed in Example \ref{exl:LEquiv}, if $u \sim_L v$, then either $u,v \in L$ or $u,v \notin L$.
It follows that every class $S_i$ is either contained in $L$ or disjoint from $L$.
