\section{Proof theory}
We will present Gentzen's sequent calculus for predicate logic.
There are other proof theories, for example Hilbert-style systems.
These theories are equivalent to each other, which means that a formula provable within one of them is also provable within any other.
(Postfactum it follows from their soundness and completeness,
but one can describe a transformation of Hilbert proof into a Gentzen proof and vice versa in a direct way.)


\subsection{Substitutions}
Let $A$ be a predicate formula, and let $t$ be a term.
Assume that $A$ contains a free variable $x$.
Then we can define a new formula $A[t/x]$ obtained by substitution of $t$ for all free occurrences of $x$.
(We don't exclude the possibility that $x$ also occurs bound in $A$.)

Intuitively, $A[t/x]$ should represent a special case of $A$, so that for example if $A$ is true for all $x$, then $A[t/x]$ is valid.
However, some substitutions do not have this property.
Take for example
\[
A = \exists y (y < x), \quad t = y.
\]
Then we have $A[t/x] = \exists y (y < y)$.
This is false in the universe of integers, although $A$ was true.
