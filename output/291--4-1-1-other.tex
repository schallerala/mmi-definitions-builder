
One can use the same set $\cV$ of variables in all signatures.
This is a countably infinite set; we use the letters $x$, $y$, $z$, $x_1, x_2, \ldots $ etc. to denote its elements.
As function symbols we will use $f$, $g$, $h$, $f_1, f_2, \ldots$
and as predicate symbols $P$, $Q$, $R$, $P_1, P_2, \ldots$.


The strange word \emph{arity} describes the number of arguments: a function or a predicate can be unary, binary, $k$-ary, and even nullary.
The arity can be indicated by the number of dots (or other placeholders) between the brackets.
Below is an example of a signature.
\begin{center}
\begin{tabular}{ll}
$\cF:\ f(\cdot)$ & \text{one unary function symbol}\\
$\cP:\ P(\cdot, \cdot), Q(\cdot)$ & \text{one binary and one unary predicate symbol}
\end{tabular}
\end{center}
Nullary functions and nullary predicates have no arguments.

