\begin{proof}
Count the vertex-edge incidences in two ways.
From the vertices viewpoint, every vertex $v$ is incident to $\deg v$ many edges.
Thus the number of incidences is the sum of the degrees of all vertices.
From the edges viewpoint, every edge is incident to two vertices.
Thus the number of incidences is twice the number of edges.
The theorem follows.
% Cut every edge in half and count the number of half-edges.
% On one hand, every vertex $v_i$ has $\deg v_i$ half-edges.
% On the other hand, the number of half-edges is twice the number of edges.
\end{proof}
The name ``handshake lemma'' suggests a reformulation of the above argument.
In a group of people, several handshakes take place. How to count the number of handshakes?
One way to do it is to ask every person how many handshakes it made and to add all these numbers.
Since every handshake is counted twice, we have to divide the result by two.
