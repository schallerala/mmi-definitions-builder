\documentclass[preview, multi=page, margin=5mm, class=report]{standalone}
\usepackage[utf8]{inputenc}

\usepackage{amsmath,amssymb,amsthm}
\usepackage{graphicx,color}
\usepackage{hyperref,url}
\graphicspath{{Fig/}}

\usepackage{mathtools}
\usepackage{bussproofs}
\usepackage{stackengine}
\def\ruleoffset{1pt}
\newcommand\specialvdash[2]{\mathrel{\ensurestackMath{
  \mkern2mu\rule[-\dp\strutbox]{.4pt}{\baselineskip}\stackon[\ruleoffset]{
    \stackunder[\dimexpr\ruleoffset-.5\ht\strutbox+.5\dp\strutbox]{
      \rule[\dimexpr.5\ht\strutbox-.5\dp\strutbox]{2.5ex}{.4pt}}{
        \scriptstyle #1}}{\scriptstyle#2}\mkern2mu}}
}

\usepackage[table]{xcolor}

\renewcommand\thesection{\arabic{section}}
\renewcommand\thefigure{\arabic{figure}}
\renewcommand\theequation{\arabic{equation}}

\newtheorem{dfn}{Definition}[section]
\newtheorem{thm}[dfn]{Theorem}
\newtheorem{lem}[dfn]{Lemma}
\newtheorem{cor}[dfn]{Corollary}


\theoremstyle{definition}
\newtheorem{exl}[dfn]{Example}
\newtheorem{rem}[dfn]{Remark}
\newtheorem{exc}{Exercise}[section]

\def\R{\mathbb{R}}
\def\N{\mathbb{N}}
\def\Z{\mathbb{Z}}
\def\C{\mathbb{C}}
\def\cP{\mathcal{P}}
\def\cV{\mathcal{V}}
\def\cF{\mathcal{F}}
\def\Th{\mathrm{Th}}

\renewcommand{\emptyset}{\varnothing}
\renewcommand{\phi}{\varphi}
\renewcommand{\epsilon}{\varepsilon}
\def\gcd{\operatorname{gcd}}

\def\Prop{\mathrm{PROP}}
\begin{document}
\setcounter{section}{1}
\setcounter{subsection}{1}
\setcounter{dfn}{0}

The alphabet of the predicate logic consists of
\begin{itemize}
\item
variables;
\item
function symbols;
\item
predicate symbols;
\item
logical connectives $\wedge, \vee, \neg, \to, \forall, \exists$;
\item
auxiliary symbols $($ and $)$;
\item
equality symbol $=$.
\end{itemize}

Informally speaking (and as indicated in the introduction), a variable is an object,
a function is an operation with objects whose result is also an object,
and a predicate is a statement about one or several object (in other words, an operation with objects whose result is a truth value).

Compared to the propositional logic, we have two new logical connectives:
the \emph{universal quantifier} $\forall$ and the \emph{existential quantifier} $\exists$.

The equality symbol is not always included in the alphabet.
Accordingly, there are two slightly different versions of predicate logic: logic with equality and logic without equality.

Before stating the rules according to which the alphabet symbols can be combined one has to fix a \emph{signature}.
This is a list of function symbols and predicate symbols together with the number of arguments for each of them.


\end{document}
