\documentclass[preview, multi=page, margin=5mm, class=report]{standalone}
\usepackage[utf8]{inputenc}

\usepackage{amsmath,amssymb,amsthm}
\usepackage{graphicx,color}
\usepackage{hyperref,url}
\graphicspath{{Fig/}}

\usepackage{mathtools}
\usepackage{bussproofs}
\usepackage{stackengine}
\def\ruleoffset{1pt}
\newcommand\specialvdash[2]{\mathrel{\ensurestackMath{
  \mkern2mu\rule[-\dp\strutbox]{.4pt}{\baselineskip}\stackon[\ruleoffset]{
    \stackunder[\dimexpr\ruleoffset-.5\ht\strutbox+.5\dp\strutbox]{
      \rule[\dimexpr.5\ht\strutbox-.5\dp\strutbox]{2.5ex}{.4pt}}{
        \scriptstyle #1}}{\scriptstyle#2}\mkern2mu}}
}

\usepackage[table]{xcolor}

\renewcommand\thesection{\arabic{section}}
\renewcommand\thefigure{\arabic{figure}}
\renewcommand\theequation{\arabic{equation}}

\newtheorem{dfn}{Definition}[section]
\newtheorem{thm}[dfn]{Theorem}
\newtheorem{lem}[dfn]{Lemma}
\newtheorem{cor}[dfn]{Corollary}


\theoremstyle{definition}
\newtheorem{exl}[dfn]{Example}
\newtheorem{rem}[dfn]{Remark}
\newtheorem{exc}{Exercise}[section]

\def\R{\mathbb{R}}
\def\N{\mathbb{N}}
\def\Z{\mathbb{Z}}
\def\C{\mathbb{C}}
\def\cP{\mathcal{P}}
\def\cV{\mathcal{V}}
\def\cF{\mathcal{F}}
\def\Th{\mathrm{Th}}

\renewcommand{\emptyset}{\varnothing}
\renewcommand{\phi}{\varphi}
\renewcommand{\epsilon}{\varepsilon}
\def\gcd{\operatorname{gcd}}

\def\Prop{\mathrm{PROP}}
\begin{document}
\setcounter{section}{1}
\setcounter{subsection}{3}
\setcounter{dfn}{6}

\begin{proof}
The set of all valuations is non-empty.
Therefore if all valuations satisfy $A$, then there is at least one valuation that satisfies $A$.

From the truth table for $\neg$ it follows that
\[
v \vDash A \text{ if and only if } v \nvDash \neg A.
\]
It follows that $A$ is satisfied by all valuations if and only if $\neg A$ is not satisfied by any.
\end{proof}

The problem of determining whether a given proposition is satisfiable is called \emph{satisfiability problem}, abbreviated $SAT$.
The problem of determining whether a given proposition is a tautology is called \emph{tautology problem}, abbreviated $TAUT$.
The satisfiability problem is $NP$-complete, that is every non-deterministically solvable in polynomial time problem can be reduced to it.
Therefore if $SAT$ can be solved in polynomial time, then every $NP$-problem can, that is $P=NP$.
On the other hand, if $TAUT$ is not $NP$, then $P \ne NP$.
For more details, see \cite[Section 3.3.5]{Gallier}.


\end{document}
