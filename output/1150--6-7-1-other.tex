\documentclass[preview, multi=page, margin=5mm, class=report]{standalone}
\usepackage[utf8]{inputenc}

\usepackage{amsmath,amssymb,amsthm}
\usepackage{graphicx,color}
\usepackage{hyperref,url}
\graphicspath{{Fig/}}

\usepackage{mathtools}
\usepackage{bussproofs}
\usepackage{stackengine}
\def\ruleoffset{1pt}
\newcommand\specialvdash[2]{\mathrel{\ensurestackMath{
  \mkern2mu\rule[-\dp\strutbox]{.4pt}{\baselineskip}\stackon[\ruleoffset]{
    \stackunder[\dimexpr\ruleoffset-.5\ht\strutbox+.5\dp\strutbox]{
      \rule[\dimexpr.5\ht\strutbox-.5\dp\strutbox]{2.5ex}{.4pt}}{
        \scriptstyle #1}}{\scriptstyle#2}\mkern2mu}}
}

\usepackage[table]{xcolor}

\renewcommand\thesection{\arabic{section}}
\renewcommand\thefigure{\arabic{figure}}
\renewcommand\theequation{\arabic{equation}}

\newtheorem{dfn}{Definition}[section]
\newtheorem{thm}[dfn]{Theorem}
\newtheorem{lem}[dfn]{Lemma}
\newtheorem{cor}[dfn]{Corollary}


\theoremstyle{definition}
\newtheorem{exl}[dfn]{Example}
\newtheorem{rem}[dfn]{Remark}
\newtheorem{exc}{Exercise}[section]

\def\R{\mathbb{R}}
\def\N{\mathbb{N}}
\def\Z{\mathbb{Z}}
\def\C{\mathbb{C}}
\def\cP{\mathcal{P}}
\def\cV{\mathcal{V}}
\def\cF{\mathcal{F}}
\def\Th{\mathrm{Th}}

\renewcommand{\emptyset}{\varnothing}
\renewcommand{\phi}{\varphi}
\renewcommand{\epsilon}{\varepsilon}
\def\gcd{\operatorname{gcd}}

\def\Prop{\mathrm{PROP}}
\begin{document}
\setcounter{section}{7}
\setcounter{subsection}{1}
\setcounter{dfn}{1}


\begin{proof}
Let $L$ be a context-free language, and let $G = (V, T, P, S)$ be a grammar that generates $L$.
We construct a grammar $G'$ by adding to $V$ a new symbol $S'$ (which will be the new start symbol) and a new production rule:
\[
V' = V \cup \{S'\}, \quad P' = P \cup \{S' \to SS' \mid \epsilon\}.
\]
In $G'$ one can derive all words of the Kleene closure $L^*$:
\[
S' \xRightarrow[G']{} SS' \xRightarrow[G']{} SSS' \xRightarrow[G']{} \cdots \xRightarrow[G']{} SS \ldots S \xRightarrow[G']{} w_1 w_2 \ldots w_n.
\]
Vice versa, every word derived from $S'$ belongs to $L^*$.

Let now $L_1$ and $L_2$ be languages generated by context-free grammars
\[
G_1 = (V_1, T_1, P_1, S_1), \quad G_2 = (V_2, T_2, P_2, S_2).
\]
Put $T = T_1 \cup T_2$.
We want to show that the languages $L_1 \cup L_2 \subset T^*$ and $L_1L_2 \subset T^*$ are context-free.
In both cases rename the variables so that $V_1 \cap V_2 = \emptyset$
and construct a new grammar $G' = (V', T, P', S')$ with $V' = V_1 \cup V_2 \cup \{S'\}$.
It is easy to see that the production rules
\[
P' = P_1 \cup P_2 \cup \{S' \to S_1 \mid S_2\}
\]
generate the language $L_1 \cup L_2$,
and the production rules
\[
P' = P_1 \cup P_2 \cup \{S'' \to S_1S_2\}.
\]
generate $L_1L_2$.
\end{proof}


\end{document}