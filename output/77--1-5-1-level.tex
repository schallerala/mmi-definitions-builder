\section{Inclusion-exclusion formula}
\subsection{The formula}
If $A \cap B \ne \emptyset$, then
\[
|A \cup B| = |A| + |B| - |A \cap B|.
\]
Indeed, in $|A|+|B|$ we have counted all elements that belong to both $A$ and $B$ twice, see Figure \ref{fig:IncExc2}, left.
Subtracting $|A \cap B|$ makes our count correct, see Figure \ref{fig:IncExc2}, right.

\begin{figure}[ht]
\begin{center}
\input{Fig/IncExc2.pdf_t}
\end{center}
\caption{Counting the elements in the union of two sets.}
\label{fig:IncExc2}
\end{figure}

Let us now count the elements in the union of three sets $A \cup B \cup C$.
In the sum
\[
|A| + |B| + |C|
\]
every element is counted as many times as to how many sets it belongs, see Figure \ref{fig:IncExc3}, left.
Let us subtract the numbers of the elements in the pairwise intersections:
\[
|A| + |B| + |C| - |A \cap B| - |B \cap C| - |A \cap C|.
\]
Now every element that belongs to one or two sets is counted exactly once,
but the elements in $A \cap B \cap C$ are not counted at all, see Figure \ref{fig:IncExc3}, middle.
So it remains to add the number of these elements to obtain the final formula:
\[
|A \cup B \cup C| = |A| + |B| + |C| - |A \cap B| - |B \cap C| - |A \cap C| + |A \cap B \cap C|.
\]

\begin{figure}[ht]
\begin{center}
\input{Fig/IncExc3.pdf_t}
\end{center}
\caption{Counting the elements in the union of three sets.}
\label{fig:IncExc3}
\end{figure}

What will the formula for the number of elements in the union of $n$ sets look like?
The formulas for $n=2$ and $n=3$ suggest that this will be the sum of the cardinalities of all sets
minus the sum of the cardinalities of pairwise intersections plus all the triple intersections
minus all the quadruple intersections, and so on.
This conjecture is true, and we will prove it now.
