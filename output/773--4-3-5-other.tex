\documentclass[preview, multi=page, margin=5mm, class=report]{standalone}
\usepackage[utf8]{inputenc}

\usepackage{amsmath,amssymb,amsthm}
\usepackage{graphicx,color}
\usepackage{hyperref,url}
\graphicspath{{Fig/}}

\usepackage{mathtools}
\usepackage{bussproofs}
\usepackage{stackengine}
\def\ruleoffset{1pt}
\newcommand\specialvdash[2]{\mathrel{\ensurestackMath{
  \mkern2mu\rule[-\dp\strutbox]{.4pt}{\baselineskip}\stackon[\ruleoffset]{
    \stackunder[\dimexpr\ruleoffset-.5\ht\strutbox+.5\dp\strutbox]{
      \rule[\dimexpr.5\ht\strutbox-.5\dp\strutbox]{2.5ex}{.4pt}}{
        \scriptstyle #1}}{\scriptstyle#2}\mkern2mu}}
}

\usepackage[table]{xcolor}

\renewcommand\thesection{\arabic{section}}
\renewcommand\thefigure{\arabic{figure}}
\renewcommand\theequation{\arabic{equation}}

\newtheorem{dfn}{Definition}[section]
\newtheorem{thm}[dfn]{Theorem}
\newtheorem{lem}[dfn]{Lemma}
\newtheorem{cor}[dfn]{Corollary}


\theoremstyle{definition}
\newtheorem{exl}[dfn]{Example}
\newtheorem{rem}[dfn]{Remark}
\newtheorem{exc}{Exercise}[section]

\def\R{\mathbb{R}}
\def\N{\mathbb{N}}
\def\Z{\mathbb{Z}}
\def\C{\mathbb{C}}
\def\cP{\mathcal{P}}
\def\cV{\mathcal{V}}
\def\cF{\mathcal{F}}
\def\Th{\mathrm{Th}}

\renewcommand{\emptyset}{\varnothing}
\renewcommand{\phi}{\varphi}
\renewcommand{\epsilon}{\varepsilon}
\def\gcd{\operatorname{gcd}}

\def\Prop{\mathrm{PROP}}
\begin{document}
\setcounter{section}{3}
\setcounter{subsection}{5}
\setcounter{dfn}{36}


\begin{proof}[Proof of Theorem \ref{thm:ConstUndecid}]
Assume that $T$ is decidable.
Consider the set
\[
X = \{(m,n) \mid m = \sharp A(\cdot), T \vdash A[\underline{n}/\cdot]\} \subset \N^2.
\]
(Here $A(\cdot)$ means that $A$ has one free variable, and $A[t/\cdot]$ means substitution of term $t$ for this variable.)
Since $T$ is decidable, the set $X$ is recursive (there is an algorithm which decides whether $(m,n) \in X$ or not).

Now let
\[
Y = \{n \mid (n,n) \notin X\}.
\]
Clearly, since $X$ is recursive, so is $Y$.
By Theorem \ref{thm:ReprThm}, the set $Y$ is representable.
That is, there is a formula $B(y)$ such that
\begin{gather*}
n \in Y \Rightarrow PA_0 \vdash B[\underline{n}/y]\\
n \notin Y \Rightarrow PA_0 \vdash \neg B[\underline{n}/y]
\end{gather*}
Since $T \supset PA_0$, for every $C$ such that $PA_0 \vdash C$ we also have $T \vdash C$.

Now let $n = \sharp B(y)$.
Let us ask ourselves if $n \in Y$ or not. Assume $n \in Y$.
By definition of $Y$, $X$, and $n$ we have
\[
n \in Y \Rightarrow (n,n) \notin X \Rightarrow T \not\vdash B[\underline{n}/y] \Rightarrow T \vdash \neg B[\underline{n}/y].
\]
On the other hand, by definition of $B(y)$ we have
\[
n \in Y \Rightarrow T \vdash B[\underline{n}/y],
\]
which implies that $T$ is incoherent.
If we assume that $n \notin Y$, then we derive similarly that $T \vdash B[\underline{n}/y]$ and $T \vdash \neg B[\underline{n}/y]$.

Thus the decidability assumption contradicts the consistency assumption, and the theorem is proved.
\end{proof}



\end{document}