
Here are the first few terms of the sequence $p_n$, starting with $p_0$ which we by definition set to be equal to $1$:
\[
1, 1, 2, 3, 5, 7, 11, 15, 22, \ldots.
\]


% Every composition can be turned into a partition by rearranging the summands in the non-increasing order.
% On the other hand, different compositions may correspond to the same partition.
% For example, the partition $5=2+2+1$ corresponds to $3$ different compositions.
% It follows that
% \[
% p_n \le 2^{n-1}
% \]
% (That the number of compositions of $n$ from an arbitrary number of summands is $2^{n-1}$
% can be easily shown by the ``stones and sticks'' method.

Unlike for the money changing problem and for Fibonacci numbers, there is no closed formula for the number of partitions.
But there are a lot of beautiful theorems about partitions, and we will prove some of them.
Often we will be using the generating function method.
