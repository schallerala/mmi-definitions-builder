\documentclass[preview, multi=page, margin=5mm, class=report]{standalone}
\usepackage[utf8]{inputenc}

\usepackage{amsmath,amssymb,amsthm}
\usepackage{graphicx,color}
\usepackage{hyperref,url}
\graphicspath{{Fig/}}

\usepackage{mathtools}
\usepackage{bussproofs}
\usepackage{stackengine}
\def\ruleoffset{1pt}
\newcommand\specialvdash[2]{\mathrel{\ensurestackMath{
  \mkern2mu\rule[-\dp\strutbox]{.4pt}{\baselineskip}\stackon[\ruleoffset]{
    \stackunder[\dimexpr\ruleoffset-.5\ht\strutbox+.5\dp\strutbox]{
      \rule[\dimexpr.5\ht\strutbox-.5\dp\strutbox]{2.5ex}{.4pt}}{
        \scriptstyle #1}}{\scriptstyle#2}\mkern2mu}}
}

\usepackage[table]{xcolor}

\renewcommand\thesection{\arabic{section}}
\renewcommand\thefigure{\arabic{figure}}
\renewcommand\theequation{\arabic{equation}}

\newtheorem{dfn}{Definition}[section]
\newtheorem{thm}[dfn]{Theorem}
\newtheorem{lem}[dfn]{Lemma}
\newtheorem{cor}[dfn]{Corollary}


\theoremstyle{definition}
\newtheorem{exl}[dfn]{Example}
\newtheorem{rem}[dfn]{Remark}
\newtheorem{exc}{Exercise}[section]

\def\R{\mathbb{R}}
\def\N{\mathbb{N}}
\def\Z{\mathbb{Z}}
\def\C{\mathbb{C}}
\def\cP{\mathcal{P}}
\def\cV{\mathcal{V}}
\def\cF{\mathcal{F}}
\def\Th{\mathrm{Th}}

\renewcommand{\emptyset}{\varnothing}
\renewcommand{\phi}{\varphi}
\renewcommand{\epsilon}{\varepsilon}
\def\gcd{\operatorname{gcd}}

\def\Prop{\mathrm{PROP}}
\begin{document}
\setcounter{section}{3}
\setcounter{subsection}{7}
\setcounter{dfn}{18}



Using relation \eqref{eqn:RecNK}, one can compute the entries of the table $p(n, \le k)$ recursively.
First, one fills the row $k=1$ and the column $n=0$ with ones.
Then, one can fill the table row after row or column after column.
One should observe that for $k > n$ one has $p(n, \le k) = p(n, \le n)$.
If one goes column after column, then in order to fill the column for $n = i$
one marks the diagonal $n+k = i$ and computes the $(n,k)$-entry as the sum of the entry immediately above it
and the entry on the intersection of the current row and the marked diagonal.

\begin{center}
\begin{tabular}{r|rrrrrrr}
$k^{\scalebox{1}{$n$}}$\hspace{-.2cm} & 0 & 1 & 2 & 3 & 4 & 5 & 6\\
\hline
1 & 1 & 1 & 1 & 1 & 1 & 1 & 1\\
2 & 1 & 1 & 2 & 2 & 3 & 3 & 4\\
3 & 1 & 1 & 2 & 3 & 4 & 5 & 7\\
4 & 1 & 1 & 2 & 3 & 5 & 6 & 9\\
5 & 1 & 1 & 2 & 3 & 5 & 7 & 10\\
6 & 1 & 1 & 2 & 3 & 5 & 7 & 11
\end{tabular}
\end{center}

There is a recurrence which allows a much faster computation of the number of partitions.


\end{document}
