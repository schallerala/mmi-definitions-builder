\documentclass[preview, multi=page, margin=5mm, class=report]{standalone}
\usepackage[utf8]{inputenc}

\usepackage{amsmath,amssymb,amsthm}
\usepackage{graphicx,color}
\usepackage{hyperref,url}
\graphicspath{{Fig/}}

\usepackage{mathtools}
\usepackage{bussproofs}
\usepackage{stackengine}
\def\ruleoffset{1pt}
\newcommand\specialvdash[2]{\mathrel{\ensurestackMath{
  \mkern2mu\rule[-\dp\strutbox]{.4pt}{\baselineskip}\stackon[\ruleoffset]{
    \stackunder[\dimexpr\ruleoffset-.5\ht\strutbox+.5\dp\strutbox]{
      \rule[\dimexpr.5\ht\strutbox-.5\dp\strutbox]{2.5ex}{.4pt}}{
        \scriptstyle #1}}{\scriptstyle#2}\mkern2mu}}
}

\usepackage[table]{xcolor}

\renewcommand\thesection{\arabic{section}}
\renewcommand\thefigure{\arabic{figure}}
\renewcommand\theequation{\arabic{equation}}

\newtheorem{dfn}{Definition}[section]
\newtheorem{thm}[dfn]{Theorem}
\newtheorem{lem}[dfn]{Lemma}
\newtheorem{cor}[dfn]{Corollary}


\theoremstyle{definition}
\newtheorem{exl}[dfn]{Example}
\newtheorem{rem}[dfn]{Remark}
\newtheorem{exc}{Exercise}[section]

\def\R{\mathbb{R}}
\def\N{\mathbb{N}}
\def\Z{\mathbb{Z}}
\def\C{\mathbb{C}}
\def\cP{\mathcal{P}}
\def\cV{\mathcal{V}}
\def\cF{\mathcal{F}}
\def\Th{\mathrm{Th}}

\renewcommand{\emptyset}{\varnothing}
\renewcommand{\phi}{\varphi}
\renewcommand{\epsilon}{\varepsilon}
\def\gcd{\operatorname{gcd}}

\def\Prop{\mathrm{PROP}}
\begin{document}
\setcounter{section}{8}
\setcounter{subsection}{5}
\setcounter{dfn}{9}

\begin{proof}
Assume the contrary.
Then there is a Turing machine $M_j$ which accepts the language $L_d$.
By inspection of the word $w_j$ we arrive to a contradiction:
\begin{itemize}
\item
if $w_j \in L_d$, then by definition of $L_d$ the word $w_j$ is not accepted by $M_j$, which by the choice of $M_j$ means that $w_j \notin L_d$;
\item
if $w_i \notin L_d$, then by definition of $L_d$ the word $w_j$ is accepted by $M_j$, which by the choice of $M_j$ means that $w_j \in L_d$.
\end{itemize}
\end{proof}


\begin{proof}[Proof of Theorem \ref{thm:HaltingProblem}]
If $L_u$ is recursive, then there is a Turing machine $A$ which always halts and accepts only pairs $(M, w)$ from $L_u$.
Let us show that then $L_d$ is recursive, which contradicts Lemma \ref{lem:LdNotRE}.
Given a word $w$ determine the integer $i$ such that $w_i = w$.
Then determine the machine $M_i$.
Feed $(M_i, w_i)$ into $A$ and accept $w$ if and only if $A$ \emph{does not} accept $(M_i,w_i)$.
This gives an algorithm which always stops and recognizes the language $L_d$.
\end{proof}


\end{document}