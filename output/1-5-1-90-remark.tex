\documentclass[preview, multi=page, margin=5mm, class=report]{standalone}
\usepackage[utf8]{inputenc}

\usepackage{amsmath,amssymb,amsthm}
\usepackage{graphicx,color}
\usepackage{hyperref,url}
\graphicspath{{Fig/}}

\usepackage{mathtools}
\usepackage{bussproofs}
\usepackage{stackengine}
\def\ruleoffset{1pt}
\newcommand\specialvdash[2]{\mathrel{\ensurestackMath{
  \mkern2mu\rule[-\dp\strutbox]{.4pt}{\baselineskip}\stackon[\ruleoffset]{
    \stackunder[\dimexpr\ruleoffset-.5\ht\strutbox+.5\dp\strutbox]{
      \rule[\dimexpr.5\ht\strutbox-.5\dp\strutbox]{2.5ex}{.4pt}}{
        \scriptstyle #1}}{\scriptstyle#2}\mkern2mu}}
}

\usepackage[table]{xcolor}

\renewcommand\thesection{\arabic{section}}
\renewcommand\thefigure{\arabic{figure}}
\renewcommand\theequation{\arabic{equation}}

\newtheorem{dfn}{Definition}[section]
\newtheorem{thm}[dfn]{Theorem}
\newtheorem{lem}[dfn]{Lemma}
\newtheorem{cor}[dfn]{Corollary}


\theoremstyle{definition}
\newtheorem{exl}[dfn]{Example}
\newtheorem{rem}[dfn]{Remark}
\newtheorem{exc}{Exercise}[section]

\def\R{\mathbb{R}}
\def\N{\mathbb{N}}
\def\Z{\mathbb{Z}}
\def\C{\mathbb{C}}
\def\cP{\mathcal{P}}
\def\cV{\mathcal{V}}
\def\cF{\mathcal{F}}
\def\Th{\mathrm{Th}}

\renewcommand{\emptyset}{\varnothing}
\renewcommand{\phi}{\varphi}
\renewcommand{\epsilon}{\varepsilon}
\def\gcd{\operatorname{gcd}}

\def\Prop{\mathrm{PROP}}
\begin{document}
\setcounter{section}{5}
\setcounter{subsection}{1}
\setcounter{dfn}{1}

\begin{rem}
One can give an exact meaning to the words ``how many times was an element counted''.
Instead of the intersections $A_i \cap A_j$ etc. consider their indicator functions $\mathbf{1}_{A_i \cap A_j}$.
Instead of summing the cardinalities, sum these functions.
The diagrams on Figures \ref{fig:IncExc2} and \ref{fig:IncExc3} show the values of certain sums of indicator functions.
The argument in the proof of Theorem \ref{thm:IncExc} shows that
\begin{multline*}
\mathbf{1}_{|A_1 \cup \cdots \cup A_n|} = \sum_{i=1}^n \mathbf{1}_{|A_i|} - \sum_{1 \le i < j \le n} \mathbf{1}_{|A_i \cap A_j|} 
+ \sum_{1 \le i < j < k \le n} \mathbf{1}_{|A_i \cap A_j \cap A_k|} - \cdots \\
\cdots +(-1)^{n-1} \mathbf{1}_{|A_1 \cap \cdots \cap A_n|}
\end{multline*}
by comparing the values of the functions on the left and on the right at every point.
The formula for the number of elements is obtained by taking the ``integrals'' of both sides,
that is replacing each function $f$ by the number $\sum_{x \in A_1 \cup \cdots \cup A_n} f(x)$.
\end{rem}

\end{document}
