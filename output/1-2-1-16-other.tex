\documentclass[preview, multi=page, margin=5mm, class=report]{standalone}
\usepackage[utf8]{inputenc}

\usepackage{amsmath,amssymb,amsthm}
\usepackage{graphicx,color}
\usepackage{hyperref,url}
\graphicspath{{Fig/}}

\usepackage{mathtools}
\usepackage{bussproofs}
\usepackage{stackengine}
\def\ruleoffset{1pt}
\newcommand\specialvdash[2]{\mathrel{\ensurestackMath{
  \mkern2mu\rule[-\dp\strutbox]{.4pt}{\baselineskip}\stackon[\ruleoffset]{
    \stackunder[\dimexpr\ruleoffset-.5\ht\strutbox+.5\dp\strutbox]{
      \rule[\dimexpr.5\ht\strutbox-.5\dp\strutbox]{2.5ex}{.4pt}}{
        \scriptstyle #1}}{\scriptstyle#2}\mkern2mu}}
}

\usepackage[table]{xcolor}

\renewcommand\thesection{\arabic{section}}
\renewcommand\thefigure{\arabic{figure}}
\renewcommand\theequation{\arabic{equation}}

\newtheorem{dfn}{Definition}[section]
\newtheorem{thm}[dfn]{Theorem}
\newtheorem{lem}[dfn]{Lemma}
\newtheorem{cor}[dfn]{Corollary}


\theoremstyle{definition}
\newtheorem{exl}[dfn]{Example}
\newtheorem{rem}[dfn]{Remark}
\newtheorem{exc}{Exercise}[section]

\def\R{\mathbb{R}}
\def\N{\mathbb{N}}
\def\Z{\mathbb{Z}}
\def\C{\mathbb{C}}
\def\cP{\mathcal{P}}
\def\cV{\mathcal{V}}
\def\cF{\mathcal{F}}
\def\Th{\mathrm{Th}}

\renewcommand{\emptyset}{\varnothing}
\renewcommand{\phi}{\varphi}
\renewcommand{\epsilon}{\varepsilon}
\def\gcd{\operatorname{gcd}}

\def\Prop{\mathrm{PROP}}
\begin{document}
\setcounter{section}{2}
\setcounter{subsection}{1}
\setcounter{dfn}{0}

A \emph{map} $f \colon X \to Y$ is a rule that associates to every element $x \in X$ a unique element of $Y$.
The element associated to $x$ is denoted by $f(x)$.

A map can be pictured as a collection of arrows going from elements of $X$ to elements of $Y$.
At every element of $X$ one and only one arrow must start.
By contrast, at an element of $Y$ several arrows or none at all may end.

\begin{figure}[ht]
\begin{center}
\input{Fig/Map.pdf_t}
\end{center}
\caption{A map $f \colon X \to Y$.}
\label{fig:Map}
\end{figure}

A map $f \colon X \to Y$ is called
\begin{itemize}
\item
\emph{injective}, if no two different elements of $X$ are sent to the same element of $Y$: for every $x_1 \ne x_2$ we have $f(x_1) \ne f(x_2)$;
\item
\emph{surjective}, if to every element of $Y$ some element of $X$ is sent: for every $y \in Y$ there is $x \in X$ such that $f(x) = y$;
\item
\emph{bijective}, if it is injective and surjective.
\end{itemize}


\end{document}
