\documentclass[preview, margin=5mm, multi=page]{standalone}
\usepackage[utf8]{inputenc}

\usepackage{amsmath,amssymb,amsthm}
\usepackage{graphicx,color}
\usepackage{hyperref,url}
\graphicspath{{Fig/}}

\usepackage{mathtools}
\usepackage{bussproofs}
\usepackage{stackengine}
\def\ruleoffset{1pt}
\newcommand\specialvdash[2]{\mathrel{\ensurestackMath{
  \mkern2mu\rule[-\dp\strutbox]{.4pt}{\baselineskip}\stackon[\ruleoffset]{
    \stackunder[\dimexpr\ruleoffset-.5\ht\strutbox+.5\dp\strutbox]{
      \rule[\dimexpr.5\ht\strutbox-.5\dp\strutbox]{2.5ex}{.4pt}}{
        \scriptstyle #1}}{\scriptstyle#2}\mkern2mu}}
}

\usepackage[table]{xcolor}

\renewcommand\thesection{\arabic{section}}
\renewcommand\thefigure{\arabic{figure}}
\renewcommand\theequation{\arabic{equation}}

\newtheorem{dfn}{Definition}[section]
\newtheorem{thm}[dfn]{Theorem}
\newtheorem{lem}[dfn]{Lemma}
\newtheorem{cor}[dfn]{Corollary}


\theoremstyle{definition}
\newtheorem{exl}[dfn]{Example}
\newtheorem{rem}[dfn]{Remark}
\newtheorem{exc}{Exercise}[section]

\def\R{\mathbb{R}}
\def\N{\mathbb{N}}
\def\Z{\mathbb{Z}}
\def\C{\mathbb{C}}
\def\cP{\mathcal{P}}
\def\cV{\mathcal{V}}
\def\cF{\mathcal{F}}
\def\Th{\mathrm{Th}}


\renewcommand{\emptyset}{\varnothing}
\renewcommand{\phi}{\varphi}
\renewcommand{\epsilon}{\varepsilon}
\def\gcd{\operatorname{gcd}}

\def\Prop{\mathrm{PROP}}



%opening
\title{{Lecture notes for the 2020/21 lectures}\\
$ $\\
$ $\\ \textsc{
Mathematical methods for Computer Science I \& II\\
and\\
Discrete Mathematics I \& II\\ }
$ $\\
$ $\\
$ $\\
$ $\\
University of Fribourg\\ Livio Liechti
$ $\\
$ $\\
$ $\\
$ $\\
$ $\\
$ $\\
$ $\\}
\date{ }

\author{Lecture notes written by Ivan Izmestiev for his 2018/19 lectures}


\begin{document}
\setcounter{section}{2}
\setcounter{subsection}{1}
\setcounter{dfn}{0}

\subsection{Definition and examples}
A regular expression is defined recursively.
The basic building blocks are the following.
\begin{enumerate}
\item
$\emptyset$ is a regular expression and denotes the language $\emptyset$.
\item
$\epsilon$ is a regular expression and denotes the language $\{\epsilon\}$.
\item
$a$ is a regular expression for every $a \in \Sigma$ and denotes the language $\{a\}$.
\end{enumerate}
Don't confuse the empty language $\emptyset$ and the language $\{\epsilon\}$ consisting of an empty word.


% \begin{figure}[ht]
% \begin{center}
% \includegraphics{EmptyEmpty.pdf}
% \end{center}
% \caption{NFAs for the languages $\emptyset$ and $\{\epsilon\}$.}
% \label{fig:EmptyEmpty}
% \end{figure}

Sometimes one uses the boldface $\bf{a}$ to denote the language $\{a\}$.
We will use the same symbol $a$.

From these building blocks one constructs more complex regular expressions by using the following operations.
If $r$ and $s$ are regular expressions denoting the languages $R$ and $S$ respectively, then
\begin{enumerate}
\item
$(r+s)$ is a regular expression and denotes the language $R \cup S$;
\item
$(rs)$ is a regular expression and denotes the language $RS = \{uv \mid u \in R, v \in S\}$;
\item
$r^*$ is a regular expression and denotes the language $R^* = \cup_{i=0}^\infty R^i$, where $R^i = \underbrace{RR \cdots R}_{i}$
(the \emph{Kleene closure} of language $R$).
\end{enumerate}


\end{document}