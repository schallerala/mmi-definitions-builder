

One can omit some of the brackets in regular expressions by adopting the convention that $*$ precedes the concatenation, and the concatenation precedes the sum.
For example, $((0(1^*))+0)$ may be written as $01^* + 0$, and we have
\[
01^* + 0 = \{0, 01, 011, 0111, \ldots\}.
\]

Two regular expressions are called equivalent if they describe the same language.
Here are some simple equivalences:
\[
(rs)t \sim rs(t), \quad (r+s)t \sim rs + rt.
\]
Instead of the equivalence sign we will use the equality sign to denote the equivalence of regular expressions.
For example,
\[
01^* + 0 = 01^*, \quad \emptyset r = \emptyset, \quad \epsilon r = r.
\]




Recall that a language is called regular if there is a finite automaton (DFA, NFA, or $\epsilon$-NFA, which does not matter, as we have shown)
that accepts this language.
The main theorem is the following.