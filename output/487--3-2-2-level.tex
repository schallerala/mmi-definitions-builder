\subsection{A Hilbert system}
A Hilbert-style deductive system has a single inference rule, the so-called \emph{modus ponens}:
\[
A, A \to B \vdash B
\]
(If $A$ is provable and $A \to B$ is provable, then $B$ is provable.)
We will abbreviate modus ponens by MP.

There are many different axyom systems.
Here is one of them, the third \L{}ukasiewicz' system.

\begin{align*}
&\vdash A \to (B \to A)\\
&\vdash (A \to (B \to C)) \to ((A \to B) \to (A \to C))\\
&\vdash (\neg A \to \neg B) \to (B \to A)
\end{align*}

Here $A$, $B$, and $C$ may be any propositional formulas.
Thus, in fact, we have infinitely many axioms (axiom \emph{instances}) that can be obtained from the above axiom \emph{schemata}
by substituting for $A$, $B$, and $C$ some particular formulas.
