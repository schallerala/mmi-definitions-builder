\begin{proof}
The proof is done by looking at the truth tables of the logical connectives.
Let us prove, for example, the ($\wedge$ right) rule:
\begin{prooftree}
\AxiomC{$\Gamma \vdash A, \Delta$}
\AxiomC{$\Gamma \vdash B, \Delta$}
\BinaryInfC{$\Gamma \vdash A \wedge B, \Delta$}
\end{prooftree}
A valuation $v$ falsifies the conclusion if and only if
it satisfies all propositions in $\Gamma$, falsifies $A \wedge B$, and falsifies all propositions in $\Delta$.
But $v$ falsifies $A \wedge B$ if and only if it either falsifies $A$ or falsifies $B$.
Thus $v$ falsifies the conclusion if and only if it
\begin{itemize}
\item
satisfies $\Gamma$ and falsifies $A$ and $\Delta$, or
\item
satisfies $\Gamma$ and falsifies $B$ and $\Delta$.
\end{itemize}
These conditions are the premises of the ($\wedge$ right) rule, therefore the statement of the lemma holds for this rule.
Similar arguments work for all of the other rules.
\end{proof}

For any proposition $A$ one can form a sequent $\vdash A$, which is falsifiable, respectively valid, if and only if
$A$ is falsifiable, respectively valid (a tautology).
Thus if we learn to prove all valid sequents, then we will be able to prove all tautologies.


