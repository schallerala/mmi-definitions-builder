\begin{proof}
Conjugation defines a self-bijection of the set of all partitions of $n$.
Diagrams with $k$ rows are conjugate to diagrams with $k$ columns,
thus the number of diagrams of the first kind is equal to the number of diagrams of the second kind.
On the other hand, diagrams with $k$ rows correspond to partitions into $k$ parts,
and diagrams with $k$ columns correspond to partitions whose largest part is equal to $k$.
Thus we have as many partitions of the first kind as partitions of the second kind.

For partitions into at most $k$ parts and partitions with all parts $\le k$ the argument is similar.
\end{proof}


Let us present another elegant statement about partitions.
First, call a partition $\lambda$ \emph{self-conjugate} if it is conjugate to itself:
the Ferrers diagram of $\lambda$ is symmetric with respect to the northwest-southeast diagonal.
Figure \ref{fig:FerrersSelfConj} shows all self-conjugate partitions of $12$.

\begin{figure}[ht]
\begin{center}
\includegraphics[width=.8\textwidth]{ConjPart12}
\end{center}
\caption{Self-conjugate partitions of $12$.}
\label{fig:FerrersSelfConj}
\end{figure}

