\documentclass[preview, multi=page, margin=5mm, class=report]{standalone}
\usepackage[utf8]{inputenc}

\usepackage{amsmath,amssymb,amsthm}
\usepackage{graphicx,color}
\usepackage{hyperref,url}
\graphicspath{{Fig/}}

\usepackage{mathtools}
\usepackage{bussproofs}
\usepackage{stackengine}
\def\ruleoffset{1pt}
\newcommand\specialvdash[2]{\mathrel{\ensurestackMath{
  \mkern2mu\rule[-\dp\strutbox]{.4pt}{\baselineskip}\stackon[\ruleoffset]{
    \stackunder[\dimexpr\ruleoffset-.5\ht\strutbox+.5\dp\strutbox]{
      \rule[\dimexpr.5\ht\strutbox-.5\dp\strutbox]{2.5ex}{.4pt}}{
        \scriptstyle #1}}{\scriptstyle#2}\mkern2mu}}
}

\usepackage[table]{xcolor}

\renewcommand\thesection{\arabic{section}}
\renewcommand\thefigure{\arabic{figure}}
\renewcommand\theequation{\arabic{equation}}

\newtheorem{dfn}{Definition}[section]
\newtheorem{thm}[dfn]{Theorem}
\newtheorem{lem}[dfn]{Lemma}
\newtheorem{cor}[dfn]{Corollary}


\theoremstyle{definition}
\newtheorem{exl}[dfn]{Example}
\newtheorem{rem}[dfn]{Remark}
\newtheorem{exc}{Exercise}[section]

\def\R{\mathbb{R}}
\def\N{\mathbb{N}}
\def\Z{\mathbb{Z}}
\def\C{\mathbb{C}}
\def\cP{\mathcal{P}}
\def\cV{\mathcal{V}}
\def\cF{\mathcal{F}}
\def\Th{\mathrm{Th}}

\renewcommand{\emptyset}{\varnothing}
\renewcommand{\phi}{\varphi}
\renewcommand{\epsilon}{\varepsilon}
\def\gcd{\operatorname{gcd}}

\def\Prop{\mathrm{PROP}}
\begin{document}
\setcounter{section}{3}
\setcounter{subsection}{2}
\setcounter{dfn}{9}

\begin{proof}
Observe that in a plane graph with $\ge 3$ vertices the degree of every face is at least $3$.
Indeed, the only way for a face of a connected graph to have degree $2$ is to enclose an edge, in which case the graph has two vertices and one edge.
A face cannot have degree $1$. And if a face of a connected graph has degree $0$, then the graph consists of a single vertex.

Then from Theorem \ref{thm:DualHandshake} and Euler's formula we get the inequality
\[
2 |E| = \sum_{f \in F} \deg f \ge 3 |F| = 3 (2 - |V| + |E|),
\]
which implies $|E| \le 3|V| - 6$.
The equality takes place only if the $\deg f = 3$ for all $f$, that is if all faces are triangles.
\end{proof}

Theorem \ref{thm:3V-6} implies that the graph $K_5$ is not planar.
Indeed, it has $5$ vertices and $10 > 3\cdot 5 - 6$ edges.
This proof of non-planarity of $K_5$ looks very nice and seems to avoid the intricacies of Jordan's curve theorem.
The simplicity is deceiving: Jordan's curve theorem is needed in the proof of Euler's formula (when we say that the cycle separates the plane).

An attempt to prove the non-planarity of $K_{3,3}$ in the same way fails:
this graph has $6$ vertices and $9 \le 3 \cdot 6 - 6$ edges.
Note however that a planar embedding of $K_{3,3}$ (if it exists) has no faces of degree $3$ (a bipartite graph contains no odd cycles).
For any plane graph without triangles we have
\[
2 |E| = \sum_{f \in F} \deg f \ge 4 |F| = 4 (2 - |V| + |E|),
\]
which implies $|E| \le 2|V| - 4$.
Since $K_{3,3}$ does not satisfy this inequality: $9 > 2 \cdot 6 - 4$, it is not planar.





\end{document}
