\documentclass[preview, multi=page, margin=5mm, class=report]{standalone}
\usepackage[utf8]{inputenc}

\usepackage{amsmath,amssymb,amsthm}
\usepackage{graphicx,color}
\usepackage{hyperref,url}
\graphicspath{{Fig/}}

\usepackage{mathtools}
\usepackage{bussproofs}
\usepackage{stackengine}
\def\ruleoffset{1pt}
\newcommand\specialvdash[2]{\mathrel{\ensurestackMath{
  \mkern2mu\rule[-\dp\strutbox]{.4pt}{\baselineskip}\stackon[\ruleoffset]{
    \stackunder[\dimexpr\ruleoffset-.5\ht\strutbox+.5\dp\strutbox]{
      \rule[\dimexpr.5\ht\strutbox-.5\dp\strutbox]{2.5ex}{.4pt}}{
        \scriptstyle #1}}{\scriptstyle#2}\mkern2mu}}
}

\usepackage[table]{xcolor}

\renewcommand\thesection{\arabic{section}}
\renewcommand\thefigure{\arabic{figure}}
\renewcommand\theequation{\arabic{equation}}

\newtheorem{dfn}{Definition}[section]
\newtheorem{thm}[dfn]{Theorem}
\newtheorem{lem}[dfn]{Lemma}
\newtheorem{cor}[dfn]{Corollary}


\theoremstyle{definition}
\newtheorem{exl}[dfn]{Example}
\newtheorem{rem}[dfn]{Remark}
\newtheorem{exc}{Exercise}[section]

\def\R{\mathbb{R}}
\def\N{\mathbb{N}}
\def\Z{\mathbb{Z}}
\def\C{\mathbb{C}}
\def\cP{\mathcal{P}}
\def\cV{\mathcal{V}}
\def\cF{\mathcal{F}}
\def\Th{\mathrm{Th}}

\renewcommand{\emptyset}{\varnothing}
\renewcommand{\phi}{\varphi}
\renewcommand{\epsilon}{\varepsilon}
\def\gcd{\operatorname{gcd}}

\def\Prop{\mathrm{PROP}}
\begin{document}
\setcounter{section}{5}
\setcounter{subsection}{2}
\setcounter{dfn}{8}

\begin{proof}
Let $G$ be a context-free grammar possibly containing $\epsilon$- and unit productions.
We construct a grammar $G'$ without $\epsilon$-productions such that $L(G') = L(G) \setminus \{\epsilon\}$.

First, identify \emph{nullable} variables, those which derive $\epsilon$.
This is done recursively.
Initialize the set of nullable variables by those $A$ for which there is a production $(A \to \epsilon) \in P$.
The recursion step adds to the set of nullable variables those $B$ for which $(B \to C_1 \cdots C_k) \in P$
and all $C_1, \ldots C_k$ are nullable.
As soon as this recursion does not find new nullable variables, the algorithm stops.

Second, remove from $P$ all $\epsilon$-productions $A \to \epsilon$ and add new productions in the following way.
Let $A \to X_1 \cdots X_n$ be a production with some of $X_i$ nullable variables
(recall that $X_i$ can stand for a variable symbol as well as for a terminal symbol).
We add all productions of the form $A \to X_1 \widehat{\cdots} X_n$,
where $\widehat{\cdots}$ means that we remove any subset of nullable variables
(with one exception: if all $X_1, \ldots, X_n$ are nullable,
then we do not add the production $A \to \epsilon$ obtained by removing all nullable variables).
That is, if $m$ symbols among $X_1, \ldots X_n$ are nullable variables, then the production $A \to X_1 \cdots X_n$
gives rise to $2^m$ productions if $m < n$ and to $2^n - 1$ productions if $m=n$.

It can be checked that the new set of productions allows to derive all words (except $\epsilon$) which were derivable
with the initial set of productions, and only those words.

Now we construct a grammar $G''$ equivalent to $G'$ but without unit productions.
Call a pair $(A, B)$ \emph{unit pair}, if $A \xRightarrow[]{*} B$.
The set of all unit pairs can be found recursively.

Remove all unit productions, and for each unit pair $(A, B)$ and each non-unit production $B \to \alpha$ add the production $A \to \alpha$.
Every word generated by the grammar $G''$ is also generated by $G'$:
the new direct productions $A \xRightarrow[G'']{} \alpha$ are compositions of two old productions
$\alpha A \beta \xRightarrow[G']{} \alpha B \beta \xRightarrow[G']{} \alpha$.
Every word generated by $G'$ is generated by $G''$: a series of unit productions must always end with a non-unit production,
so if we had $\alpha A \gamma \xRightarrow[G']{*} \alpha B \gamma \xRightarrow[G']{} \alpha\beta\gamma$,
then we have $\alpha A \gamma \xRightarrow[G''] \alpha\beta\gamma$.
Thus the new set of productions generates the same language as before.
\end{proof}
It is important to remove first the $\epsilon$-productions and then the unit productions.
If first the unit, and then $\epsilon$-productions are removed, then the result might contain unit productions.


\end{document}
