\documentclass[preview, multi=page, margin=5mm, class=report]{standalone}
\usepackage[utf8]{inputenc}

\usepackage{amsmath,amssymb,amsthm}
\usepackage{graphicx,color}
\usepackage{hyperref,url}
\graphicspath{{Fig/}}

\usepackage{mathtools}
\usepackage{bussproofs}
\usepackage{stackengine}
\def\ruleoffset{1pt}
\newcommand\specialvdash[2]{\mathrel{\ensurestackMath{
  \mkern2mu\rule[-\dp\strutbox]{.4pt}{\baselineskip}\stackon[\ruleoffset]{
    \stackunder[\dimexpr\ruleoffset-.5\ht\strutbox+.5\dp\strutbox]{
      \rule[\dimexpr.5\ht\strutbox-.5\dp\strutbox]{2.5ex}{.4pt}}{
        \scriptstyle #1}}{\scriptstyle#2}\mkern2mu}}
}

\usepackage[table]{xcolor}

\renewcommand\thesection{\arabic{section}}
\renewcommand\thefigure{\arabic{figure}}
\renewcommand\theequation{\arabic{equation}}

\newtheorem{dfn}{Definition}[section]
\newtheorem{thm}[dfn]{Theorem}
\newtheorem{lem}[dfn]{Lemma}
\newtheorem{cor}[dfn]{Corollary}


\theoremstyle{definition}
\newtheorem{exl}[dfn]{Example}
\newtheorem{rem}[dfn]{Remark}
\newtheorem{exc}{Exercise}[section]

\def\R{\mathbb{R}}
\def\N{\mathbb{N}}
\def\Z{\mathbb{Z}}
\def\C{\mathbb{C}}
\def\cP{\mathcal{P}}
\def\cV{\mathcal{V}}
\def\cF{\mathcal{F}}
\def\Th{\mathrm{Th}}

\renewcommand{\emptyset}{\varnothing}
\renewcommand{\phi}{\varphi}
\renewcommand{\epsilon}{\varepsilon}
\def\gcd{\operatorname{gcd}}

\def\Prop{\mathrm{PROP}}
\begin{document}
\setcounter{section}{1}
\setcounter{subsection}{2}
\setcounter{dfn}{2}


Once the truth values of all proposition symbols are known, one can determine the truth value of any propositional formula.
This is done with the help of the truth tables for logical connectives given below.

\begin{center}
\begin{tabular}{|c||c|}
\hline
$A$ & $\neg A$\\\hline
$0$ & $1$\\\hline
$1$ & $0$\\\hline
\end{tabular}
\hspace{2cm}
\begin{tabular}{|c|c||c|c|c|}
\hline
$A$ & $B$ & $A \wedge B$ & $A \vee B$ & $A \to B$\\\hline
$0$ & $0$ & $0$ & $0$ & $1$\\\hline
$0$ & $1$ & $0$ & $1$ & $1$\\\hline
$1$ & $0$ & $0$ & $1$ & $0$\\\hline
$1$ & $1$ & $1$ & $1$ & $1$\\\hline
\end{tabular}
\end{center}

Here $A$ and $B$ are arbitrary propositions.
The tables tell you the truth values of $\neg A$, $A \wedge B$, $A \vee B$, $A \to B$ once the truth values of $A$ and $B$ are known.
They allow to write the truth table for any propositional formula recursively by ``climbing'' the parse tree.


\end{document}