\documentclass[preview, margin=5mm, multi=page]{standalone}
\usepackage[utf8]{inputenc}

\usepackage{amsmath,amssymb,amsthm}
\usepackage{graphicx,color}
\usepackage{hyperref,url}
\graphicspath{{Fig/}}

\usepackage{mathtools}
\usepackage{bussproofs}
\usepackage{stackengine}
\def\ruleoffset{1pt}
\newcommand\specialvdash[2]{\mathrel{\ensurestackMath{
  \mkern2mu\rule[-\dp\strutbox]{.4pt}{\baselineskip}\stackon[\ruleoffset]{
    \stackunder[\dimexpr\ruleoffset-.5\ht\strutbox+.5\dp\strutbox]{
      \rule[\dimexpr.5\ht\strutbox-.5\dp\strutbox]{2.5ex}{.4pt}}{
        \scriptstyle #1}}{\scriptstyle#2}\mkern2mu}}
}

\usepackage[table]{xcolor}

\renewcommand\thesection{\arabic{section}}
\renewcommand\thefigure{\arabic{figure}}
\renewcommand\theequation{\arabic{equation}}

\newtheorem{dfn}{Definition}[section]
\newtheorem{thm}[dfn]{Theorem}
\newtheorem{lem}[dfn]{Lemma}
\newtheorem{cor}[dfn]{Corollary}


\theoremstyle{definition}
\newtheorem{exl}[dfn]{Example}
\newtheorem{rem}[dfn]{Remark}
\newtheorem{exc}{Exercise}[section]

\def\R{\mathbb{R}}
\def\N{\mathbb{N}}
\def\Z{\mathbb{Z}}
\def\C{\mathbb{C}}
\def\cP{\mathcal{P}}
\def\cV{\mathcal{V}}
\def\cF{\mathcal{F}}
\def\Th{\mathrm{Th}}


\renewcommand{\emptyset}{\varnothing}
\renewcommand{\phi}{\varphi}
\renewcommand{\epsilon}{\varepsilon}
\def\gcd{\operatorname{gcd}}

\def\Prop{\mathrm{PROP}}



%opening
\title{{Lecture notes for the 2020/21 lectures}\\
$ $\\
$ $\\ \textsc{
Mathematical methods for Computer Science I \& II\\
and\\
Discrete Mathematics I \& II\\ }
$ $\\
$ $\\
$ $\\
$ $\\
University of Fribourg\\ Livio Liechti
$ $\\
$ $\\
$ $\\
$ $\\
$ $\\
$ $\\
$ $\\}
\date{ }

\author{Lecture notes written by Ivan Izmestiev for his 2018/19 lectures}


\begin{document}
\setcounter{section}{1}
\setcounter{subsection}{4}
\setcounter{dfn}{13}

\begin{proof}
Exercise.
% Induction on $k$. For $k=1$, a walk of length $1$ is just an edge, and the statement holds by definition of the adjacency matrix.
% 
% The induction step: assume the statement holds for $k-1$, prove it for $k$.
% Take any two vertices $v_i, v_j \in V$.
% A walk starts with the first step.
% Every walk of length $k$ from $v_i$ to $v_j$ consists of an edge from $v_i$ to some neighbor $v_l$
% followed by a walk of length $k-1$ from $v_l$ to $v_j$.
% Hence, in order to compute the number of walks of length $k$ from $v_i$ to $v_j$
% we have to sum the number of walks of length $k-1$ from all neighbors of $v_i$ to $v_j$.
% Since by the induction assumption there are $a^{(k-1)}_{lj}$ walks of length $k-1$ from $v_l$ to $v_j$, this sum is equal to
% \[
% \sum_{\{v_i,v_l\} \in E} a^{(k-1)}_{lj} = \sum_{l=1}^n a_{il} a^{(k-1)}_{lj},
% \]
% which is exactly the $(i,j)$-entry in the matrix $A \cdot A^{k-1} = A^k$.
\end{proof}

The reader is invited to define analogs of the adjacency matrix for directed and weighted graphs.



\end{document}
