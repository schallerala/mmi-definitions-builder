This and many other relations between Fibonacci numbers can be proved by induction,
sometimes in a not very straightforward way.
When the Binet formula is used, the proof consists of simple algebraic manipulations.

\begin{proof}
We have $a_n = \frac{1}{\sqrt{5}}(\lambda_1^n - \lambda_2^n)$ with $\lambda_i$ as in \eqref{eqn:FibRoots}.
Taking into account that $\lambda_1\lambda_2 = -1$, one computes
\[
a_n^2 = \frac15(\lambda_1^{2n} - 2 \lambda_1^n \lambda_2^n + \lambda_2^{2n}) = \frac15(\lambda_1^{2n} + \lambda_2^{2n} - 2(-1)^n).
\]
On the other hand,
\begin{multline*}
a_{n-1} a_{n+1} = \frac15(\lambda_1^{n-1} - \lambda_2^{n-1})(\lambda_1^{n+1} - \lambda_2^{n+1})\\
= \frac15(\lambda_1^{2n} - \lambda_1^{n-1}\lambda_2^{n+1} - \lambda_2^{n-1}\lambda_1^{n+1} + \lambda_2^{2n})\\
= \frac15(\lambda_1^{2n} + \lambda_2^{2n} - \lambda_1^{n-1}\lambda_2^{n-1}(\lambda_1^2 + \lambda_2^2))
\end{multline*}
One computes
\[
\lambda_1^2 + \lambda_2^2 = \frac{1 + 2\sqrt{5} + 5}4 + \frac{1 - 2\sqrt{5} + 5}4 = 3,
\]
which implies
\[
a_{n-1} a_{n+1} = \frac15(\lambda_1^{2n} + \lambda_2^{2n} - 3(-1)^{n-1}) = a_n^2 + (-1)^n.
\]
\end{proof}



