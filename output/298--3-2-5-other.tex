A parent vertex in a rooted tree is a vertex with the out-degree $\ge 1$ in the canonical orientation of the edges, see Figure \ref{fig:RootedTreeOrient}.
Vertices of out-degree $0$ will be called \emph{leaves} of a rooted tree.
Since the in-degree of every non-root vertex is $1$, non-root leaves are leaves in the usual sense.
However, the root is a leaf if and only if the tree has only one vertex.

A standalone sequent is a simplest deduction tree (tree with one vertex).
Next to it are the deduction trees copied from the inference rules as shown in Figure \ref{fig:DeductionTree}.
Note that a deduction tree the root is at the bottom, and the children of every vertex are situated above the vertex.

\begin{figure}[ht]
\begin{center}
\raisebox{1cm}{
\AxiomC{$\Gamma \vdash A, \Delta$}
\AxiomC{$\Gamma \vdash B, \Delta$}
\BinaryInfC{$\Gamma \vdash A \wedge B, \Delta$}
\DisplayProof
}
\hspace{1cm}
\input{Fig/DeductionTree.pdf_t}
\end{center}
\caption{The ($\wedge$: right) inference rule as a deduction tree.}
\label{fig:DeductionTree}
\end{figure}
