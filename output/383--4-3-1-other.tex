
The first-order theory of graphs is incomplete, and this is good: one has many different graphs with different properties.
But if we want to know everything about the arithmetics of $\N$, the natural numbers, then we need a complete theory.
The theory should contain some axioms for addition and multiplication and maybe some other axioms,
so that every sentence has a well-defined truth value
(that is, all structures that satisfy our axioms assign to it the same value).
For example, the question ``Can every integer greater than $2$ be expressed as the sum of two primes?'' should have a definite answer.
%If our set of axioms is incomplete, then there is a sentence which is true in one model and false in another model.

One might use the result of the above exercise: if we assume that $\N$ is something objectively real,
then there is the corresponding complete theory $T_{\N}$.
But this theory has ``too many'' axioms, and we have no explicit description for it, apart from saying ``what is true, is true''.
%An obvious requirement to a theory is that it should be possible to describe its axioms in some explicit way.
G\"odel's incompleteness theorem says that there is no complete theory of $\N$
such that for every sentence one can decide in finite time is it an axiom of the theory or not.

Before we say more about the incompleteness theorem, we must discuss proofs, a constructive way to derive consequences of a theory.


