\begin{proof}
The right hand side is equal to
\begin{multline*}
(1 + x^9 + x^{18} + \cdots)(1 + x^{17} + x^{34} + \cdots)(1 + x^{31} + x^{62} + \cdots)\\
= \sum_{k,l,m\ge 0} x^{9k+17l+31m}.
\end{multline*}
(We pick $x^{9k}$ from the first brackets, $x^{17l}$ from the second brackets, and $x^{31m}$ from the third brackets).
Thus the coefficient at $x^n$ is the number of solutions of the equation
\[
9k + 17l + 31m = n, \quad k, l, m \ge 0,
\]
that is $a_n$.
\end{proof}

One can represent the quotient $\frac{1}{(1-x^9)(1-x^{17})(1-x^{31})}$ as the sum of partial fractions.
For this one has to factorize $1-x^9$.
Complex roots of unity will appear.
It is ultimately possible (but very time-consuming) to write a closed formula for the number of ways to change $n$ dollars.
What is easier to prove is the asymptotics of the number $a_n$:
\[
a_n \sim \frac{n^2}{9 \cdot 17 \cdot 31} = \frac{n^2}{4743}, \text{ that is } \lim_{n \to 0} \frac{a_n}{n^2} = \frac{1}{4743}.
\]



