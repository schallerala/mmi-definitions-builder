\documentclass[preview, multi=page, margin=5mm, class=report]{standalone}
\usepackage[utf8]{inputenc}

\usepackage{amsmath,amssymb,amsthm}
\usepackage{graphicx,color}
\usepackage{hyperref,url}
\graphicspath{{Fig/}}

\usepackage{mathtools}
\usepackage{bussproofs}
\usepackage{stackengine}
\def\ruleoffset{1pt}
\newcommand\specialvdash[2]{\mathrel{\ensurestackMath{
  \mkern2mu\rule[-\dp\strutbox]{.4pt}{\baselineskip}\stackon[\ruleoffset]{
    \stackunder[\dimexpr\ruleoffset-.5\ht\strutbox+.5\dp\strutbox]{
      \rule[\dimexpr.5\ht\strutbox-.5\dp\strutbox]{2.5ex}{.4pt}}{
        \scriptstyle #1}}{\scriptstyle#2}\mkern2mu}}
}

\usepackage[table]{xcolor}

\renewcommand\thesection{\arabic{section}}
\renewcommand\thefigure{\arabic{figure}}
\renewcommand\theequation{\arabic{equation}}

\newtheorem{dfn}{Definition}[section]
\newtheorem{thm}[dfn]{Theorem}
\newtheorem{lem}[dfn]{Lemma}
\newtheorem{cor}[dfn]{Corollary}


\theoremstyle{definition}
\newtheorem{exl}[dfn]{Example}
\newtheorem{rem}[dfn]{Remark}
\newtheorem{exc}{Exercise}[section]

\def\R{\mathbb{R}}
\def\N{\mathbb{N}}
\def\Z{\mathbb{Z}}
\def\C{\mathbb{C}}
\def\cP{\mathcal{P}}
\def\cV{\mathcal{V}}
\def\cF{\mathcal{F}}
\def\Th{\mathrm{Th}}

\renewcommand{\emptyset}{\varnothing}
\renewcommand{\phi}{\varphi}
\renewcommand{\epsilon}{\varepsilon}
\def\gcd{\operatorname{gcd}}

\def\Prop{\mathrm{PROP}}
\begin{document}
\setcounter{section}{3}
\setcounter{subsection}{0}
\setcounter{dfn}{17}


It follows that one can extract $q$-th root from any formal power series with non-zero constant term.

\begin{proof}[Proof of Theorem \ref{thm:Vandermonde}]
Step 1. If $\alpha = m$ and $\beta = n$ are positive integers, then
\[
\binom{m + n}{k} = \sum_{i=0}^k \binom{m}{i} \binom{n}{k-i}
\]
can be proved by a combinatorial argument: $\binom{m+n}{k}$ is the number of different choices of $k$ elements from the set $\{1, \ldots, m+n\}$.
To choose $k$ elements, one has to choose $i$ elements among $\{1, \ldots, m\}$ and $k-i$ elements among $\{m+1, \ldots, m+n\}$ for some $i$ between $0$ and $k$.
The number of such choices is $\binom{m}{i} \binom{n}{k-i}$.
Summing over $i$ we obtain the desired formula.

Step 2. Let us prove
\[
\binom{\alpha + n}{k} = \sum_{i=0}^k \binom{\alpha}{i} \binom{n}{k-i}
\]
for all $\alpha \in \R$ and all positive integers $n$.
For fixed $n$ and $k$ the left hand side of the above formula is a polynomial in $\alpha$ of degree $k$;
the right hand side is also a polynomial in $\alpha$ of degree $k$.
Both polynomials take equal values at $\alpha = m$ for all positive integers $m$.
It follows that the polynomials are identical
(if two polynomials of degree $k$ coincide at $k+1$ points, then their difference is a polynomial of degree $\le k$ with $> k$ roots, hence identically zero).

Step 3. Finally let us prove
\[
\binom{\alpha + \beta}{k} = \sum_{i=0}^k \binom{\alpha}{i} \binom{\beta}{k-i}
\]
for all $\alpha, \beta \in \R$.
Fix $\alpha \in \R$ and $k \in \Z_{\ge 0}$. Then the left and the right hand sides are polynomials in $\beta$ of degree $k$.
By Step 2, the values of these polynomials coincide whenever $\beta$ is a positive integer.
Thus the polynomials are identically equal.
(In particular, evaluating the left hand side and the right hand side for any values of $\alpha, k, \beta$ leads to the same results.)
\end{proof}






\end{document}