\begin{exl}
\label{exl:DFAForReg1}
(Example 2.13. from \cite{HU79}.)
Find a regular expression for the language accepted by the automaton on Figure \ref{fig:DFAForReg1}.
\begin{figure}[ht]
\begin{center}
\input{Fig/DFAForReg1.pdf_t}
\end{center}
\caption{Automaton for Example \ref{exl:DFAForReg1}.}
\label{fig:DFAForReg1}
\end{figure}

One fills the table in Figure \ref{fig:TableRegExp} column after column according to the above algorithm.

\begin{figure}[ht]
\begin{center}
\begin{tabular}{c|ccc}
& $k=0$ & $k=1$ & $k=2$\\
\hline
$r_{11}^k$ & $\epsilon$ & $\epsilon$ & $(00)^*$\\
$r_{12}^k$ & $0$ & $0$ & $0(00)^*$\\
$r_{13}^k$ & $1$ & $1$ & $0^*1$\\
$r_{21}^k$ & $0$ & $0$ & $0(00)^*$\\
$r_{22}^k$ & $\epsilon$ & $\epsilon + 00$ & $(00)^*$\\
$r_{23}^k$ & $1$ & $1 + 01$ & $0^*1$\\
$r_{31}^k$ & $\emptyset$ & $\emptyset$ & $(0+1)(00)^*0$\\
$r_{32}^k$ & $0+1$ & $0+1$ & $(0+1)(00)^*$\\
$r_{33}^k$ & $\epsilon$ & $\epsilon$ & $\epsilon + (0+1)0^*1$
\end{tabular}
\end{center}
\caption{Finding a regular expression for Example \ref{exl:DFAForReg1}.}
\label{fig:TableRegExp}
\end{figure}

The first column is easy.
For the second and the third column use the recursive formula.
Sometimes a regular expression can be simplified, and this was done at several places in this table.
For example,
\[
r_{22}^1 = r_{22}^0 + r_{21}^0(r_{11}^0)^*r_{12}^0 = \epsilon + 0(\epsilon)^*0 = \epsilon + 00.
\]
More interesting things happen to $r_{13}^2$, which by the direct application of the recursive formula is equal to
\[
r_{13}^2 = r_{12}^1(r_{22}^1)^*r_{23}^1 + r_{13}^1 = 1+ 0(\epsilon + 00)^*(1+01).
\]
Because of $(\epsilon + 00)^* = (00)^*$ and $1+01 = (\epsilon + 0)1$ this can be rewritten as
\[
r_{13}^2 = 1 + 0(00)^*(\epsilon + 0)1.
\]
Further, one has $(00)^*(\epsilon + 0) = 0^*$, so that
\[
r_{13}^2 = 1 + 00^*1 = 0^*1.
\]
A regular expression for the language accepted by this automaton is $r_{12}^3 + r_{13}^3$.
Each of the summands is a lengthy expression. After some simplifications one obtains
\[
r = 0^*1((0+1)0^*1)^*(\epsilon+(0+1)(00)^*) + 0(00)^*.
\]

It should be noted that one does not need all of the table \ref{fig:TableRegExp} to compute the expression $r$.
\end{exl}