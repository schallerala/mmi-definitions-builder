\documentclass[preview, multi=page, margin=5mm, class=report]{standalone}
\usepackage[utf8]{inputenc}

\usepackage{amsmath,amssymb,amsthm}
\usepackage{graphicx,color}
\usepackage{hyperref,url}
\graphicspath{{Fig/}}

\usepackage{mathtools}
\usepackage{bussproofs}
\usepackage{stackengine}
\def\ruleoffset{1pt}
\newcommand\specialvdash[2]{\mathrel{\ensurestackMath{
  \mkern2mu\rule[-\dp\strutbox]{.4pt}{\baselineskip}\stackon[\ruleoffset]{
    \stackunder[\dimexpr\ruleoffset-.5\ht\strutbox+.5\dp\strutbox]{
      \rule[\dimexpr.5\ht\strutbox-.5\dp\strutbox]{2.5ex}{.4pt}}{
        \scriptstyle #1}}{\scriptstyle#2}\mkern2mu}}
}

\usepackage[table]{xcolor}

\renewcommand\thesection{\arabic{section}}
\renewcommand\thefigure{\arabic{figure}}
\renewcommand\theequation{\arabic{equation}}

\newtheorem{dfn}{Definition}[section]
\newtheorem{thm}[dfn]{Theorem}
\newtheorem{lem}[dfn]{Lemma}
\newtheorem{cor}[dfn]{Corollary}


\theoremstyle{definition}
\newtheorem{exl}[dfn]{Example}
\newtheorem{rem}[dfn]{Remark}
\newtheorem{exc}{Exercise}[section]

\def\R{\mathbb{R}}
\def\N{\mathbb{N}}
\def\Z{\mathbb{Z}}
\def\C{\mathbb{C}}
\def\cP{\mathcal{P}}
\def\cV{\mathcal{V}}
\def\cF{\mathcal{F}}
\def\Th{\mathrm{Th}}

\renewcommand{\emptyset}{\varnothing}
\renewcommand{\phi}{\varphi}
\renewcommand{\epsilon}{\varepsilon}
\def\gcd{\operatorname{gcd}}

\def\Prop{\mathrm{PROP}}
\begin{document}
\setcounter{section}{4}
\setcounter{subsection}{1}
\setcounter{dfn}{1}

\begin{proof}
Make the balls of the same color distinguishable (by writing on them numbers, for example).
Then all $n$ balls can be arranged in $n!$ different ways.
Now apply the quotient rule.
There is a forgetful map from the set $X$ of arrangements with balls of the same color distinguishable
to the set $Y$ of arrangements where balls of the same color are undistinguishable.
What is the multiplicity of this map, that is the cardinalities of the preimages $|f^{-1}(y)|$?
The balls of the $i$-th color can be permuted amongst themselves in $k_i!$ different ways.
Permutations within every color can be performed independently, which gives us $k_1! \cdots k_m!$ elements in $Y$
all of which correspond to the same element in $X$.
Thus we have $|f^{-1}(y)| = k_1! \cdots k_m!$, and hence
\[
|Y| = \frac{|X|}{k_1! \cdots k_m!} = \frac{n!}{k_1! \cdots k_m!}.
\]
\end{proof}




\end{document}
