\documentclass[preview, margin=5mm, multi=page]{standalone}
\usepackage[utf8]{inputenc}

\usepackage{amsmath,amssymb,amsthm}
\usepackage{graphicx,color}
\usepackage{hyperref,url}
\graphicspath{{Fig/}}

\usepackage{mathtools}
\usepackage{bussproofs}
\usepackage{stackengine}
\def\ruleoffset{1pt}
\newcommand\specialvdash[2]{\mathrel{\ensurestackMath{
  \mkern2mu\rule[-\dp\strutbox]{.4pt}{\baselineskip}\stackon[\ruleoffset]{
    \stackunder[\dimexpr\ruleoffset-.5\ht\strutbox+.5\dp\strutbox]{
      \rule[\dimexpr.5\ht\strutbox-.5\dp\strutbox]{2.5ex}{.4pt}}{
        \scriptstyle #1}}{\scriptstyle#2}\mkern2mu}}
}

\usepackage[table]{xcolor}

\renewcommand\thesection{\arabic{section}}
\renewcommand\thefigure{\arabic{figure}}
\renewcommand\theequation{\arabic{equation}}

\newtheorem{dfn}{Definition}[section]
\newtheorem{thm}[dfn]{Theorem}
\newtheorem{lem}[dfn]{Lemma}
\newtheorem{cor}[dfn]{Corollary}


\theoremstyle{definition}
\newtheorem{exl}[dfn]{Example}
\newtheorem{rem}[dfn]{Remark}
\newtheorem{exc}{Exercise}[section]

\def\R{\mathbb{R}}
\def\N{\mathbb{N}}
\def\Z{\mathbb{Z}}
\def\C{\mathbb{C}}
\def\cP{\mathcal{P}}
\def\cV{\mathcal{V}}
\def\cF{\mathcal{F}}
\def\Th{\mathrm{Th}}


\renewcommand{\emptyset}{\varnothing}
\renewcommand{\phi}{\varphi}
\renewcommand{\epsilon}{\varepsilon}
\def\gcd{\operatorname{gcd}}

\def\Prop{\mathrm{PROP}}



%opening
\title{{Lecture notes for the 2020/21 lectures}\\
$ $\\
$ $\\ \textsc{
Mathematical methods for Computer Science I \& II\\
and\\
Discrete Mathematics I \& II\\ }
$ $\\
$ $\\
$ $\\
$ $\\
University of Fribourg\\ Livio Liechti
$ $\\
$ $\\
$ $\\
$ $\\
$ $\\
$ $\\
$ $\\}
\date{ }

\author{Lecture notes written by Ivan Izmestiev for his 2018/19 lectures}


\begin{document}
\setcounter{section}{3}
\setcounter{subsection}{3}
\setcounter{dfn}{11}


A minor of a graph $G$ is any graph obtained by repeated vertex deletions, edge deletions and edge contractions.
See Figure \ref{fig:K5Contraction} for an example of an edge contraction.
If an edge contraction results in a multiple edge, then we replace a multiple edge by a simple edge.
If it results in a loop, then we remove a loop.

\begin{figure}[ht]
\begin{center}
\input{Fig/K5Contraction.pdf_t}
\end{center}
\caption{An example of edge contraction.}
\label{fig:K5Contraction}
\end{figure}

Note that a minor of $G$ is not necessarily isomorphic to a subgraph of $G$.
For example, the cycle $C_3$ is a minor of $C_4$ but not its subgraph.

Similarly to the Kuratowski theorem, one direction of the Wagner theorem is easy to prove, but not the other.




\end{document}
