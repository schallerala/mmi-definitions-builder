\begin{proof}
The equation $A(x)B(x) = 1$ consists of an infinite sequence of equations
\begin{gather*}
a_0b_0 = 1\\
a_0b_1 + a_1b_0 = 0\\
a_0b_2 + a_1b_1 + a_2b_0 = 0\\
\cdots
\end{gather*}
with unknowns $b_0, b_1, \ldots$.
The first equation implies $b_0 = \frac{1}{a_0}$ (which is defined because $a_0 \ne 0$).
Knowing $b_0$ we can express $b_1$ from the second equation:
\[
b_1 = -\frac{a_1b_0}{a_0}
\]
and continue in the same spirit, because $(k+1)$-st equation can be solved for $b_k$:
\[
b_k = -\frac{1}{a_0}\sum_{i=1}^k a_ib_{k-i}.
\]
This shows that the inverse series $B(x)$ exists and is unique.
\end{proof}

We denote the inverse series to $A(x)$ by $(A(x))^{-1}$ or $\frac{1}{A(x)}$.
