\documentclass[preview, multi=page, margin=5mm, class=report]{standalone}
\usepackage[utf8]{inputenc}

\usepackage{amsmath,amssymb,amsthm}
\usepackage{graphicx,color}
\usepackage{hyperref,url}
\graphicspath{{Fig/}}

\usepackage{mathtools}
\usepackage{bussproofs}
\usepackage{stackengine}
\def\ruleoffset{1pt}
\newcommand\specialvdash[2]{\mathrel{\ensurestackMath{
  \mkern2mu\rule[-\dp\strutbox]{.4pt}{\baselineskip}\stackon[\ruleoffset]{
    \stackunder[\dimexpr\ruleoffset-.5\ht\strutbox+.5\dp\strutbox]{
      \rule[\dimexpr.5\ht\strutbox-.5\dp\strutbox]{2.5ex}{.4pt}}{
        \scriptstyle #1}}{\scriptstyle#2}\mkern2mu}}
}

\usepackage[table]{xcolor}

\renewcommand\thesection{\arabic{section}}
\renewcommand\thefigure{\arabic{figure}}
\renewcommand\theequation{\arabic{equation}}

\newtheorem{dfn}{Definition}[section]
\newtheorem{thm}[dfn]{Theorem}
\newtheorem{lem}[dfn]{Lemma}
\newtheorem{cor}[dfn]{Corollary}


\theoremstyle{definition}
\newtheorem{exl}[dfn]{Example}
\newtheorem{rem}[dfn]{Remark}
\newtheorem{exc}{Exercise}[section]

\def\R{\mathbb{R}}
\def\N{\mathbb{N}}
\def\Z{\mathbb{Z}}
\def\C{\mathbb{C}}
\def\cP{\mathcal{P}}
\def\cV{\mathcal{V}}
\def\cF{\mathcal{F}}
\def\Th{\mathrm{Th}}

\renewcommand{\emptyset}{\varnothing}
\renewcommand{\phi}{\varphi}
\renewcommand{\epsilon}{\varepsilon}
\def\gcd{\operatorname{gcd}}

\def\Prop{\mathrm{PROP}}
\begin{document}
\setcounter{section}{7}
\setcounter{subsection}{2}
\setcounter{dfn}{5}

\begin{exl}
\label{exl:WW}
The language $L = \{ww \mid w \in \{0,1\}^*\}$ is not context-free.
Assume it is, and let $n$ be a pumping bound for this language.
Take the word $z = 0^n1^n0^n1^n \in L$ and write it as $z = uvwxy$ according to the pumping lemma.
By the lemma, the word $z' = uwy$ is in $L$, we will however show that this is impossible.

The subword $vwx$ is contained within some two consecutive blocks (inside $0^n1^n$ or $1^n0^n$).
% If it is contained in the first half, then look at the word $uv^2wx^2y$.
% It is longer than $z$ by not more than $n$ symbols,
% therefore its middle ``moved to the left'' by not more than $n/2$ symbols with respect to the middle of $z$.
% Then the right half of $z'$ begins with $1$ (the pumping changes the structure of the word but does not decrease the length of the block $1^n$).
% But the left half of $z'$ begins as before with $0$, which is a contradiction.
% A similar argument works if $vwx$ is contained in the second half of $z$.
If $vwx$ is contained in the first half, then while transforming $z$ to $z'$ we removed some symbols from the first half.
It follows that $uwy = 0^k1^l0^n1^n$, where $k, l \le n$ and at least one of them is $<n$.
This word does not belong to $L$.
Similarly, for the other situations of $vwx$ the depumped word has the form $0^n1^k0^l1^n$ or $0^n1^n0^k1^l$ and does not belong to $L$ either.
\end{exl}

\end{document}