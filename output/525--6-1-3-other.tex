\begin{proof}
The proof is based on the interpretation of an NFA as ``being in several states at the same time''.
One constructs a deterministic automaton whose states correspond to sets of states of NFA.

Take any NFA $M = (Q, \Sigma, \delta, q_0, F)$.
Define a DFA $M' = (Q', \Sigma, \delta', q'_0, F')$ as follows.
\begin{itemize}
\item
$Q' = 2^Q$: the states of $M'$ are all subsets of the set of states of $M$.
\item
$q'_0 = \{q_0\}$, a one-element set.
\item
$F' = \{P \subset Q \mid P \cap F \ne \emptyset\}$, all subsets of $Q$ that contain at least one final state of $M$.
\item
For any $P \subset Q$ put $\delta'(P, a) = \bigcup\limits_{p\in P}\delta(p, a)$.
\end{itemize}

We claim that
\[
\widehat{\delta'}(\{q\}, u) = \widehat{\delta}(q, u)
\]
for all $q \in Q$ and all $u \in \Sigma^*$ and prove it by induction on the length of $u$.
If $u = \epsilon$, then both sides are equal to $\{q\}$.
Take any word of length at least~$1$, and let $a$ be its last letter.
Then this word can be written as $wa$, and we have
\begin{gather*}
\widehat{\delta'}(\{q\}, wa) = \delta'(\widehat{\delta'}(\{q\}, w), a) = \bigcup\limits_{p \in \widehat{\delta'}(\{q\}, w)} \delta(p, a)\\
\widehat{\delta}(q, wa) = \bigcup\limits_{p \in \widehat{\delta}(q,w)} \delta(p, a)
\end{gather*}
By induction hypothesis, $\widehat{\delta'}(\{q\}, w) = \widehat{\delta}(q, w)$, and the induction step is proved.

Definition of $F'$ and definition \eqref{eqn:NFALanguage} imply that $L(M) = L(M')$.
\end{proof}
