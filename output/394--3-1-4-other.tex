\documentclass[preview, multi=page, margin=5mm, class=report]{standalone}
\usepackage[utf8]{inputenc}

\usepackage{amsmath,amssymb,amsthm}
\usepackage{graphicx,color}
\usepackage{hyperref,url}
\graphicspath{{Fig/}}

\usepackage{mathtools}
\usepackage{bussproofs}
\usepackage{stackengine}
\def\ruleoffset{1pt}
\newcommand\specialvdash[2]{\mathrel{\ensurestackMath{
  \mkern2mu\rule[-\dp\strutbox]{.4pt}{\baselineskip}\stackon[\ruleoffset]{
    \stackunder[\dimexpr\ruleoffset-.5\ht\strutbox+.5\dp\strutbox]{
      \rule[\dimexpr.5\ht\strutbox-.5\dp\strutbox]{2.5ex}{.4pt}}{
        \scriptstyle #1}}{\scriptstyle#2}\mkern2mu}}
}

\usepackage[table]{xcolor}

\renewcommand\thesection{\arabic{section}}
\renewcommand\thefigure{\arabic{figure}}
\renewcommand\theequation{\arabic{equation}}

\newtheorem{dfn}{Definition}[section]
\newtheorem{thm}[dfn]{Theorem}
\newtheorem{lem}[dfn]{Lemma}
\newtheorem{cor}[dfn]{Corollary}


\theoremstyle{definition}
\newtheorem{exl}[dfn]{Example}
\newtheorem{rem}[dfn]{Remark}
\newtheorem{exc}{Exercise}[section]

\def\R{\mathbb{R}}
\def\N{\mathbb{N}}
\def\Z{\mathbb{Z}}
\def\C{\mathbb{C}}
\def\cP{\mathcal{P}}
\def\cV{\mathcal{V}}
\def\cF{\mathcal{F}}
\def\Th{\mathrm{Th}}

\renewcommand{\emptyset}{\varnothing}
\renewcommand{\phi}{\varphi}
\renewcommand{\epsilon}{\varepsilon}
\def\gcd{\operatorname{gcd}}

\def\Prop{\mathrm{PROP}}
\begin{document}
\setcounter{section}{1}
\setcounter{subsection}{4}
\setcounter{dfn}{14}

\begin{proof}
By Theorem \ref{thm:BooleNF} every Boolean function can be represented by a formula using $\neg, \wedge, \vee$ only.
Replace every occurrence of $\vee$ using a De Morgan law and the double negation:
\[
A \vee B \simeq \neg(\neg A \wedge \neg B).
\]
This gives an equivalent formula with connectives $\neg$ and $\wedge$ only.
One removes conjunctions in a similar way with the help of the equivalence
\[
A \wedge B \simeq \neg (\neg A \vee \neg B).
\]
\end{proof}

For example,
\begin{align*}
(\neg p \wedge q) \vee (p \wedge \neg q) &\simeq \neg(\neg(\neg p \wedge q) \wedge \neg (p \wedge \neg q))\\
&\simeq \neg(p \vee \neg q) \vee \neg(\neg p \vee q)
\end{align*}

One can achieve an absolute minimalism by introducing a logical connective $\uparrow$ with the truth table
\begin{center}
\begin{tabular}{|c|c||c|c|c|}
\hline
$A$ & $B$ & $A \uparrow B$\\\hline
$0$ & $0$ & $1$\\\hline
$0$ & $1$ & $1$\\\hline
$1$ & $0$ & $1$\\\hline
$1$ & $1$ & $0$\\\hline
\end{tabular}
\end{center}
(also called $NAND$ for obvious reasons).


\end{document}