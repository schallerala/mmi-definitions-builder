\documentclass[preview, multi=page, margin=5mm, class=report]{standalone}
\usepackage[utf8]{inputenc}

\usepackage{amsmath,amssymb,amsthm}
\usepackage{graphicx,color}
\usepackage{hyperref,url}
\graphicspath{{Fig/}}

\usepackage{mathtools}
\usepackage{bussproofs}
\usepackage{stackengine}
\def\ruleoffset{1pt}
\newcommand\specialvdash[2]{\mathrel{\ensurestackMath{
  \mkern2mu\rule[-\dp\strutbox]{.4pt}{\baselineskip}\stackon[\ruleoffset]{
    \stackunder[\dimexpr\ruleoffset-.5\ht\strutbox+.5\dp\strutbox]{
      \rule[\dimexpr.5\ht\strutbox-.5\dp\strutbox]{2.5ex}{.4pt}}{
        \scriptstyle #1}}{\scriptstyle#2}\mkern2mu}}
}

\usepackage[table]{xcolor}

\renewcommand\thesection{\arabic{section}}
\renewcommand\thefigure{\arabic{figure}}
\renewcommand\theequation{\arabic{equation}}

\newtheorem{dfn}{Definition}[section]
\newtheorem{thm}[dfn]{Theorem}
\newtheorem{lem}[dfn]{Lemma}
\newtheorem{cor}[dfn]{Corollary}


\theoremstyle{definition}
\newtheorem{exl}[dfn]{Example}
\newtheorem{rem}[dfn]{Remark}
\newtheorem{exc}{Exercise}[section]

\def\R{\mathbb{R}}
\def\N{\mathbb{N}}
\def\Z{\mathbb{Z}}
\def\C{\mathbb{C}}
\def\cP{\mathcal{P}}
\def\cV{\mathcal{V}}
\def\cF{\mathcal{F}}
\def\Th{\mathrm{Th}}

\renewcommand{\emptyset}{\varnothing}
\renewcommand{\phi}{\varphi}
\renewcommand{\epsilon}{\varepsilon}
\def\gcd{\operatorname{gcd}}

\def\Prop{\mathrm{PROP}}
\begin{document}
\setcounter{section}{3}
\setcounter{subsection}{3}
\setcounter{dfn}{24}

The signature consists of a nullary function $0$, a unary function $s$ (successor), two binary functions $+$ and $\cdot$.
The equality predicate $=$.
Axioms of $PA$:
\begin{enumerate}
\item
$\forall x \neg(s(x) = 0)$
\item
$\forall x \forall y (s(x)=s(y) \to x=y)$
\item
$\forall x (x = 0 \vee \exists y (s(y)=x))$
\item
$\forall x (x+0=x)$
\item
$\forall x \forall y (x+s(y) = s(x+y))$
\item
$\forall x (x \cdot 0 = 0)$
\item
$\forall x \forall y (x \cdot s(y) = x \cdot y + x)$
\item
$\forall \bar y \left((A(0, \bar y) \wedge \forall x (A(x, \bar y) \to A(s(x), \bar y)) \to \forall x A(x, \bar y)\right)$
\end{enumerate}
The last item is the \emph{induction schema}, that is it encodes infinitely many sentences.
Here $\bar y$ denotes $y_1, \ldots, y_n$, and $\forall \bar y$ denotes $\forall y_1 \ldots \forall y_n$.
The number $n$ can be any non-negative integer, and $A$ can be any formula with $n+1$ free variables $x, y_1, \ldots, y_n$.


The theory consisting of the first seven axioms is called Robinson arithmetic, we will denote it by $PA_0$.
It does not imply that the addition is commutative.

Since we assume that $\N$ is an objective reality and the operations with positive integers satisfy the above properties,
theory $PA$ is satisfiable and hence consistent.






\end{document}
