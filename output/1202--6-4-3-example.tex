\begin{exl}
Consider the DFA in Figure \ref{fig:DFAMinim}.
It has one inaccessible state~$q_3$.

\begin{figure}[ht]
\begin{center}
\input{Fig/DFAMinim.pdf_t}
\end{center}
\caption{A non-minimal DFA.}
\label{fig:DFAMinim}
\end{figure}

We thus draw a $7 \times 7$ table and fill it step by step.
It suffices to fill only a half of the table, on one side of the diagonal.

\renewcommand{\arraystretch}{1.7}
% \begin{figure}[ht]
\begin{center}
\begin{tabular}{c|c|c|c|c|c|c|c|}
& $q_0$ & $q_1$ & $q_2$ & $q_4$ & $q_5$ & $q_6$ & $q_7$\\\hline
$q_0$ & \cellcolor{lightgray} &&&&&&\\\hline
$q_1$ & $\times_1$ & \cellcolor{lightgray} &&&&&\\\hline
$q_2$ & $\times_0$ & $\times_0$ & \cellcolor{lightgray} &&&&\\\hline
$q_4$ &  & $\times_1$ & $\times_0$ & \cellcolor{lightgray} &&&\\\hline
$q_5$ & $\times_1$ & $\times_1$ & $\times_0$ & $\times_1$ & \cellcolor{lightgray} &&\\\hline
$q_6$ & $\times_2$ & $\times_1$ & $\times_0$ & $\times_2$ & $\times_1$ & \cellcolor{lightgray} &\\\hline
$q_7$ & $\times_1$ &  & $\times_0$ & $\times_1$ & $\times_1$ & $\times_1$ & \cellcolor{lightgray}\\\hline
\end{tabular}
% \caption{Calculation of equivalent states. The mark $\times_n$ means that the corresponding pair of states was recognized as non-equivalent at the step $n$.}
% \label{fig:DFAMinimTable}
\end{center}
% \end{figure}

A pair of states is marked with $\times_n$ if these states were recognized as non-equivalent at Step $n$.
As initialization (Step 0) we mark all pairs $(q_2, q_i)$ because $q_2$ is the only final state.

A lot of cells are marked at Step 1.
These are all pairs $(q_i, q_j)$ such that either the $0$-arrows or the $1$-arrows lead from $q_i$ to a final and from $q_j$ to a non-final state or vice versa.
For example, we mark $(q_0, q_1)$ because $\delta(q_0, 1) = q_5$ is non-final and $\delta(q_1, 1) = q_2$ is final.

At Step 2 two cells are marked.
For example, we mark $(q_4, q_6)$ because $\delta(q_4, 0) = q_7$, $\delta(q_6, 0) = q_6$, and the pair $(q_6, q_7)$ is already marked (it was marked at Step 1).

At Step 3 we check all pairs of unmarked cells, by looking where the $0$- and $1$-arrows lead, but do not find anything that should be marked.
Thus the algorithm stops, and the minimal DFA is obtained from the one in Figure~\ref{fig:DFAMinim} by removing the state $q_3$,
merging $q_0$ with $q_4$ and merging $q_1$ with $q_7$. See Figure \ref{fig:DFAMinimResult}.

\begin{figure}[ht]
\begin{center}
\input{Fig/DFAMinimResult.pdf_t}
\end{center}
\caption{The minimal DFA equivalent to one in Figure \ref{fig:DFAMinim}.}
\label{fig:DFAMinimResult}
\end{figure}
\end{exl}