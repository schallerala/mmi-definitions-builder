\documentclass[preview, multi=page, margin=5mm, class=report]{standalone}
\usepackage[utf8]{inputenc}

\usepackage{amsmath,amssymb,amsthm}
\usepackage{graphicx,color}
\usepackage{hyperref,url}
\graphicspath{{Fig/}}

\usepackage{mathtools}
\usepackage{bussproofs}
\usepackage{stackengine}
\def\ruleoffset{1pt}
\newcommand\specialvdash[2]{\mathrel{\ensurestackMath{
  \mkern2mu\rule[-\dp\strutbox]{.4pt}{\baselineskip}\stackon[\ruleoffset]{
    \stackunder[\dimexpr\ruleoffset-.5\ht\strutbox+.5\dp\strutbox]{
      \rule[\dimexpr.5\ht\strutbox-.5\dp\strutbox]{2.5ex}{.4pt}}{
        \scriptstyle #1}}{\scriptstyle#2}\mkern2mu}}
}

\usepackage[table]{xcolor}

\renewcommand\thesection{\arabic{section}}
\renewcommand\thefigure{\arabic{figure}}
\renewcommand\theequation{\arabic{equation}}

\newtheorem{dfn}{Definition}[section]
\newtheorem{thm}[dfn]{Theorem}
\newtheorem{lem}[dfn]{Lemma}
\newtheorem{cor}[dfn]{Corollary}


\theoremstyle{definition}
\newtheorem{exl}[dfn]{Example}
\newtheorem{rem}[dfn]{Remark}
\newtheorem{exc}{Exercise}[section]

\def\R{\mathbb{R}}
\def\N{\mathbb{N}}
\def\Z{\mathbb{Z}}
\def\C{\mathbb{C}}
\def\cP{\mathcal{P}}
\def\cV{\mathcal{V}}
\def\cF{\mathcal{F}}
\def\Th{\mathrm{Th}}

\renewcommand{\emptyset}{\varnothing}
\renewcommand{\phi}{\varphi}
\renewcommand{\epsilon}{\varepsilon}
\def\gcd{\operatorname{gcd}}

\def\Prop{\mathrm{PROP}}
\begin{document}
\setcounter{section}{1}
\setcounter{subsection}{2}
\setcounter{dfn}{0}

\subsection{Alphabets, words, and languages}
An \emph{alphabet} is any finite set of symbols.
Examples:
\begin{itemize}
\item
the binary alphabet $\{0,1\}$;
\item
the alphabet of a single symbol $\{0\}$;
\item
the alphabet $\{p_1, \ldots, p_n\} \cup \{\neg, \wedge, \vee, \to, (, )\}$ of the propositional logic.
\end{itemize}

A \emph{string} or a \emph{word} is a finite sequence of symbols from a given alphabet.
The set of words of length $n$ in the alphabet $\Sigma$ is denoted by $\Sigma^n$:
\[
\Sigma^n = \{x_1 \ldots x_n \mid x_i \in \Sigma\ \forall i\}.
\]
This is the same as the Cartesian power $\Sigma^n$, with only a notational difference: $x_1 \ldots x_n$
instead of $(x_1, \ldots, x_n)$.

The concatenation of two words defines a map $\Sigma^m \times \Sigma^n \to \Sigma^{m+n}$.
Clearly, $uv \ne vu$ in general.
Denote by $\Sigma^* = \Sigma^0 \cup \Sigma^1 \cup \Sigma^2 \cup \cdots$ the set of all words in the alphabet $\Sigma$.

There is a unique element in $\Sigma^0$: the word of zero length; it is denoted by $\epsilon$.
One has
\[
\epsilon w = w = w\epsilon \text{ for all } w \in \Sigma^*.
\]

A \emph{language} is a subset of $\Sigma^*$.
Here are some examples of languages.

\begin{itemize}
\item
The set of all sequences of zeros of prime length:
\[
\{0^p \mid p \text{ is a prime number}\}.
\]
\item
The set of all binary palindromes (binary sequences that read the same forward and backward):
\[
\{\epsilon, 0, 1, 00, 11, 000, 010, \ldots\}.
\]
\item
In the alphabet of propositional logic, the set of all propositional formulas.
\item
In the same alphabet, the set of all propositional formulas which are tautologies.
\end{itemize}






\end{document}