
Axioms of the Gentzen system are, as before, sequents $\Gamma \vdash \Delta$
with some formula occurring on both sides: $A \in \Gamma \cap \Delta$.
Axioms are valid (because no structure can at the same time satisfy and falsify $A$).
Inference rules for the logical connectives $\wedge, \vee, \to, \neg$ are the same.
There are four new rules involving the quantifiers.

\begin{align*}
&(\forall\text{ \rm left}):\
\AxiomC{$A[t/x], \forall x A, \Gamma \vdash \Delta$}
\UnaryInfC{$\forall x A, \Gamma \vdash \Delta$}
\DisplayProof
&&(\forall\text{ \rm right}):\
\AxiomC{$\Gamma \vdash A[y/x], \Delta$}
\UnaryInfC{$\Gamma \vdash \forall x A, \Delta$}
\DisplayProof
\\
&(\exists\text{ \rm left}):\
\AxiomC{$A[y/x], \Gamma \vdash \Delta$}
\UnaryInfC{$\exists x A, \Gamma \vdash \Delta$}
\DisplayProof
&&(\exists\text{ \rm right}):\
\AxiomC{$\Gamma \vdash A[t/x], \exists x A, \Delta$}
\UnaryInfC{$\Gamma \vdash \exists x A, \Delta$}
\DisplayProof
\end{align*}

Here $t$ is any term free for $x$ in $A$, and $y$ is any variable free for $x$ in $A$ and not occurring freely in the conclusion
(that is, in the sequent $\Gamma \vdash \forall x A, \Delta$ for the right $\forall$ rule and in the sequent $\exists x A, \Gamma \vdash \Delta$
for the left $\exists$ rule).
In particular, one can substitute $x$ for itself if $x$ does not occur freely in $\Gamma$ and~$\Delta$.

As before, a deduction tree is a proof tree if all of its leaves are axioms.
A sequent is called provable if there is a proof tree with this sequent as a root.
