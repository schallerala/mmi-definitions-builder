\subsection{Peano arithmetic}
The signature consists of a nullary function $0$, a unary function $s$ (successor), two binary functions $+$ and $\cdot$.
The equality predicate $=$.
Axioms of $PA$:
\begin{enumerate}
\item
$\forall x \neg(s(x) = 0)$
\item
$\forall x \forall y (s(x)=s(y) \to x=y)$
\item
$\forall x (x = 0 \vee \exists y (s(y)=x))$
\item
$\forall x (x+0=x)$
\item
$\forall x \forall y (x+s(y) = s(x+y))$
\item
$\forall x (x \cdot 0 = 0)$
\item
$\forall x \forall y (x \cdot s(y) = x \cdot y + x)$
\item
$\forall \bar y \left((A(0, \bar y) \wedge \forall x (A(x, \bar y) \to A(s(x), \bar y)) \to \forall x A(x, \bar y)\right)$
\end{enumerate}
The last item is the \emph{induction schema}, that is it encodes infinitely many sentences.
Here $\bar y$ denotes $y_1, \ldots, y_n$, and $\forall \bar y$ denotes $\forall y_1 \ldots \forall y_n$.
The number $n$ can be any non-negative integer, and $A$ can be any formula with $n+1$ free variables $x, y_1, \ldots, y_n$.


The theory consisting of the first seven axioms is called Robinson arithmetic, we will denote it by $PA_0$.
It does not imply that the addition is commutative.

Since we assume that $\N$ is an objective reality and the operations with positive integers satisfy the above properties,
theory $PA$ is satisfiable and hence consistent.




