\begin{proof}[Sketch of proof]
Let $v_1, \ldots, v_5$ be the vertices of $K_5$.
We denote by the same letters their images in the plane.
The vertices $v_1, v_2, v_3, v_4$ and the edges between them form a planar embedding of $K_4$.
Up to relabeling of vertices and isotopy there is a unique embedding of $K_4$ in the plane, see Figure \ref{fig:K4Embedding}, left.
The graph $K_4$ separates the plane into four regions.
The fifth vertex $v_5$ must lie in one of these regions.
In whatever region it lies, it will be separated from one of the first four vertices.
For example, if it lies in the outer region,
then the edge $v_5v_1$ must intersect the contour $v_2v_3v_4$ at least once, see Figure \ref{fig:K4Embedding}, right.
\end{proof}

\begin{figure}[ht]
\begin{center}
\input{Fig/K4Embedding.pdf_t}
\end{center}
\caption{Proof of the non-planarity of $K_5$.}
\label{fig:K4Embedding}
\end{figure}

The above is only a sketch of the proof, because the assertions
``there is a unique embedding of $K_4$'' and ``the edge must intersect the contour'' require formal proofs.
They follow from the Jordan curve theorem: any embedded closed curve in the plane separates the plane in two connected components.
This might seem trivial but you should take into account that there are curves which are neither smooth nor polygonal
(maybe you have heard about fractals).

On the other hand, there is the following theorem (whose proof relies on the Jordan curve theorem).