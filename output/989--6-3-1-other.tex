\documentclass[preview, multi=page, margin=5mm, class=report]{standalone}
\usepackage[utf8]{inputenc}

\usepackage{amsmath,amssymb,amsthm}
\usepackage{graphicx,color}
\usepackage{hyperref,url}
\graphicspath{{Fig/}}

\usepackage{mathtools}
\usepackage{bussproofs}
\usepackage{stackengine}
\def\ruleoffset{1pt}
\newcommand\specialvdash[2]{\mathrel{\ensurestackMath{
  \mkern2mu\rule[-\dp\strutbox]{.4pt}{\baselineskip}\stackon[\ruleoffset]{
    \stackunder[\dimexpr\ruleoffset-.5\ht\strutbox+.5\dp\strutbox]{
      \rule[\dimexpr.5\ht\strutbox-.5\dp\strutbox]{2.5ex}{.4pt}}{
        \scriptstyle #1}}{\scriptstyle#2}\mkern2mu}}
}

\usepackage[table]{xcolor}

\renewcommand\thesection{\arabic{section}}
\renewcommand\thefigure{\arabic{figure}}
\renewcommand\theequation{\arabic{equation}}

\newtheorem{dfn}{Definition}[section]
\newtheorem{thm}[dfn]{Theorem}
\newtheorem{lem}[dfn]{Lemma}
\newtheorem{cor}[dfn]{Corollary}


\theoremstyle{definition}
\newtheorem{exl}[dfn]{Example}
\newtheorem{rem}[dfn]{Remark}
\newtheorem{exc}{Exercise}[section]

\def\R{\mathbb{R}}
\def\N{\mathbb{N}}
\def\Z{\mathbb{Z}}
\def\C{\mathbb{C}}
\def\cP{\mathcal{P}}
\def\cV{\mathcal{V}}
\def\cF{\mathcal{F}}
\def\Th{\mathrm{Th}}

\renewcommand{\emptyset}{\varnothing}
\renewcommand{\phi}{\varphi}
\renewcommand{\epsilon}{\varepsilon}
\def\gcd{\operatorname{gcd}}

\def\Prop{\mathrm{PROP}}
\begin{document}
\setcounter{section}{3}
\setcounter{subsection}{1}
\setcounter{dfn}{1}

\begin{proof}
If $r_1$ is a regular expression for $R_1$, and $r_2$ is a regular expression for $R_2$,
then the regular expression $r_1 + r_2$ describes the language $R_1 \cup R_2$, which is therefore regular.

Given regular expressions $r_1$, $r_2$, $r$,
it is very difficult to find regular expressions for the intersection $R_1 \cap R_2$ and the complement $\Sigma^* \setminus R$.
Let us approach the problem from a different direction.

Let $M = (Q, \Sigma, \delta, q_0, F)$ be a DFA accepting the language $R$.
Then $\overline{M} := (Q, \Sigma, \delta, q_0, Q \setminus F)$ accepts the language $\Sigma^* \setminus R$.
Indeed,
\[
w \in \Sigma^* \setminus R \Leftrightarrow w \notin R \Leftrightarrow \widehat{\delta}(q_0, w) \notin F
\Leftrightarrow \widehat{\delta}(q_0, w) \in Q \setminus F \Leftrightarrow w \in L(\overline{M}).
\]
Therefore $\Sigma^* \setminus R$ is regular.

With the intersection we are helped by de Morgan's rule:
\[
R_1 \cap R_2 = \overline{\overline{R_1} \cup \overline{R_2}},
\]
where the overline denotes the complement.
Since the operations applied on the right hand side preserve regularity, the intersection of two regular languages is regular.
\end{proof}



\end{document}