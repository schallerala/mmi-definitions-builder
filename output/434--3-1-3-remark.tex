\documentclass[preview, multi=page, margin=5mm, class=report]{standalone}
\usepackage[utf8]{inputenc}

\usepackage{amsmath,amssymb,amsthm}
\usepackage{graphicx,color}
\usepackage{hyperref,url}
\graphicspath{{Fig/}}

\usepackage{mathtools}
\usepackage{bussproofs}
\usepackage{stackengine}
\def\ruleoffset{1pt}
\newcommand\specialvdash[2]{\mathrel{\ensurestackMath{
  \mkern2mu\rule[-\dp\strutbox]{.4pt}{\baselineskip}\stackon[\ruleoffset]{
    \stackunder[\dimexpr\ruleoffset-.5\ht\strutbox+.5\dp\strutbox]{
      \rule[\dimexpr.5\ht\strutbox-.5\dp\strutbox]{2.5ex}{.4pt}}{
        \scriptstyle #1}}{\scriptstyle#2}\mkern2mu}}
}

\usepackage[table]{xcolor}

\renewcommand\thesection{\arabic{section}}
\renewcommand\thefigure{\arabic{figure}}
\renewcommand\theequation{\arabic{equation}}

\newtheorem{dfn}{Definition}[section]
\newtheorem{thm}[dfn]{Theorem}
\newtheorem{lem}[dfn]{Lemma}
\newtheorem{cor}[dfn]{Corollary}


\theoremstyle{definition}
\newtheorem{exl}[dfn]{Example}
\newtheorem{rem}[dfn]{Remark}
\newtheorem{exc}{Exercise}[section]

\def\R{\mathbb{R}}
\def\N{\mathbb{N}}
\def\Z{\mathbb{Z}}
\def\C{\mathbb{C}}
\def\cP{\mathcal{P}}
\def\cV{\mathcal{V}}
\def\cF{\mathcal{F}}
\def\Th{\mathrm{Th}}

\renewcommand{\emptyset}{\varnothing}
\renewcommand{\phi}{\varphi}
\renewcommand{\epsilon}{\varepsilon}
\def\gcd{\operatorname{gcd}}

\def\Prop{\mathrm{PROP}}
\begin{document}
\setcounter{section}{1}
\setcounter{subsection}{3}
\setcounter{dfn}{9}

\begin{rem}
The symbol $\simeq$ of the logical equivalence does not belong to the alphabet of propositional logic, so that $A \simeq B$ is not a propositional formula.
The symbols $\simeq$ and $\vDash$, and also symbols $A$ and $B$ used to denote arbitrary propositions,
all belong to the \emph{metalanguage}, a language that we use to describe propositional logic and prove its properties.
Also the symbol $\Leftrightarrow$ that we will use in our arguments as an abbreviation for ``if and only if'' is a metasymbol.
The definitions and statements in this section define notions of a metalanguage and formulate metatheorems.

When we prove some properties of propositional logic,
we are implicitly using another, more complicated, logical system with its own semantics (what is true and what is not true).
If, in turn, we want to discuss this more complicated system, then we need a metametalanguage and so on.

Imagine a computer program which is able to recognize propositional formulas and compute their truth values for any valuation.
This program speaks the language of propositional logic, and it cannot analize its own actions.
The programmer speaks a metalanguage and can predict the behavior of the program.
\end{rem}

\end{document}