\begin{proof}
The length of $z$ is the number of leaves of the derivation tree.
Deleting the terminal nodes does not change the number of leaves but decreases the depth of the tree by one.
A full binary tree of depth $i-1$ has at most $2^{i-1}$ leaves.
\end{proof}

\begin{proof}[Proof of Theorem \ref{thm:PumpCFL}]
Let $G$ be a grammar in Chomsky form generating the language $L \setminus \{\epsilon\}$.
Put $n = 2^{k-1}+1$, where $k$ is the number of variables in $G$.
We will show that this $n$ satisfies the conditions of the pumping lemma.

Let $z \in L(G)$ be such that $|z| \ge n$, and let $T$ be any derivation tree for $z$.
By Lemma \ref{lem:ChomskyTree}, $T$ contains a path of length at least $k+1$.
This path contains at least $k+2$ vertices.
The last vertex of the path is labeled by a terminal, all of the other vertices are labeled by variables.
Hence there is a variable that appears on this path at least twice.

Take any of the longest paths in $T$ and, ascending it from the leaf, find the first repetition of variables.
This gives us two vertices $v_1$ and $v_2$ such that
\begin{enumerate}
\item
$v_1$ and $v_2$ have the same label, say $A$;
\item
$v_1$ is closer to the root than $v_2$;
\item
the portion of the path from $v_1$ to the leaf has length at most $k+1$.
\end{enumerate}

The subtree $T_1$ with the root $v_1$ represents a derivation from $A$ of a word $z'$ which is a subword of $z$ and has length at most $2^k$.
Indeed, the portion of our path from $v_1$ down to the leaf has length at most $k+1$, and there is no longer path starting from $v_1$
because we have chosen a longest path.


\begin{figure}[ht]
\begin{center}
\input{Fig/PumpCFL1.pdf_t}
\end{center}
\caption{Proof of the pumping lemma for CFLs: structuring a word.}
\label{fig:PumpCFL1}
\end{figure}

The tree $T_1$ contains a subtree $T_2$ growing from $v_2$.
This subtree derives from $A$ a word $w$ which is a subword of $z'$:
\[
z' = vwx,
\]
see Figure \ref{fig:PumpCFL1}.
The words $v$ and $x$ cannot both be $\epsilon$, because $v_1$ has two children, only one of which is an ancestor of $v_2$,
so that the other child generates a subword of $z'$ disjoint from $w$.

To summarize, we have
\[
S \xRightarrow[]{*} uAy, \quad \xRightarrow[]{*} vAx, \quad A \xRightarrow[]{*} w.
\]
It follows that $A \xRightarrow[]{*} v^iwx^i$ and $S \xRightarrow[]{*} uv^iwx^iy$ for all $i \ge 0$.
A derivation tree for $uv^iwx^iy$ can be obtained from the tree $T$ by cut, copy, and paste, see Figure \ref{fig:PumpCFL2}.
\end{proof}

\begin{figure}[ht]
\begin{center}
\includegraphics[width=\textwidth]{PumpCFL2.pdf}
\end{center}
\caption{Proof of the pumping lemma for CFLs: pumping a word. These trees represent the words $uv^3wx^3y$ and $uwy$ respectively.}
\label{fig:PumpCFL2}
\end{figure}

