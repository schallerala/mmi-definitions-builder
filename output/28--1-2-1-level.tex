\subsection{Maps}
A \emph{map} $f \colon X \to Y$ is a rule that associates to every element $x \in X$ a unique element of $Y$.
The element associated to $x$ is denoted by $f(x)$.

A map can be pictured as a collection of arrows going from elements of $X$ to elements of $Y$.
At every element of $X$ one and only one arrow must start.
By contrast, at an element of $Y$ several arrows or none at all may end.

\begin{figure}[ht]
\begin{center}
\input{Fig/Map.pdf_t}
\end{center}
\caption{A map $f \colon X \to Y$.}
\label{fig:Map}
\end{figure}

A map $f \colon X \to Y$ is called
\begin{itemize}
\item
\emph{injective}, if no two different elements of $X$ are sent to the same element of $Y$: for every $x_1 \ne x_2$ we have $f(x_1) \ne f(x_2)$;
\item
\emph{surjective}, if to every element of $Y$ some element of $X$ is sent: for every $y \in Y$ there is $x \in X$ such that $f(x) = y$;
\item
\emph{bijective}, if it is injective and surjective.
\end{itemize}
