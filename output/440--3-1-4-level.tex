\subsection{Boolean functions}
Although the set of all proposition symbols is infinite,
every proposition contains only a finite number of distinct proposition symbols.
Assume that $A \in \Prop$ contains only symbols from the set $\{p_1, p_2, \ldots, p_n\}$.
Then $A$ defines a map
\[
f_A \colon \{0,1\}^n \to \{0,1\}
\]
in the following way.
Every element $(x_1, \ldots, x_n) \in \{0,1\}^n$ can be viewed as a partial valuation, giving $p_i$ the truth value $x_i$ for $i = 1, \ldots, n$.
Then $f_A(x_1, \ldots, x_n)$ is the truth value of $A$ corresponding to this valuation:
\[
f_A(x_1, \ldots, x_n) = \hat{v}(A), \text{ where } v(p_i) = x_i.
\]

The function $f_A$ is described by the truth table of proposition $A$.
As an immediate reformulation of Definition \ref{dfn:LogEqProp},
\[
A \simeq B \Leftrightarrow f_A = f_B.
\]

In fact, $0$-$1$-valued functions of $0$-$1$-valued agruments is a very basic object which can be studied irrespective of the propositional logic.