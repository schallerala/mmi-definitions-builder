\chapter{Basic combinatorics}

\section{How to count}
All sets in this chapter are finite.

For a finite set $X$, by $|X|$ we denote the number of elements in $X$, also called the \emph{cardinality} of $X$.

\subsection{Sum rule}
\begin{center}
\emph{If $X \cap Y = \emptyset$, then $|X \cup Y| = |X| + |Y|$.}
\end{center}

More generally,
\begin{center}
\parbox{.9\textwidth}{\emph{If the sets $X_1, \ldots, X_n$ are pairwise disjoint
(that is $X_i \cap X_j = \emptyset$ for all $i \ne j$),
then $|X_1 \cup X_2 \cup \cdots \cup X_n| = |X_1| + |X_2| + \cdots + |X_n|$.}}
\end{center}
Formally, this follows by induction on $n$ from the sum rule for two sets.

Later we will learn how to proceed if the sets are not disjoint.

\subsection{Product rule}
First, let us state a special product rule:
\[
|X \times Y| = |X| \cdot |Y|.
\]
Here $X \times Y$, the \emph{Cartesian product} of $X$ and $Y$, denotes the set of ordered pairs $(x,y)$ with $x \in X$, $y \in Y$.

The elements of $X \times Y$ can be written in a table whose rows correspond to the elements of $X$,
and the columns correspond to the elements of $Y$.
This justifies the product rule.

Again, there is an extension to several sets:
\[
|X_1 \times \cdots X_n| = |X_1| \cdots |X_n|.
\]
