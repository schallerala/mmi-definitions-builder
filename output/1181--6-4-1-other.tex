\documentclass[preview, multi=page, margin=5mm, class=report]{standalone}
\usepackage[utf8]{inputenc}

\usepackage{amsmath,amssymb,amsthm}
\usepackage{graphicx,color}
\usepackage{hyperref,url}
\graphicspath{{Fig/}}

\usepackage{mathtools}
\usepackage{bussproofs}
\usepackage{stackengine}
\def\ruleoffset{1pt}
\newcommand\specialvdash[2]{\mathrel{\ensurestackMath{
  \mkern2mu\rule[-\dp\strutbox]{.4pt}{\baselineskip}\stackon[\ruleoffset]{
    \stackunder[\dimexpr\ruleoffset-.5\ht\strutbox+.5\dp\strutbox]{
      \rule[\dimexpr.5\ht\strutbox-.5\dp\strutbox]{2.5ex}{.4pt}}{
        \scriptstyle #1}}{\scriptstyle#2}\mkern2mu}}
}

\usepackage[table]{xcolor}

\renewcommand\thesection{\arabic{section}}
\renewcommand\thefigure{\arabic{figure}}
\renewcommand\theequation{\arabic{equation}}

\newtheorem{dfn}{Definition}[section]
\newtheorem{thm}[dfn]{Theorem}
\newtheorem{lem}[dfn]{Lemma}
\newtheorem{cor}[dfn]{Corollary}


\theoremstyle{definition}
\newtheorem{exl}[dfn]{Example}
\newtheorem{rem}[dfn]{Remark}
\newtheorem{exc}{Exercise}[section]

\def\R{\mathbb{R}}
\def\N{\mathbb{N}}
\def\Z{\mathbb{Z}}
\def\C{\mathbb{C}}
\def\cP{\mathcal{P}}
\def\cV{\mathcal{V}}
\def\cF{\mathcal{F}}
\def\Th{\mathrm{Th}}

\renewcommand{\emptyset}{\varnothing}
\renewcommand{\phi}{\varphi}
\renewcommand{\epsilon}{\varepsilon}
\def\gcd{\operatorname{gcd}}

\def\Prop{\mathrm{PROP}}
\begin{document}
\setcounter{section}{4}
\setcounter{subsection}{1}
\setcounter{dfn}{3}

\begin{proof}
The reflexivity and the symmetry are obvious.
To prove the transitivity, let $u \sim_L v$, $v \sim_L w$, and assume that $u \not\sim_L w$.
Then there is a distinguishing extension $x$ for $u$ and $w$.
Without loss of generality, $ux \in L$, $wx \notin L$.
Then if $vx \in L$, we have $v \not\sim_L w$, and if $vx \notin L$, we have $u \not\sim_L v$.
\end{proof}

An equivalence relation splits the set $\Sigma^*$ of all words into \emph{equivalence classes}:
\begin{equation}
\label{eqn:SigmaEqClasses}
\Sigma^* = S_0 \cup S_1 \cup S_2 \cup \cdots
\end{equation}
where $u \sim_L v$ if and only if $u$ and $v$ belong to the same class.
As we noticed in Example \ref{exl:LEquiv}, if $u \sim_L v$, then either $u,v \in L$ or $u,v \notin L$.
It follows that every class $S_i$ is either contained in $L$ or disjoint from $L$.


\end{document}