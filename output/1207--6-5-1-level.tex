\subsection{Generating a language by a grammar}
\label{sec:LangGram}
Consider the sentence
\begin{center}
\emph{Colorless green ideas sleep furiously.}
\end{center}
Although it does not make any sense, it is syntactically correct.
One distinguishes a noun phrase ``colorless green ideas'' and a verb phrase ``sleep furiously''.
The noun phrase itself consists of a noun preceeded by adjectives, and the verb phrase consists of a verb followed by adverbs.

More generally, one can formulate the following rules of production of simple English sentences:
\begin{align*}
&S \to NV &&\text{a sentence consists of a noun phrase and a verb phrase}\\
&N \to AdjN &&\text{a noun phrase may start with one or more adjectives}\\
&V \to VAdv &&\text{a verb phrase may end with one or more adverbs}
\end{align*}
At any stage one can substitute for $N$, $V$, $Adj$, $Adv$ a word from a dictionary:
\begin{align*}
&N \to \text{list of nouns}\\
&V \to \text{list of verbs}\\
&Adj \to \text{list of adjectives}\\
&Adv \to \text{list of adverbs}
\end{align*}
The result is a syntactically correct (but mostly meaningless) sentence.

A production can be represented linearly:
\begin{multline*}
S \to NV \to AdjNV \to AdjAdjNV \to \text{colorless }AdjNV\\
\to \text{colorless }AdjNVAdv \to \cdots 
\end{multline*}
or by a \emph{derivation tree} or \emph{parse tree}, see Figure \ref{fig:Chomsky}.

\begin{figure}[ht]
\begin{center}
\includegraphics{Chomsky.pdf}
\end{center}
\caption{Derivation tree for the Chomsky example.}
\label{fig:Chomsky}
\end{figure}

The sentence at the beginning of this section is from a book of Noam Chomsky.
In mid-1950's he proposed the above principles as description of the structure of human languages.
(Of course one needs more production rules in order to be able to generate more complicated sentences.)
A couple of years later John Backus, a programming language designer acquainted with Chomsky's ideas,
described the syntax of the ALGOL programming language in a similar way.

Let us now give an exact definition.