\subsection{Types of graphs}
Intuitively, a graph is a set of points, some of which are joined by lines.
The points are called \emph{vertices} of the graph, the lines are called \emph{edges} of the graph.
Quite often, we represent a graph by drawing it in the plane.
For some graphs, one can draw the edges in such a way that they do not intersect.
But for other graphs self-intersections are inavoidable.
In order not to confuse the intersection points with the ``true'' vertices, we draw the vertices as small disks.
See Figure \ref{fig:TwoDrawings} for two drawings of the same graph: one with, the other without self-intersections.

\begin{figure}[ht]
\begin{center}
\includegraphics[width=.7\textwidth]{HouseTwoDrawings.pdf}
\end{center}
\caption{Two drawings of the same graph.}
\label{fig:TwoDrawings}
\end{figure}

Let us give a formal definition.