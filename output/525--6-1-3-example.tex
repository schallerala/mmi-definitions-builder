\begin{exl}
Let us construct a DFA equivalent to the NFA in Figure~\ref{fig:NFA}.
For convenience, write first the table of our NFA.
\begin{center}
\begin{tabular}{c|cc}
$\delta$ & $0$ & $1$\\
\hline
$q_0$ & $\{q_0, q_1\}$ & $\{q_0\}$\\
$q_1$ & $\{q_2\}$ & $\emptyset$\\
$q_2$ & $\{q_2\}$ & $\{q_2\}$
\end{tabular}
\end{center}
The set $Q = \{q_0, q_1, q_2\}$ has $8$ subsets, so if we follow the construction given in the theorem literally, we must write a table with $8$ rows.
However, not all of the $8$ states will be accessible from the initial state.
The inaccessible states can be removed from the automaton without affecting the language.
Therefore we will introduce new rows in our table for $M'$ only as soon as they are needed.
Also, for a better distinction we will use in $M'$ the $[\ ]$ brackets instead of the set brackets $\{\ \}$.
The result is the following table:
\begin{center}
\begin{tabular}{c|cc}
$\delta'$ & $0$ & $1$\\
\hline
$[q_0]$ & $[q_0,q_1]$ & $[q_0]$\\
$[q_0,q_1]$ & $[q_0,q_1,q_2]$ & $[q_0]$\\
$[q_0, q_1, q_2]$ & $[q_0,q_1,q_2]$ & $[q_0,q_2]$\\
$[q_0,q_2]$ & $[q_0,q_1,q_2]$ & $[q_0,q_2]$
\end{tabular}
\end{center}
For brevity, rename the states so that the table takes the form
\begin{center}
\begin{tabular}{c|cc}
$\delta'$ & $0$ & $1$\\
\hline
$q'_0$ & $q'_1$ & $q'_0$\\
$q'_1$ & $q'_2$ & $q'_0$\\
$q'_2$ & $q'_2$ & $q'_3$\\
$q'_3$ & $q'_2$ & $q'_3$
\end{tabular}
\end{center}
The corresponding transition diagram is shown in Figure \ref{fig:NFAtoDFA}.
The final states are $q'_2$ and $q'_3$ because they correspond to the sets which contain the final state $q_2$ of $M'$.
\end{exl}