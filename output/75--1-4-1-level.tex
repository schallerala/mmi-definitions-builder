\subsection{Words with repeating letters}
Consider the following problem:
\begin{quote}
\emph{How many different words of length $n=k+l+m$ can be written with $k$ letters $a$, $l$ letters $b$, and $m$ letters $c$?}
\end{quote}

We have $n$ places for the letters.
First choose $k$ places where to put the letters $a$.
This can be done in $\binom{n}{k}$ different ways.
Then from $n-k$ remaining places choose $l$ places where to put the letters $b$.
This can be done in $\binom{n-k}{l}$ different ways.
Thus by the (general) product rule the number of different words is
\[
\binom{n}{k} \binom{n-k}{l} = \frac{n!}{k!(n-k)!} \frac{(n-k)!}{l!(n-k-l)!} = \frac{n!}{k!l!m!}.
\]

We might have started by choosing $l$ places for the letters $b$, and then, say, choose $m$ places for the letters $c$.
Then we would compute the product
\[
\binom{n}{l} \binom{n-l}{m}
\]
which is the same.

The following notation is used:
\[
\frac{n!}{k!l!m!} =: \binom{n}{k, l, m}.
\]

Let us now consider a more general problem and solve it in a different way.
