\documentclass[preview, multi=page, margin=5mm, class=report]{standalone}
\usepackage[utf8]{inputenc}

\usepackage{amsmath,amssymb,amsthm}
\usepackage{graphicx,color}
\usepackage{hyperref,url}
\graphicspath{{Fig/}}

\usepackage{mathtools}
\usepackage{bussproofs}
\usepackage{stackengine}
\def\ruleoffset{1pt}
\newcommand\specialvdash[2]{\mathrel{\ensurestackMath{
  \mkern2mu\rule[-\dp\strutbox]{.4pt}{\baselineskip}\stackon[\ruleoffset]{
    \stackunder[\dimexpr\ruleoffset-.5\ht\strutbox+.5\dp\strutbox]{
      \rule[\dimexpr.5\ht\strutbox-.5\dp\strutbox]{2.5ex}{.4pt}}{
        \scriptstyle #1}}{\scriptstyle#2}\mkern2mu}}
}

\usepackage[table]{xcolor}

\renewcommand\thesection{\arabic{section}}
\renewcommand\thefigure{\arabic{figure}}
\renewcommand\theequation{\arabic{equation}}

\newtheorem{dfn}{Definition}[section]
\newtheorem{thm}[dfn]{Theorem}
\newtheorem{lem}[dfn]{Lemma}
\newtheorem{cor}[dfn]{Corollary}


\theoremstyle{definition}
\newtheorem{exl}[dfn]{Example}
\newtheorem{rem}[dfn]{Remark}
\newtheorem{exc}{Exercise}[section]

\def\R{\mathbb{R}}
\def\N{\mathbb{N}}
\def\Z{\mathbb{Z}}
\def\C{\mathbb{C}}
\def\cP{\mathcal{P}}
\def\cV{\mathcal{V}}
\def\cF{\mathcal{F}}
\def\Th{\mathrm{Th}}

\renewcommand{\emptyset}{\varnothing}
\renewcommand{\phi}{\varphi}
\renewcommand{\epsilon}{\varepsilon}
\def\gcd{\operatorname{gcd}}

\def\Prop{\mathrm{PROP}}
\begin{document}
\setcounter{section}{0}
\setcounter{subsection}{0}
\setcounter{dfn}{8}

\begin{proof}
Let $G = (X \cup Y, E)$ be a $k$-regular bipartite graph.
Since every edge is incident to exactly one vertex from $X$, and every vertex is incident to exactly $k$ edges, we have $|E| = k|X|$.
Similarly, $|E| = k|Y|$. Thus we have $|X| = |Y|$.

Let us show that $k$-regularity implies condition \eqref{eqn:Hall}.
Take any $S \subset X$ and consider the bipartite graph $G' = (S \cup N(S), E')$, where $E'$ is the set of all edges incident to a vertex from $S$.
By the above argument, $|E'| = k|S|$.
On the other hand, for every $v \in N(S)$ we have $\deg_{G'} v \le k$, which implies $|E'| \le k|N(S)|$.
It follows that
\[
|N(S)| \ge \frac{|E'|}{k} = |S|,
\]
and we are done.
\end{proof}








\end{document}