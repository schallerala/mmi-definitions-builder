\subsection{Duality for embedded graphs}
Let $G$ be a connected plane graph.
Define a new plane graph $G^*$ as follows.
Inside every face $f$ of $G$ choose a point $f^*$.
For every edge $e$ of $G$ draw an arc $e^*$ crossing the edge $e$ and joining the points $f_1^*$ and $f_2^*$ inside the faces incident with $e$.
(If $e$ is incident to one face only, then the arc $e^*$ is a loop.)

It is possible to draw all arcs $e^*$ so that they do not intersect each other.
(Mark a point in the interior of every edge;
inside every face $f$, join the point $f^*$ to the points marked on the incident edges in a non-self-intersecting way.)

See Figure \ref{fig:DualGraph} for an example.

\begin{figure}[ht]
\begin{center}
\includegraphics[width=.6\textwidth]{DualGraph.pdf}
\end{center}
\caption{A graph and its dual.}
\label{fig:DualGraph}
\end{figure}

Different planar embeddings of the same graph may have different duals.
For example, consider the duals of the graphs on Figure \ref{fig:TwoEmbeddings}.

Let us describe those graphs whose duals have no loops and multiple edges.
