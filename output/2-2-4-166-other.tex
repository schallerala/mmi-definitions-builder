\documentclass[preview, margin=5mm, multi=page]{standalone}
\usepackage[utf8]{inputenc}

\usepackage{amsmath,amssymb,amsthm}
\usepackage{graphicx,color}
\usepackage{hyperref,url}
\graphicspath{{Fig/}}

\usepackage{mathtools}
\usepackage{bussproofs}
\usepackage{stackengine}
\def\ruleoffset{1pt}
\newcommand\specialvdash[2]{\mathrel{\ensurestackMath{
  \mkern2mu\rule[-\dp\strutbox]{.4pt}{\baselineskip}\stackon[\ruleoffset]{
    \stackunder[\dimexpr\ruleoffset-.5\ht\strutbox+.5\dp\strutbox]{
      \rule[\dimexpr.5\ht\strutbox-.5\dp\strutbox]{2.5ex}{.4pt}}{
        \scriptstyle #1}}{\scriptstyle#2}\mkern2mu}}
}

\usepackage[table]{xcolor}

\renewcommand\thesection{\arabic{section}}
\renewcommand\thefigure{\arabic{figure}}
\renewcommand\theequation{\arabic{equation}}

\newtheorem{dfn}{Definition}[section]
\newtheorem{thm}[dfn]{Theorem}
\newtheorem{lem}[dfn]{Lemma}
\newtheorem{cor}[dfn]{Corollary}


\theoremstyle{definition}
\newtheorem{exl}[dfn]{Example}
\newtheorem{rem}[dfn]{Remark}
\newtheorem{exc}{Exercise}[section]

\def\R{\mathbb{R}}
\def\N{\mathbb{N}}
\def\Z{\mathbb{Z}}
\def\C{\mathbb{C}}
\def\cP{\mathcal{P}}
\def\cV{\mathcal{V}}
\def\cF{\mathcal{F}}
\def\Th{\mathrm{Th}}


\renewcommand{\emptyset}{\varnothing}
\renewcommand{\phi}{\varphi}
\renewcommand{\epsilon}{\varepsilon}
\def\gcd{\operatorname{gcd}}

\def\Prop{\mathrm{PROP}}



%opening
\title{{Lecture notes for the 2020/21 lectures}\\
$ $\\
$ $\\ \textsc{
Mathematical methods for Computer Science I \& II\\
and\\
Discrete Mathematics I \& II\\ }
$ $\\
$ $\\
$ $\\
$ $\\
University of Fribourg\\ Livio Liechti
$ $\\
$ $\\
$ $\\
$ $\\
$ $\\
$ $\\
$ $\\}
\date{ }

\author{Lecture notes written by Ivan Izmestiev for his 2018/19 lectures}


\begin{document}
\setcounter{section}{2}
\setcounter{subsection}{4}
\setcounter{dfn}{11}

We will find a spanning tree by deleting edges from the graph one by one while taking care that the graph remains connected.
For any graph $G = (V, E)$ and any its edge $e \in E$ denote by $G - e$ the graph $(V, E \setminus \{e\})$.
(Note that we are not removing any vertices, even if after deletion of $e$ an isolated vertex appears.)
This is the operation of \emph{edge deletion}.
\begin{proof}
Let $G$ be a connected graph.
If $G$ contains a cycle $C$, then let $e$ be any edge of $C$.
I claim that the graph $G - e$ is connected.
Indeed, let $v, w \in V$ be any two vertices of $G$.
Since $G$ is connected, there is a path in $G$ between $v$ and $w$.
If this path never uses the edge $e$, then this is also a path in $G - e$.
If it does use $e$, then instead going on $e$, take a detour via the path $C - e$.
This produces a walk in $G - e$ from $v$ to $w$.
A walk can be transformed into a path by removing cycles.

Thus $G - e$ is connected.
If it is acyclic, then it is a spanning tree.
Otherwise repeat the operation: take another cycle and remove an edge from it etc.
until we arrive at an acyclic connected subgraph with the same vertex set as $G$.
\end{proof}

The following theorem is a strengthening of Theorem \ref{thm:TreeEdges}.

\end{document}
