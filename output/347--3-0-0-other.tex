\documentclass[preview, margin=5mm, multi=page]{standalone}
\usepackage[utf8]{inputenc}

\usepackage{amsmath,amssymb,amsthm}
\usepackage{graphicx,color}
\usepackage{hyperref,url}
\graphicspath{{Fig/}}

\usepackage{mathtools}
\usepackage{bussproofs}
\usepackage{stackengine}
\def\ruleoffset{1pt}
\newcommand\specialvdash[2]{\mathrel{\ensurestackMath{
  \mkern2mu\rule[-\dp\strutbox]{.4pt}{\baselineskip}\stackon[\ruleoffset]{
    \stackunder[\dimexpr\ruleoffset-.5\ht\strutbox+.5\dp\strutbox]{
      \rule[\dimexpr.5\ht\strutbox-.5\dp\strutbox]{2.5ex}{.4pt}}{
        \scriptstyle #1}}{\scriptstyle#2}\mkern2mu}}
}

\usepackage[table]{xcolor}

\renewcommand\thesection{\arabic{section}}
\renewcommand\thefigure{\arabic{figure}}
\renewcommand\theequation{\arabic{equation}}

\newtheorem{dfn}{Definition}[section]
\newtheorem{thm}[dfn]{Theorem}
\newtheorem{lem}[dfn]{Lemma}
\newtheorem{cor}[dfn]{Corollary}


\theoremstyle{definition}
\newtheorem{exl}[dfn]{Example}
\newtheorem{rem}[dfn]{Remark}
\newtheorem{exc}{Exercise}[section]

\def\R{\mathbb{R}}
\def\N{\mathbb{N}}
\def\Z{\mathbb{Z}}
\def\C{\mathbb{C}}
\def\cP{\mathcal{P}}
\def\cV{\mathcal{V}}
\def\cF{\mathcal{F}}
\def\Th{\mathrm{Th}}


\renewcommand{\emptyset}{\varnothing}
\renewcommand{\phi}{\varphi}
\renewcommand{\epsilon}{\varepsilon}
\def\gcd{\operatorname{gcd}}

\def\Prop{\mathrm{PROP}}



%opening
\title{{Lecture notes for the 2020/21 lectures}\\
$ $\\
$ $\\ \textsc{
Mathematical methods for Computer Science I \& II\\
and\\
Discrete Mathematics I \& II\\ }
$ $\\
$ $\\
$ $\\
$ $\\
University of Fribourg\\ Livio Liechti
$ $\\
$ $\\
$ $\\
$ $\\
$ $\\
$ $\\
$ $\\}
\date{ }

\author{Lecture notes written by Ivan Izmestiev for his 2018/19 lectures}


\begin{document}
\setcounter{section}{1}
\setcounter{subsection}{0}
\setcounter{dfn}{8}

\chapter{Propositional logic}
\section*{Introduction}
We will study two logical systems: \emph{propositional logic} and \emph{predicate logic}.
The predicate logic is also know as first-order logic; there is also second-order logic and higher order logics.

Every logic has two aspects: \emph{syntax} and \emph{semantics}.
On the syntactic side, logic consists of a \emph{language}, which is a set of expressions built according to certain rules.
The semantics interprets each of these expressions as true or false.

There are obvious parallels to human languages, but also important differences.
Every human language has syntax rules, but firstly a real language is a living thing, so that the vocabulary and syntax are constantly changing,
and secondly one may intentionally break the rules in order to create an artistic effect (you will easily find examples in literature or cinema).
Also, a human language contains sentences without truth values, such as ``What's for lunch?'' or ``Mind the gap''.

Thus, if we apply logic to a human language, then we can deal only with declarative sentences.
An example of a declarative sentence is ``It will snow for Christmas''.
This is certainly either true or false, although we do not know it at the moment.

Here we come to an important point.
Although we have said that every logical expression has a well-defined truth value, it may be not clear what this value is.
A \emph{proof theory} provides tools for determining this truth value.
Every proof theory contains a set of \emph{inference rules} or \emph{deduction rules},
which allow to determine the truth value of an expression if the truth values of some other expressions are known.
This is, of course, the way we are using logic in the everyday life, when we are trying to convince someone.
A classical example of deduction is
\begin{quote}
All men are mortal.\\
Socrates is a man.\\
Therefore, Socrates is mortal.
\end{quote}
(If the first two statements are true, then so is the third one.)

Formalization of human reasoning was at the origin of logic and stimulated its development over the centuries.
Another important motivation was the search for foundations of mathematics (end of XIX -- beginning of XX century)
as one has tried to axiomatize mathematics and formalize the mathematical reasoning.
Every mathematical proposition is either true or false, but it can be difficult to determine which way it is.
The four-color theorem and Fermat's Last Theorem are famous examples.

The following logic textbooks are recommended: \cite{Gallier, CL1, CL2, Dalen}.




\end{document}