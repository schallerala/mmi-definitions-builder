
It is convenient to represent a DFA in the form of a \emph{transition diagram}.
A transition diagram is a graph whose vertex set is the set of states, and edges are directed and labeled by the alphabet symbols.
The edges describe the transition function: if $\delta(q_i,a) = q_j$, then we draw a directed edge from $q_i$ to $q_j$ and label it with $a$.
The initial state is indicated by an incoming arrow starting at nowhere.
The final states are indicated by double circles.
Figure \ref{fig:DFAOdd} shows an example of a DFA.

\begin{figure}[htb]
\begin{center}
\input{Fig/DFAOdd.pdf_t}
\caption{A deterministic finite automaton.}
\label{fig:DFAOdd}
\end{center}
\end{figure}

At the beginning the automaton is in the initial state.
When it receives an input word $w \in \Sigma^*$, it reads it from left to right and changes its state after each letter according to the transition function.
If after reading the input the automaton is in one of the final states, then one says that the word $w$ is \emph{accepted} by the automaton.
(Because of this, the final states are sometimes called \emph{accepting states}.)

One describes it formally by extending the transition function $\delta$ to a function $\widehat{\delta} \colon Q \times \Sigma^* \to Q$.
The value of $\widehat{\delta}(q,w)$ is the state in which the automaton ends if it starts at $q$ and reads the word $w$.
The definition is recursive:
\begin{enumerate}
\item
$\widehat{\delta}(q, \epsilon) = q$ for every state $q$;
\item
$\widehat{\delta}(q, wa) = \delta(\widehat{\delta}(q,w), a)$ for every state $q$, every word $w$, and every letter $a$.
\end{enumerate}
