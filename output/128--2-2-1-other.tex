\begin{proof}
Take any two vertices of a tree.
Since a tree is connected, there is at least one path between these two vertices.
If there is more than one path, then this implies the existence of a cycle.
Namely, take the first vertex where the two paths diverge and the first vertex where they meet again;
the union of the segments of our paths between these vertices will be a cycle.
(We are not working out the details here.)
This contradicts the assumption that our graph is a tree, thus there cannot be more than one path between two vertices.
\end{proof}

A \emph{rooted} tree is a tree $T$ with a specified vertex $x$, called the \emph{root} of~$T$.
The edges of a tree can be equipped with orientation so that for every vertex $v$ the (unique) path from $x$ to $v$ always follows the directions of edges.
(Again, this looks intuitively clear, but requires a formal proof.)
See Figure~\ref{fig:RootedTreeOrient} for an example.

\begin{figure}[ht]
\begin{center}
\includegraphics[width=.6\textwidth]{RootedTreeOrient.pdf}
\end{center}
\caption{A canonically oriented rooted tree.}
\label{fig:RootedTreeOrient}
\end{figure}

