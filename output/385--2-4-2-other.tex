\documentclass[preview, multi=page, margin=5mm, class=report]{standalone}
\usepackage[utf8]{inputenc}

\usepackage{amsmath,amssymb,amsthm}
\usepackage{graphicx,color}
\usepackage{hyperref,url}
\graphicspath{{Fig/}}

\usepackage{mathtools}
\usepackage{bussproofs}
\usepackage{stackengine}
\def\ruleoffset{1pt}
\newcommand\specialvdash[2]{\mathrel{\ensurestackMath{
  \mkern2mu\rule[-\dp\strutbox]{.4pt}{\baselineskip}\stackon[\ruleoffset]{
    \stackunder[\dimexpr\ruleoffset-.5\ht\strutbox+.5\dp\strutbox]{
      \rule[\dimexpr.5\ht\strutbox-.5\dp\strutbox]{2.5ex}{.4pt}}{
        \scriptstyle #1}}{\scriptstyle#2}\mkern2mu}}
}

\usepackage[table]{xcolor}

\renewcommand\thesection{\arabic{section}}
\renewcommand\thefigure{\arabic{figure}}
\renewcommand\theequation{\arabic{equation}}

\newtheorem{dfn}{Definition}[section]
\newtheorem{thm}[dfn]{Theorem}
\newtheorem{lem}[dfn]{Lemma}
\newtheorem{cor}[dfn]{Corollary}


\theoremstyle{definition}
\newtheorem{exl}[dfn]{Example}
\newtheorem{rem}[dfn]{Remark}
\newtheorem{exc}{Exercise}[section]

\def\R{\mathbb{R}}
\def\N{\mathbb{N}}
\def\Z{\mathbb{Z}}
\def\C{\mathbb{C}}
\def\cP{\mathcal{P}}
\def\cV{\mathcal{V}}
\def\cF{\mathcal{F}}
\def\Th{\mathrm{Th}}

\renewcommand{\emptyset}{\varnothing}
\renewcommand{\phi}{\varphi}
\renewcommand{\epsilon}{\varepsilon}
\def\gcd{\operatorname{gcd}}

\def\Prop{\mathrm{PROP}}
\begin{document}
\setcounter{section}{4}
\setcounter{subsection}{3}
\setcounter{dfn}{5}

\begin{proof}
Let us prove that if $M$ is a maximum matching, then there is no $M$-augmenting path.
By contraposition we have to show that if $G$ contains an $M$-augmenting path, then $M$ is not a maximum matching.
This follows from the observation we made before stating the theorem:
an augmenting path can be used to increase the number of edges in a matching.

For the opposite direction we have to show that if $M$ is not a maximum matching, then there is an $M$-augmenting path.
Let $M'$ be a maximum matching in $G$.
Then $|M'| > |M|$.
Let $H$ be the graph formed by those edges that belong to exactly one of $M$ and $M'$: the edge set of $H$ is
\[
(M \setminus M') \cup (M \setminus M').
\]
(This is called symmetric difference of $M$ and $M'$.)
Every vertex of $H$ has degree $1$ or $2$ because it is incident to at most one edge from $M$ and at most one edge from $M'$.
Therefore each connected component of $H$ is either an even cycle with edges alternately in $M$ and $M'$ or a path with edges alternately in $M$ and $M'$,
see Figure \ref{fig:MM'}.

\begin{figure}[ht]
\begin{center}
\includegraphics[width=.7\textwidth]{TwoMatchings.pdf}
\end{center}
\caption{Symmetric difference of a non-maximum matching $M$ (blue) and a maximum matching $M'$ (red).}
\label{fig:MM'}
\end{figure}

Due to $|M'| > |M|$, in $H$ there are more edges from $M'$ than from $M$.
Therefore there is a path that starts and ends with an $M'$-edge.
The endpoints of this path have no incident $M$-edges, otherwise such an edge would also belong to $H$, and the path would not stop here.
This path is an $M$-augmenting path and the theorem is proved.
\end{proof}


In order to find a maximum matching in a graph, start with any matching (for example, an empty set of edges).
Then, recursively, find an augmenting path and modify the current matching.
If no augmenting path can be found, then the current matching is a maximum one.
Algorithms for finding an augmenting path are described in \cite[Section 16.5]{BM}.



\end{document}