\documentclass[preview, multi=page, margin=5mm, class=report]{standalone}
\usepackage[utf8]{inputenc}

\usepackage{amsmath,amssymb,amsthm}
\usepackage{graphicx,color}
\usepackage{hyperref,url}
\graphicspath{{Fig/}}

\usepackage{mathtools}
\usepackage{bussproofs}
\usepackage{stackengine}
\def\ruleoffset{1pt}
\newcommand\specialvdash[2]{\mathrel{\ensurestackMath{
  \mkern2mu\rule[-\dp\strutbox]{.4pt}{\baselineskip}\stackon[\ruleoffset]{
    \stackunder[\dimexpr\ruleoffset-.5\ht\strutbox+.5\dp\strutbox]{
      \rule[\dimexpr.5\ht\strutbox-.5\dp\strutbox]{2.5ex}{.4pt}}{
        \scriptstyle #1}}{\scriptstyle#2}\mkern2mu}}
}

\usepackage[table]{xcolor}

\renewcommand\thesection{\arabic{section}}
\renewcommand\thefigure{\arabic{figure}}
\renewcommand\theequation{\arabic{equation}}

\newtheorem{dfn}{Definition}[section]
\newtheorem{thm}[dfn]{Theorem}
\newtheorem{lem}[dfn]{Lemma}
\newtheorem{cor}[dfn]{Corollary}


\theoremstyle{definition}
\newtheorem{exl}[dfn]{Example}
\newtheorem{rem}[dfn]{Remark}
\newtheorem{exc}{Exercise}[section]

\def\R{\mathbb{R}}
\def\N{\mathbb{N}}
\def\Z{\mathbb{Z}}
\def\C{\mathbb{C}}
\def\cP{\mathcal{P}}
\def\cV{\mathcal{V}}
\def\cF{\mathcal{F}}
\def\Th{\mathrm{Th}}

\renewcommand{\emptyset}{\varnothing}
\renewcommand{\phi}{\varphi}
\renewcommand{\epsilon}{\varepsilon}
\def\gcd{\operatorname{gcd}}

\def\Prop{\mathrm{PROP}}
\begin{document}
\setcounter{section}{3}
\setcounter{subsection}{2}
\setcounter{dfn}{8}


\begin{proof}
Induction on the number of edges.

Let $|E| = 0$. The only connected graph without edges is the graph with one vertex.
It has one face, and we have $1 - 0 + 1 = 2$. The induction base is proved.

Now take a graph with $n$ edges, $n \ge 1$.
Consider two cases.

\noindent 1) The graph is acyclic.
Then it is a tree. A tree does not separate the plane, so we have $|F| = 1$ in this case.
By Theorem \ref{thm:TreeEdges}, $|E| = |V|-1$.
Thus we have
\[
|V| - |E| + |F| = |V| - (|V|-1) + 1 = 2.
\]

\noindent 2) The graph contains a cycle.
Let $C \subset G$ be any cycle and let $e$ be any edge of $C$.
The graph $G - e$ is still connected and has one edge less.
Let us show that it also has one face less.
Indeed, the points on different sides of $e$ belong to different faces of $G$:
any arc connecting them must intersect the cycle $C$.
These two faces are merged to one face in $G - e$; all other faces are unchanged.
Thus $G$ has the same number of vertices, one edge less, and one face less than $G - e$.
By the induction assumption, Euler's formula holds for $G - e$.
Thus it also holds for $G$.
\end{proof}



\end{document}