\documentclass[preview, multi=page, margin=5mm, class=report]{standalone}
\usepackage[utf8]{inputenc}

\usepackage{amsmath,amssymb,amsthm}
\usepackage{graphicx,color}
\usepackage{hyperref,url}
\graphicspath{{Fig/}}

\usepackage{mathtools}
\usepackage{bussproofs}
\usepackage{stackengine}
\def\ruleoffset{1pt}
\newcommand\specialvdash[2]{\mathrel{\ensurestackMath{
  \mkern2mu\rule[-\dp\strutbox]{.4pt}{\baselineskip}\stackon[\ruleoffset]{
    \stackunder[\dimexpr\ruleoffset-.5\ht\strutbox+.5\dp\strutbox]{
      \rule[\dimexpr.5\ht\strutbox-.5\dp\strutbox]{2.5ex}{.4pt}}{
        \scriptstyle #1}}{\scriptstyle#2}\mkern2mu}}
}

\usepackage[table]{xcolor}

\renewcommand\thesection{\arabic{section}}
\renewcommand\thefigure{\arabic{figure}}
\renewcommand\theequation{\arabic{equation}}

\newtheorem{dfn}{Definition}[section]
\newtheorem{thm}[dfn]{Theorem}
\newtheorem{lem}[dfn]{Lemma}
\newtheorem{cor}[dfn]{Corollary}


\theoremstyle{definition}
\newtheorem{exl}[dfn]{Example}
\newtheorem{rem}[dfn]{Remark}
\newtheorem{exc}{Exercise}[section]

\def\R{\mathbb{R}}
\def\N{\mathbb{N}}
\def\Z{\mathbb{Z}}
\def\C{\mathbb{C}}
\def\cP{\mathcal{P}}
\def\cV{\mathcal{V}}
\def\cF{\mathcal{F}}
\def\Th{\mathrm{Th}}

\renewcommand{\emptyset}{\varnothing}
\renewcommand{\phi}{\varphi}
\renewcommand{\epsilon}{\varepsilon}
\def\gcd{\operatorname{gcd}}

\def\Prop{\mathrm{PROP}}
\begin{document}
\setcounter{section}{4}
\setcounter{subsection}{0}
\setcounter{dfn}{17}

\begin{proof}
Let $L \subset \Delta^*$ be a regular language, and $h \colon \Sigma^* \to \Delta^*$ a homomorphism.
Let $M = (Q, \Delta, \delta, q_0, F)$ be a DFA accepting $L$.
We will construct a DFA accepting $h^{-1}(L)$ thus proving that this language is regular.
The idea is to use the same set of states and the same set of final states, but interpret each symbol $a \in \Sigma$ as $h(a) \in \Delta$.
Then a word $w \in \Sigma^*$ will be accepted by the new automaton if and only if $h(w)$ was accepted by the old one.
Formally, put
\[
M' = (Q, \Sigma, \delta', q_0, F), \text{ where }\delta'(q, a) = \widehat{\delta}(q, h(a)).
\]
It can be shown by induction on the length of a word $w$ that $\widehat{\delta'}(q, w) = \widehat{\delta}(q, h(w))$.
Thus we have
\[
w \in L(M') \Leftrightarrow \widehat{\delta'}(q, w) \in F \Leftrightarrow \widehat{\delta}(q, h(w)) \in F \Leftrightarrow h(w) \in L
\Leftrightarrow w \in h^{-1}(L),
\]
which means $L(M') = h^{-1}(L)$.
\end{proof}






\end{document}