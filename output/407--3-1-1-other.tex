The last condition can be replaced by the requirement that $\Prop$ is the smallest (in the sense of inclusion) set satisfying the first three conditions.

A propositional formula has a recursive structure that can be represented with a \emph{parse tree}.
See Figure \ref{fig:ParseTreeProp} for an example.

\begin{figure}[ht]
\begin{center}
\input{Fig/ParseTreeProp.pdf_t}
\end{center}
\caption{Parse tree for $((p \vee q) \to \neg q)$.}
\label{fig:ParseTreeProp}
\end{figure}

Parentheses in a propositional formula ensure that the parse tree is unique.
If the root of the parsing tree is labeled with $\wedge, \vee$, or $\to$, then the formula starts with $($ and ends with $)$.
This pair of parentheses is not needed for parsing, and we will often omit it.
Strictly speaking, this is syntactically wrong, but should not lead to confusion.

