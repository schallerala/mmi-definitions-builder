The above definition is highly informal.
The formal definition says that a recursive function is one that can be obtained from basic functions
(constants, $x \mapsto x+1$, projections) by finitely many operations such as composition,
primitive recursion (which allows to construct $f(x,y) = x+y$, for example), and minimization.
(Functions which can be obtained without using minimization are called primitive recursive functions.)
It can be shown that recursive functions are exactly those which can be computed by Turing machines.
The input of a machine is a $p$-tuple of positive integers $x = (x_1, \ldots, x_p)$;
if $x \in X$, then the machine outputs $f(x)$, if $x \notin X$, then the machine outputs an error or never stops.
(Without loss of generality, one can assume that for $x \notin X$ the machine never stops: if one has a machine that outputs an error,
attach to it a machine which reacts on this error by starting an infinite loop.)

