\documentclass[preview, margin=5mm, multi=page]{standalone}
\usepackage[utf8]{inputenc}

\usepackage{amsmath,amssymb,amsthm}
\usepackage{graphicx,color}
\usepackage{hyperref,url}
\graphicspath{{Fig/}}

\usepackage{mathtools}
\usepackage{bussproofs}
\usepackage{stackengine}
\def\ruleoffset{1pt}
\newcommand\specialvdash[2]{\mathrel{\ensurestackMath{
  \mkern2mu\rule[-\dp\strutbox]{.4pt}{\baselineskip}\stackon[\ruleoffset]{
    \stackunder[\dimexpr\ruleoffset-.5\ht\strutbox+.5\dp\strutbox]{
      \rule[\dimexpr.5\ht\strutbox-.5\dp\strutbox]{2.5ex}{.4pt}}{
        \scriptstyle #1}}{\scriptstyle#2}\mkern2mu}}
}

\usepackage[table]{xcolor}

\renewcommand\thesection{\arabic{section}}
\renewcommand\thefigure{\arabic{figure}}
\renewcommand\theequation{\arabic{equation}}

\newtheorem{dfn}{Definition}[section]
\newtheorem{thm}[dfn]{Theorem}
\newtheorem{lem}[dfn]{Lemma}
\newtheorem{cor}[dfn]{Corollary}


\theoremstyle{definition}
\newtheorem{exl}[dfn]{Example}
\newtheorem{rem}[dfn]{Remark}
\newtheorem{exc}{Exercise}[section]

\def\R{\mathbb{R}}
\def\N{\mathbb{N}}
\def\Z{\mathbb{Z}}
\def\C{\mathbb{C}}
\def\cP{\mathcal{P}}
\def\cV{\mathcal{V}}
\def\cF{\mathcal{F}}
\def\Th{\mathrm{Th}}


\renewcommand{\emptyset}{\varnothing}
\renewcommand{\phi}{\varphi}
\renewcommand{\epsilon}{\varepsilon}
\def\gcd{\operatorname{gcd}}

\def\Prop{\mathrm{PROP}}



%opening
\title{{Lecture notes for the 2020/21 lectures}\\
$ $\\
$ $\\ \textsc{
Mathematical methods for Computer Science I \& II\\
and\\
Discrete Mathematics I \& II\\ }
$ $\\
$ $\\
$ $\\
$ $\\
University of Fribourg\\ Livio Liechti
$ $\\
$ $\\
$ $\\
$ $\\
$ $\\
$ $\\
$ $\\}
\date{ }

\author{Lecture notes written by Ivan Izmestiev for his 2018/19 lectures}


\begin{document}
\setcounter{section}{4}
\setcounter{subsection}{4}
\setcounter{dfn}{5}

\begin{proof}
We establish a bijection between binary trees with $n+1$ leaves and bracket-variable expressions with $n+1$ variables.

Any tree with $n+1$ leaves is the parse tree of some bracket-variable expression.
Put the variables $x_1, \ldots, x_{n+1}$ at the leaves of the tree, in the order from the left to the right.
Then mark the non-leaf vertices of the tree in the following way:
if the children of a vertex are marked with $A$ and $B$, then mark the vertex with $(AB)$.
The expression which appears at the root is the bracket-variable expression parsed by the tree.
(It has extra brackets around it, which can be removed.)

Thus one has a map from the set of binary trees to the set of bracket-variable expressions.
In order to show that this map is a bijection, one has to show that from \emph{every} bracket-variable expression
one can reconstruct \emph{uniquely} the tree which produces this expression by the above procedure.

This reconstruction (the inverse map from expressions to trees) is described as follows.
Consider the last multiplication to be performed and split the expression at this place.
Draw a root with two children and write the left part of the expression at the left child, and the right part at the right child.
Split in the same way the expressions at the child vertices and continue until all leaves will be marked with variables.
\end{proof}




\end{document}