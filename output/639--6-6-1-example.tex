\begin{exl}
\label{exl:Palindromes}
A PDA accepting the language $\{w \bar{w} \mid w \in \{0,1\}^*\}$ by the empty stack.
\[
M = (\{q_1, q_2\}, \{0, 1\}, \{A, B, Z_0\}, \delta, q_1, Z_0, \emptyset\}
\]
The principle is the same: while in the state $q_1$, we encode the input by putting into the stack $A$ for the input symbol $0$ and $B$ for the input $1$:
\begin{gather*}
\delta(q_1, 0, Z_0) = (q_1, AZ_0), \quad \delta(q_1, 0, B) = (q_1, AB)\\
\delta(q_1, 1, Z_0) = (q_1, BZ_0), \quad \delta(q_1, 1, A) = (q_1, BA)
\end{gather*}
However, if the input symbol agrees with the top stack symbol, then this might be the middle of the palindrome (but also might be not).
So, we make a guess and allow a multiple transition:
\[
\delta(q_1, 0, A) = \{(q_1, AA), (q_2, \epsilon)\}, \quad \delta(q_1, 1, B) = \{(q_1, BB), (q_2, \epsilon)\}.
\]
While in the state $q_2$, we compare the input with the content of the stack:
\[
\delta(q_2, 0, A) = (q_2, \epsilon) \quad \delta(q_2, 1, B) = (q_2, \epsilon)
\]
Finally, we have the possibility to empty the stack spontaneously if its top symbol is $Z_0$,
because this can happen only in the case if the input word was a palindrome (including the empty input):
\[
\delta(q_1, \epsilon, Z_0) = (q_2, \epsilon) \quad \delta(q_2, \epsilon, Z_0) = (q_2, \epsilon)
\]
\end{exl}