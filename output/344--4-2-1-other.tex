In other words, when we are substituting $t$ for free occurrences of $x$,
we also want all parts of $t$ to remain free in the resulting formula.

In order to determine the truth value of a non-closed formula $A$, one needs a structure $M = (U,I)$
and an assignment $\mu$ for all free variables in $A$ (that is, $\mu$ is a map from the set of free variables in $A$ to $U$).
A formula $A$ is called satisfiable (respectively, falsifiable) in $M$
if there is an assignment $\mu$ such that $(M, \mu) \vDash A$ (respectively, $(M, \mu) \not\vDash A$).
