\documentclass[preview, margin=5mm, multi=page]{standalone}
\usepackage[utf8]{inputenc}

\usepackage{amsmath,amssymb,amsthm}
\usepackage{graphicx,color}
\usepackage{hyperref,url}
\graphicspath{{Fig/}}

\usepackage{mathtools}
\usepackage{bussproofs}
\usepackage{stackengine}
\def\ruleoffset{1pt}
\newcommand\specialvdash[2]{\mathrel{\ensurestackMath{
  \mkern2mu\rule[-\dp\strutbox]{.4pt}{\baselineskip}\stackon[\ruleoffset]{
    \stackunder[\dimexpr\ruleoffset-.5\ht\strutbox+.5\dp\strutbox]{
      \rule[\dimexpr.5\ht\strutbox-.5\dp\strutbox]{2.5ex}{.4pt}}{
        \scriptstyle #1}}{\scriptstyle#2}\mkern2mu}}
}

\usepackage[table]{xcolor}

\renewcommand\thesection{\arabic{section}}
\renewcommand\thefigure{\arabic{figure}}
\renewcommand\theequation{\arabic{equation}}

\newtheorem{dfn}{Definition}[section]
\newtheorem{thm}[dfn]{Theorem}
\newtheorem{lem}[dfn]{Lemma}
\newtheorem{cor}[dfn]{Corollary}


\theoremstyle{definition}
\newtheorem{exl}[dfn]{Example}
\newtheorem{rem}[dfn]{Remark}
\newtheorem{exc}{Exercise}[section]

\def\R{\mathbb{R}}
\def\N{\mathbb{N}}
\def\Z{\mathbb{Z}}
\def\C{\mathbb{C}}
\def\cP{\mathcal{P}}
\def\cV{\mathcal{V}}
\def\cF{\mathcal{F}}
\def\Th{\mathrm{Th}}


\renewcommand{\emptyset}{\varnothing}
\renewcommand{\phi}{\varphi}
\renewcommand{\epsilon}{\varepsilon}
\def\gcd{\operatorname{gcd}}

\def\Prop{\mathrm{PROP}}



%opening
\title{{Lecture notes for the 2020/21 lectures}\\
$ $\\
$ $\\ \textsc{
Mathematical methods for Computer Science I \& II\\
and\\
Discrete Mathematics I \& II\\ }
$ $\\
$ $\\
$ $\\
$ $\\
University of Fribourg\\ Livio Liechti
$ $\\
$ $\\
$ $\\
$ $\\
$ $\\
$ $\\
$ $\\}
\date{ }

\author{Lecture notes written by Ivan Izmestiev for his 2018/19 lectures}


\begin{document}
\setcounter{section}{4}
\setcounter{subsection}{3}
\setcounter{dfn}{4}

\subsection{Bracket-variable expressions}
Assume that we have to multiply three variables $x$, $y$, and $z$.
Here by ``multiplication'' we mean any binary operation.
If this operation is commutative and associative, then neither the order of variables nor the order of operations is important:
\[
(xy)z = x(yz) = x(zy).
\]
If the operation is associative but not commutative (as multiplication of matrices for example),
then the order of variables matters, but the order of operations does not:
\[
(xy)z = x(yz) \ne x(zy).
\]
Finally, if the operation is neither commutative nor associative (as, for example, $xy := x^y$),
then we must take care both of the order of variables and the order of operations:
\[
(xy)z \ne x(yz).
\]
In how many ways can one multiply a given sequence of variables without changing their order?
For two variables there is only one way, for three variables two: $(xy)z$ and $x(yz)$,
below are all expressions with four variables:
\[
((x_1x_2)x_3)x_4, \quad (x_1(x_2x_3))x_4, \quad (x_1x_2)(x_3x_4), \quad x_1((x_2x_3)x_4), \quad x_1(x_2(x_3x_4)).
\]


\end{document}