\subsection{Rooted binary trees}
Recall that a \emph{rooted tree} is a tree with a marked vertex, the root.
We have used rooted trees (with labels at the vertices) as parse trees of propositional formulas
and as proof structures in the sequent calculus.
Edges or a rooted tree have a natural orientation such that the path from the root to every vertex goes in the direction of edges.
The \emph{out-degree} of a vertex in a rooted tree is the number of outward-directed edges incident to this vertex.
Similarly, the \emph{in-degree} is the number of inward-directed edges; the in-degree of the root is zero, and the in-degrees of all other vertices are one.

A \emph{binary rooted tree} is a rooted tree where the out-degrees of all vertices are at most two.
A \emph{full binary rooted tree} is a rooted tree where the out-degree of each vertex is either two or zero.
(Vertices with out-degree zero are the leaves of the tree.)

With the help of the handshake lemma and the relation $|V| = |E| + 1$ one can show that
a full binary rooted tree with $n+1$ leaves has $2n+1$ vertices and $2n$ edges.
