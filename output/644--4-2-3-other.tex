\documentclass[preview, margin=5mm, multi=page]{standalone}
\usepackage[utf8]{inputenc}

\usepackage{amsmath,amssymb,amsthm}
\usepackage{graphicx,color}
\usepackage{hyperref,url}
\graphicspath{{Fig/}}

\usepackage{mathtools}
\usepackage{bussproofs}
\usepackage{stackengine}
\def\ruleoffset{1pt}
\newcommand\specialvdash[2]{\mathrel{\ensurestackMath{
  \mkern2mu\rule[-\dp\strutbox]{.4pt}{\baselineskip}\stackon[\ruleoffset]{
    \stackunder[\dimexpr\ruleoffset-.5\ht\strutbox+.5\dp\strutbox]{
      \rule[\dimexpr.5\ht\strutbox-.5\dp\strutbox]{2.5ex}{.4pt}}{
        \scriptstyle #1}}{\scriptstyle#2}\mkern2mu}}
}

\usepackage[table]{xcolor}

\renewcommand\thesection{\arabic{section}}
\renewcommand\thefigure{\arabic{figure}}
\renewcommand\theequation{\arabic{equation}}

\newtheorem{dfn}{Definition}[section]
\newtheorem{thm}[dfn]{Theorem}
\newtheorem{lem}[dfn]{Lemma}
\newtheorem{cor}[dfn]{Corollary}


\theoremstyle{definition}
\newtheorem{exl}[dfn]{Example}
\newtheorem{rem}[dfn]{Remark}
\newtheorem{exc}{Exercise}[section]

\def\R{\mathbb{R}}
\def\N{\mathbb{N}}
\def\Z{\mathbb{Z}}
\def\C{\mathbb{C}}
\def\cP{\mathcal{P}}
\def\cV{\mathcal{V}}
\def\cF{\mathcal{F}}
\def\Th{\mathrm{Th}}


\renewcommand{\emptyset}{\varnothing}
\renewcommand{\phi}{\varphi}
\renewcommand{\epsilon}{\varepsilon}
\def\gcd{\operatorname{gcd}}

\def\Prop{\mathrm{PROP}}



%opening
\title{{Lecture notes for the 2020/21 lectures}\\
$ $\\
$ $\\ \textsc{
Mathematical methods for Computer Science I \& II\\
and\\
Discrete Mathematics I \& II\\ }
$ $\\
$ $\\
$ $\\
$ $\\
University of Fribourg\\ Livio Liechti
$ $\\
$ $\\
$ $\\
$ $\\
$ $\\
$ $\\
$ $\\}
\date{ }

\author{Lecture notes written by Ivan Izmestiev for his 2018/19 lectures}


\begin{document}
\setcounter{section}{2}
\setcounter{subsection}{3}
\setcounter{dfn}{9}

We will not give a detailed proof of this theorem, but will describe a procedure which, for every closed predicate formula $A$,
constructs a deduction tree with the root $\vdash A$ with one of the following two properties.
The algorithm assumes that the signature does not contain functions; if it does, some modifications are needed.

\begin{itemize}
\item
It is finite and each of its leaves is an axiom.
\item
It is finite and has a leaf from which a counterexample can be read off.
\item
It contains an infinite path producing a counterexample.
\end{itemize}

One cannot exclude the possibility of an infinite tree because there are formulas satisfied by all finite structures
but falsified by some infinite structure (see the exercises).

\begin{proof}[Algorithm description]
We assume that the signature contains no function symbols.
Given a closed formula $A$, put the sequent $\vdash A$ at the root.

From the set of variables choose an infinite sequence of symbols $u_1, u_2, \ldots, $ not occurring in $A$.
(The set $U = \{u_1, u_2, \ldots\}$ or a finite initial segment of it will later be our universe.)
During the algorithm, the variables from the sequence $u_1, u_2, \ldots$ will be \emph{activated} in their natural order.
At the beginning, only $u_1$ is active, the rest is not.

At every step of the algorithm we have some previously constructed deduction tree
and are going to apply an inference rule to one of its leaves.
Each inference rule consists in copying most of the formulas of the sequent and changing only one of them, called the \emph{principal formula}.
The type of the inference rule is determined uniquely by the principal formula: this is the rule for the outermost connective in the principal formula,
and it is the left or the right rule according on which side of $\vdash$ the principal formula is situated.
For the connectives of propositional logic the inference rules are unique.
For the quantifiers, one needs to specify the substitutions.
This is done as follows.

\begin{itemize}
\item
Right $\forall$ or left $\exists$ rule:
substitute the first inactive variable from the sequence $\{u_1, u_2, \ldots\}$.
\item
Left $\forall$ or right $\exists$ rule:
substitute all active $u_i$ not previously substituted into this formula.
\end{itemize}
That means, if there are several active variables not yet substituted into $\forall \vdash$ or $\vdash \exists$,
then the corresponding inference rule is applied several times.
We abbreviate this sequence of steps as
\begin{prooftree}
\AxiomC{$A[u_{k+1}/x], \ldots, A[u_l/x], \forall x A, \Gamma \vdash \Delta$}
\UnaryInfC{$\forall x A, \Gamma \vdash \Delta$}
\end{prooftree}
(and similarly for $\exists x A$ on the right),
where $u_{k+1}, \ldots, u_l$ are all active variables not substituted into $\forall x A$ at some previous step.

Let us note that if the signature contains function symbols, then in the left $\forall$ and right $\exists$ case one
must substitute all possible terms with currently active variables (this is a finite but possibly quite large set).

The inference rules are applied in a breadth-first way:
in one round we go over all leaves, and every non-atomic formula inside a leaf is used as a principal formula.
More exactly, if $A_1, \ldots, A_m \vdash B_1, \ldots, B_n$ is a leaf sequent,
then we first apply the inference rule with $A_1$ as the principal formula.
After that we apply the inference rule with $A_2$ as the principal formula to all ``newly grown'' leaves.
Then we work with $A_3$ in all new leaves, and so on.
Only after we finished the work with the formula $B_n$, we proceed to the next leaf sequent.

A leaf is called closed if one of the following occurs:
\begin{itemize}
\item
its sequent contains only atomic formulas;
\item
its sequent contains only atomic formulas and $\forall x$-formulas on the left or $\exists x$-formulas on the right
for which all substitutions (of active variables, see the instructions above) have already been performed.
\end{itemize}

A closed leaf with only atomic formulas is either an axiom leaf or a counterexample leaf (see Example \ref{exl:CounterexamplePred}).

Example \ref{exl:CounterexampleFinite} illustrates the second possibility.

It can happen that after every round there are non-closed leaves.
In this case the algorithm never terminates, and the sequent at its root is falsifiable.
Namely, the tree will contain an infinite path, and a counterexample can be read off this path as illustrated in Example \ref{exl:CounterexampleInfinite} below.
\end{proof}


\end{document}