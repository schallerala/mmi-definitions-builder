\documentclass[preview, multi=page, margin=5mm, class=report]{standalone}
\usepackage[utf8]{inputenc}

\usepackage{amsmath,amssymb,amsthm}
\usepackage{graphicx,color}
\usepackage{hyperref,url}
\graphicspath{{Fig/}}

\usepackage{mathtools}
\usepackage{bussproofs}
\usepackage{stackengine}
\def\ruleoffset{1pt}
\newcommand\specialvdash[2]{\mathrel{\ensurestackMath{
  \mkern2mu\rule[-\dp\strutbox]{.4pt}{\baselineskip}\stackon[\ruleoffset]{
    \stackunder[\dimexpr\ruleoffset-.5\ht\strutbox+.5\dp\strutbox]{
      \rule[\dimexpr.5\ht\strutbox-.5\dp\strutbox]{2.5ex}{.4pt}}{
        \scriptstyle #1}}{\scriptstyle#2}\mkern2mu}}
}

\usepackage[table]{xcolor}

\renewcommand\thesection{\arabic{section}}
\renewcommand\thefigure{\arabic{figure}}
\renewcommand\theequation{\arabic{equation}}

\newtheorem{dfn}{Definition}[section]
\newtheorem{thm}[dfn]{Theorem}
\newtheorem{lem}[dfn]{Lemma}
\newtheorem{cor}[dfn]{Corollary}


\theoremstyle{definition}
\newtheorem{exl}[dfn]{Example}
\newtheorem{rem}[dfn]{Remark}
\newtheorem{exc}{Exercise}[section]

\def\R{\mathbb{R}}
\def\N{\mathbb{N}}
\def\Z{\mathbb{Z}}
\def\C{\mathbb{C}}
\def\cP{\mathcal{P}}
\def\cV{\mathcal{V}}
\def\cF{\mathcal{F}}
\def\Th{\mathrm{Th}}

\renewcommand{\emptyset}{\varnothing}
\renewcommand{\phi}{\varphi}
\renewcommand{\epsilon}{\varepsilon}
\def\gcd{\operatorname{gcd}}

\def\Prop{\mathrm{PROP}}
\begin{document}
\setcounter{section}{1}
\setcounter{subsection}{4}
\setcounter{dfn}{17}

\begin{exl}
Let us construct a DFA equivalent to the $\epsilon$-NFA from Example \ref{exl:ENFA}.

The transition table is obtained by consulting the table from Example \ref{exl:ENFA}
and applying the rule $\delta'(P, a) = \overline{\delta(P, a)}$.
As in the construction of an NFA out of a DFA, it might be not necessary to consider all subsets of $Q$.
We start with the row corresponding to the initial state,
and add a new row for every state which appeared in one of the previous rows.
The construction ends when no new states appear.

The initial state in our case is $\overline{\{q_0\}} = \{q_0, q_1\}$.
\begin{center}
\begin{tabular}{c|ccc}
& $-$ & $0$ & $1$-$9$\\\hline
$\{q_0, q_1\}$ & $\{q_1\}$ & $\{q_3\}$ & $\{q_2, q_3\}$\\
$\{q_1\}$ & $\emptyset$ & $\emptyset$ & $\{q_2, q_3\}$\\
$\{q_3\}$ & $\emptyset$ & $\emptyset$ & $\emptyset$\\
$\{q_2, q_3\}$ & $\emptyset$ & $\{q_2, q_3\}$ & $\{q_2, q_3\}$\\
$\emptyset$ & $\emptyset$ & $\emptyset$ & $\emptyset$
\end{tabular}
\end{center}

\begin{figure}[ht]
\begin{center}
\input{Fig/DFAFromENFA.pdf_t}
\end{center}
\caption{A DFA equivalent to the $\epsilon$-NFA from Example \ref{exl:ENFA}.}
\label{fig:DFAFromENFA}
\end{figure}

The diagram of this automaton is shown in Figure \ref{fig:DFAFromENFA}.
\end{exl}

\end{document}