\documentclass[preview, margin=5mm, multi=page]{standalone}
\usepackage[utf8]{inputenc}

\usepackage{amsmath,amssymb,amsthm}
\usepackage{graphicx,color}
\usepackage{hyperref,url}
\graphicspath{{Fig/}}

\usepackage{mathtools}
\usepackage{bussproofs}
\usepackage{stackengine}
\def\ruleoffset{1pt}
\newcommand\specialvdash[2]{\mathrel{\ensurestackMath{
  \mkern2mu\rule[-\dp\strutbox]{.4pt}{\baselineskip}\stackon[\ruleoffset]{
    \stackunder[\dimexpr\ruleoffset-.5\ht\strutbox+.5\dp\strutbox]{
      \rule[\dimexpr.5\ht\strutbox-.5\dp\strutbox]{2.5ex}{.4pt}}{
        \scriptstyle #1}}{\scriptstyle#2}\mkern2mu}}
}

\usepackage[table]{xcolor}

\renewcommand\thesection{\arabic{section}}
\renewcommand\thefigure{\arabic{figure}}
\renewcommand\theequation{\arabic{equation}}

\newtheorem{dfn}{Definition}[section]
\newtheorem{thm}[dfn]{Theorem}
\newtheorem{lem}[dfn]{Lemma}
\newtheorem{cor}[dfn]{Corollary}


\theoremstyle{definition}
\newtheorem{exl}[dfn]{Example}
\newtheorem{rem}[dfn]{Remark}
\newtheorem{exc}{Exercise}[section]

\def\R{\mathbb{R}}
\def\N{\mathbb{N}}
\def\Z{\mathbb{Z}}
\def\C{\mathbb{C}}
\def\cP{\mathcal{P}}
\def\cV{\mathcal{V}}
\def\cF{\mathcal{F}}
\def\Th{\mathrm{Th}}


\renewcommand{\emptyset}{\varnothing}
\renewcommand{\phi}{\varphi}
\renewcommand{\epsilon}{\varepsilon}
\def\gcd{\operatorname{gcd}}

\def\Prop{\mathrm{PROP}}



%opening
\title{{Lecture notes for the 2020/21 lectures}\\
$ $\\
$ $\\ \textsc{
Mathematical methods for Computer Science I \& II\\
and\\
Discrete Mathematics I \& II\\ }
$ $\\
$ $\\
$ $\\
$ $\\
University of Fribourg\\ Livio Liechti
$ $\\
$ $\\
$ $\\
$ $\\
$ $\\
$ $\\
$ $\\}
\date{ }

\author{Lecture notes written by Ivan Izmestiev for his 2018/19 lectures}


\begin{document}
\setcounter{section}{3}
\setcounter{subsection}{3}
\setcounter{dfn}{2}

Write the generating function for the Fibonacci sequence in the following way:
\[
\frac{x}{1-x-x^2} = \frac{x}{1-(x+x^2)} = x( 1 + (x+x^2) + (x+x^2)^2 + \cdots)
\]
Multiplying out the power $(x+x^2)^k$, one obtains monomials of the form $x^{m_1+\cdots+m_k}$,
where each of $m_i$ is equal $1$ or $2$.
Summing $(x+x^2)^k$ over all $k$ (and not forgetting the factor $x$ on the right hand side of the above equation) one obtains
\[
\frac{x}{1-x-x^2} = \sum_{n=1}^\infty a_{n-1} x^n,
\]
where $a_n$ is the number of ways to represent $n$ as a sum of ones and twos.
The number of summands is not prescribed, and representations that differ in the order of summands are counted separately.

The above interpretation of Fibonacci numbers is equivalent to the following one:
the $n$-th Fibonacci number is the number of domino tilings of the $2 \times (n-1)$ rectangle.
See Figure \ref{fig:FibDomino}.

\begin{figure}[ht]
\begin{center}
\includegraphics{2nTiling}
\end{center}
\caption{This tiling corresponds to the representation $7 = 1 + 2 + 1 + 1 + 2$.}
\label{fig:FibDomino}
\end{figure}




\end{document}
