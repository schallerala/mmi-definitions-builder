\begin{proof}
We will construct a matching on the set of partitions into distinct summands
such that partitions with an even number of summands are matched to partitions with an odd number of summands.
For $n$ different from $\frac{3k^2 \pm k}2$ this matching will be perfect,
and for $n = \frac{3k^2 \pm k}2$ exactly one partition will remain unmatched.

For every partition $\lambda$ denote by $i(\lambda)$ the size of its smallest part.
By $j(\lambda)$ denote the number of consecutive parts of $\lambda$ starting with the largest part.
On the Ferrers diagram, $i(\lambda)$ is the length of the last row,
and $j(\lambda)$ is the length of the diagonal starting from the top right dot (let us call it the rightmost diagonal),
see Figure \ref{fig:IandJ}.

\begin{figure}[ht]
\begin{center}
\input{Fig/IandJ.pdf_t}
\end{center}
\caption{Definition of $i(\lambda)$ and $j(\lambda)$.}
\label{fig:IandJ}
\end{figure}

Take a partition $\lambda$ with $i(\lambda) \le j(\lambda)$ and look at its diagram.
Construct a new diagram by removing the last row and adding a dot to each of the first $i$ rows.
(In other words, move the last row so that it becomes a new rightmost diagonal.)
This yields a new partition $\lambda'$ with $j(\lambda') = i(\lambda)$ (because the new rightmost diagonal is the old last row)
and $i(\lambda') > i(\lambda)$ (because the new last row is longer than the old one).
In particular, $i(\lambda') > j(\lambda')$.
See Figure~\ref{fig:PentaMatching}.

\begin{figure}[ht]
\begin{center}
\includegraphics[width=.7\textwidth]{PentaMatching}
\end{center}
\caption{Transforming a $(i \le j)$-partition into a $(i > j)$-partition.}
\label{fig:PentaMatching}
\end{figure}

This is an (almost) perfect matching between partitions with $i \le j$ and partitions with $i > j$.
Indeed, the above transformation is (almost always) invertible:
take a partition with $i > j$ and move its rightmost diagonal to the bottom so that it becomes the new last row.

Also, this matching fulfills our needs: when the last row is moved, the number of parts changes by $1$,
so that partitions with an even number of parts are matched to partitions with an odd number of parts.

Why is it only almost perfect?
The transformation of a $(i \le j)$-partition into a $(i > j)$-partition is not going to work if $i=j$ and the last row intersects the rightmost diagonal
(in this case the new rightmost diagonal will ``stick out'').
A partition with this property looks as shown on Figure \ref{fig:FerrersPenta}, left.
Thus it is only possible for $n = \frac{3k^2 - k}2$.
The inverse transformation of a $(i > j)$-partition into a $(i \le j)$-partition will not work
if $i = j+1$ and the last row intersects the rightmost diagonal
(in this case after the transformation the two last rows have equal length, so the partition does not have distinct parts).
Such partitions look as shown on Figure \ref{fig:FerrersPenta}, right, and are only possible for $n = \frac{3k^2 + k}2$.

The ``lonely'' partition has $k$ rows.
Thus if $k$ is even, then there is one even partition more than odd,
and if $k$ is odd, then there is one odd partition more than even.
\end{proof}




