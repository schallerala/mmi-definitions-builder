In Figure \ref{fig:PentaTriangulations} all triangulations of a regular pentagon are presented.
As one can see, triangulations which differ by rotation or reflection of the polygon are counted separately.

\begin{figure}[ht]
\begin{center}
\includegraphics[width=\textwidth]{PentaTriangulations}
\end{center}
\caption{All $5$ triangulations of the pentagon.}
\label{fig:PentaTriangulations}
\end{figure}


\begin{proof}
We establish a bijection between triangulations and binary trees.

Place a dot inside every triangle of the triangulation and place a dot near every edge of the polygon just outside of the polygon.
Then draw a segment between every pair of vertices separated by an edge of the triangulation or by an edge of the polygon.
The result is a tree; every vertex inside of a triangle has degree $3$, and the vertices outside of the polygon are leaves.
Remove the dot at the base edge of the polygon and the edge incident to it.
The result is a rooted binary tree.
See Figure \ref{fig:TriangToTree} for an example.

\begin{figure}[ht]
\begin{center}
\includegraphics[width=.9\textwidth]{TriangToTree}
\end{center}
\caption{From a triangulation to a binary tree.}
\label{fig:TriangToTree}
\end{figure}

The base edge is chosen in advance.
For example, if the polygon ``stands'' on a line, one can declare the lowest edge the base edge.
Distinguishing the base edge reflects the fact that the polygon is not allowed to rotate.

From every binary tree one can reconstruct the corresponding triangulation in a unique way.
First, add an edge to the root so that all non-leaf vertices have degree $3$.
Then draw a triangle around each vertex so that each of the sides of the triangle intersects one edge of the tree.
Finally, for every edge of the tree glue the triangles surrounding the incident vertices along their sides intersecting this edge.
\end{proof}


