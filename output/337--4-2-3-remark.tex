\begin{rem}
It is important to work in a breadth-first way, otherwise one can obtain an infinite tree for a valid formula.
This happens, for example, in the following deduction tree for the formula $(P \vee \neg P) \vee \exists x \forall y P(x,y)$:
\begin{prooftree}
\AxiomC{$\cdots$}
\UnaryInfC{$\vdash Q, \neg Q, P(u_1,u_2), P(u_2,u_3), \exists x \forall y P(x,y)$}
\UnaryInfC{$\cdots$}
\UnaryInfC{$\vdash Q, \neg Q, \exists x \forall y P(x,y)$}
\UnaryInfC{$\vdash (Q \vee \neg Q) \vee \exists x \forall y P(x,y)$}
\UnaryInfC{$\vdash (Q \vee \neg Q) \vee \exists x \forall y P(x,y)$}
\end{prooftree}
Here we are neglecting the non-atomic formula $\neg Q$ and working with $\exists x \forall y P(x,y)$ only,
which produces an infinite path from Example \ref{exl:CounterexampleInfinite}.
Constructed in a breadth-first way, this tree will close with an axiom leaf with $Q$ on both sides of $\vdash$.
\end{rem}