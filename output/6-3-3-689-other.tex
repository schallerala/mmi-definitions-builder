\documentclass[preview, margin=5mm, multi=page]{standalone}
\usepackage[utf8]{inputenc}

\usepackage{amsmath,amssymb,amsthm}
\usepackage{graphicx,color}
\usepackage{hyperref,url}
\graphicspath{{Fig/}}

\usepackage{mathtools}
\usepackage{bussproofs}
\usepackage{stackengine}
\def\ruleoffset{1pt}
\newcommand\specialvdash[2]{\mathrel{\ensurestackMath{
  \mkern2mu\rule[-\dp\strutbox]{.4pt}{\baselineskip}\stackon[\ruleoffset]{
    \stackunder[\dimexpr\ruleoffset-.5\ht\strutbox+.5\dp\strutbox]{
      \rule[\dimexpr.5\ht\strutbox-.5\dp\strutbox]{2.5ex}{.4pt}}{
        \scriptstyle #1}}{\scriptstyle#2}\mkern2mu}}
}

\usepackage[table]{xcolor}

\renewcommand\thesection{\arabic{section}}
\renewcommand\thefigure{\arabic{figure}}
\renewcommand\theequation{\arabic{equation}}

\newtheorem{dfn}{Definition}[section]
\newtheorem{thm}[dfn]{Theorem}
\newtheorem{lem}[dfn]{Lemma}
\newtheorem{cor}[dfn]{Corollary}


\theoremstyle{definition}
\newtheorem{exl}[dfn]{Example}
\newtheorem{rem}[dfn]{Remark}
\newtheorem{exc}{Exercise}[section]

\def\R{\mathbb{R}}
\def\N{\mathbb{N}}
\def\Z{\mathbb{Z}}
\def\C{\mathbb{C}}
\def\cP{\mathcal{P}}
\def\cV{\mathcal{V}}
\def\cF{\mathcal{F}}
\def\Th{\mathrm{Th}}


\renewcommand{\emptyset}{\varnothing}
\renewcommand{\phi}{\varphi}
\renewcommand{\epsilon}{\varepsilon}
\def\gcd{\operatorname{gcd}}

\def\Prop{\mathrm{PROP}}



%opening
\title{{Lecture notes for the 2020/21 lectures}\\
$ $\\
$ $\\ \textsc{
Mathematical methods for Computer Science I \& II\\
and\\
Discrete Mathematics I \& II\\ }
$ $\\
$ $\\
$ $\\
$ $\\
University of Fribourg\\ Livio Liechti
$ $\\
$ $\\
$ $\\
$ $\\
$ $\\
$ $\\
$ $\\}
\date{ }

\author{Lecture notes written by Ivan Izmestiev for his 2018/19 lectures}


\begin{document}
\setcounter{section}{3}
\setcounter{subsection}{3}
\setcounter{dfn}{12}

\begin{proof}
Let $L$ be a regular language, and let $r$ be a regular expression representing $L$.
Make a substitution in $r$, replacing every symbol by its image under the homomorphism.
The result is a regular expression in the alphabet $\Delta$; denote it by $h(r)$.
Then the language defined by $h(r)$ is $h(L)$, thus $h(L)$ is regular.

The claim $L(h(r)) = h(L(r))$ is proved by induction on the complexity of the expression $r$.
Here is the induction step to $h = h_1 + h_2$:
\begin{multline*}
L(h(r_1+r_2)) = L(h(r_1)+h(r_2)) = L(h(r_1)) \cup L(h(r_2))\\
= h(L(r_1)) \cup h(L(r_2)) = h(L(r_1) \cup L(r_2)) = h(L(r_1+r_2)).
\end{multline*}
\end{proof}

For example, if $h(0) = 0$ and $h(1) = 10$, then $h((0+1)^*) = (0+10)^*$.

The next example shows that a homomorphic image of a non-regular language can be regular.


\end{document}
