\documentclass[preview, multi=page, margin=5mm, class=report]{standalone}
\usepackage[utf8]{inputenc}

\usepackage{amsmath,amssymb,amsthm}
\usepackage{graphicx,color}
\usepackage{hyperref,url}
\graphicspath{{Fig/}}

\usepackage{mathtools}
\usepackage{bussproofs}
\usepackage{stackengine}
\def\ruleoffset{1pt}
\newcommand\specialvdash[2]{\mathrel{\ensurestackMath{
  \mkern2mu\rule[-\dp\strutbox]{.4pt}{\baselineskip}\stackon[\ruleoffset]{
    \stackunder[\dimexpr\ruleoffset-.5\ht\strutbox+.5\dp\strutbox]{
      \rule[\dimexpr.5\ht\strutbox-.5\dp\strutbox]{2.5ex}{.4pt}}{
        \scriptstyle #1}}{\scriptstyle#2}\mkern2mu}}
}

\usepackage[table]{xcolor}

\renewcommand\thesection{\arabic{section}}
\renewcommand\thefigure{\arabic{figure}}
\renewcommand\theequation{\arabic{equation}}

\newtheorem{dfn}{Definition}[section]
\newtheorem{thm}[dfn]{Theorem}
\newtheorem{lem}[dfn]{Lemma}
\newtheorem{cor}[dfn]{Corollary}


\theoremstyle{definition}
\newtheorem{exl}[dfn]{Example}
\newtheorem{rem}[dfn]{Remark}
\newtheorem{exc}{Exercise}[section]

\def\R{\mathbb{R}}
\def\N{\mathbb{N}}
\def\Z{\mathbb{Z}}
\def\C{\mathbb{C}}
\def\cP{\mathcal{P}}
\def\cV{\mathcal{V}}
\def\cF{\mathcal{F}}
\def\Th{\mathrm{Th}}

\renewcommand{\emptyset}{\varnothing}
\renewcommand{\phi}{\varphi}
\renewcommand{\epsilon}{\varepsilon}
\def\gcd{\operatorname{gcd}}

\def\Prop{\mathrm{PROP}}
\begin{document}
\setcounter{section}{3}
\setcounter{subsection}{4}
\setcounter{dfn}{5}

\begin{proof}
This is similar to the money changing problem, but with banknotes of any denomination available.

The right hand side is equal to
\[
(1+x+x^2+\cdots)(1+x^2+x^4+\cdots)(1+x^3+x^6+\cdots)\cdots = \sum x^{m_1 + 2m_2 + 3m_3 + \cdots + km_k},
\]
where the sum is taken over all $k$ and all collections of non-negative integers $m_1, \ldots, m_k$.
When we collect the like terms, the coefficient at $x^n$ will be equal to the number of representations of $n$ in the form
$m_1 + 2m_2 + \cdots + km_k$.
This corresponds to a unique partition, namely to
\[
n = \underbrace{k+\cdots+k}_{m_k} + \cdots + \underbrace{1+\cdots+1}_{m_1}.
\]
Thus the product on the right hand side is equal to the generating function of the number of partitions.
\end{proof}

In the above proof we met an infinite product of power series.
This product is again a power series because in order to compute the coefficient at $x^n$
only finitely many factors from the infinite product are needed (the first $n$ factors in the above case).
Definitions below formalize this.


\end{document}