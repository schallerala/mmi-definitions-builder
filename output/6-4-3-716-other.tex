\documentclass[preview, multi=page, margin=5mm, class=report]{standalone}
\usepackage[utf8]{inputenc}

\usepackage{amsmath,amssymb,amsthm}
\usepackage{graphicx,color}
\usepackage{hyperref,url}
\graphicspath{{Fig/}}

\usepackage{mathtools}
\usepackage{bussproofs}
\usepackage{stackengine}
\def\ruleoffset{1pt}
\newcommand\specialvdash[2]{\mathrel{\ensurestackMath{
  \mkern2mu\rule[-\dp\strutbox]{.4pt}{\baselineskip}\stackon[\ruleoffset]{
    \stackunder[\dimexpr\ruleoffset-.5\ht\strutbox+.5\dp\strutbox]{
      \rule[\dimexpr.5\ht\strutbox-.5\dp\strutbox]{2.5ex}{.4pt}}{
        \scriptstyle #1}}{\scriptstyle#2}\mkern2mu}}
}

\usepackage[table]{xcolor}

\renewcommand\thesection{\arabic{section}}
\renewcommand\thefigure{\arabic{figure}}
\renewcommand\theequation{\arabic{equation}}

\newtheorem{dfn}{Definition}[section]
\newtheorem{thm}[dfn]{Theorem}
\newtheorem{lem}[dfn]{Lemma}
\newtheorem{cor}[dfn]{Corollary}


\theoremstyle{definition}
\newtheorem{exl}[dfn]{Example}
\newtheorem{rem}[dfn]{Remark}
\newtheorem{exc}{Exercise}[section]

\def\R{\mathbb{R}}
\def\N{\mathbb{N}}
\def\Z{\mathbb{Z}}
\def\C{\mathbb{C}}
\def\cP{\mathcal{P}}
\def\cV{\mathcal{V}}
\def\cF{\mathcal{F}}
\def\Th{\mathrm{Th}}

\renewcommand{\emptyset}{\varnothing}
\renewcommand{\phi}{\varphi}
\renewcommand{\epsilon}{\varepsilon}
\def\gcd{\operatorname{gcd}}

\def\Prop{\mathrm{PROP}}
\begin{document}
\setcounter{section}{4}
\setcounter{subsection}{3}
\setcounter{dfn}{8}



From any DFA one can construct the minimum DFA accepting the same language by merging certain sets of states into one state.
Two states $q_i$, $q_j$ must be merged if the corresponding sets $T_i$, $T_j$ are contained in the same equivalence class $S_k$.
That is, $q_i \sim q_j$ if for all $x \in \Sigma^*$ either both $\widehat{\delta}(q_i, x)$ and $\widehat{\delta}(q_j, x)$ belong to $F$ or both do not.

One can certify non-equivalence of two states by finding a word $x$ such that $\widehat{\delta}(q_i, x) \in F$ and $\widehat{\delta}(q_j, x) \notin F$
or vice versa.
The algorithm marks pairs of distinguishable states recursively.

At the very beginning one removes all inaccessible states.
Obviously, this does not change the accepted language.

Then one draws a table whose rows and columns are marked by the remaining states.
As initialization, one marks all pairs $(q_i,q_j)$ such that $q_i \in F$ and $q_j \notin F$ or vice versa.
At each of the following steps one goes through all pairs $(q_i,q_j)$, and for each of them one considers all alphabet letters $a$.
If the pair $(\delta(q_i,a), \delta(q_j,a))$ was marked at one of the previous steps, then one marks the pair $(q_i, q_j)$.
If at some step no new pairs are marked, then the algorithm stops.
All pairs of states which are unmarked are merged into a single state
(it is also possible that several states are merged into one state).


\end{document}
