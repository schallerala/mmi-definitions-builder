\documentclass[preview, multi=page, margin=5mm, class=report]{standalone}
\usepackage[utf8]{inputenc}

\usepackage{amsmath,amssymb,amsthm}
\usepackage{graphicx,color}
\usepackage{hyperref,url}
\graphicspath{{Fig/}}

\usepackage{mathtools}
\usepackage{bussproofs}
\usepackage{stackengine}
\def\ruleoffset{1pt}
\newcommand\specialvdash[2]{\mathrel{\ensurestackMath{
  \mkern2mu\rule[-\dp\strutbox]{.4pt}{\baselineskip}\stackon[\ruleoffset]{
    \stackunder[\dimexpr\ruleoffset-.5\ht\strutbox+.5\dp\strutbox]{
      \rule[\dimexpr.5\ht\strutbox-.5\dp\strutbox]{2.5ex}{.4pt}}{
        \scriptstyle #1}}{\scriptstyle#2}\mkern2mu}}
}

\usepackage[table]{xcolor}

\renewcommand\thesection{\arabic{section}}
\renewcommand\thefigure{\arabic{figure}}
\renewcommand\theequation{\arabic{equation}}

\newtheorem{dfn}{Definition}[section]
\newtheorem{thm}[dfn]{Theorem}
\newtheorem{lem}[dfn]{Lemma}
\newtheorem{cor}[dfn]{Corollary}


\theoremstyle{definition}
\newtheorem{exl}[dfn]{Example}
\newtheorem{rem}[dfn]{Remark}
\newtheorem{exc}{Exercise}[section]

\def\R{\mathbb{R}}
\def\N{\mathbb{N}}
\def\Z{\mathbb{Z}}
\def\C{\mathbb{C}}
\def\cP{\mathcal{P}}
\def\cV{\mathcal{V}}
\def\cF{\mathcal{F}}
\def\Th{\mathrm{Th}}

\renewcommand{\emptyset}{\varnothing}
\renewcommand{\phi}{\varphi}
\renewcommand{\epsilon}{\varepsilon}
\def\gcd{\operatorname{gcd}}

\def\Prop{\mathrm{PROP}}
\begin{document}
\setcounter{section}{1}
\setcounter{subsection}{4}
\setcounter{dfn}{12}

\begin{proof}
First let us find a formula that represents a very simple function: a function that takes value $1$ at one point only.
Take any vector $x \in \{0,1\}^n$ and put
\[
f(x) = 1, \quad f(y) = 0 \text{ for all }y \ne x.
\]
We need a propositional formula with symbols $p_1, \ldots, p_n$ that evaluates to $1$ only when $v(p_i) = x_i$ for all $i$.
This is achieved by a conjunction
\[
B = C_1 \wedge C_2 \wedge \cdots \wedge C_n,
\]
where
\[
C_i =
\begin{cases}
p_i, &\text{ if } x_1 = 1,\\
\neg p_i, &\text{ if }x_1 = 0.
\end{cases}
\]
Let us prove that $f_B = f$ in a formal way.
Take any $y \in \{0,1\}^n$.
By definition, $f_B(y) = \hat{v}(B)$ for the valuation $v(p_i) = y_i$.
Apply the recursive definition of $\hat{v}$:
\[
f_B(y) = \hat{v}(B) = \hat{v}(C_1) \wedge \hat{v}(C_2) \wedge \cdots \wedge \hat{v}(C_n).
\]
The right hand side is equal to $1$ if and only if $\hat{v}(C_i) = 1$ for all $i$.
By definition of $C_i$ we have
\begin{align*}
\text{if }x_i = 1,\ &\text{then }\hat{v}(C_i) = \hat{v}(p_i) = v(p_i) = y_i\\
\text{if }x_i = 0,\ &\text{then } \hat{v}(C_i) = \hat{v}(\neg p_i) = \neg \hat{v}(p_i) = \neg y_i,
\end{align*}
which implies that $\hat{v}(C_i) = 1$ if and only if $y_i = x_i$.
It follows that $f_B(y) = 1$ if and only if $y_i = x_i$ for all $i$, that is $f_B = f$.

Now let $f$ be an arbitrary Boolean function of $n$ arguments.
If $f(x) = 0$ for all $x$, then one can represent $f$ by the formula $p_1 \wedge \neg p_1$.
Assume that there is at least one $x$ such that $f(x) = 1$.
For every $x \in \{0,1\}^n$ such that $f(x) = 1$ construct a conjunction $B_x$ as above and take the disjunction of all such $B_x$:
\[
A = \bigvee_{f(x)=1} B_x.
\]
We claim that $f_A = f$.
Indeed, $A$ evaluates to $1$ if and only if at least one of $B_x$ evaluates to $1$
and, as we know, $B_x$ evaluates to $1$ only at $x$.
Thus $A$ evaluates to $1$ exactly at those points where $f(x) = 1$.

Formally, for every $y$ we have
\[
f_A(y) = \hat{v}(A) = \bigvee_{f(x)=1} \hat{v}(B_x)
\]
(where $v(p_i) = y_i$).
The right hand side equals $1$ if and only if $\hat{v}(B_x) = 1$ for some $x$.
But $\hat{v}(B_x) = 1$ if and only if $y=x$.
Thus $f_A(y) = 1$ if and only if $y=x$ for some $x$ such that $f(x)=1$, which is just a complicated way to say if and only if $f(y) = 1$.
Hence $f_A = f$, and the theorem is proved.
\end{proof}

The above argument provides a procedure of writing a formula with a given truth table.
We will illustrate it on the example of the function given by the table below.

\begin{center}
\begin{tabular}{|c|c|c|}
\hline
$p$ & $q$ & ?\\\hline
$0$ & $0$ & $0$\\\hline
$0$ & $1$ & $1$\\\hline
$1$ & $0$ & $1$\\\hline
$1$ & $1$ & $0$\\\hline
\end{tabular}
\end{center}

Mark the rows with $1$ at the end.
For each of these rows write a conjunction of all proposition symbols, negated or not depending on whether the value of this symbol in this row is $0$ or $1$.
In our case these are $\neg p \wedge q$ and $p \wedge \neg q$.
Then write the disjunction of the obtained conjunctions:
\[
(\neg p \wedge q) \vee (p \wedge \neg q).
\]

Observe that the formula constructed in Theorem \ref{thm:BooleNF} does not use the connective $\to$.

\end{document}