
We don't give a proof of the above theorem, but there is a simple algorithm that allows to compute the coefficients $c_i$, respectively $c_{ij}$.
Bring the equation to a common denominator, it becomes an equation between the numerators.
The numerators are polynomials; they are equal if and only if their corresponding coefficients are equal.
This yields a system of linear equations on the unknowns $c_i$ (respectively $c_{ij}$).

Because of
\[
\frac{1}{1-\lambda_i x} = 1 + \lambda_i x + \lambda_i^2 x^2 + \cdots
\]
the first part of the above theorem implies that every linear recursive sequence has the form $a_n = \sum_i c_i \lambda_i^n$,
if the characteristic polynomial has only simple roots $\lambda_i$.
In the case of multiple roots we need to represent the quotient $\frac{1}{(1-\lambda x)^j}$ as a formal power series.

