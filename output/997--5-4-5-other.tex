\documentclass[preview, multi=page, margin=5mm, class=report]{standalone}
\usepackage[utf8]{inputenc}

\usepackage{amsmath,amssymb,amsthm}
\usepackage{graphicx,color}
\usepackage{hyperref,url}
\graphicspath{{Fig/}}

\usepackage{mathtools}
\usepackage{bussproofs}
\usepackage{stackengine}
\def\ruleoffset{1pt}
\newcommand\specialvdash[2]{\mathrel{\ensurestackMath{
  \mkern2mu\rule[-\dp\strutbox]{.4pt}{\baselineskip}\stackon[\ruleoffset]{
    \stackunder[\dimexpr\ruleoffset-.5\ht\strutbox+.5\dp\strutbox]{
      \rule[\dimexpr.5\ht\strutbox-.5\dp\strutbox]{2.5ex}{.4pt}}{
        \scriptstyle #1}}{\scriptstyle#2}\mkern2mu}}
}

\usepackage[table]{xcolor}

\renewcommand\thesection{\arabic{section}}
\renewcommand\thefigure{\arabic{figure}}
\renewcommand\theequation{\arabic{equation}}

\newtheorem{dfn}{Definition}[section]
\newtheorem{thm}[dfn]{Theorem}
\newtheorem{lem}[dfn]{Lemma}
\newtheorem{cor}[dfn]{Corollary}


\theoremstyle{definition}
\newtheorem{exl}[dfn]{Example}
\newtheorem{rem}[dfn]{Remark}
\newtheorem{exc}{Exercise}[section]

\def\R{\mathbb{R}}
\def\N{\mathbb{N}}
\def\Z{\mathbb{Z}}
\def\C{\mathbb{C}}
\def\cP{\mathcal{P}}
\def\cV{\mathcal{V}}
\def\cF{\mathcal{F}}
\def\Th{\mathrm{Th}}

\renewcommand{\emptyset}{\varnothing}
\renewcommand{\phi}{\varphi}
\renewcommand{\epsilon}{\varepsilon}
\def\gcd{\operatorname{gcd}}

\def\Prop{\mathrm{PROP}}
\begin{document}
\setcounter{section}{4}
\setcounter{subsection}{5}
\setcounter{dfn}{8}

\begin{proof}
Let us show that the sequence $d_n$ of numbers of Dyck paths satisfies the same recurrence relation that the sequence of Catalan numbers.
Take a path from $(0,0)$ to $(n+1,n+1)$ and let $(k+1,k+1)$ be the first point after $(0,0)$ where it touches the diagonal.
The number $k$ can take any value between $0$ and $n$.
The point $(k+1,k+1)$ separates the path into two parts.
The first part never touches the diagonal except at the endpoints.
If we remove from it the initial and the terminal segments, then we get a Dyck path from $(1,0)$ to $(k+1,k)$.
It can be identified by translation with a Dyck path from $(0,0)$ to $(k,k)$.
The part of the path after the point $(k+1,k+1)$ can be identified with a Dyck path from $(0,0)$ to $(n-k,n-k)$.

\begin{figure}[ht]
\begin{center}
\input{Fig/DyckInduction.pdf_t}
\end{center}
\caption{Proving the recursive relation for the number of Dyck paths.}
\label{fig:DyckInduction}
\end{figure}

Conversely, from any $k$-Dyck path and any $(n-k)$-Dyck path one can build a $(n+1)$-Dyck path
by ``lifting up'' the $k$-path and concatenating it with the $(n-k)$-path.
Thus the number of $(n+1)$-Dyck paths whose first contact with the diagonal is at $(k+1, k+1)$ is $d_k d_{n-k}$,
and the total number of $(n+1)$-Dyck paths is
\[
d_{n+1} = \sum_{k=0}^n d_k d_{n-k}.
\]
Thus we have for the sequence $d_n$ the same recursive relation, and also the same starting value $d_0 = 1$.
It follows that $d_n = c_n$ for all $n$.
\end{proof}

% \begin{proof}[Proof by bijection]
% Take a full binary tree with $n+1$ leaves.
% Write its vertices in the depth-first order $v_0v_1\ldots v_{2n}$.
% Associate to it a sequence $a_1a_2 \ldots a_{2n}$ of brackets in the following way:
% \[
% a_i =
% \begin{cases}
% (, &\text{if }v_i\text{ is a child of }v_{i-1},\\
% ), &\text{otherwise}.
% \end{cases}
% \]
% Let us show that this sequence of brackets is balanced.
% Every vertex except the root is either the left child or the right child of its parent.
% The definition of the sequence $a_i$ is equivalent to the following:
% \[
% a_i =
% \begin{cases}
% (, &\text{if }v_i\text{ is the left child of its parent},\\
% ), &\text{if }v_i\text{ is the right child of its parent}.
% \end{cases}
% \]
% In every pair of siblings, the left one precedes in our ordering the right one.
% This implies that the number of closing brackets never exceeds the number of opening brackets.
% \end{proof}



\end{document}