\documentclass[preview, multi=page, margin=5mm, class=report]{standalone}
\usepackage[utf8]{inputenc}

\usepackage{amsmath,amssymb,amsthm}
\usepackage{graphicx,color}
\usepackage{hyperref,url}
\graphicspath{{Fig/}}

\usepackage{mathtools}
\usepackage{bussproofs}
\usepackage{stackengine}
\def\ruleoffset{1pt}
\newcommand\specialvdash[2]{\mathrel{\ensurestackMath{
  \mkern2mu\rule[-\dp\strutbox]{.4pt}{\baselineskip}\stackon[\ruleoffset]{
    \stackunder[\dimexpr\ruleoffset-.5\ht\strutbox+.5\dp\strutbox]{
      \rule[\dimexpr.5\ht\strutbox-.5\dp\strutbox]{2.5ex}{.4pt}}{
        \scriptstyle #1}}{\scriptstyle#2}\mkern2mu}}
}

\usepackage[table]{xcolor}

\renewcommand\thesection{\arabic{section}}
\renewcommand\thefigure{\arabic{figure}}
\renewcommand\theequation{\arabic{equation}}

\newtheorem{dfn}{Definition}[section]
\newtheorem{thm}[dfn]{Theorem}
\newtheorem{lem}[dfn]{Lemma}
\newtheorem{cor}[dfn]{Corollary}


\theoremstyle{definition}
\newtheorem{exl}[dfn]{Example}
\newtheorem{rem}[dfn]{Remark}
\newtheorem{exc}{Exercise}[section]

\def\R{\mathbb{R}}
\def\N{\mathbb{N}}
\def\Z{\mathbb{Z}}
\def\C{\mathbb{C}}
\def\cP{\mathcal{P}}
\def\cV{\mathcal{V}}
\def\cF{\mathcal{F}}
\def\Th{\mathrm{Th}}

\renewcommand{\emptyset}{\varnothing}
\renewcommand{\phi}{\varphi}
\renewcommand{\epsilon}{\varepsilon}
\def\gcd{\operatorname{gcd}}

\def\Prop{\mathrm{PROP}}
\begin{document}
\setcounter{section}{2}
\setcounter{subsection}{2}
\setcounter{dfn}{3}

\begin{proof}
The equation $A(x)B(x) = 1$ consists of an infinite sequence of equations
\begin{gather*}
a_0b_0 = 1\\
a_0b_1 + a_1b_0 = 0\\
a_0b_2 + a_1b_1 + a_2b_0 = 0\\
\cdots
\end{gather*}
with unknowns $b_0, b_1, \ldots$.
The first equation implies $b_0 = \frac{1}{a_0}$ (which is defined because $a_0 \ne 0$).
Knowing $b_0$ we can express $b_1$ from the second equation:
\[
b_1 = -\frac{a_1b_0}{a_0}
\]
and continue in the same spirit, because $(k+1)$-st equation can be solved for $b_k$:
\[
b_k = -\frac{1}{a_0}\sum_{i=1}^k a_ib_{k-i}.
\]
This shows that the inverse series $B(x)$ exists and is unique.
\end{proof}

We denote the inverse series to $A(x)$ by $(A(x))^{-1}$ or $\frac{1}{A(x)}$.


\end{document}