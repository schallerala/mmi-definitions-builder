\subsection{Propositional logic inside predicate logic}
Consider the signature without functions and with nullary predicates only:
\begin{equation}
\label{eqn:NullSign}
\cP:\ P_1(), P_2(), \ldots
\end{equation}
Formulas in this signature contain no terms, because a term must occur as an argument of a predicate, but nullary predicates have no arguments.
We can introduce variables in the formulas only with quantifiers by writing something like $\forall x \exists y P \wedge Q$,
but in any structure this formula evaluates in the same way as $P \wedge Q$.
Thus the formulas in signature \eqref{eqn:NullSign} look like propositional formulas with variable symbols $P_i$.

How does $P \wedge Q$ actually evaluate?
By definition, a first-order structure $(U,I)$ assigns to each nullary predicate $P$ a truth value $I(P)$.
Then the truth value of $P \wedge Q$ is $I(P) \wedge I(Q)$ (and the universe $U$ has no significance).
Similarly for every other formula: an evaluation with respect to interpretation $I$ is the same as evaluation of a propositional formula
with $I$ viewed as valuation $v$.

Thus signature \eqref{eqn:NullSign} realizes the propositional logic as a special case of the predicate logic.
One can state this as follows.
