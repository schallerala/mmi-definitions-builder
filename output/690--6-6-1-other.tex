At the beginning, the automaton is in the state $q_0$, and the stack contains the symbol $Z_0$.

The transition function $\delta$ takes three arguments:
the current state of the automaton, an input symbol or $\epsilon$, and the top symbol of the stack.
The value of the transition function
\[
\delta(q, a, Z) = \{(p_1, \gamma_1), \ldots, (p_m, \gamma_m)\}, \quad \gamma_i \in \Gamma^*,
\]
is interpreted as follows.
If the automaton in the state $q$ reads the input symbol $a$ and sees the symbol $Z$ on the top of the stack,
then, for a random $i$, it enters the state $p_i$ and replaces the symbol $Z$ by the string $\gamma_i$.
