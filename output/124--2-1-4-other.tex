\begin{proof}
Exercise.
% Induction on $k$. For $k=1$, a walk of length $1$ is just an edge, and the statement holds by definition of the adjacency matrix.
% 
% The induction step: assume the statement holds for $k-1$, prove it for $k$.
% Take any two vertices $v_i, v_j \in V$.
% A walk starts with the first step.
% Every walk of length $k$ from $v_i$ to $v_j$ consists of an edge from $v_i$ to some neighbor $v_l$
% followed by a walk of length $k-1$ from $v_l$ to $v_j$.
% Hence, in order to compute the number of walks of length $k$ from $v_i$ to $v_j$
% we have to sum the number of walks of length $k-1$ from all neighbors of $v_i$ to $v_j$.
% Since by the induction assumption there are $a^{(k-1)}_{lj}$ walks of length $k-1$ from $v_l$ to $v_j$, this sum is equal to
% \[
% \sum_{\{v_i,v_l\} \in E} a^{(k-1)}_{lj} = \sum_{l=1}^n a_{il} a^{(k-1)}_{lj},
% \]
% which is exactly the $(i,j)$-entry in the matrix $A \cdot A^{k-1} = A^k$.
\end{proof}

The reader is invited to define analogs of the adjacency matrix for directed and weighted graphs.

