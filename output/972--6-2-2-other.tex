\documentclass[preview, multi=page, margin=5mm, class=report]{standalone}
\usepackage[utf8]{inputenc}

\usepackage{amsmath,amssymb,amsthm}
\usepackage{graphicx,color}
\usepackage{hyperref,url}
\graphicspath{{Fig/}}

\usepackage{mathtools}
\usepackage{bussproofs}
\usepackage{stackengine}
\def\ruleoffset{1pt}
\newcommand\specialvdash[2]{\mathrel{\ensurestackMath{
  \mkern2mu\rule[-\dp\strutbox]{.4pt}{\baselineskip}\stackon[\ruleoffset]{
    \stackunder[\dimexpr\ruleoffset-.5\ht\strutbox+.5\dp\strutbox]{
      \rule[\dimexpr.5\ht\strutbox-.5\dp\strutbox]{2.5ex}{.4pt}}{
        \scriptstyle #1}}{\scriptstyle#2}\mkern2mu}}
}

\usepackage[table]{xcolor}

\renewcommand\thesection{\arabic{section}}
\renewcommand\thefigure{\arabic{figure}}
\renewcommand\theequation{\arabic{equation}}

\newtheorem{dfn}{Definition}[section]
\newtheorem{thm}[dfn]{Theorem}
\newtheorem{lem}[dfn]{Lemma}
\newtheorem{cor}[dfn]{Corollary}


\theoremstyle{definition}
\newtheorem{exl}[dfn]{Example}
\newtheorem{rem}[dfn]{Remark}
\newtheorem{exc}{Exercise}[section]

\def\R{\mathbb{R}}
\def\N{\mathbb{N}}
\def\Z{\mathbb{Z}}
\def\C{\mathbb{C}}
\def\cP{\mathcal{P}}
\def\cV{\mathcal{V}}
\def\cF{\mathcal{F}}
\def\Th{\mathrm{Th}}

\renewcommand{\emptyset}{\varnothing}
\renewcommand{\phi}{\varphi}
\renewcommand{\epsilon}{\varepsilon}
\def\gcd{\operatorname{gcd}}

\def\Prop{\mathrm{PROP}}
\begin{document}
\setcounter{section}{2}
\setcounter{subsection}{2}
\setcounter{dfn}{7}

\begin{proof}
We describe a construction algorithm of an $\epsilon$-NFA that accepts the language described by a given regular expression $r$.
Moreover, the resulting automaton will have a unique accepting state.
The construction uses the recursive structure of the regular expression.

Let $r$ be any regular expression.
By definition, $r$ is either basic or is obtained from one or two simpler expressions through sum, concatenation or closure.

If $r$ is basic, then the corresponding language is accepted by one of the automata shown in Figure \ref{fig:AutomBasicRegExpr}.

\begin{figure}[ht]
\begin{center}
\input{Fig/AutomBasicRegExpr.pdf_t}
\end{center}
\caption{Automata for basic regular expressions.}
\label{fig:AutomBasicRegExpr}
\end{figure}

If $r = r_1 + r_2$, then by assumption there are $\epsilon$-NFAs $M_1$ and $M_2$,
each with a unique final state, for the languages represented by $r_1$ and $r_2$.
The automaton in Figure \ref{fig:AutomSum} accepts the language of the expression $r_1 + r_2$.

\begin{figure}[ht]
\begin{center}
\input{Fig/AutomSum.pdf_t}
\end{center}
\caption{Automaton realizing the union of two languages.}
\label{fig:AutomSum}
\end{figure}

If $r = r_1r_2$, then we combine the automata for $r_1$ and $r_2$ as shown in Figure \ref{fig:AutomConcat}.

\begin{figure}[ht]
\begin{center}
\input{Fig/AutomConcat.pdf_t}
\end{center}
\caption{Automaton realizing the concatenation of two languages.}
\label{fig:AutomConcat}
\end{figure}

Finally, the automaton in Figure \ref{fig:AutomClosure} accepts the language of $(r_1)^*$.

\begin{figure}[ht]
\begin{center}
\input{Fig/AutomClosure.pdf_t}
\end{center}
\caption{Automaton realizing the Kleene closure of a language.}
\label{fig:AutomClosure}
\end{figure}

In order to show that these automata do what they are meant to do, one has to prove two things:
first, each word from the language $R_1 + R_2$ (respectively, $R_1R_2$, or $R_1^*$) is accepted by the automaton;
second, each word accepted by the automaton belongs to the respective language.
The arguments proving this are rather straightforward, and we omit them.
\end{proof}


\end{document}