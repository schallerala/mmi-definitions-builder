\documentclass[preview, multi=page, margin=5mm, class=report]{standalone}
\usepackage[utf8]{inputenc}

\usepackage{amsmath,amssymb,amsthm}
\usepackage{graphicx,color}
\usepackage{hyperref,url}
\graphicspath{{Fig/}}

\usepackage{mathtools}
\usepackage{bussproofs}
\usepackage{stackengine}
\def\ruleoffset{1pt}
\newcommand\specialvdash[2]{\mathrel{\ensurestackMath{
  \mkern2mu\rule[-\dp\strutbox]{.4pt}{\baselineskip}\stackon[\ruleoffset]{
    \stackunder[\dimexpr\ruleoffset-.5\ht\strutbox+.5\dp\strutbox]{
      \rule[\dimexpr.5\ht\strutbox-.5\dp\strutbox]{2.5ex}{.4pt}}{
        \scriptstyle #1}}{\scriptstyle#2}\mkern2mu}}
}

\usepackage[table]{xcolor}

\renewcommand\thesection{\arabic{section}}
\renewcommand\thefigure{\arabic{figure}}
\renewcommand\theequation{\arabic{equation}}

\newtheorem{dfn}{Definition}[section]
\newtheorem{thm}[dfn]{Theorem}
\newtheorem{lem}[dfn]{Lemma}
\newtheorem{cor}[dfn]{Corollary}


\theoremstyle{definition}
\newtheorem{exl}[dfn]{Example}
\newtheorem{rem}[dfn]{Remark}
\newtheorem{exc}{Exercise}[section]

\def\R{\mathbb{R}}
\def\N{\mathbb{N}}
\def\Z{\mathbb{Z}}
\def\C{\mathbb{C}}
\def\cP{\mathcal{P}}
\def\cV{\mathcal{V}}
\def\cF{\mathcal{F}}
\def\Th{\mathrm{Th}}

\renewcommand{\emptyset}{\varnothing}
\renewcommand{\phi}{\varphi}
\renewcommand{\epsilon}{\varepsilon}
\def\gcd{\operatorname{gcd}}

\def\Prop{\mathrm{PROP}}
\begin{document}
\setcounter{section}{2}
\setcounter{subsection}{2}
\setcounter{dfn}{2}

\begin{proof}
Take any two vertices of a tree.
Since a tree is connected, there is at least one path between these two vertices.
If there is more than one path, then this implies the existence of a cycle.
Namely, take the first vertex where the two paths diverge and the first vertex where they meet again;
the union of the segments of our paths between these vertices will be a cycle.
(We are not working out the details here.)
This contradicts the assumption that our graph is a tree, thus there cannot be more than one path between two vertices.
\end{proof}

A \emph{rooted} tree is a tree $T$ with a specified vertex $x$, called the \emph{root} of~$T$.
The edges of a tree can be equipped with orientation so that for every vertex $v$ the (unique) path from $x$ to $v$ always follows the directions of edges.
(Again, this looks intuitively clear, but requires a formal proof.)
See Figure~\ref{fig:RootedTreeOrient} for an example.

\begin{figure}[ht]
\begin{center}
\includegraphics[width=.6\textwidth]{RootedTreeOrient.pdf}
\end{center}
\caption{A canonically oriented rooted tree.}
\label{fig:RootedTreeOrient}
\end{figure}



\end{document}