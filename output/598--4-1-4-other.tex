\documentclass[preview, multi=page, margin=5mm, class=report]{standalone}
\usepackage[utf8]{inputenc}

\usepackage{amsmath,amssymb,amsthm}
\usepackage{graphicx,color}
\usepackage{hyperref,url}
\graphicspath{{Fig/}}

\usepackage{mathtools}
\usepackage{bussproofs}
\usepackage{stackengine}
\def\ruleoffset{1pt}
\newcommand\specialvdash[2]{\mathrel{\ensurestackMath{
  \mkern2mu\rule[-\dp\strutbox]{.4pt}{\baselineskip}\stackon[\ruleoffset]{
    \stackunder[\dimexpr\ruleoffset-.5\ht\strutbox+.5\dp\strutbox]{
      \rule[\dimexpr.5\ht\strutbox-.5\dp\strutbox]{2.5ex}{.4pt}}{
        \scriptstyle #1}}{\scriptstyle#2}\mkern2mu}}
}

\usepackage[table]{xcolor}

\renewcommand\thesection{\arabic{section}}
\renewcommand\thefigure{\arabic{figure}}
\renewcommand\theequation{\arabic{equation}}

\newtheorem{dfn}{Definition}[section]
\newtheorem{thm}[dfn]{Theorem}
\newtheorem{lem}[dfn]{Lemma}
\newtheorem{cor}[dfn]{Corollary}


\theoremstyle{definition}
\newtheorem{exl}[dfn]{Example}
\newtheorem{rem}[dfn]{Remark}
\newtheorem{exc}{Exercise}[section]

\def\R{\mathbb{R}}
\def\N{\mathbb{N}}
\def\Z{\mathbb{Z}}
\def\C{\mathbb{C}}
\def\cP{\mathcal{P}}
\def\cV{\mathcal{V}}
\def\cF{\mathcal{F}}
\def\Th{\mathrm{Th}}

\renewcommand{\emptyset}{\varnothing}
\renewcommand{\phi}{\varphi}
\renewcommand{\epsilon}{\varepsilon}
\def\gcd{\operatorname{gcd}}

\def\Prop{\mathrm{PROP}}
\begin{document}
\setcounter{section}{1}
\setcounter{subsection}{4}
\setcounter{dfn}{12}


Given a structure, one can evaluate every closed formula.
For example, the formula $\forall x \neg(f(x) = a)$ acquires the meaning
\[
\forall x \in \N \  x+1 \ne 1,
\]
which we identify as a true statement.

It should be intuitively clear how to find the truth value of a given closed formula.
But one needs a formal definition.
Of course, it proceeds by recursion because formulas have a recursive structure.
This means that, even if we want to determine the truth values only of closed formulas,
we also need to determine the truth values of formulas with free variables (which appear as intermediate steps when building a closed formula).

A \emph{variable assignment} is a map $\mu \colon \cV \to U$ which associated to every variable an element from the universe.
Together with interpretation $I$ it allows to evaluate every term to an element of $U$ and every formula to a truth value.
\begin{itemize}
\item Given terms $t_1, \ldots, t_k$ which evaluate to $u_1, \ldots, u_k \in U$ and a $k$-ary function symbol $f$,
the term $f(t_1, \ldots, t_k)$ evaluates to $I(f)(u_1, \ldots, u_k)$.
\item Similarly, a formula $P(t_1, \ldots, t_k)$ evaluates to $I(P)(u_1, \ldots, u_k)$ if $t_i$ evaluates to $u_i$.
\item Given terms $t_1$ and $t_2$ which evaluate to $u_1$ and $u_2$, the formula $t_1 = t_2$ evaluates to true if and only if $u_1 = u_2$.
\item Formulas of the form $\phi \wedge \psi$, $\phi \vee \psi$, $\phi \to \psi$, $\neg\phi$ evaluate according to the truth tables
for logical connectives.
\item A formula $\exists x \phi$ evaluates to true if there exists an evaluation $\mu'$ that differs from $\mu$ only in the value of $x$
such that $\phi$ evaluates under $\mu'$ to true.
\item A formula $\forall x \phi$ evaluates to true if $\phi$ evaluates to true under all assignments $\mu'$ that differ from $\mu$ only in the value of $x$.
\end{itemize}
One sees that the truth values of $\exists x \phi$ and $\forall x \phi$ do not depend on the assignment value of the variable $x$.
It follows that the truth values of closed formulas are independent of the variable assignments, hence determined by $M = (U,I)$ only.



\end{document}