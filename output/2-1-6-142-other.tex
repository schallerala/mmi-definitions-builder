\documentclass[preview, margin=5mm, multi=page]{standalone}
\usepackage[utf8]{inputenc}

\usepackage{amsmath,amssymb,amsthm}
\usepackage{graphicx,color}
\usepackage{hyperref,url}
\graphicspath{{Fig/}}

\usepackage{mathtools}
\usepackage{bussproofs}
\usepackage{stackengine}
\def\ruleoffset{1pt}
\newcommand\specialvdash[2]{\mathrel{\ensurestackMath{
  \mkern2mu\rule[-\dp\strutbox]{.4pt}{\baselineskip}\stackon[\ruleoffset]{
    \stackunder[\dimexpr\ruleoffset-.5\ht\strutbox+.5\dp\strutbox]{
      \rule[\dimexpr.5\ht\strutbox-.5\dp\strutbox]{2.5ex}{.4pt}}{
        \scriptstyle #1}}{\scriptstyle#2}\mkern2mu}}
}

\usepackage[table]{xcolor}

\renewcommand\thesection{\arabic{section}}
\renewcommand\thefigure{\arabic{figure}}
\renewcommand\theequation{\arabic{equation}}

\newtheorem{dfn}{Definition}[section]
\newtheorem{thm}[dfn]{Theorem}
\newtheorem{lem}[dfn]{Lemma}
\newtheorem{cor}[dfn]{Corollary}


\theoremstyle{definition}
\newtheorem{exl}[dfn]{Example}
\newtheorem{rem}[dfn]{Remark}
\newtheorem{exc}{Exercise}[section]

\def\R{\mathbb{R}}
\def\N{\mathbb{N}}
\def\Z{\mathbb{Z}}
\def\C{\mathbb{C}}
\def\cP{\mathcal{P}}
\def\cV{\mathcal{V}}
\def\cF{\mathcal{F}}
\def\Th{\mathrm{Th}}


\renewcommand{\emptyset}{\varnothing}
\renewcommand{\phi}{\varphi}
\renewcommand{\epsilon}{\varepsilon}
\def\gcd{\operatorname{gcd}}

\def\Prop{\mathrm{PROP}}



%opening
\title{{Lecture notes for the 2020/21 lectures}\\
$ $\\
$ $\\ \textsc{
Mathematical methods for Computer Science I \& II\\
and\\
Discrete Mathematics I \& II\\ }
$ $\\
$ $\\
$ $\\
$ $\\
University of Fribourg\\ Livio Liechti
$ $\\
$ $\\
$ $\\
$ $\\
$ $\\
$ $\\
$ $\\}
\date{ }

\author{Lecture notes written by Ivan Izmestiev for his 2018/19 lectures}


\begin{document}
\setcounter{section}{1}
\setcounter{subsection}{6}
\setcounter{dfn}{20}

The somewhat awkward condition ``all vertices of positive degree belong to the same connected component''
is equivalent to ``the graph becomes connected after deleting all isolated vertices''.
Another way to deal with the isolated vertices is to define the notion of a Eulerian circuit/walk differently:
it must visit all vertices.
Then a graph is (semi-)Eulerian if and only if it is connected and the degree evenness condition is satisfied.
\begin{proof}
The ``only if'' direction. Assume that the graph has a Eulerian walk or a Eulerian circuit.
Then there is a walk between any two vertices of positive degree:
one can use a piece of the walk or circuit to get from one to the other.
In order to prove that the degrees of all vertices (except two in the semi-Eulerian case) are even,
orient the edges in the direction of the walk: orient $e_i$ from $v_{i-1}$ to $v_i$.
We obtain a directed graph.
In a directed graph, every vertex $v$ has the \emph{in-degree} $\deg_+ v$ and the \emph{out-degree} $\deg_- v$:
the number of edges entering $v$ and the number of edges leaving $v$.
Clearly, $\deg v = \deg_+ v + \deg_- v$.
On the other hand, a Eulerian circuit enters and leaves every vertex equal number of times: $\deg_+ v = \deg_- v$.
This implies that $\deg v$ is even for every vertex of a Eulerian graph.

In a semi-Eulerian graph we have $\deg_+ v = \deg_- v$ for every intermediate vertex of the walk, but
\[
\deg_+ v_0 = \deg_- v_0 - 1, \quad \deg_+ v_m = \deg_- v_m + 1
\]
for the initial and the final vertices of the walk, respectively.
It follows that the degrees of all vertices except $v_0$ and $v_m$ are even.

The ``if'' direction.
First, consider the case when all vertex degrees are even.
Start to walk from any vertex without going along any edge twice.
At some point we must stop because all edges incident to the current vertex are used.
This can only happen at the initial vertex of our walk because if you stop at a different vertex, then its in-degree will be one bigger than the out-degree,
which contradicts the assumption that all vertex degrees are even.
Thus we obtain a circuit (which does not necessarily cover all edges).
Remove this circuit from the graph.
We obtain a possibly disconnected graph where all vertex degrees are even.
Repeat the procedure until the edge set of our graph will be partitioned into circuits.
Then start to merge the circuits: if two circuits have a common vertex, then they can be replaced by a single circuit.
If some circuit has no common vertices with the other circuits,
then its vertices and edges form a connected component of the graph,
and we will have at least two non-trivial connected components.
Thus all circuits can be merged to a circuit covering all edges of the graph.

If the graph has two vertices of odd degree, then start our first walk from one of these vertices.
This walk will necessarily stop at the other odd vertex (again, by consideration of the in- and out-degrees).
Removing this walk from the graph yields a graph with all vertices of even degree, and we proceed as in the previous case.
\end{proof}

Try this algorithm on the graph on Figure \ref{fig:EulerianGraph}.

\begin{figure}[h]
\begin{center}
\includegraphics[width=.4\textwidth]{Fig/EulerianGraph.pdf}
\end{center}
\caption{Find a Eulerian circuit in this graph.}
\label{fig:EulerianGraph}
\end{figure}

Note that for graphs with two vertices of odd degree we have proved a bit more:
every Eulerian circuit starts in one of the odd vertices and ends in the other one.


% The skeleton of the cube is Hamiltonian but not Eulerian.
% The graph $K_{4,6}$ is Eulerian but not Hamiltonian (why?).





% Would be good to discuss depth-first search here. Connected components, existence of cycles, planarity testing.





\end{document}
