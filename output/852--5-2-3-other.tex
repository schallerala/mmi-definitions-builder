\documentclass[preview, margin=5mm, multi=page]{standalone}
\usepackage[utf8]{inputenc}

\usepackage{amsmath,amssymb,amsthm}
\usepackage{graphicx,color}
\usepackage{hyperref,url}
\graphicspath{{Fig/}}

\usepackage{mathtools}
\usepackage{bussproofs}
\usepackage{stackengine}
\def\ruleoffset{1pt}
\newcommand\specialvdash[2]{\mathrel{\ensurestackMath{
  \mkern2mu\rule[-\dp\strutbox]{.4pt}{\baselineskip}\stackon[\ruleoffset]{
    \stackunder[\dimexpr\ruleoffset-.5\ht\strutbox+.5\dp\strutbox]{
      \rule[\dimexpr.5\ht\strutbox-.5\dp\strutbox]{2.5ex}{.4pt}}{
        \scriptstyle #1}}{\scriptstyle#2}\mkern2mu}}
}

\usepackage[table]{xcolor}

\renewcommand\thesection{\arabic{section}}
\renewcommand\thefigure{\arabic{figure}}
\renewcommand\theequation{\arabic{equation}}

\newtheorem{dfn}{Definition}[section]
\newtheorem{thm}[dfn]{Theorem}
\newtheorem{lem}[dfn]{Lemma}
\newtheorem{cor}[dfn]{Corollary}


\theoremstyle{definition}
\newtheorem{exl}[dfn]{Example}
\newtheorem{rem}[dfn]{Remark}
\newtheorem{exc}{Exercise}[section]

\def\R{\mathbb{R}}
\def\N{\mathbb{N}}
\def\Z{\mathbb{Z}}
\def\C{\mathbb{C}}
\def\cP{\mathcal{P}}
\def\cV{\mathcal{V}}
\def\cF{\mathcal{F}}
\def\Th{\mathrm{Th}}


\renewcommand{\emptyset}{\varnothing}
\renewcommand{\phi}{\varphi}
\renewcommand{\epsilon}{\varepsilon}
\def\gcd{\operatorname{gcd}}

\def\Prop{\mathrm{PROP}}



%opening
\title{{Lecture notes for the 2020/21 lectures}\\
$ $\\
$ $\\ \textsc{
Mathematical methods for Computer Science I \& II\\
and\\
Discrete Mathematics I \& II\\ }
$ $\\
$ $\\
$ $\\
$ $\\
University of Fribourg\\ Livio Liechti
$ $\\
$ $\\
$ $\\
$ $\\
$ $\\
$ $\\
$ $\\}
\date{ }

\author{Lecture notes written by Ivan Izmestiev for his 2018/19 lectures}


\begin{document}
\setcounter{section}{2}
\setcounter{subsection}{3}
\setcounter{dfn}{10}


\begin{proof}
Exercise.
% One computes
% \begin{multline*}
% \overline{P}(x)A(x) =
% (1 - r_1 x - r_2 x^2 - \cdots - r_k x^k)(a_0 + a_1x + a_2x^2 + \cdots)\\
% \sum_{i=0}^{k-1} (a_i - r_1 a_{i-1} - \cdots - r_i a_0) x^i + \sum_{n=k}^\infty x^n(a_n - r_1 a_{n-1} - \cdots - r_k a_{n-k}) = B(x).
% \end{multline*}
% Thus we have
% \[
% A(x) = \frac{B(x)}{\overline{P}(x)},
% \]
% where $B(x)$ is a polynomial of degree at most $k-1$.
\end{proof}

Note that $\overline{P}(x)$ is related to the characteristic polynomial $P(x)$ through
\[
\overline{P}(x) = x^k P\left(\frac1x\right).
\]
It follows that the roots of the polynomial $\overline{P}(x)$ are reciprocals of the roots of $P(x)$.
More exactly, if $P(x) = (x - \lambda_1)^{k_1} \cdots (x - \lambda_m)^{k_m}$, then
\[
\overline{P}(x) = (1 - \lambda_1 x)^{k_1} \cdots (1- \lambda_m)^{k_m}.
\]

The following theorem generalizes our representation of the fraction $\frac{x}{1-x-x^2}$ as a sum of two simpler fractions.

\end{document}