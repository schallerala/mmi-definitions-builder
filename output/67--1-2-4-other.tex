\documentclass[preview, multi=page, margin=5mm, class=report]{standalone}
\usepackage[utf8]{inputenc}

\usepackage{amsmath,amssymb,amsthm}
\usepackage{graphicx,color}
\usepackage{hyperref,url}
\graphicspath{{Fig/}}

\usepackage{mathtools}
\usepackage{bussproofs}
\usepackage{stackengine}
\def\ruleoffset{1pt}
\newcommand\specialvdash[2]{\mathrel{\ensurestackMath{
  \mkern2mu\rule[-\dp\strutbox]{.4pt}{\baselineskip}\stackon[\ruleoffset]{
    \stackunder[\dimexpr\ruleoffset-.5\ht\strutbox+.5\dp\strutbox]{
      \rule[\dimexpr.5\ht\strutbox-.5\dp\strutbox]{2.5ex}{.4pt}}{
        \scriptstyle #1}}{\scriptstyle#2}\mkern2mu}}
}

\usepackage[table]{xcolor}

\renewcommand\thesection{\arabic{section}}
\renewcommand\thefigure{\arabic{figure}}
\renewcommand\theequation{\arabic{equation}}

\newtheorem{dfn}{Definition}[section]
\newtheorem{thm}[dfn]{Theorem}
\newtheorem{lem}[dfn]{Lemma}
\newtheorem{cor}[dfn]{Corollary}


\theoremstyle{definition}
\newtheorem{exl}[dfn]{Example}
\newtheorem{rem}[dfn]{Remark}
\newtheorem{exc}{Exercise}[section]

\def\R{\mathbb{R}}
\def\N{\mathbb{N}}
\def\Z{\mathbb{Z}}
\def\C{\mathbb{C}}
\def\cP{\mathcal{P}}
\def\cV{\mathcal{V}}
\def\cF{\mathcal{F}}
\def\Th{\mathrm{Th}}

\renewcommand{\emptyset}{\varnothing}
\renewcommand{\phi}{\varphi}
\renewcommand{\epsilon}{\varepsilon}
\def\gcd{\operatorname{gcd}}

\def\Prop{\mathrm{PROP}}
\begin{document}
\setcounter{section}{2}
\setcounter{subsection}{5}
\setcounter{dfn}{11}

\begin{proof}
Let $X$ be the set of all possible ordered choices of $k$ balls out of $n$,
and $Y$ be the set of all unordered choices of $k$ balls.
There is a map $f \colon X \to Y$ (the ``forgetful map'') that associates to
an ordered collection of $k$ balls the same set of balls, but unordered.
(The balls lying in a line are put into another bag.)

For $y \in Y$, what is the cardinality of its preimage $f^{-1}(y)$?
This is the number of ways to order an unordered set of $k$ balls.
An ordering is a bijection to the set $\{1, 2, \ldots, k\}$, and from Corollary \ref{cor:NumberOfBijections} we know that there are $k!$ of them.
Therefore by the quotient rule we have
\[
|Y| = \frac{|X|}{k!} = \frac{n(n-1)\cdot \ldots \cdot (n-k+1)}{k!}.
\]
\end{proof}

As we already said, an unordered choice of $k$ balls out of $n$ is also called a $k$-combination.
Yet another name of this is a \emph{$k$-element subset} of a given $n$-element set.
(By definition, a set is an unordered collection of elements.)

\smallskip

\noindent\textbf{Notation.}
The number of $k$-element subsets of an $n$-element set is denoted by $\binom{n}{k}$
(pronounced ``$n$ choose $k$'').




\end{document}