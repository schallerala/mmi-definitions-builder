In other words, $\overline{P}$ is $P$ together with all states that can be reached from $P$ by sequences of $\epsilon$-transitions.

Let us modify and extend the transition function so that it will tell us what states are accessible from a given state
for a given input.
\begin{enumerate}
\item
$\widehat{\delta}(q, \epsilon) = \overline{\{q\}}$
\item
$\widehat{\delta}(q, wa) = \overline{\delta(\widehat{\delta}(q,w), a)} \text{ for all }w \in \Sigma^*$
\end{enumerate}
(Note that $\widehat\delta(q,w)$ is a set, so that $\delta(\widehat{\delta}(q,w), a)$ denotes
the union of $\delta(p,w)$ over all $p \in \widehat{\delta}(q,w)$.)

Observe that, contrarily to the case of DFA and NFA, $\widehat{\delta}(q,a) \ne \delta(q,a)$, but rather
\[
\widehat{\delta}(q,a) = \overline{\delta(\overline{\{q\}}, a)} \supset \delta(q,a).
\]
It is not hard to see that $\widehat{\delta}(q,w)$ consists of all states reachable from $q$ on the input $w$
with arbitrarily many $\epsilon$-transitions before $w$, in the middle of $w$, and after $w$.

In terms of the extended transition function the language accepted by an $\epsilon$-NFA is defined as follows.