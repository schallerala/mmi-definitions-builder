\subsection{Propositional formulas}
\label{sec:PropFormulas}
The language of propositional logic consists of strings of symbols, where each of the symbols is one of the following:
\begin{itemize}
\item
A proposition symbol $p$, $q$, $r$, $s$, $p_1, p_2, \ldots$.
(A countably infinite set.)
\item
A logical connective $\wedge$, $\vee$, $\to$, $\neg$.
\item
An auxiliary symbol $($ or $)$.
\end{itemize}
Sometimes to the list of logical connectives one adds $\leftrightarrow$ and $\perp$.
We will abstain from this.

The logical connectives have the following names.

\begin{center}
\begin{tabular}[c]{l@{\hspace{1cm}}l@{\hspace{1cm}}l}
$\wedge$ & and & conjunction\\
$\vee$ & or & disjunction\\
$\to$ & if ..., then ... & implication\\
$\neg$ & not & negation
\end{tabular}
\end{center}

Now there come the syntax rules describing what strings of symbols are allowed.
