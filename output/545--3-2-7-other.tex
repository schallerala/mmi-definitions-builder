\documentclass[preview, multi=page, margin=5mm, class=report]{standalone}
\usepackage[utf8]{inputenc}

\usepackage{amsmath,amssymb,amsthm}
\usepackage{graphicx,color}
\usepackage{hyperref,url}
\graphicspath{{Fig/}}

\usepackage{mathtools}
\usepackage{bussproofs}
\usepackage{stackengine}
\def\ruleoffset{1pt}
\newcommand\specialvdash[2]{\mathrel{\ensurestackMath{
  \mkern2mu\rule[-\dp\strutbox]{.4pt}{\baselineskip}\stackon[\ruleoffset]{
    \stackunder[\dimexpr\ruleoffset-.5\ht\strutbox+.5\dp\strutbox]{
      \rule[\dimexpr.5\ht\strutbox-.5\dp\strutbox]{2.5ex}{.4pt}}{
        \scriptstyle #1}}{\scriptstyle#2}\mkern2mu}}
}

\usepackage[table]{xcolor}

\renewcommand\thesection{\arabic{section}}
\renewcommand\thefigure{\arabic{figure}}
\renewcommand\theequation{\arabic{equation}}

\newtheorem{dfn}{Definition}[section]
\newtheorem{thm}[dfn]{Theorem}
\newtheorem{lem}[dfn]{Lemma}
\newtheorem{cor}[dfn]{Corollary}


\theoremstyle{definition}
\newtheorem{exl}[dfn]{Example}
\newtheorem{rem}[dfn]{Remark}
\newtheorem{exc}{Exercise}[section]

\def\R{\mathbb{R}}
\def\N{\mathbb{N}}
\def\Z{\mathbb{Z}}
\def\C{\mathbb{C}}
\def\cP{\mathcal{P}}
\def\cV{\mathcal{V}}
\def\cF{\mathcal{F}}
\def\Th{\mathrm{Th}}

\renewcommand{\emptyset}{\varnothing}
\renewcommand{\phi}{\varphi}
\renewcommand{\epsilon}{\varepsilon}
\def\gcd{\operatorname{gcd}}

\def\Prop{\mathrm{PROP}}
\begin{document}
\setcounter{section}{2}
\setcounter{subsection}{8}
\setcounter{dfn}{19}

\begin{proof}
For a given sequent, consider its closed deduction tree.
It has leaves of two kinds:
\begin{itemize}
\item
Leaves labeled with axioms $p_1, \ldots, p_k \vdash q_1, \ldots, q_l$ such that $p_i = q_j$ for some $i$, $j$.
\item
Leaves with labels $p_1, \ldots, p_k \vdash q_1, \ldots, q_l$ such that $p_i \ne q_j$ for all $i$, $j$.
These are called counterexample leaves.
\end{itemize}
A counterexample leaf is falsifiable: it suffices to set $v(p_i) = 1$ for all $i$ and $v(q_j) = 0$ for all $j$.
Thus if the closed deduction tree has a counterexample leaf, then the sequent at the root is also falsifiable.

If our sequent is valid, then all of the leaves are of the first kind, and the tree is a proof tree.
\end{proof}

Note that we get more than just completeness: constructing a closed deduction tree provides a concrete counterexample to a falsifiable sequent.



\end{document}