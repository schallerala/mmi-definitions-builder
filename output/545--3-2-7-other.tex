\begin{proof}
For a given sequent, consider its closed deduction tree.
It has leaves of two kinds:
\begin{itemize}
\item
Leaves labeled with axioms $p_1, \ldots, p_k \vdash q_1, \ldots, q_l$ such that $p_i = q_j$ for some $i$, $j$.
\item
Leaves with labels $p_1, \ldots, p_k \vdash q_1, \ldots, q_l$ such that $p_i \ne q_j$ for all $i$, $j$.
These are called counterexample leaves.
\end{itemize}
A counterexample leaf is falsifiable: it suffices to set $v(p_i) = 1$ for all $i$ and $v(q_j) = 0$ for all $j$.
Thus if the closed deduction tree has a counterexample leaf, then the sequent at the root is also falsifiable.

If our sequent is valid, then all of the leaves are of the first kind, and the tree is a proof tree.
\end{proof}

Note that we get more than just completeness: constructing a closed deduction tree provides a concrete counterexample to a falsifiable sequent.

