\documentclass[preview, multi=page, margin=5mm, class=report]{standalone}
\usepackage[utf8]{inputenc}

\usepackage{amsmath,amssymb,amsthm}
\usepackage{graphicx,color}
\usepackage{hyperref,url}
\graphicspath{{Fig/}}

\usepackage{mathtools}
\usepackage{bussproofs}
\usepackage{stackengine}
\def\ruleoffset{1pt}
\newcommand\specialvdash[2]{\mathrel{\ensurestackMath{
  \mkern2mu\rule[-\dp\strutbox]{.4pt}{\baselineskip}\stackon[\ruleoffset]{
    \stackunder[\dimexpr\ruleoffset-.5\ht\strutbox+.5\dp\strutbox]{
      \rule[\dimexpr.5\ht\strutbox-.5\dp\strutbox]{2.5ex}{.4pt}}{
        \scriptstyle #1}}{\scriptstyle#2}\mkern2mu}}
}

\usepackage[table]{xcolor}

\renewcommand\thesection{\arabic{section}}
\renewcommand\thefigure{\arabic{figure}}
\renewcommand\theequation{\arabic{equation}}

\newtheorem{dfn}{Definition}[section]
\newtheorem{thm}[dfn]{Theorem}
\newtheorem{lem}[dfn]{Lemma}
\newtheorem{cor}[dfn]{Corollary}


\theoremstyle{definition}
\newtheorem{exl}[dfn]{Example}
\newtheorem{rem}[dfn]{Remark}
\newtheorem{exc}{Exercise}[section]

\def\R{\mathbb{R}}
\def\N{\mathbb{N}}
\def\Z{\mathbb{Z}}
\def\C{\mathbb{C}}
\def\cP{\mathcal{P}}
\def\cV{\mathcal{V}}
\def\cF{\mathcal{F}}
\def\Th{\mathrm{Th}}

\renewcommand{\emptyset}{\varnothing}
\renewcommand{\phi}{\varphi}
\renewcommand{\epsilon}{\varepsilon}
\def\gcd{\operatorname{gcd}}

\def\Prop{\mathrm{PROP}}
\begin{document}
\setcounter{section}{3}
\setcounter{subsection}{6}
\setcounter{dfn}{13}

\begin{proof}[First proof]
Let us establish a bijection between the weak compositions of $n$ from $k$ parts and the compositions of $n+k$ from $k$ parts.
Indeed, adding $1$ to every summand in a weak composition of $n$ transforms it into a (strong) composition of $n+k$.
In the opposite direction, subtracting $1$ from every summand of a composition of $n+k$ transforms it into
a weak composition of $n$.
This is a one-to-one correspondence (a bijection).
By Theorem \ref{thm:Compositions}, the number of (strong) compositions of $n+k$ from $k$ parts is $\binom{n+k-1}{k-1}$, and we are done.
\end{proof}

\begin{proof}[Second proof]
You have $n$ stones and $k-1$ sticks.
Mark $n+k-1$ spots on the ground.
You have to choose $k-1$ among them where you lay sticks, then you will lay your stones on the remaining spots.
There are $\binom{n+k-1}{k-1}$ different arrangements, and they correspond to weak compositions of $n$ from $k$ parts.
For example, if the stones are on the first $k-1$ spots, then the first $k-1$ summands are equal to zero, and the $k$-th summand equals $n$.
\end{proof}



\end{document}