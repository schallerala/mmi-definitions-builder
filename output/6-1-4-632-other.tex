\documentclass[preview, multi=page, margin=5mm, class=report]{standalone}
\usepackage[utf8]{inputenc}

\usepackage{amsmath,amssymb,amsthm}
\usepackage{graphicx,color}
\usepackage{hyperref,url}
\graphicspath{{Fig/}}

\usepackage{mathtools}
\usepackage{bussproofs}
\usepackage{stackengine}
\def\ruleoffset{1pt}
\newcommand\specialvdash[2]{\mathrel{\ensurestackMath{
  \mkern2mu\rule[-\dp\strutbox]{.4pt}{\baselineskip}\stackon[\ruleoffset]{
    \stackunder[\dimexpr\ruleoffset-.5\ht\strutbox+.5\dp\strutbox]{
      \rule[\dimexpr.5\ht\strutbox-.5\dp\strutbox]{2.5ex}{.4pt}}{
        \scriptstyle #1}}{\scriptstyle#2}\mkern2mu}}
}

\usepackage[table]{xcolor}

\renewcommand\thesection{\arabic{section}}
\renewcommand\thefigure{\arabic{figure}}
\renewcommand\theequation{\arabic{equation}}

\newtheorem{dfn}{Definition}[section]
\newtheorem{thm}[dfn]{Theorem}
\newtheorem{lem}[dfn]{Lemma}
\newtheorem{cor}[dfn]{Corollary}


\theoremstyle{definition}
\newtheorem{exl}[dfn]{Example}
\newtheorem{rem}[dfn]{Remark}
\newtheorem{exc}{Exercise}[section]

\def\R{\mathbb{R}}
\def\N{\mathbb{N}}
\def\Z{\mathbb{Z}}
\def\C{\mathbb{C}}
\def\cP{\mathcal{P}}
\def\cV{\mathcal{V}}
\def\cF{\mathcal{F}}
\def\Th{\mathrm{Th}}

\renewcommand{\emptyset}{\varnothing}
\renewcommand{\phi}{\varphi}
\renewcommand{\epsilon}{\varepsilon}
\def\gcd{\operatorname{gcd}}

\def\Prop{\mathrm{PROP}}
\begin{document}
\setcounter{section}{1}
\setcounter{subsection}{4}
\setcounter{dfn}{14}

In other words, $\overline{P}$ is $P$ together with all states that can be reached from $P$ by sequences of $\epsilon$-transitions.

Let us modify and extend the transition function so that it will tell us what states are accessible from a given state
for a given input.
\begin{enumerate}
\item
$\widehat{\delta}(q, \epsilon) = \overline{\{q\}}$
\item
$\widehat{\delta}(q, wa) = \overline{\delta(\widehat{\delta}(q,w), a)} \text{ for all }w \in \Sigma^*$
\end{enumerate}
(Note that $\widehat\delta(q,w)$ is a set, so that $\delta(\widehat{\delta}(q,w), a)$ denotes
the union of $\delta(p,w)$ over all $p \in \widehat{\delta}(q,w)$.)

Observe that, contrarily to the case of DFA and NFA, $\widehat{\delta}(q,a) \ne \delta(q,a)$, but rather
\[
\widehat{\delta}(q,a) = \overline{\delta(\overline{\{q\}}, a)} \supset \delta(q,a).
\]
It is not hard to see that $\widehat{\delta}(q,w)$ consists of all states reachable from $q$ on the input $w$
with arbitrarily many $\epsilon$-transitions before $w$, in the middle of $w$, and after $w$.

In terms of the extended transition function the language accepted by an $\epsilon$-NFA is defined as follows.

\end{document}
