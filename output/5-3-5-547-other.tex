\documentclass[preview, multi=page, margin=5mm, class=report]{standalone}
\usepackage[utf8]{inputenc}

\usepackage{amsmath,amssymb,amsthm}
\usepackage{graphicx,color}
\usepackage{hyperref,url}
\graphicspath{{Fig/}}

\usepackage{mathtools}
\usepackage{bussproofs}
\usepackage{stackengine}
\def\ruleoffset{1pt}
\newcommand\specialvdash[2]{\mathrel{\ensurestackMath{
  \mkern2mu\rule[-\dp\strutbox]{.4pt}{\baselineskip}\stackon[\ruleoffset]{
    \stackunder[\dimexpr\ruleoffset-.5\ht\strutbox+.5\dp\strutbox]{
      \rule[\dimexpr.5\ht\strutbox-.5\dp\strutbox]{2.5ex}{.4pt}}{
        \scriptstyle #1}}{\scriptstyle#2}\mkern2mu}}
}

\usepackage[table]{xcolor}

\renewcommand\thesection{\arabic{section}}
\renewcommand\thefigure{\arabic{figure}}
\renewcommand\theequation{\arabic{equation}}

\newtheorem{dfn}{Definition}[section]
\newtheorem{thm}[dfn]{Theorem}
\newtheorem{lem}[dfn]{Lemma}
\newtheorem{cor}[dfn]{Corollary}


\theoremstyle{definition}
\newtheorem{exl}[dfn]{Example}
\newtheorem{rem}[dfn]{Remark}
\newtheorem{exc}{Exercise}[section]

\def\R{\mathbb{R}}
\def\N{\mathbb{N}}
\def\Z{\mathbb{Z}}
\def\C{\mathbb{C}}
\def\cP{\mathcal{P}}
\def\cV{\mathcal{V}}
\def\cF{\mathcal{F}}
\def\Th{\mathrm{Th}}

\renewcommand{\emptyset}{\varnothing}
\renewcommand{\phi}{\varphi}
\renewcommand{\epsilon}{\varepsilon}
\def\gcd{\operatorname{gcd}}

\def\Prop{\mathrm{PROP}}
\begin{document}
\setcounter{section}{3}
\setcounter{subsection}{5}
\setcounter{dfn}{11}

\begin{proof}[Algebraic proof]
The generating function for partitions into distinct parts is
\[
(1+x)(1+x^2)(1+x^3)\cdots
\]
The generating function for partitions into odd parts is
\[
(1+x+x^2+\cdots)(1+x^3+x^6+\cdots)(1+x^5+x^{10}+\cdots)\cdots = \frac1{1-x} \frac1{1-x^3} \frac1{1-x^5} \cdots
\]
Let us show that the first formal power series is equal to the second one.
\begin{equation}
\label{eqn:OddDist}
(1+x)(1+x^2)(1+x^3)\cdots = \frac{1-x^2}{1-x} \frac{1-x^4}{1-x^2} \frac{1-x^6}{1-x^3} \cdots = \frac1{1-x} \frac1{1-x^3} \frac1{1-x^5} \cdots
\end{equation}
This equation is a bit more subtle than it appears.
We have
\[
(1+x) \cdots (1+x^{2k}) = \frac{1-x^2}{1-x} \cdots \frac{1-x^{4k}}{1-x^{2k}} = \frac{(1-x^{2k+2}) \cdots (1-x^{4k})}{(1-x) \cdots (1-x^{2k-1})}
\]
The right hand side has the same coefficient at $x^i$ for $i \le 2k$ as the infinite product on the right hand side of \eqref{eqn:OddDist}.
And the left hand side has the same coefficient at $x^i$ for $i \le 2k$ as the infinite product on the left hand side of \eqref{eqn:OddDist}.
\end{proof}

\begin{proof}[Bijective proof]
Take a partition of $n$ into odd parts:
\[
n = 1 \cdot m_1 + 3 \cdot m_3 + 5 \cdot m_5 + \cdots
\]
It can be transformed into a partition into distinct parts as follows.
Write $m_{2k+1}$ in the binary system:
\[
m_{2k+1} = 2^{d_1} + \cdots + 2^{d_s}, \quad d_i \ne d_j.
\]
Then replace $(2k+1) \cdot m_{2k+1}$ by
\[
(2k+1)(2^{d_1} + \cdots + 2^{d_s}) = 2^{d_1}(2k+1) + \cdots + 2^{d_s}(2k+1).
\]
Being done for all $k$, this gives a new partition of $n$.
The parts of the new partition are different.
Indeed, any two parts that come from the same $k$ are distinct: $2^{d_i}(2k+1) \ne 2^{d_j}(2k+1)$ because $d_i \ne d_j$.
Any two parts that come from different $k$ are also distinct: $2^{d_i(k)}(2k+1) \ne 2^{d_j(l)}(2l+1)$
because they have different greatest odd divisors $2k+1 \ne 2l+1$.

In the opposite direction, we transform every partition of $n$ into distinct parts as follows:
\begin{multline*}
n = k_1 + \cdots + k_t, \quad k_i \ne k_j\\
= o_1 2^{d_1} + \cdots + o_t 2^{d_t}, \quad o_i \text{ odd}\\
= 1 \cdot m_1 + 3 \cdot m_3 + \cdots, \quad m_{2k+1} = \sum_{o_i = 2k+1} 2^{d_i}
\end{multline*}
It can easily be shown that this transformation is inverse to the first one.
Thus we have a bijection between partitions into odd and partitions into distinct parts.
\end{proof}


\end{document}
