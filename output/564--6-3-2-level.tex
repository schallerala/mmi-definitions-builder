\subsection{The pumping lemma}
Let $\Sigma$ be a finite alphabet.
The set $\Sigma^*$ of all words in $\Sigma$ is countably infinite.
The set $2^{\Sigma^*}$ of all languages in $\Sigma$ is uncountable: it has the cardinality of the continuum.
On the other hand, the set of regular languages is countable, because the set of all regular expressions (and of all finite automata) is countable.
(Note that different automata or different regular expressions can define the same language, but this is not a problem.)
It follows that ``most'' languages are non-regular.

The above argument is a pure existence proof.
In this section we prove the \emph{pumping lemma}, a tool that allows to prove the non-regularity of some languages.

Before stating the lemma, let us explain the underlying idea.
Let $M$ be a DFA.
Any word $z$ accepted by $M$ determines a path from the initial state $q_0$ to one of the finals states $q \in F$.
If the word $z$ is long enough (namely if its length is bigger than the number of states of $M$),
then the corresponding path contains a cycle.
This cycle gives rize to infinitely many other words accepted by $M$, because one can run along it several times.
For example, in Figure \ref{fig:LongWord} one has $z = a_1a_2a_3a_4a_5a_6a_7 \in L(M)$.
By repeating or removing the cycle contained in the path, one obtains
\[
a_1a_2(a_3a_4a_5)^ka_6a_7 \in L(M)
\]
for all $k$, including $k = 0$.

\begin{figure}[ht]
\begin{center}
\input{Fig/LongWord.pdf_t}
\end{center}
\caption{Idea of the pumping lemma: a long word contains a subword which can be repeated.}
\label{fig:LongWord}
\end{figure}

The existence of a cycle can be stated as follows.
