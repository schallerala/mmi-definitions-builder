\documentclass[preview, multi=page, margin=5mm, class=report]{standalone}
\usepackage[utf8]{inputenc}

\usepackage{amsmath,amssymb,amsthm}
\usepackage{graphicx,color}
\usepackage{hyperref,url}
\graphicspath{{Fig/}}

\usepackage{mathtools}
\usepackage{bussproofs}
\usepackage{stackengine}
\def\ruleoffset{1pt}
\newcommand\specialvdash[2]{\mathrel{\ensurestackMath{
  \mkern2mu\rule[-\dp\strutbox]{.4pt}{\baselineskip}\stackon[\ruleoffset]{
    \stackunder[\dimexpr\ruleoffset-.5\ht\strutbox+.5\dp\strutbox]{
      \rule[\dimexpr.5\ht\strutbox-.5\dp\strutbox]{2.5ex}{.4pt}}{
        \scriptstyle #1}}{\scriptstyle#2}\mkern2mu}}
}

\usepackage[table]{xcolor}

\renewcommand\thesection{\arabic{section}}
\renewcommand\thefigure{\arabic{figure}}
\renewcommand\theequation{\arabic{equation}}

\newtheorem{dfn}{Definition}[section]
\newtheorem{thm}[dfn]{Theorem}
\newtheorem{lem}[dfn]{Lemma}
\newtheorem{cor}[dfn]{Corollary}


\theoremstyle{definition}
\newtheorem{exl}[dfn]{Example}
\newtheorem{rem}[dfn]{Remark}
\newtheorem{exc}{Exercise}[section]

\def\R{\mathbb{R}}
\def\N{\mathbb{N}}
\def\Z{\mathbb{Z}}
\def\C{\mathbb{C}}
\def\cP{\mathcal{P}}
\def\cV{\mathcal{V}}
\def\cF{\mathcal{F}}
\def\Th{\mathrm{Th}}

\renewcommand{\emptyset}{\varnothing}
\renewcommand{\phi}{\varphi}
\renewcommand{\epsilon}{\varepsilon}
\def\gcd{\operatorname{gcd}}

\def\Prop{\mathrm{PROP}}
\begin{document}
\setcounter{section}{2}
\setcounter{subsection}{0}
\setcounter{dfn}{8}

This and many other relations between Fibonacci numbers can be proved by induction,
sometimes in a not very straightforward way.
When the Binet formula is used, the proof consists of simple algebraic manipulations.

\begin{proof}
We have $a_n = \frac{1}{\sqrt{5}}(\lambda_1^n - \lambda_2^n)$ with $\lambda_i$ as in \eqref{eqn:FibRoots}.
Taking into account that $\lambda_1\lambda_2 = -1$, one computes
\[
a_n^2 = \frac15(\lambda_1^{2n} - 2 \lambda_1^n \lambda_2^n + \lambda_2^{2n}) = \frac15(\lambda_1^{2n} + \lambda_2^{2n} - 2(-1)^n).
\]
On the other hand,
\begin{multline*}
a_{n-1} a_{n+1} = \frac15(\lambda_1^{n-1} - \lambda_2^{n-1})(\lambda_1^{n+1} - \lambda_2^{n+1})\\
= \frac15(\lambda_1^{2n} - \lambda_1^{n-1}\lambda_2^{n+1} - \lambda_2^{n-1}\lambda_1^{n+1} + \lambda_2^{2n})\\
= \frac15(\lambda_1^{2n} + \lambda_2^{2n} - \lambda_1^{n-1}\lambda_2^{n-1}(\lambda_1^2 + \lambda_2^2))
\end{multline*}
One computes
\[
\lambda_1^2 + \lambda_2^2 = \frac{1 + 2\sqrt{5} + 5}4 + \frac{1 - 2\sqrt{5} + 5}4 = 3,
\]
which implies
\[
a_{n-1} a_{n+1} = \frac15(\lambda_1^{2n} + \lambda_2^{2n} - 3(-1)^{n-1}) = a_n^2 + (-1)^n.
\]
\end{proof}





\end{document}