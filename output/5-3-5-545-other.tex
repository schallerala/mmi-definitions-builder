\documentclass[preview, multi=page, margin=5mm, class=report]{standalone}
\usepackage[utf8]{inputenc}

\usepackage{amsmath,amssymb,amsthm}
\usepackage{graphicx,color}
\usepackage{hyperref,url}
\graphicspath{{Fig/}}

\usepackage{mathtools}
\usepackage{bussproofs}
\usepackage{stackengine}
\def\ruleoffset{1pt}
\newcommand\specialvdash[2]{\mathrel{\ensurestackMath{
  \mkern2mu\rule[-\dp\strutbox]{.4pt}{\baselineskip}\stackon[\ruleoffset]{
    \stackunder[\dimexpr\ruleoffset-.5\ht\strutbox+.5\dp\strutbox]{
      \rule[\dimexpr.5\ht\strutbox-.5\dp\strutbox]{2.5ex}{.4pt}}{
        \scriptstyle #1}}{\scriptstyle#2}\mkern2mu}}
}

\usepackage[table]{xcolor}

\renewcommand\thesection{\arabic{section}}
\renewcommand\thefigure{\arabic{figure}}
\renewcommand\theequation{\arabic{equation}}

\newtheorem{dfn}{Definition}[section]
\newtheorem{thm}[dfn]{Theorem}
\newtheorem{lem}[dfn]{Lemma}
\newtheorem{cor}[dfn]{Corollary}


\theoremstyle{definition}
\newtheorem{exl}[dfn]{Example}
\newtheorem{rem}[dfn]{Remark}
\newtheorem{exc}{Exercise}[section]

\def\R{\mathbb{R}}
\def\N{\mathbb{N}}
\def\Z{\mathbb{Z}}
\def\C{\mathbb{C}}
\def\cP{\mathcal{P}}
\def\cV{\mathcal{V}}
\def\cF{\mathcal{F}}
\def\Th{\mathrm{Th}}

\renewcommand{\emptyset}{\varnothing}
\renewcommand{\phi}{\varphi}
\renewcommand{\epsilon}{\varepsilon}
\def\gcd{\operatorname{gcd}}

\def\Prop{\mathrm{PROP}}
\begin{document}
\setcounter{section}{3}
\setcounter{subsection}{5}
\setcounter{dfn}{10}

% \begin{proof}[Algebraic proof]
% Let $q_n$ be the number of partitions of $n$ without parts of size $1$.
% In the same way as we derived the generating function for the number of partitions,
% we can derive the generating function for the sequence $q_n$:
% \begin{multline*}
% \sum_{n=0}^\infty q_nx^n = (1+x^2+x^4+\cdots)(1+x^3+x^6+\cdots)\\
% = \frac{1}{(1-x^2)(1-x^3)\cdots} = (1-x) \sum_{n=0}^\infty p_nx^n
% \end{multline*}
% Thus we have
% \begin{multline*}
% q_0 + q_1 x + q_2 x^2 + \cdots = (1-x)(p_0 + p_1x + p_2x^2 +\cdots\\
% = p_0 + p_1x + p_2x^2 + \cdots\\
% \quad\quad - p_0x - p_1x_2 - \cdots\\
% = p_0 + (p_1-p_0)x + (p_2-p_1)x^2 + \cdots,
% \end{multline*}
% that is $q_n = p_n - p_{n-1}$ for $n \ge 1$.
% \end{proof}
% \begin{proof}[Bijective proof]
% Let $P_n$ denote the set of all partitions of $n$.
% Consider the map
% \[
% f \colon P_{n-1} \to P_n, \quad f(m_1+\cdots+m_k) = m_1+\cdots+m_k+1.
% \]
% This map is injective: different partitions are sent to different partitions.
% Thus $|f(P_{n-1})| = |P_{n-1}| = p_{n-1}$.
% Observe that
% \[
% P_n \setminus f(P_{n-1}) = \{\text{partitions of }n \text{ without parts of size }1\}.
% \]
% Thus the number of partitions of $n$ without parts of size $1$ is equal to
% \[
% |P_n \setminus f(P_{n-1})| = |P_n| - |f(P_{n-1})| = p_n - p_{n-1},
% \]
% and the theorem is proved.
% \end{proof}

Here is the first amazing fact about partitions.


\end{document}
