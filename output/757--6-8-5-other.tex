\begin{proof}
Assume the contrary.
Then there is a Turing machine $M_j$ which accepts the language $L_d$.
By inspection of the word $w_j$ we arrive to a contradiction:
\begin{itemize}
\item
if $w_j \in L_d$, then by definition of $L_d$ the word $w_j$ is not accepted by $M_j$, which by the choice of $M_j$ means that $w_j \notin L_d$;
\item
if $w_i \notin L_d$, then by definition of $L_d$ the word $w_j$ is accepted by $M_j$, which by the choice of $M_j$ means that $w_j \in L_d$.
\end{itemize}
\end{proof}


\begin{proof}[Proof of Theorem \ref{thm:HaltingProblem}]
If $L_u$ is recursive, then there is a Turing machine $A$ which always halts and accepts only pairs $(M, w)$ from $L_u$.
Let us show that then $L_d$ is recursive, which contradicts Lemma \ref{lem:LdNotRE}.
Given a word $w$ determine the integer $i$ such that $w_i = w$.
Then determine the machine $M_i$.
Feed $(M_i, w_i)$ into $A$ and accept $w$ if and only if $A$ \emph{does not} accept $(M_i,w_i)$.
This gives an algorithm which always stops and recognizes the language $L_d$.
\end{proof}
