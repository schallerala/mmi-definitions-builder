\documentclass[preview, multi=page, margin=5mm, class=report]{standalone}
\usepackage[utf8]{inputenc}

\usepackage{amsmath,amssymb,amsthm}
\usepackage{graphicx,color}
\usepackage{hyperref,url}
\graphicspath{{Fig/}}

\usepackage{mathtools}
\usepackage{bussproofs}
\usepackage{stackengine}
\def\ruleoffset{1pt}
\newcommand\specialvdash[2]{\mathrel{\ensurestackMath{
  \mkern2mu\rule[-\dp\strutbox]{.4pt}{\baselineskip}\stackon[\ruleoffset]{
    \stackunder[\dimexpr\ruleoffset-.5\ht\strutbox+.5\dp\strutbox]{
      \rule[\dimexpr.5\ht\strutbox-.5\dp\strutbox]{2.5ex}{.4pt}}{
        \scriptstyle #1}}{\scriptstyle#2}\mkern2mu}}
}

\usepackage[table]{xcolor}

\renewcommand\thesection{\arabic{section}}
\renewcommand\thefigure{\arabic{figure}}
\renewcommand\theequation{\arabic{equation}}

\newtheorem{dfn}{Definition}[section]
\newtheorem{thm}[dfn]{Theorem}
\newtheorem{lem}[dfn]{Lemma}
\newtheorem{cor}[dfn]{Corollary}


\theoremstyle{definition}
\newtheorem{exl}[dfn]{Example}
\newtheorem{rem}[dfn]{Remark}
\newtheorem{exc}{Exercise}[section]

\def\R{\mathbb{R}}
\def\N{\mathbb{N}}
\def\Z{\mathbb{Z}}
\def\C{\mathbb{C}}
\def\cP{\mathcal{P}}
\def\cV{\mathcal{V}}
\def\cF{\mathcal{F}}
\def\Th{\mathrm{Th}}

\renewcommand{\emptyset}{\varnothing}
\renewcommand{\phi}{\varphi}
\renewcommand{\epsilon}{\varepsilon}
\def\gcd{\operatorname{gcd}}

\def\Prop{\mathrm{PROP}}
\begin{document}
\setcounter{section}{4}
\setcounter{subsection}{1}
\setcounter{dfn}{0}

\subsection{Rooted binary trees}
Recall that a \emph{rooted tree} is a tree with a marked vertex, the root.
We have used rooted trees (with labels at the vertices) as parse trees of propositional formulas
and as proof structures in the sequent calculus.
Edges or a rooted tree have a natural orientation such that the path from the root to every vertex goes in the direction of edges.
The \emph{out-degree} of a vertex in a rooted tree is the number of outward-directed edges incident to this vertex.
Similarly, the \emph{in-degree} is the number of inward-directed edges; the in-degree of the root is zero, and the in-degrees of all other vertices are one.

A \emph{binary rooted tree} is a rooted tree where the out-degrees of all vertices are at most two.
A \emph{full binary rooted tree} is a rooted tree where the out-degree of each vertex is either two or zero.
(Vertices with out-degree zero are the leaves of the tree.)

With the help of the handshake lemma and the relation $|V| = |E| + 1$ one can show that
a full binary rooted tree with $n+1$ leaves has $2n+1$ vertices and $2n$ edges.


\end{document}