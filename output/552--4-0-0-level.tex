\documentclass[preview, margin=5mm, multi=page]{standalone}
\usepackage[utf8]{inputenc}

\usepackage{amsmath,amssymb,amsthm}
\usepackage{graphicx,color}
\usepackage{hyperref,url}
\graphicspath{{Fig/}}

\usepackage{mathtools}
\usepackage{bussproofs}
\usepackage{stackengine}
\def\ruleoffset{1pt}
\newcommand\specialvdash[2]{\mathrel{\ensurestackMath{
  \mkern2mu\rule[-\dp\strutbox]{.4pt}{\baselineskip}\stackon[\ruleoffset]{
    \stackunder[\dimexpr\ruleoffset-.5\ht\strutbox+.5\dp\strutbox]{
      \rule[\dimexpr.5\ht\strutbox-.5\dp\strutbox]{2.5ex}{.4pt}}{
        \scriptstyle #1}}{\scriptstyle#2}\mkern2mu}}
}

\usepackage[table]{xcolor}

\renewcommand\thesection{\arabic{section}}
\renewcommand\thefigure{\arabic{figure}}
\renewcommand\theequation{\arabic{equation}}

\newtheorem{dfn}{Definition}[section]
\newtheorem{thm}[dfn]{Theorem}
\newtheorem{lem}[dfn]{Lemma}
\newtheorem{cor}[dfn]{Corollary}


\theoremstyle{definition}
\newtheorem{exl}[dfn]{Example}
\newtheorem{rem}[dfn]{Remark}
\newtheorem{exc}{Exercise}[section]

\def\R{\mathbb{R}}
\def\N{\mathbb{N}}
\def\Z{\mathbb{Z}}
\def\C{\mathbb{C}}
\def\cP{\mathcal{P}}
\def\cV{\mathcal{V}}
\def\cF{\mathcal{F}}
\def\Th{\mathrm{Th}}


\renewcommand{\emptyset}{\varnothing}
\renewcommand{\phi}{\varphi}
\renewcommand{\epsilon}{\varepsilon}
\def\gcd{\operatorname{gcd}}

\def\Prop{\mathrm{PROP}}



%opening
\title{{Lecture notes for the 2020/21 lectures}\\
$ $\\
$ $\\ \textsc{
Mathematical methods for Computer Science I \& II\\
and\\
Discrete Mathematics I \& II\\ }
$ $\\
$ $\\
$ $\\
$ $\\
University of Fribourg\\ Livio Liechti
$ $\\
$ $\\
$ $\\
$ $\\
$ $\\
$ $\\
$ $\\}
\date{ }

\author{Lecture notes written by Ivan Izmestiev for his 2018/19 lectures}


\begin{document}
\setcounter{section}{1}
\setcounter{subsection}{0}
\setcounter{dfn}{20}

\chapter{Predicate logic}
\section*{Introduction}
Predicate logic is a more complicated and powerful system than the propositional logic.

Before proceeding to formal definitions, let us have a glimpse at how it works.
Consider the statement
\begin{quote}
For every number $x$ one has $x < x+1$.
\end{quote}
It can be expressed by a predicate formula
\begin{equation}
\label{eqn:PredForm}
\forall x P(x, f(x)),
\end{equation}
where $f(x) = x+1$, and $P(x,y)$ means $x < y$.
Functions of one or several arguments that take truth values (such as $x < y$ or ``$x$ is blue'') are called \emph{predicates}.
Formula \eqref{eqn:PredForm} contains all the main building blocks of the predicate logic: a variable $x$, a function $f$, and a predicate $P$.

Let us now forget the origin of formula \eqref{eqn:PredForm}.
In order to make sense of it, its elements must be interpreted:
what kind of objects are represented by the variable $x$, how is the function $f$ defined, and what does $P(x,y)$ mean.
This interpretation can be as above, but can also be different.
For example, the same formula can be interpreted as
\begin{quote}
It gets colder every day.
\end{quote}
Now $x$ is a day, $f(x)$ is the day after day $x$, and $P(x,y)$ means ``$y$ is colder than $x$''.

Our first interpretation evaluates the formula \eqref{eqn:PredForm} to true, while the second evaluates it to false.
A formula of the predicate logic is \emph{valid} if it evaluates to true in all interpretations.
Similarly to the propositional logic, one aims at finding a method (a proof theory) to establish the validity of a formula.




\end{document}