\documentclass[preview, multi=page, margin=5mm, class=report]{standalone}
\usepackage[utf8]{inputenc}

\usepackage{amsmath,amssymb,amsthm}
\usepackage{graphicx,color}
\usepackage{hyperref,url}
\graphicspath{{Fig/}}

\usepackage{mathtools}
\usepackage{bussproofs}
\usepackage{stackengine}
\def\ruleoffset{1pt}
\newcommand\specialvdash[2]{\mathrel{\ensurestackMath{
  \mkern2mu\rule[-\dp\strutbox]{.4pt}{\baselineskip}\stackon[\ruleoffset]{
    \stackunder[\dimexpr\ruleoffset-.5\ht\strutbox+.5\dp\strutbox]{
      \rule[\dimexpr.5\ht\strutbox-.5\dp\strutbox]{2.5ex}{.4pt}}{
        \scriptstyle #1}}{\scriptstyle#2}\mkern2mu}}
}

\usepackage[table]{xcolor}

\renewcommand\thesection{\arabic{section}}
\renewcommand\thefigure{\arabic{figure}}
\renewcommand\theequation{\arabic{equation}}

\newtheorem{dfn}{Definition}[section]
\newtheorem{thm}[dfn]{Theorem}
\newtheorem{lem}[dfn]{Lemma}
\newtheorem{cor}[dfn]{Corollary}


\theoremstyle{definition}
\newtheorem{exl}[dfn]{Example}
\newtheorem{rem}[dfn]{Remark}
\newtheorem{exc}{Exercise}[section]

\def\R{\mathbb{R}}
\def\N{\mathbb{N}}
\def\Z{\mathbb{Z}}
\def\C{\mathbb{C}}
\def\cP{\mathcal{P}}
\def\cV{\mathcal{V}}
\def\cF{\mathcal{F}}
\def\Th{\mathrm{Th}}

\renewcommand{\emptyset}{\varnothing}
\renewcommand{\phi}{\varphi}
\renewcommand{\epsilon}{\varepsilon}
\def\gcd{\operatorname{gcd}}

\def\Prop{\mathrm{PROP}}
\begin{document}
\setcounter{section}{3}
\setcounter{subsection}{3}
\setcounter{dfn}{10}


A graph $G'$ is a \emph{subdivision} of a graph $G$ if $G'$ is obtained from $G$ by repeated \emph{edge subdivisions}.
To subdivide an edge $e = \{v,w\}$ of a graph $G$ means to introduce a new vertex $x$, delete $e$, and introduce two new edges $\{x,v\}$ and $\{x,w\}$.
Figure \ref{fig:KSubdivisions} shows a subdivision of $K_5$ and a subdivision of $K_{3,3}$.

\begin{figure}[ht]
\begin{center}
\includegraphics[width=.8\textwidth]{KSubdivisions.pdf}
\end{center}
\caption{Some subdivisions of $K_5$ and $K_{3,3}$.}
\label{fig:KSubdivisions}
\end{figure}

One direction of the Kuratowski theorem is easy to prove: If a graph contains a subdivision of $K_5$ or $K_{3,3}$, then it cannot be planar.
Indeed, an embedding of the graph would contain an embedding of a subdivision of $K_5$ or $K_{3,3}$,
and hence an embedding of $K_5$ or $K_{3,3}$.
It is the opposite direction which is the most interesting and non-obvious:
the only obstacles to existence of a planar embedding of $G$ are graphs $K_5$ or $K_{3,3}$ contained in $G$ (in the form of subdivisions).


There is a similar planarity criterion that uses the notion of a \emph{minor}.


\end{document}
