\documentclass[preview, multi=page, margin=5mm, class=report]{standalone}
\usepackage[utf8]{inputenc}

\usepackage{amsmath,amssymb,amsthm}
\usepackage{graphicx,color}
\usepackage{hyperref,url}
\graphicspath{{Fig/}}

\usepackage{mathtools}
\usepackage{bussproofs}
\usepackage{stackengine}
\def\ruleoffset{1pt}
\newcommand\specialvdash[2]{\mathrel{\ensurestackMath{
  \mkern2mu\rule[-\dp\strutbox]{.4pt}{\baselineskip}\stackon[\ruleoffset]{
    \stackunder[\dimexpr\ruleoffset-.5\ht\strutbox+.5\dp\strutbox]{
      \rule[\dimexpr.5\ht\strutbox-.5\dp\strutbox]{2.5ex}{.4pt}}{
        \scriptstyle #1}}{\scriptstyle#2}\mkern2mu}}
}

\usepackage[table]{xcolor}

\renewcommand\thesection{\arabic{section}}
\renewcommand\thefigure{\arabic{figure}}
\renewcommand\theequation{\arabic{equation}}

\newtheorem{dfn}{Definition}[section]
\newtheorem{thm}[dfn]{Theorem}
\newtheorem{lem}[dfn]{Lemma}
\newtheorem{cor}[dfn]{Corollary}


\theoremstyle{definition}
\newtheorem{exl}[dfn]{Example}
\newtheorem{rem}[dfn]{Remark}
\newtheorem{exc}{Exercise}[section]

\def\R{\mathbb{R}}
\def\N{\mathbb{N}}
\def\Z{\mathbb{Z}}
\def\C{\mathbb{C}}
\def\cP{\mathcal{P}}
\def\cV{\mathcal{V}}
\def\cF{\mathcal{F}}
\def\Th{\mathrm{Th}}

\renewcommand{\emptyset}{\varnothing}
\renewcommand{\phi}{\varphi}
\renewcommand{\epsilon}{\varepsilon}
\def\gcd{\operatorname{gcd}}

\def\Prop{\mathrm{PROP}}
\begin{document}
\setcounter{section}{2}
\setcounter{subsection}{3}
\setcounter{dfn}{12}

\begin{rem}
It is important to work in a breadth-first way, otherwise one can obtain an infinite tree for a valid formula.
This happens, for example, in the following deduction tree for the formula $(P \vee \neg P) \vee \exists x \forall y P(x,y)$:
\begin{prooftree}
\AxiomC{$\cdots$}
\UnaryInfC{$\vdash Q, \neg Q, P(u_1,u_2), P(u_2,u_3), \exists x \forall y P(x,y)$}
\UnaryInfC{$\cdots$}
\UnaryInfC{$\vdash Q, \neg Q, \exists x \forall y P(x,y)$}
\UnaryInfC{$\vdash (Q \vee \neg Q) \vee \exists x \forall y P(x,y)$}
\UnaryInfC{$\vdash (Q \vee \neg Q) \vee \exists x \forall y P(x,y)$}
\end{prooftree}
Here we are neglecting the non-atomic formula $\neg Q$ and working with $\exists x \forall y P(x,y)$ only,
which produces an infinite path from Example \ref{exl:CounterexampleInfinite}.
Constructed in a breadth-first way, this tree will close with an axiom leaf with $Q$ on both sides of $\vdash$.
\end{rem}

\end{document}