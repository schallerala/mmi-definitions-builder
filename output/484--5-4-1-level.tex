\subsection{More about partitions}
\begin{itemize}
\item 
An infinite (but fast convergent) series that computes $p_n$ was found by Ramanujan and Hardy and later improved by Rademacher.
A consequences of the latter is the asymptotics for the number of partitions:
\[
p_n \sim \frac{1}{4n\sqrt{3}} e^{\pi\sqrt{2n/3}}.
\]
\item
Ramanujan observed and later proved that
\begin{gather*}
p_{5n+4} \text{ is divisible by } 5,\\
p_{7n+5} \text{ is divisible by } 7,\\
p_{11n+6} \text{ is divisible by } 11.
\end{gather*}
\item
Erd\"os and Lehner proved that a ``random'' partition of $n$ has $\frac{2\pi}{\sqrt{6}} \sqrt{n} \log n$ summands.
\end{itemize}

Both Ramanujan and Erd\"os were extraordinary figures.
For the biography of Ramanujan see, for example,
\url{http://www-history.mcs.st-andrews.ac.uk/Biographies/Ramanujan.html}.

Further reading about partitions: \cite{AE04}.


\newpage

\section{Catalan numbers}
\subsection{Rooted binary trees}
Recall that a \emph{rooted tree} is a tree with a marked vertex, the root.
We have used rooted trees (with labels at the vertices) as parse trees of propositional formulas
and as proof structures in the sequent calculus.
Edges or a rooted tree have a natural orientation such that the path from the root to every vertex goes in the direction of edges.
The \emph{out-degree} of a vertex in a rooted tree is the number of outward-directed edges incident to this vertex.
Similarly, the \emph{in-degree} is the number of inward-directed edges; the in-degree of the root is zero, and the in-degrees of all other vertices are one.

A \emph{binary rooted tree} is a rooted tree where the out-degrees of all vertices are at most two.
A \emph{full binary rooted tree} is a rooted tree where the out-degree of each vertex is either two or zero.
(Vertices with out-degree zero are the leaves of the tree.)

With the help of the handshake lemma and the relation $|V| = |E| + 1$ one can show that
a full binary rooted tree with $n+1$ leaves has $2n+1$ vertices and $2n$ edges.
