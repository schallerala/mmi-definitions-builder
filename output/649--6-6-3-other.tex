\begin{proof}[Idea of the proof]
Introduce two new states: a final state $q_f$ and a new initial state $q'_0$, and also a new start symbol $X_0$.
As the first step, the automaton $M'$ puts the old start symbol $Z_0$ on the top of the new start symbol and goes to the old initial state:
\[
\delta(q'_0, \epsilon, X_0) = (q_0, Z_0X_0).
\]
Then one lets the old PDA $M$ do its job.
If one sees $X_0$ on the top of the stack, then it means that $M$ has emptied its stack.
One then goes to the final state:
\[
\delta(q, \epsilon, X_0) = (q_f, \epsilon) \text{ for all }q \ne q'_0.
\]
\end{proof}

