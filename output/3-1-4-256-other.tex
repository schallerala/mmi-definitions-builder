\documentclass[preview, margin=5mm, multi=page]{standalone}
\usepackage[utf8]{inputenc}

\usepackage{amsmath,amssymb,amsthm}
\usepackage{graphicx,color}
\usepackage{hyperref,url}
\graphicspath{{Fig/}}

\usepackage{mathtools}
\usepackage{bussproofs}
\usepackage{stackengine}
\def\ruleoffset{1pt}
\newcommand\specialvdash[2]{\mathrel{\ensurestackMath{
  \mkern2mu\rule[-\dp\strutbox]{.4pt}{\baselineskip}\stackon[\ruleoffset]{
    \stackunder[\dimexpr\ruleoffset-.5\ht\strutbox+.5\dp\strutbox]{
      \rule[\dimexpr.5\ht\strutbox-.5\dp\strutbox]{2.5ex}{.4pt}}{
        \scriptstyle #1}}{\scriptstyle#2}\mkern2mu}}
}

\usepackage[table]{xcolor}

\renewcommand\thesection{\arabic{section}}
\renewcommand\thefigure{\arabic{figure}}
\renewcommand\theequation{\arabic{equation}}

\newtheorem{dfn}{Definition}[section]
\newtheorem{thm}[dfn]{Theorem}
\newtheorem{lem}[dfn]{Lemma}
\newtheorem{cor}[dfn]{Corollary}


\theoremstyle{definition}
\newtheorem{exl}[dfn]{Example}
\newtheorem{rem}[dfn]{Remark}
\newtheorem{exc}{Exercise}[section]

\def\R{\mathbb{R}}
\def\N{\mathbb{N}}
\def\Z{\mathbb{Z}}
\def\C{\mathbb{C}}
\def\cP{\mathcal{P}}
\def\cV{\mathcal{V}}
\def\cF{\mathcal{F}}
\def\Th{\mathrm{Th}}


\renewcommand{\emptyset}{\varnothing}
\renewcommand{\phi}{\varphi}
\renewcommand{\epsilon}{\varepsilon}
\def\gcd{\operatorname{gcd}}

\def\Prop{\mathrm{PROP}}



%opening
\title{{Lecture notes for the 2020/21 lectures}\\
$ $\\
$ $\\ \textsc{
Mathematical methods for Computer Science I \& II\\
and\\
Discrete Mathematics I \& II\\ }
$ $\\
$ $\\
$ $\\
$ $\\
University of Fribourg\\ Livio Liechti
$ $\\
$ $\\
$ $\\
$ $\\
$ $\\
$ $\\
$ $\\}
\date{ }

\author{Lecture notes written by Ivan Izmestiev for his 2018/19 lectures}


\begin{document}
\setcounter{section}{1}
\setcounter{subsection}{4}
\setcounter{dfn}{10}

Although the set of all proposition symbols is infinite,
every proposition contains only a finite number of distinct proposition symbols.
Assume that $A \in \Prop$ contains only symbols from the set $\{p_1, p_2, \ldots, p_n\}$.
Then $A$ defines a map
\[
f_A \colon \{0,1\}^n \to \{0,1\}
\]
in the following way.
Every element $(x_1, \ldots, x_n) \in \{0,1\}^n$ can be viewed as a partial valuation, giving $p_i$ the truth value $x_i$ for $i = 1, \ldots, n$.
Then $f_A(x_1, \ldots, x_n)$ is the truth value of $A$ corresponding to this valuation:
\[
f_A(x_1, \ldots, x_n) = \hat{v}(A), \text{ where } v(p_i) = x_i.
\]

The function $f_A$ is described by the truth table of proposition $A$.
As an immediate reformulation of Definition \ref{dfn:LogEqProp},
\[
A \simeq B \Leftrightarrow f_A = f_B.
\]

In fact, $0$-$1$-valued functions of $0$-$1$-valued agruments is a very basic object which can be studied irrespective of the propositional logic.

\end{document}
