
It follows that an $M$-augmenting path can look only as the bottom path on Figure \ref{fig:AltPath}.
In addition, there must be no edge of $M$ incident to the initial and terminal vertices of the path.
See Figure \ref{fig:AugmPathDef}.

\begin{figure}[ht]
\begin{center}
\includegraphics[width=.6\textwidth]{AugmPathDef.pdf}
\end{center}
\caption{An augmenting path.}
\label{fig:AugmPathDef}
\end{figure}

An $M$-augmenting path can be used to modify $M$ to a bigger matching by ``switching'' the edges along the path.
This is illustrated in Figure \ref{fig:AugmPath}.


\begin{figure}[ht]
\begin{center}
\includegraphics[width=.8\textwidth]{AugmPath1.pdf}
\end{center}
\caption{Modifying a matching with the help of an augmenting path.}
\label{fig:AugmPath}
\end{figure}


The following theorem provides a basis to an algorithm for finding a maximum matching.
