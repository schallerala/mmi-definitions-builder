\documentclass[preview, multi=page, margin=5mm, class=report]{standalone}
\usepackage[utf8]{inputenc}

\usepackage{amsmath,amssymb,amsthm}
\usepackage{graphicx,color}
\usepackage{hyperref,url}
\graphicspath{{Fig/}}

\usepackage{mathtools}
\usepackage{bussproofs}
\usepackage{stackengine}
\def\ruleoffset{1pt}
\newcommand\specialvdash[2]{\mathrel{\ensurestackMath{
  \mkern2mu\rule[-\dp\strutbox]{.4pt}{\baselineskip}\stackon[\ruleoffset]{
    \stackunder[\dimexpr\ruleoffset-.5\ht\strutbox+.5\dp\strutbox]{
      \rule[\dimexpr.5\ht\strutbox-.5\dp\strutbox]{2.5ex}{.4pt}}{
        \scriptstyle #1}}{\scriptstyle#2}\mkern2mu}}
}

\usepackage[table]{xcolor}

\renewcommand\thesection{\arabic{section}}
\renewcommand\thefigure{\arabic{figure}}
\renewcommand\theequation{\arabic{equation}}

\newtheorem{dfn}{Definition}[section]
\newtheorem{thm}[dfn]{Theorem}
\newtheorem{lem}[dfn]{Lemma}
\newtheorem{cor}[dfn]{Corollary}


\theoremstyle{definition}
\newtheorem{exl}[dfn]{Example}
\newtheorem{rem}[dfn]{Remark}
\newtheorem{exc}{Exercise}[section]

\def\R{\mathbb{R}}
\def\N{\mathbb{N}}
\def\Z{\mathbb{Z}}
\def\C{\mathbb{C}}
\def\cP{\mathcal{P}}
\def\cV{\mathcal{V}}
\def\cF{\mathcal{F}}
\def\Th{\mathrm{Th}}

\renewcommand{\emptyset}{\varnothing}
\renewcommand{\phi}{\varphi}
\renewcommand{\epsilon}{\varepsilon}
\def\gcd{\operatorname{gcd}}

\def\Prop{\mathrm{PROP}}
\begin{document}
\setcounter{section}{5}
\setcounter{subsection}{2}
\setcounter{dfn}{2}

\subsection{De Montmort problem, or counting the derangements}
The problem was originally posed by Pierre R\'emond de Montmort in 1708, and was solved by him and, independently, Nicholas Bernoulli.

De Montmort stated it in terms of a game of cards, later it became popular in the following formulation:

\begin{quote}
\emph{The guests leaving a party are taking their hats in the garderobe.
In the darkness they cannot tell the hats one from the other, so everybody takes a hat by chance.
What is the probability that nobody will get his own hat?}
\end{quote}

Here is a formal description of the problem.
Consider a bijection
\[
f \colon \{1, 2, \ldots, n\} \to \{1, 2, \ldots, n\}.
\]
Such a bijection is called a permutation.
An element $x \in \{1, 2, \ldots, n\}$ is called a \emph{fixed point} of $f$ if $f(x) = x$.
A permutation without fixed points is sometimes called a \emph{derangement}.
The probability to be computed is equal to
\[
\frac{\#\text{derangements}}{\#\text{permutations}}.
\]
The number of permutations is known: it is $n!$.
Thus we have to count the number of derangements.


\end{document}