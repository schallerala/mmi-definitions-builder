\subsection{De Montmort problem, or counting the derangements}
The problem was originally posed by Pierre R\'emond de Montmort in 1708, and was solved by him and, independently, Nicholas Bernoulli.

De Montmort stated it in terms of a game of cards, later it became popular in the following formulation:

\begin{quote}
\emph{The guests leaving a party are taking their hats in the garderobe.
In the darkness they cannot tell the hats one from the other, so everybody takes a hat by chance.
What is the probability that nobody will get his own hat?}
\end{quote}

Here is a formal description of the problem.
Consider a bijection
\[
f \colon \{1, 2, \ldots, n\} \to \{1, 2, \ldots, n\}.
\]
Such a bijection is called a permutation.
An element $x \in \{1, 2, \ldots, n\}$ is called a \emph{fixed point} of $f$ if $f(x) = x$.
A permutation without fixed points is sometimes called a \emph{derangement}.
The probability to be computed is equal to
\[
\frac{\#\text{derangements}}{\#\text{permutations}}.
\]
The number of permutations is known: it is $n!$.
Thus we have to count the number of derangements.
