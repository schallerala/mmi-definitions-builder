\subsection{Birthday problem}
Given $k$ people, what is the probability that some two of them have the same birthday?
How big should $k$ be for this probability to exceed $\frac12$?










\section{Many faces of $\binom{n}{k}$}
The number $\binom{n}{k}$ that we have introduced in the previous lecture has several interpretations.
\subsection{Subsets or unordered choices}
This is our original definition: $\binom{n}{k}$ is the number of unordered choices of $k$ elements out of $n$.
In a more abstract language, this is the number of $k$-element subsets of an $n$-element set.

We have proved that
\begin{equation}
\label{eqn:NChooseK}
\binom{n}{k} = \frac{n(n-1)\cdot \ldots \cdot (n-k+1)}{k!} = \frac{n!}{k!(n-k)!}.
\end{equation}
