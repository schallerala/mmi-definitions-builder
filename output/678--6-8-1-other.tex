At each moment of time, the machine is situated opposite to some cell of the input tape.
The input symbol $X$ and the current state $p$ of the machine determine its move $\delta(p, X) = (q, Y, D)$
which consists in:
\begin{itemize}
\item
changing the state to $q$,
\item
replacing the symbol $X$ in the current cell by $Y$,
\item
and moving in the direction $D$ (one cell to the left if $D = L$ or one cell to the right if $D = R$).
\end{itemize}
See Figure \ref{fig:TuringMachine}.

\begin{figure}[ht]
\begin{center}
\input{Fig/TuringMachine.pdf_t}
\end{center}
\caption{A Turing machine.}
\label{fig:TuringMachine}
\end{figure}


At the beginning, the machine is placed at the leftmost cell of the tape and is in the state $q_0$.
The \emph{language accepted by} $M$ is the set of all input words for which $M$ enters a final state at some moment of time.
After entering a final state, the machine \emph{halts}, that is $q(f,X)$ is undefined for all $f \in F$.
If the input word is not accepted, then the machine either halts in a non-final state or runs forever
(in an infinite loop or by increasing the data volume on the tape to infinity).
