\begin{proof}
If $r_1$ is a regular expression for $R_1$, and $r_2$ is a regular expression for $R_2$,
then the regular expression $r_1 + r_2$ describes the language $R_1 \cup R_2$, which is therefore regular.

Given regular expressions $r_1$, $r_2$, $r$,
it is very difficult to find regular expressions for the intersection $R_1 \cap R_2$ and the complement $\Sigma^* \setminus R$.
Let us approach the problem from a different direction.

Let $M = (Q, \Sigma, \delta, q_0, F)$ be a DFA accepting the language $R$.
Then $\overline{M} := (Q, \Sigma, \delta, q_0, Q \setminus F)$ accepts the language $\Sigma^* \setminus R$.
Indeed,
\[
w \in \Sigma^* \setminus R \Leftrightarrow w \notin R \Leftrightarrow \widehat{\delta}(q_0, w) \notin F
\Leftrightarrow \widehat{\delta}(q_0, w) \in Q \setminus F \Leftrightarrow w \in L(\overline{M}).
\]
Therefore $\Sigma^* \setminus R$ is regular.

With the intersection we are helped by de Morgan's rule:
\[
R_1 \cap R_2 = \overline{\overline{R_1} \cup \overline{R_2}},
\]
where the overline denotes the complement.
Since the operations applied on the right hand side preserve regularity, the intersection of two regular languages is regular.
\end{proof}

