
In order to describe formally what it means that a word can be derived from the start symbol, let us fix some notations and terminology.
If $A \to \beta$ is any production in $P$, and $\alpha, \gamma \in (V \cup T)^*$,
then we write $\alpha A \gamma \xRightarrow[G]{} \alpha \beta \gamma$
and say that $\alpha A \gamma$ \emph{directly derives} $\alpha\beta\gamma$ in grammar~$G$.
If $\alpha_1, \ldots, \alpha_n \in (V \cup T)^*$ are such that
\[
\alpha_1 \xRightarrow[G]{} \alpha_2, \quad \alpha_2 \xRightarrow[G]{} \alpha_3, \quad \ldots, \quad  \alpha_{n-1} \xRightarrow[G]{} \alpha_n,
\]
then we write $\alpha_1 \xRightarrow[G]{*} \alpha_n$ and say that $\alpha_1$ \emph{derives} $\alpha_n$ in $G$.
When it is clear which grammar we use, then we omit $G$ and write $\xRightarrow[]{}$ and $\xRightarrow[]{*}$, respectively.

Now we can describe the language generated by $G$ as
\[
L(G) = \{w \in T^* \mid S \xRightarrow[G]{*} w\}.
\]
A language is called a \emph{context-free language} (CFL) if it is generated by some context-free grammar.
