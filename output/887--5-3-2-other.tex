\documentclass[preview, multi=page, margin=5mm, class=report]{standalone}
\usepackage[utf8]{inputenc}

\usepackage{amsmath,amssymb,amsthm}
\usepackage{graphicx,color}
\usepackage{hyperref,url}
\graphicspath{{Fig/}}

\usepackage{mathtools}
\usepackage{bussproofs}
\usepackage{stackengine}
\def\ruleoffset{1pt}
\newcommand\specialvdash[2]{\mathrel{\ensurestackMath{
  \mkern2mu\rule[-\dp\strutbox]{.4pt}{\baselineskip}\stackon[\ruleoffset]{
    \stackunder[\dimexpr\ruleoffset-.5\ht\strutbox+.5\dp\strutbox]{
      \rule[\dimexpr.5\ht\strutbox-.5\dp\strutbox]{2.5ex}{.4pt}}{
        \scriptstyle #1}}{\scriptstyle#2}\mkern2mu}}
}

\usepackage[table]{xcolor}

\renewcommand\thesection{\arabic{section}}
\renewcommand\thefigure{\arabic{figure}}
\renewcommand\theequation{\arabic{equation}}

\newtheorem{dfn}{Definition}[section]
\newtheorem{thm}[dfn]{Theorem}
\newtheorem{lem}[dfn]{Lemma}
\newtheorem{cor}[dfn]{Corollary}


\theoremstyle{definition}
\newtheorem{exl}[dfn]{Example}
\newtheorem{rem}[dfn]{Remark}
\newtheorem{exc}{Exercise}[section]

\def\R{\mathbb{R}}
\def\N{\mathbb{N}}
\def\Z{\mathbb{Z}}
\def\C{\mathbb{C}}
\def\cP{\mathcal{P}}
\def\cV{\mathcal{V}}
\def\cF{\mathcal{F}}
\def\Th{\mathrm{Th}}

\renewcommand{\emptyset}{\varnothing}
\renewcommand{\phi}{\varphi}
\renewcommand{\epsilon}{\varepsilon}
\def\gcd{\operatorname{gcd}}

\def\Prop{\mathrm{PROP}}
\begin{document}
\setcounter{section}{3}
\setcounter{subsection}{3}
\setcounter{dfn}{2}

\begin{proof}
One has
\[
\frac{1}{(1-x)^k} = \underbrace{(1+x+x^2+\cdots)\cdots(1+x+x^2+\cdots)}_{k}
\]
When one expands the brackets in the product on the right hand side,
one picks from the first brackets a monomial $x^{m_1}$, from the second $x^{m_2}$ and so on up to $x^{m_k}$ from the last brackets.
The product of these monomials is $x^{m_1+\cdots+m_k}$.
When one collects the monomials of degree $n$, one obtains a term $a_n x^n$,
where $n$ is the number of solutions of the equation
\[
m_1 + \cdots + m_k = n,
\]
where the unknowns $m_1, \ldots, m_k$ can take only non-negative integer values.
The number of solutions is exactly the number of weak compositions of $n$ from $k$ parts.
\end{proof}

Now, by applying the generalized binomial theorem one obtains
\[
\sum_{n=0}^\infty a_n x^n = \frac{1}{(1-x)^k} = (1-x)^{-k} = \sum_{n=0}^\infty (-1)^n \binom{-k}{n} x^n
\]
Thus we have
\[
a_n = (-1)^n \binom{-k}{n} = \binom{n+k-1}{n} = \binom{n+k-1}{k-1}.
\]




\end{document}