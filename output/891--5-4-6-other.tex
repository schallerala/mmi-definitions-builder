\documentclass[preview, multi=page, margin=5mm, class=report]{standalone}
\usepackage[utf8]{inputenc}

\usepackage{amsmath,amssymb,amsthm}
\usepackage{graphicx,color}
\usepackage{hyperref,url}
\graphicspath{{Fig/}}

\usepackage{mathtools}
\usepackage{bussproofs}
\usepackage{stackengine}
\def\ruleoffset{1pt}
\newcommand\specialvdash[2]{\mathrel{\ensurestackMath{
  \mkern2mu\rule[-\dp\strutbox]{.4pt}{\baselineskip}\stackon[\ruleoffset]{
    \stackunder[\dimexpr\ruleoffset-.5\ht\strutbox+.5\dp\strutbox]{
      \rule[\dimexpr.5\ht\strutbox-.5\dp\strutbox]{2.5ex}{.4pt}}{
        \scriptstyle #1}}{\scriptstyle#2}\mkern2mu}}
}

\usepackage[table]{xcolor}

\renewcommand\thesection{\arabic{section}}
\renewcommand\thefigure{\arabic{figure}}
\renewcommand\theequation{\arabic{equation}}

\newtheorem{dfn}{Definition}[section]
\newtheorem{thm}[dfn]{Theorem}
\newtheorem{lem}[dfn]{Lemma}
\newtheorem{cor}[dfn]{Corollary}


\theoremstyle{definition}
\newtheorem{exl}[dfn]{Example}
\newtheorem{rem}[dfn]{Remark}
\newtheorem{exc}{Exercise}[section]

\def\R{\mathbb{R}}
\def\N{\mathbb{N}}
\def\Z{\mathbb{Z}}
\def\C{\mathbb{C}}
\def\cP{\mathcal{P}}
\def\cV{\mathcal{V}}
\def\cF{\mathcal{F}}
\def\Th{\mathrm{Th}}

\renewcommand{\emptyset}{\varnothing}
\renewcommand{\phi}{\varphi}
\renewcommand{\epsilon}{\varepsilon}
\def\gcd{\operatorname{gcd}}

\def\Prop{\mathrm{PROP}}
\begin{document}
\setcounter{section}{0}
\setcounter{subsection}{0}
\setcounter{dfn}{10}


If we prove this lemma, then the formula for $c_n$ follows immediately:
\begin{multline*}
c_n = \binom{2n}{n} - \binom{2n}{n-1} = \frac{(2n)!}{n!n!} - \frac{(2n)!}{(n+1)!(n-1)!}\\
= \frac{(2n)!}{(n+1)!n!}((n+1) - n) = \frac{1}{n+1} \binom{2n}{n}.
\end{multline*}


\begin{proof}[Proof of Lemma \ref{lem:NonDyck}]
Let $(k,k-1)$ be the first point where a non-Dyck path enters the triangle below the diagonal.
The number $k$ can take any value between $1$ (which means that the first step goes below the diagonal) and $n$.
Reflect the part of the path from $(k,k-1)$ to $(n,n)$ as shown in Figure \ref{fig:NonDyckCount}.

\begin{figure}[ht]
\begin{center}
\includegraphics{NonDyckCount}
\end{center}
\caption{Counting non-Dyck paths.}
\label{fig:NonDyckCount}
\end{figure}

This transforms every non-Dyck path to a path from $(0,0)$ to $(n+1,n-1)$.
For every path from $(0,0)$ to $(n+1,n-1)$ there is a unique non-Dyck path that produces it.
To reconstruct this non-Dyck path, apply the same operation: take the first point of the form $(k,k-1)$ on the path to $(n+1,n-1)$
and reflect the part of the path after this point.
\end{proof}




\end{document}