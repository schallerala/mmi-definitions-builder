\documentclass[preview, multi=page, margin=5mm, class=report]{standalone}
\usepackage[utf8]{inputenc}

\usepackage{amsmath,amssymb,amsthm}
\usepackage{graphicx,color}
\usepackage{hyperref,url}
\graphicspath{{Fig/}}

\usepackage{mathtools}
\usepackage{bussproofs}
\usepackage{stackengine}
\def\ruleoffset{1pt}
\newcommand\specialvdash[2]{\mathrel{\ensurestackMath{
  \mkern2mu\rule[-\dp\strutbox]{.4pt}{\baselineskip}\stackon[\ruleoffset]{
    \stackunder[\dimexpr\ruleoffset-.5\ht\strutbox+.5\dp\strutbox]{
      \rule[\dimexpr.5\ht\strutbox-.5\dp\strutbox]{2.5ex}{.4pt}}{
        \scriptstyle #1}}{\scriptstyle#2}\mkern2mu}}
}

\usepackage[table]{xcolor}

\renewcommand\thesection{\arabic{section}}
\renewcommand\thefigure{\arabic{figure}}
\renewcommand\theequation{\arabic{equation}}

\newtheorem{dfn}{Definition}[section]
\newtheorem{thm}[dfn]{Theorem}
\newtheorem{lem}[dfn]{Lemma}
\newtheorem{cor}[dfn]{Corollary}


\theoremstyle{definition}
\newtheorem{exl}[dfn]{Example}
\newtheorem{rem}[dfn]{Remark}
\newtheorem{exc}{Exercise}[section]

\def\R{\mathbb{R}}
\def\N{\mathbb{N}}
\def\Z{\mathbb{Z}}
\def\C{\mathbb{C}}
\def\cP{\mathcal{P}}
\def\cV{\mathcal{V}}
\def\cF{\mathcal{F}}
\def\Th{\mathrm{Th}}

\renewcommand{\emptyset}{\varnothing}
\renewcommand{\phi}{\varphi}
\renewcommand{\epsilon}{\varepsilon}
\def\gcd{\operatorname{gcd}}

\def\Prop{\mathrm{PROP}}
\begin{document}
\setcounter{section}{5}
\setcounter{subsection}{2}
\setcounter{dfn}{10}




\begin{proof}[Proof of Theorem \ref{thm:ChomskyForm}]
Let $L$ be a language without $\epsilon$.
By Lemma \ref{lem:NoENoUnit} there is a grammar $G$ without $\epsilon$ and unit productions such that $L = L(G)$.
If a production of $G$ has a single symbol on the right, then this symbol is a terminal, so the production is of the form $A \to a$.

Any other production of $G$ has the form
\[
A \to X_1 X_2 \cdots X_n, \quad n \ge 2,
\]
where every $X_i$ is either a variable or a terminal.
If $X_i = a$ is a terminal, then introduce a new variable $C_a$ and a new production $C_a \to a$.
In the ``long'' ($n \ge 2$) right hand sides of all productions replace $a$ by $C_a$.
Clearly, the new grammar $G'$ generates the same language as the old one.

In the grammar $G'$, all productions are of the form $A \to a$ or $A \to B_1 \cdots B_n$, $n \ge 2$.
Create a new grammar $G''$ by introducing for each production of a ``long'' ($n \ge 3$) word a new set of variables $D_1, \ldots, D_{n-2}$
and replacing this production by a set of productions
\[
A \to B_1D_1, \quad D_1 \to B_2D_2, \ldots, \quad D_{n-3} \to B_{n-2}D_{n-2}, \quad D_{n-2} \to B_{n-1}B_n.
\]
Again, it is not hard to convince yourself that the new grammar generates the same language.
\end{proof}


\end{document}