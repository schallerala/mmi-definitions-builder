\begin{exl}
Let us construct a DFA equivalent to the $\epsilon$-NFA from Example \ref{exl:ENFA}.

The transition table is obtained by consulting the table from Example \ref{exl:ENFA}
and applying the rule $\delta'(P, a) = \overline{\delta(P, a)}$.
As in the construction of an NFA out of a DFA, it might be not necessary to consider all subsets of $Q$.
We start with the row corresponding to the initial state,
and add a new row for every state which appeared in one of the previous rows.
The construction ends when no new states appear.

The initial state in our case is $\overline{\{q_0\}} = \{q_0, q_1\}$.
\begin{center}
\begin{tabular}{c|ccc}
& $-$ & $0$ & $1$-$9$\\\hline
$\{q_0, q_1\}$ & $\{q_1\}$ & $\{q_3\}$ & $\{q_2, q_3\}$\\
$\{q_1\}$ & $\emptyset$ & $\emptyset$ & $\{q_2, q_3\}$\\
$\{q_3\}$ & $\emptyset$ & $\emptyset$ & $\emptyset$\\
$\{q_2, q_3\}$ & $\emptyset$ & $\{q_2, q_3\}$ & $\{q_2, q_3\}$\\
$\emptyset$ & $\emptyset$ & $\emptyset$ & $\emptyset$
\end{tabular}
\end{center}

\begin{figure}[ht]
\begin{center}
\input{Fig/DFAFromENFA.pdf_t}
\end{center}
\caption{A DFA equivalent to the $\epsilon$-NFA from Example \ref{exl:ENFA}.}
\label{fig:DFAFromENFA}
\end{figure}

The diagram of this automaton is shown in Figure \ref{fig:DFAFromENFA}.
\end{exl}