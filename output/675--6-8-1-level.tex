\section{Turing machines}
\subsection{Definition}
Similarly to all automata we considered before,
a Turing machine has a finite number of states and changes its states according to the input.
The input is a finite word on an infinite tape, and there are two new aspects in how the machine interacts with the input:
\begin{itemize}
\item
the machine can move along the tape;
\item
the machine can write on the tape.
\end{itemize}

Any sort of input: a number, a list, a table, a combinatorial structure such as a graph,
can be encoded as a sequence of symbols on a tape (one just needs to choose an encoding convention).
For any algorithm: multiplication of integers, search for a path between two vertices in a graph,
there is a Turing machine which applies this algorithm to any given input.
The last sentence is, in fact, a definition of the algorithm and a form of the Church thesis:
everything which ``can be computed'' can be done with some Turing machine.
