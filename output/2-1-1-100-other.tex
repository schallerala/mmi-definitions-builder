\documentclass[preview, multi=page, margin=5mm, class=report]{standalone}
\usepackage[utf8]{inputenc}

\usepackage{amsmath,amssymb,amsthm}
\usepackage{graphicx,color}
\usepackage{hyperref,url}
\graphicspath{{Fig/}}

\usepackage{mathtools}
\usepackage{bussproofs}
\usepackage{stackengine}
\def\ruleoffset{1pt}
\newcommand\specialvdash[2]{\mathrel{\ensurestackMath{
  \mkern2mu\rule[-\dp\strutbox]{.4pt}{\baselineskip}\stackon[\ruleoffset]{
    \stackunder[\dimexpr\ruleoffset-.5\ht\strutbox+.5\dp\strutbox]{
      \rule[\dimexpr.5\ht\strutbox-.5\dp\strutbox]{2.5ex}{.4pt}}{
        \scriptstyle #1}}{\scriptstyle#2}\mkern2mu}}
}

\usepackage[table]{xcolor}

\renewcommand\thesection{\arabic{section}}
\renewcommand\thefigure{\arabic{figure}}
\renewcommand\theequation{\arabic{equation}}

\newtheorem{dfn}{Definition}[section]
\newtheorem{thm}[dfn]{Theorem}
\newtheorem{lem}[dfn]{Lemma}
\newtheorem{cor}[dfn]{Corollary}


\theoremstyle{definition}
\newtheorem{exl}[dfn]{Example}
\newtheorem{rem}[dfn]{Remark}
\newtheorem{exc}{Exercise}[section]

\def\R{\mathbb{R}}
\def\N{\mathbb{N}}
\def\Z{\mathbb{Z}}
\def\C{\mathbb{C}}
\def\cP{\mathcal{P}}
\def\cV{\mathcal{V}}
\def\cF{\mathcal{F}}
\def\Th{\mathrm{Th}}

\renewcommand{\emptyset}{\varnothing}
\renewcommand{\phi}{\varphi}
\renewcommand{\epsilon}{\varepsilon}
\def\gcd{\operatorname{gcd}}

\def\Prop{\mathrm{PROP}}
\begin{document}
\setcounter{section}{1}
\setcounter{subsection}{1}
\setcounter{dfn}{1}


Examples of graphs: transport networks, neural networks, ``friendship'' graphs (social networks).
Usually the set $V$ of vertices is assumed to be finite.
However it can be quite large (and it is in some of the above examples).

In some situations one is lead to consider graphs with \emph{loops} (lines joining a vertex to itself)
and \emph{multiple edges} (several lines between the same pair of vertices), see Figure \ref{fig:OtherGraphs}, left.
In some other situations one wants to draw arrows instead of lines, see Figure \ref{fig:OtherGraphs}, right.
Graphs with oriented edges are called \emph{directed graphs}.
Another type of graphs are \emph{weighted graphs}: here to every edge a number is assigned.
When we say ``a graph'', we mean it in the sense of definition \ref{dfn:Graph}: an undirected graph without loops and multiple edges
and without assignment of weights.

\begin{figure}[ht]
\begin{center}
\includegraphics[width=.8\textwidth]{OtherGraphs.pdf}
\end{center}
\caption{A graph with loops and multiple edges; a directed graph.}
\label{fig:OtherGraphs}
\end{figure}



\end{document}
