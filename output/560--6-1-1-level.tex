\subsection{Alphabets, words, and languages}
An \emph{alphabet} is any finite set of symbols.
Examples:
\begin{itemize}
\item
the binary alphabet $\{0,1\}$;
\item
the alphabet of a single symbol $\{0\}$;
\item
the alphabet $\{p_1, \ldots, p_n\} \cup \{\neg, \wedge, \vee, \to, (, )\}$ of the propositional logic.
\end{itemize}

A \emph{string} or a \emph{word} is a finite sequence of symbols from a given alphabet.
The set of words of length $n$ in the alphabet $\Sigma$ is denoted by $\Sigma^n$:
\[
\Sigma^n = \{x_1 \ldots x_n \mid x_i \in \Sigma\ \forall i\}.
\]
This is the same as the Cartesian power $\Sigma^n$, with only a notational difference: $x_1 \ldots x_n$
instead of $(x_1, \ldots, x_n)$.

The concatenation of two words defines a map $\Sigma^m \times \Sigma^n \to \Sigma^{m+n}$.
Clearly, $uv \ne vu$ in general.
Denote by $\Sigma^* = \Sigma^0 \cup \Sigma^1 \cup \Sigma^2 \cup \cdots$ the set of all words in the alphabet $\Sigma$.

There is a unique element in $\Sigma^0$: the word of zero length; it is denoted by $\epsilon$.
One has
\[
\epsilon w = w = w\epsilon \text{ for all } w \in \Sigma^*.
\]

A \emph{language} is a subset of $\Sigma^*$.
Here are some examples of languages.

\begin{itemize}
\item
The set of all sequences of zeros of prime length:
\[
\{0^p \mid p \text{ is a prime number}\}.
\]
\item
The set of all binary palindromes (binary sequences that read the same forward and backward):
\[
\{\epsilon, 0, 1, 00, 11, 000, 010, \ldots\}.
\]
\item
In the alphabet of propositional logic, the set of all propositional formulas.
\item
In the same alphabet, the set of all propositional formulas which are tautologies.
\end{itemize}




