\documentclass[preview, multi=page, margin=5mm, class=report]{standalone}
\usepackage[utf8]{inputenc}

\usepackage{amsmath,amssymb,amsthm}
\usepackage{graphicx,color}
\usepackage{hyperref,url}
\graphicspath{{Fig/}}

\usepackage{mathtools}
\usepackage{bussproofs}
\usepackage{stackengine}
\def\ruleoffset{1pt}
\newcommand\specialvdash[2]{\mathrel{\ensurestackMath{
  \mkern2mu\rule[-\dp\strutbox]{.4pt}{\baselineskip}\stackon[\ruleoffset]{
    \stackunder[\dimexpr\ruleoffset-.5\ht\strutbox+.5\dp\strutbox]{
      \rule[\dimexpr.5\ht\strutbox-.5\dp\strutbox]{2.5ex}{.4pt}}{
        \scriptstyle #1}}{\scriptstyle#2}\mkern2mu}}
}

\usepackage[table]{xcolor}

\renewcommand\thesection{\arabic{section}}
\renewcommand\thefigure{\arabic{figure}}
\renewcommand\theequation{\arabic{equation}}

\newtheorem{dfn}{Definition}[section]
\newtheorem{thm}[dfn]{Theorem}
\newtheorem{lem}[dfn]{Lemma}
\newtheorem{cor}[dfn]{Corollary}


\theoremstyle{definition}
\newtheorem{exl}[dfn]{Example}
\newtheorem{rem}[dfn]{Remark}
\newtheorem{exc}{Exercise}[section]

\def\R{\mathbb{R}}
\def\N{\mathbb{N}}
\def\Z{\mathbb{Z}}
\def\C{\mathbb{C}}
\def\cP{\mathcal{P}}
\def\cV{\mathcal{V}}
\def\cF{\mathcal{F}}
\def\Th{\mathrm{Th}}

\renewcommand{\emptyset}{\varnothing}
\renewcommand{\phi}{\varphi}
\renewcommand{\epsilon}{\varepsilon}
\def\gcd{\operatorname{gcd}}

\def\Prop{\mathrm{PROP}}
\begin{document}
\setcounter{section}{2}
\setcounter{subsection}{5}
\setcounter{dfn}{13}

A parent vertex in a rooted tree is a vertex with the out-degree $\ge 1$ in the canonical orientation of the edges, see Figure \ref{fig:RootedTreeOrient}.
Vertices of out-degree $0$ will be called \emph{leaves} of a rooted tree.
Since the in-degree of every non-root vertex is $1$, non-root leaves are leaves in the usual sense.
However, the root is a leaf if and only if the tree has only one vertex.

A standalone sequent is a simplest deduction tree (tree with one vertex).
Next to it are the deduction trees copied from the inference rules as shown in Figure \ref{fig:DeductionTree}.
Note that a deduction tree the root is at the bottom, and the children of every vertex are situated above the vertex.

\begin{figure}[ht]
\begin{center}
\raisebox{1cm}{
\AxiomC{$\Gamma \vdash A, \Delta$}
\AxiomC{$\Gamma \vdash B, \Delta$}
\BinaryInfC{$\Gamma \vdash A \wedge B, \Delta$}
\DisplayProof
}
\hspace{1cm}
\input{Fig/DeductionTree.pdf_t}
\end{center}
\caption{The ($\wedge$: right) inference rule as a deduction tree.}
\label{fig:DeductionTree}
\end{figure}


\end{document}