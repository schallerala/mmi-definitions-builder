\documentclass[preview, multi=page, margin=5mm, class=report]{standalone}
\usepackage[utf8]{inputenc}

\usepackage{amsmath,amssymb,amsthm}
\usepackage{graphicx,color}
\usepackage{hyperref,url}
\graphicspath{{Fig/}}

\usepackage{mathtools}
\usepackage{bussproofs}
\usepackage{stackengine}
\def\ruleoffset{1pt}
\newcommand\specialvdash[2]{\mathrel{\ensurestackMath{
  \mkern2mu\rule[-\dp\strutbox]{.4pt}{\baselineskip}\stackon[\ruleoffset]{
    \stackunder[\dimexpr\ruleoffset-.5\ht\strutbox+.5\dp\strutbox]{
      \rule[\dimexpr.5\ht\strutbox-.5\dp\strutbox]{2.5ex}{.4pt}}{
        \scriptstyle #1}}{\scriptstyle#2}\mkern2mu}}
}

\usepackage[table]{xcolor}

\renewcommand\thesection{\arabic{section}}
\renewcommand\thefigure{\arabic{figure}}
\renewcommand\theequation{\arabic{equation}}

\newtheorem{dfn}{Definition}[section]
\newtheorem{thm}[dfn]{Theorem}
\newtheorem{lem}[dfn]{Lemma}
\newtheorem{cor}[dfn]{Corollary}


\theoremstyle{definition}
\newtheorem{exl}[dfn]{Example}
\newtheorem{rem}[dfn]{Remark}
\newtheorem{exc}{Exercise}[section]

\def\R{\mathbb{R}}
\def\N{\mathbb{N}}
\def\Z{\mathbb{Z}}
\def\C{\mathbb{C}}
\def\cP{\mathcal{P}}
\def\cV{\mathcal{V}}
\def\cF{\mathcal{F}}
\def\Th{\mathrm{Th}}

\renewcommand{\emptyset}{\varnothing}
\renewcommand{\phi}{\varphi}
\renewcommand{\epsilon}{\varepsilon}
\def\gcd{\operatorname{gcd}}

\def\Prop{\mathrm{PROP}}
\begin{document}
\setcounter{section}{2}
\setcounter{subsection}{2}
\setcounter{dfn}{0}

\subsection{Fibonacci again}
\label{sec:FibAgain}
Take the Fibonacci sequence
\[
(a_0, a_1, a_2, a_3, \ldots) = (0, 1, 1, 2, \ldots)
\]
and write a power series
\[
A(x) = \sum_{k=0}^\infty a_k x^k = a_0 + a_1 x + a_2 x^2 + \cdots.
\]

Because of
\begin{align*}
xA(x)  = a_0x + & a_1x^2 + a_2x^3 + \cdots\\
x^2A(x)  = & a_0x^2 + a_1x^3 + \cdots
\end{align*}
we have
\begin{multline*}
xA(x) + x^2A(x) = a_0 x + (a_1+a_0)x^2 + (a_2+a_1)x^3 + \cdots\\
= a_2x^2 + a_3x^3 + \cdots = A(x) - a_0 - a_1 x = A(x) - x.
\end{multline*}
This implies
\[
A(x)(1-x-x^2) = x \Rightarrow A(x) = \frac{x}{1-x-x^2}.
\]
(At the moment it is not clear what this equation means and why can we perform with the power series $A(x)$ the above algebraic manipulations.
A justification will be given later. Now let us continue to do whatever looks reasonable.)

We claim that there are real numbers $A, B$ such that
\[
\frac{x}{1-x-x^2} = \frac{x}{(1-\lambda_1x)(1-\lambda_2x)} = \frac{A}{1-\lambda_1x} + \frac{B}{1-\lambda_2x}
\]
Here $\lambda_1 = \frac{1+\sqrt{5}}2$, and $\lambda_2 = \frac{1-\sqrt{5}}2$.

The numbers $A$ and $B$ can be found by a smart guess:
\begin{multline*}
\frac{x}{(1-\lambda_1x)(1-\lambda_2x)} =
\frac{1}{\lambda_1 - \lambda_2} \frac{(1-\lambda_2x) - (1-\lambda_1x)}{(1-\lambda_1x)(1-\lambda_2x)}\\
= \frac{1}{\sqrt{5}} \left( \frac{1}{1-\lambda_1x} - \frac{1}{1-\lambda_2x} \right)
\end{multline*}
Or they can be found by writing down a system of linear equations:
\begin{multline*}
\frac{x}{(1-\lambda_1x)(1-\lambda_2x)} = \frac{A}{1-\lambda_1x} + \frac{B}{1-\lambda_2x}\\
= \frac{A(1-\lambda_2x) + B(1-\lambda_1x)}{(1-\lambda_1x)(1-\lambda_2x)}
= \frac{(A+B) - (A\lambda_2 + B\lambda_1)x}{(1-\lambda_1x)(1-\lambda_2x)}\\
\Rightarrow \begin{cases} A+B = 0\\ A\lambda_2 + B\lambda_1 = -1 \end{cases}
\end{multline*}

Anyway, we have
\[
A(x) = \frac{x}{1-x-x^2} = \frac{1}{\sqrt{5}} \left( \frac{1}{1-\lambda_1x} - \frac{1}{1-\lambda_2x} \right).
\]
Now, from the formula for geometric progression
\[
\frac{1}{1-y} = 1 + y + y^2 + y^3 + \cdots
\]
by substituting $y = \lambda x$ we get
\[
\frac{1}{1 - \lambda x} = 1 + \lambda x + \lambda^2 x^2 + \lambda^3 x^3 + \cdots.
\]
Thus we have
\[
A(x) = \frac{1}{\sqrt{5}} \left( \sum_{k=0}^\infty \lambda_1^kx^k - \sum_{k=0}^\infty \lambda_2^kx^k \right)
= \sum_{k=0}^\infty \frac{\lambda_1^k - \lambda_2^k}{\sqrt{5}} x^k,
\]
which means that
\[
a_k = \frac{\lambda_1^k - \lambda_2^k}{\sqrt{5}},
\]
the Binet formula again.



\end{document}