\documentclass[preview, multi=page, margin=5mm, class=report]{standalone}
\usepackage[utf8]{inputenc}

\usepackage{amsmath,amssymb,amsthm}
\usepackage{graphicx,color}
\usepackage{hyperref,url}
\graphicspath{{Fig/}}

\usepackage{mathtools}
\usepackage{bussproofs}
\usepackage{stackengine}
\def\ruleoffset{1pt}
\newcommand\specialvdash[2]{\mathrel{\ensurestackMath{
  \mkern2mu\rule[-\dp\strutbox]{.4pt}{\baselineskip}\stackon[\ruleoffset]{
    \stackunder[\dimexpr\ruleoffset-.5\ht\strutbox+.5\dp\strutbox]{
      \rule[\dimexpr.5\ht\strutbox-.5\dp\strutbox]{2.5ex}{.4pt}}{
        \scriptstyle #1}}{\scriptstyle#2}\mkern2mu}}
}

\usepackage[table]{xcolor}

\renewcommand\thesection{\arabic{section}}
\renewcommand\thefigure{\arabic{figure}}
\renewcommand\theequation{\arabic{equation}}

\newtheorem{dfn}{Definition}[section]
\newtheorem{thm}[dfn]{Theorem}
\newtheorem{lem}[dfn]{Lemma}
\newtheorem{cor}[dfn]{Corollary}


\theoremstyle{definition}
\newtheorem{exl}[dfn]{Example}
\newtheorem{rem}[dfn]{Remark}
\newtheorem{exc}{Exercise}[section]

\def\R{\mathbb{R}}
\def\N{\mathbb{N}}
\def\Z{\mathbb{Z}}
\def\C{\mathbb{C}}
\def\cP{\mathcal{P}}
\def\cV{\mathcal{V}}
\def\cF{\mathcal{F}}
\def\Th{\mathrm{Th}}

\renewcommand{\emptyset}{\varnothing}
\renewcommand{\phi}{\varphi}
\renewcommand{\epsilon}{\varepsilon}
\def\gcd{\operatorname{gcd}}

\def\Prop{\mathrm{PROP}}
\begin{document}
\setcounter{section}{2}
\setcounter{subsection}{2}
\setcounter{dfn}{1}


A map is bijective iff at every element of $Y$ ends exactly one arrow.
By inverting the arrows we obtain the \emph{inverse map} $f^{-1} \colon Y \to X$,
which has the properties $f^{-1}(f(x)) = x$ for all $x \in X$ and $f(f^{-1}(y)) = y$ for all $y \in Y$.

If $f$ is not bijective, then there is no inverse map $f^{-1}$.
However, by abuse of notation one uses $f^{-1}(y)$ to denote the \emph{preimage} of $y$:
\[
f^{-1}(y) = \{x \in X \mid f(x) = y\}.
\]
Similarly one can define the preimage $f^{-1}(B)$ of any subset $B \subset Y$.

Observe that
\begin{itemize}
\item
$f$ injective $\Leftrightarrow$ $|f^{-1}(y)| \le 1$ for all $y$;
\item
$f$ surjective $\Leftrightarrow$ $f^{-1}(y) \ne \emptyset$ for all $y$.
\end{itemize}

We can now formulate the quotient rule in the mathematical language.

\begin{center}
\parbox{.75\textwidth}{\emph{If a map $f \colon X \to Y$ satisfies $|f^{-1}(y)| = k$ for all $y \in Y$, then $|Y| = \frac{|X|}{k}$.}}
\end{center}

A special case of this is the bijection principle:
\begin{center}
\parbox{.75\textwidth}{\emph{If a map $f \colon X \to Y$ is a bijection, then $|X| = |Y|$.}}
\end{center}



\end{document}