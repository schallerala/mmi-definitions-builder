\subsection{Definition and examples}
A regular expression is defined recursively.
The basic building blocks are the following.
\begin{enumerate}
\item
$\emptyset$ is a regular expression and denotes the language $\emptyset$.
\item
$\epsilon$ is a regular expression and denotes the language $\{\epsilon\}$.
\item
$a$ is a regular expression for every $a \in \Sigma$ and denotes the language $\{a\}$.
\end{enumerate}
Don't confuse the empty language $\emptyset$ and the language $\{\epsilon\}$ consisting of an empty word.


% \begin{figure}[ht]
% \begin{center}
% \includegraphics{EmptyEmpty.pdf}
% \end{center}
% \caption{NFAs for the languages $\emptyset$ and $\{\epsilon\}$.}
% \label{fig:EmptyEmpty}
% \end{figure}

Sometimes one uses the boldface $\bf{a}$ to denote the language $\{a\}$.
We will use the same symbol $a$.

From these building blocks one constructs more complex regular expressions by using the following operations.
If $r$ and $s$ are regular expressions denoting the languages $R$ and $S$ respectively, then
\begin{enumerate}
\item
$(r+s)$ is a regular expression and denotes the language $R \cup S$;
\item
$(rs)$ is a regular expression and denotes the language $RS = \{uv \mid u \in R, v \in S\}$;
\item
$r^*$ is a regular expression and denotes the language $R^* = \cup_{i=0}^\infty R^i$, where $R^i = \underbrace{RR \cdots R}_{i}$
(the \emph{Kleene closure} of language $R$).
\end{enumerate}
