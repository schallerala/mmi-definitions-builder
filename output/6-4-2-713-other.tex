\documentclass[preview, margin=5mm, multi=page]{standalone}
\usepackage[utf8]{inputenc}

\usepackage{amsmath,amssymb,amsthm}
\usepackage{graphicx,color}
\usepackage{hyperref,url}
\graphicspath{{Fig/}}

\usepackage{mathtools}
\usepackage{bussproofs}
\usepackage{stackengine}
\def\ruleoffset{1pt}
\newcommand\specialvdash[2]{\mathrel{\ensurestackMath{
  \mkern2mu\rule[-\dp\strutbox]{.4pt}{\baselineskip}\stackon[\ruleoffset]{
    \stackunder[\dimexpr\ruleoffset-.5\ht\strutbox+.5\dp\strutbox]{
      \rule[\dimexpr.5\ht\strutbox-.5\dp\strutbox]{2.5ex}{.4pt}}{
        \scriptstyle #1}}{\scriptstyle#2}\mkern2mu}}
}

\usepackage[table]{xcolor}

\renewcommand\thesection{\arabic{section}}
\renewcommand\thefigure{\arabic{figure}}
\renewcommand\theequation{\arabic{equation}}

\newtheorem{dfn}{Definition}[section]
\newtheorem{thm}[dfn]{Theorem}
\newtheorem{lem}[dfn]{Lemma}
\newtheorem{cor}[dfn]{Corollary}


\theoremstyle{definition}
\newtheorem{exl}[dfn]{Example}
\newtheorem{rem}[dfn]{Remark}
\newtheorem{exc}{Exercise}[section]

\def\R{\mathbb{R}}
\def\N{\mathbb{N}}
\def\Z{\mathbb{Z}}
\def\C{\mathbb{C}}
\def\cP{\mathcal{P}}
\def\cV{\mathcal{V}}
\def\cF{\mathcal{F}}
\def\Th{\mathrm{Th}}


\renewcommand{\emptyset}{\varnothing}
\renewcommand{\phi}{\varphi}
\renewcommand{\epsilon}{\varepsilon}
\def\gcd{\operatorname{gcd}}

\def\Prop{\mathrm{PROP}}



%opening
\title{{Lecture notes for the 2020/21 lectures}\\
$ $\\
$ $\\ \textsc{
Mathematical methods for Computer Science I \& II\\
and\\
Discrete Mathematics I \& II\\ }
$ $\\
$ $\\
$ $\\
$ $\\
University of Fribourg\\ Livio Liechti
$ $\\
$ $\\
$ $\\
$ $\\
$ $\\
$ $\\
$ $\\}
\date{ }

\author{Lecture notes written by Ivan Izmestiev for his 2018/19 lectures}


\begin{document}
\setcounter{section}{4}
\setcounter{subsection}{2}
\setcounter{dfn}{7}

\begin{proof}
Assume that $L$ is regular.
Take a DFA that accepts $L$.
Let $q_0, \ldots, q_n$ be its states, and $q_0$ be the initial state.
Denote
\[
T_i = \{w \in \Sigma^* \mid \widehat{\delta}(q_0, w) = q_i\}.
\]
One has
\[
\Sigma^* = T_0 \cup T_1 \cup \cdots T_n.
\]
We claim that every $T_i$ is a subset of some $S_j$ from the decomposition \eqref{eqn:SigmaEqClasses}.
In other words, every $S_j$ is the union of one or several $T_i$,
which means that the number of $\sim_L$-equivalence classes is at most $n+1$ and implies the first part of the theorem.

In order to prove the claim it suffices to show that if $u$ and $v$ belong to the same $T_i$, then $u \sim_L v$.
Then for every $x \in \Sigma^*$ one has
\[
\widehat{\delta}(q_0, ux) = \widehat{\delta}(\widehat{\delta}(q_0,u), x) = \widehat{\delta}(q_i, x)
= \widehat{\delta}(\widehat{\delta}(q_0,v), x) = \widehat{\delta}(q_0, vx).
\]
Since both words $ux$ and $vx$ bring us to the same state, they either both belong to $L$ (if this state is final)
or both not belong to $L$ (if this state is not final).
Thus $u \sim_L v$.


In the opposite direction, let $\Sigma^* = S_0 \cup \cdots \cup S_n$, where $\epsilon \in S_0$.
Construct a DFA with states $q_0, \ldots, q_n$, the initial state $q_0$, and the transition function defined as follows.
To find $\delta(q_i, a)$, take some $u \in S_i$ and look in which class the word $ua$ lies.
If $ua \in S_j$, then put $\delta(q_i, a) = q_j$.
The result is independent of the choice of a representative $u \in S_i$.
Indeed, by Lemma \ref{lem:LEquivRInvar} $u \sim_L v \Rightarrow ua \sim_L va$.
A state $q_i$ is designated as final if and only if $S_i \subset L$.
It is easy to see that the language accepted by this automaton is $L$.
\end{proof}



\end{document}
