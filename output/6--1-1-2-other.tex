
Up to now we have considered independent choices,
when the result of one choice does not influence the sets of subsequent choices.
However, it is not the \textbf{set} of choices what matters, but rather the \textbf{number} of choices.
This leads us to the general product rule:
\begin{center}
\parbox{.9\textwidth}{\emph{If two consecutive choices are made, with $m$ possibilities for the first choice and $n$ possibilities for the second choice,
then the number of all possible outcomes is equal to $mn$.}}
\end{center}
Of course, this can be generalized to several consecutive choices
if the number of possibilities for each choice is independent of the results of all previous choices.
