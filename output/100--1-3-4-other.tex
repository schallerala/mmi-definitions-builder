
The coefficients in the above formula come from the $n$-th row of the Pascal triangle.
For example, by looking at the Pascal triangle we can conclude
\[
(a+b)^5 = a^5 + 5a^4b + 10a^3b^2 + 10a^2b^3 + 5ab^4 + b^5.
\]

\begin{proof}
How do we prove $(a+b)^2 = a^2 + 2ab + b^2$?
For this, we write $(a+b)^2 = (a+b)(a+b)$, multiply
every term inside the first pair of brackets with every term inside the second pair of brackets,
and finally collect the like terms:
\[
(a+b)^2 = (a+b)(a+b) = a^2 + ab + ba + b^2 = a^2 + 2ab + b^2.
\]
What happens when we multiply out $n$ pairs of brackets $(a+b)$?
\[
(a+b)^n = \underbrace{(a+b)(a+b) \cdots (a+b)}_{n \text{ pairs of brackets}}
\]
Before collecting the like terms, we obtain a sum of products of $n$ factors,
every factor being $a$ or $b$.
That is to say, we are writing down all words of length $n$ consisting of letters $a$ and $b$.
When collecting the like terms, we ignore the order of letters in each word, counting only
the number of $a$'s and $b$'s.
That is to say, the term $a^{n-k}b^k$ occurs in our sum as often as
there are $(a,b)$-words of length $n$ with $n-k$ letters $a$ and $k$ letters $b$.
But we know that there are $\binom{n}{k}$ such words, and the theorem follows.
\end{proof}

Instead of $a$ and $b$ we can substitute any numbers or expressions.
For example, we have
\[
(1+x)^n = \sum_{k=0}^n \binom{n}{k} x^k.
\]
Note that by substituting $x=1$ we obtain a new (but a quite intricate) proof of Theorem \ref{thm:BinomSum}.
