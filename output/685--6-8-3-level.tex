\subsection{Modifications of Turing machines}
There are several variations of Turing machines, e.~g.:
\begin{itemize}
\item
multi-tape: several tapes, each with its own head for reading and writing;
\item
multidimensional: a square grid instead of a tape, the head can move not only left and right, but also up and down;
\item
non-deterministic.
\end{itemize}
Although they seem more powerful, each of them can be simulated by an ordinary Turing machine.



\subsection{Problems and languages}
Consider a problem like ``Is a given finite graph connected?''
An \emph{instance} of a problem is any finite graph.
Since graphs can be encoded as words in an alphabet, this problem determines a language:
the set of all words encoding connected graphs.
This holds for any question which depends on some countable parameter and has a yes/no answer:
encode the parameter values by words, and consider the set of all words for which the answer to the question is ``yes''.
