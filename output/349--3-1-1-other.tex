\documentclass[preview, multi=page, margin=5mm, class=report]{standalone}
\usepackage[utf8]{inputenc}

\usepackage{amsmath,amssymb,amsthm}
\usepackage{graphicx,color}
\usepackage{hyperref,url}
\graphicspath{{Fig/}}

\usepackage{mathtools}
\usepackage{bussproofs}
\usepackage{stackengine}
\def\ruleoffset{1pt}
\newcommand\specialvdash[2]{\mathrel{\ensurestackMath{
  \mkern2mu\rule[-\dp\strutbox]{.4pt}{\baselineskip}\stackon[\ruleoffset]{
    \stackunder[\dimexpr\ruleoffset-.5\ht\strutbox+.5\dp\strutbox]{
      \rule[\dimexpr.5\ht\strutbox-.5\dp\strutbox]{2.5ex}{.4pt}}{
        \scriptstyle #1}}{\scriptstyle#2}\mkern2mu}}
}

\usepackage[table]{xcolor}

\renewcommand\thesection{\arabic{section}}
\renewcommand\thefigure{\arabic{figure}}
\renewcommand\theequation{\arabic{equation}}

\newtheorem{dfn}{Definition}[section]
\newtheorem{thm}[dfn]{Theorem}
\newtheorem{lem}[dfn]{Lemma}
\newtheorem{cor}[dfn]{Corollary}


\theoremstyle{definition}
\newtheorem{exl}[dfn]{Example}
\newtheorem{rem}[dfn]{Remark}
\newtheorem{exc}{Exercise}[section]

\def\R{\mathbb{R}}
\def\N{\mathbb{N}}
\def\Z{\mathbb{Z}}
\def\C{\mathbb{C}}
\def\cP{\mathcal{P}}
\def\cV{\mathcal{V}}
\def\cF{\mathcal{F}}
\def\Th{\mathrm{Th}}

\renewcommand{\emptyset}{\varnothing}
\renewcommand{\phi}{\varphi}
\renewcommand{\epsilon}{\varepsilon}
\def\gcd{\operatorname{gcd}}

\def\Prop{\mathrm{PROP}}
\begin{document}
\setcounter{section}{1}
\setcounter{subsection}{1}
\setcounter{dfn}{0}

\subsection{Propositional formulas}
\label{sec:PropFormulas}
The language of propositional logic consists of strings of symbols, where each of the symbols is one of the following:
\begin{itemize}
\item
A proposition symbol $p$, $q$, $r$, $s$, $p_1, p_2, \ldots$.
(A countably infinite set.)
\item
A logical connective $\wedge$, $\vee$, $\to$, $\neg$.
\item
An auxiliary symbol $($ or $)$.
\end{itemize}
Sometimes to the list of logical connectives one adds $\leftrightarrow$ and $\perp$.
We will abstain from this.

The logical connectives have the following names.

\begin{center}
\begin{tabular}[c]{l@{\hspace{1cm}}l@{\hspace{1cm}}l}
$\wedge$ & and & conjunction\\
$\vee$ & or & disjunction\\
$\to$ & if ..., then ... & implication\\
$\neg$ & not & negation
\end{tabular}
\end{center}

Now there come the syntax rules describing what strings of symbols are allowed.


\end{document}