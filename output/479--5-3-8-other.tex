
This result is remarkable for two reasons.
Firstly, the coefficients on the right hand side take only values $0$ and $\pm 1$ (and more often $0$ than $\pm 1$).
This means that there is a lot of cancellation happening during this bracket expansion.
Secondly, we know the values of these coefficients exactly.

The proof of the Euler identity is very elegant.
Before learning it, let us see how this identity implies MacMahon's recurrence.

\begin{proof}[Proof of MacMahon's recurrence]
The formula for generating function of the number of partitions
\[
\sum_{n=0}^\infty p_n x^n = \frac{1}{(1-x)(1-x^2)(1-x^3) \cdots}
\]
and the Euler identity imply that
\[
(p_0 + p_1 x + p_2 x^2 + \cdots)(1 - x - x^2 + x^5 + x^7 - x^{12} - x^{15} + \cdots) = 1.
\]
The coefficient at $x^n$, $n \ge 1$, on the left hand side is
\[
p_n - p_{n-1} - p_{n-2} + p_{n-5} + p_{n-7} - \cdots,
\]
while on the right hand side the term $x^n$ is missing.
This implies the MacMahon recurrence.
\end{proof}


The numbers $\frac{3k^2 \pm k}{2}$ are called \emph{pentagonal numbers}
because they count the numbers of dots on Figure \ref{fig:PentaNumber}.
If one substitutes $-k$ for $k$, then $\frac{3k^2 - k}{2}$ becomes $\frac{3k^2 + k}{2}$,
this is why the latter number is also called pentagonal.

\begin{figure}[ht]
\begin{center}
\input{Fig/PentaNumbers.pdf_t}
\end{center}
\caption{Pentagonal numbers $\frac{3k^2 - k}{2}$.}
\label{fig:PentaNumber}
\end{figure}

What is of importance for us is not the arrangement of dots on Figure~\ref{fig:PentaNumber}
but Ferrers diagrams on Figure \ref{fig:FerrersPenta}.
They show that pentagonal numbers have partitions of the following form:
\begin{gather*}
\frac{3k^2-k}{2} = k + (k+1) + \cdots + (2k-1),\\
\frac{3k^2+k}{2} = (k+1) + (k+2) + \cdots + 2k.
\end{gather*}

\begin{figure}[ht]
\begin{center}
\includegraphics[width=.7\textwidth]{FerrersPenta}
\end{center}
\caption{Arrangements of $\frac{3k^2 - k}{2}$ or $\frac{3k^2 + k}{2}$ dots.}
\label{fig:FerrersPenta}
\end{figure}


Let us proceed to the proof of Theorem \ref{thm:EulerPenta}.
Denote by $q_n$ the number of partitions of $n$ into distinct parts.
Also, denote by $q_{n,\mathrm{even}}$ and $q_{n,\mathrm{odd}}$ the number of partitions of $n$ into an even,
respectively odd number of distinct parts.
