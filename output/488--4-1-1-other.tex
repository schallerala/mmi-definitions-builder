\documentclass[preview, multi=page, margin=5mm, class=report]{standalone}
\usepackage[utf8]{inputenc}

\usepackage{amsmath,amssymb,amsthm}
\usepackage{graphicx,color}
\usepackage{hyperref,url}
\graphicspath{{Fig/}}

\usepackage{mathtools}
\usepackage{bussproofs}
\usepackage{stackengine}
\def\ruleoffset{1pt}
\newcommand\specialvdash[2]{\mathrel{\ensurestackMath{
  \mkern2mu\rule[-\dp\strutbox]{.4pt}{\baselineskip}\stackon[\ruleoffset]{
    \stackunder[\dimexpr\ruleoffset-.5\ht\strutbox+.5\dp\strutbox]{
      \rule[\dimexpr.5\ht\strutbox-.5\dp\strutbox]{2.5ex}{.4pt}}{
        \scriptstyle #1}}{\scriptstyle#2}\mkern2mu}}
}

\usepackage[table]{xcolor}

\renewcommand\thesection{\arabic{section}}
\renewcommand\thefigure{\arabic{figure}}
\renewcommand\theequation{\arabic{equation}}

\newtheorem{dfn}{Definition}[section]
\newtheorem{thm}[dfn]{Theorem}
\newtheorem{lem}[dfn]{Lemma}
\newtheorem{cor}[dfn]{Corollary}


\theoremstyle{definition}
\newtheorem{exl}[dfn]{Example}
\newtheorem{rem}[dfn]{Remark}
\newtheorem{exc}{Exercise}[section]

\def\R{\mathbb{R}}
\def\N{\mathbb{N}}
\def\Z{\mathbb{Z}}
\def\C{\mathbb{C}}
\def\cP{\mathcal{P}}
\def\cV{\mathcal{V}}
\def\cF{\mathcal{F}}
\def\Th{\mathrm{Th}}

\renewcommand{\emptyset}{\varnothing}
\renewcommand{\phi}{\varphi}
\renewcommand{\epsilon}{\varepsilon}
\def\gcd{\operatorname{gcd}}

\def\Prop{\mathrm{PROP}}
\begin{document}
\setcounter{section}{1}
\setcounter{subsection}{1}
\setcounter{dfn}{1}


One can use the same set $\cV$ of variables in all signatures.
This is a countably infinite set; we use the letters $x$, $y$, $z$, $x_1, x_2, \ldots $ etc. to denote its elements.
As function symbols we will use $f$, $g$, $h$, $f_1, f_2, \ldots$
and as predicate symbols $P$, $Q$, $R$, $P_1, P_2, \ldots$.


The strange word \emph{arity} describes the number of arguments: a function or a predicate can be unary, binary, $k$-ary, and even nullary.
The arity can be indicated by the number of dots (or other placeholders) between the brackets.
Below is an example of a signature.
\begin{center}
\begin{tabular}{ll}
$\cF:\ f(\cdot)$ & \text{one unary function symbol}\\
$\cP:\ P(\cdot, \cdot), Q(\cdot)$ & \text{one binary and one unary predicate symbol}
\end{tabular}
\end{center}
Nullary functions and nullary predicates have no arguments.



\end{document}