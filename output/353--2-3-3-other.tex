
A graph $G'$ is a \emph{subdivision} of a graph $G$ if $G'$ is obtained from $G$ by repeated \emph{edge subdivisions}.
To subdivide an edge $e = \{v,w\}$ of a graph $G$ means to introduce a new vertex $x$, delete $e$, and introduce two new edges $\{x,v\}$ and $\{x,w\}$.
Figure \ref{fig:KSubdivisions} shows a subdivision of $K_5$ and a subdivision of $K_{3,3}$.

\begin{figure}[ht]
\begin{center}
\includegraphics[width=.8\textwidth]{KSubdivisions.pdf}
\end{center}
\caption{Some subdivisions of $K_5$ and $K_{3,3}$.}
\label{fig:KSubdivisions}
\end{figure}

One direction of the Kuratowski theorem is easy to prove: If a graph contains a subdivision of $K_5$ or $K_{3,3}$, then it cannot be planar.
Indeed, an embedding of the graph would contain an embedding of a subdivision of $K_5$ or $K_{3,3}$,
and hence an embedding of $K_5$ or $K_{3,3}$.
It is the opposite direction which is the most interesting and non-obvious:
the only obstacles to existence of a planar embedding of $G$ are graphs $K_5$ or $K_{3,3}$ contained in $G$ (in the form of subdivisions).


There is a similar planarity criterion that uses the notion of a \emph{minor}.
