\documentclass[preview, multi=page, margin=5mm, class=report]{standalone}
\usepackage[utf8]{inputenc}

\usepackage{amsmath,amssymb,amsthm}
\usepackage{graphicx,color}
\usepackage{hyperref,url}
\graphicspath{{Fig/}}

\usepackage{mathtools}
\usepackage{bussproofs}
\usepackage{stackengine}
\def\ruleoffset{1pt}
\newcommand\specialvdash[2]{\mathrel{\ensurestackMath{
  \mkern2mu\rule[-\dp\strutbox]{.4pt}{\baselineskip}\stackon[\ruleoffset]{
    \stackunder[\dimexpr\ruleoffset-.5\ht\strutbox+.5\dp\strutbox]{
      \rule[\dimexpr.5\ht\strutbox-.5\dp\strutbox]{2.5ex}{.4pt}}{
        \scriptstyle #1}}{\scriptstyle#2}\mkern2mu}}
}

\usepackage[table]{xcolor}

\renewcommand\thesection{\arabic{section}}
\renewcommand\thefigure{\arabic{figure}}
\renewcommand\theequation{\arabic{equation}}

\newtheorem{dfn}{Definition}[section]
\newtheorem{thm}[dfn]{Theorem}
\newtheorem{lem}[dfn]{Lemma}
\newtheorem{cor}[dfn]{Corollary}


\theoremstyle{definition}
\newtheorem{exl}[dfn]{Example}
\newtheorem{rem}[dfn]{Remark}
\newtheorem{exc}{Exercise}[section]

\def\R{\mathbb{R}}
\def\N{\mathbb{N}}
\def\Z{\mathbb{Z}}
\def\C{\mathbb{C}}
\def\cP{\mathcal{P}}
\def\cV{\mathcal{V}}
\def\cF{\mathcal{F}}
\def\Th{\mathrm{Th}}

\renewcommand{\emptyset}{\varnothing}
\renewcommand{\phi}{\varphi}
\renewcommand{\epsilon}{\varepsilon}
\def\gcd{\operatorname{gcd}}

\def\Prop{\mathrm{PROP}}
\begin{document}
\setcounter{section}{6}
\setcounter{subsection}{1}
\setcounter{dfn}{5}

\begin{exl}
\label{exl:Palindromes}
A PDA accepting the language $\{w \bar{w} \mid w \in \{0,1\}^*\}$ by the empty stack.
\[
M = (\{q_1, q_2\}, \{0, 1\}, \{A, B, Z_0\}, \delta, q_1, Z_0, \emptyset\}
\]
The principle is the same: while in the state $q_1$, we encode the input by putting into the stack $A$ for the input symbol $0$ and $B$ for the input $1$:
\begin{gather*}
\delta(q_1, 0, Z_0) = (q_1, AZ_0), \quad \delta(q_1, 0, B) = (q_1, AB)\\
\delta(q_1, 1, Z_0) = (q_1, BZ_0), \quad \delta(q_1, 1, A) = (q_1, BA)
\end{gather*}
However, if the input symbol agrees with the top stack symbol, then this might be the middle of the palindrome (but also might be not).
So, we make a guess and allow a multiple transition:
\[
\delta(q_1, 0, A) = \{(q_1, AA), (q_2, \epsilon)\}, \quad \delta(q_1, 1, B) = \{(q_1, BB), (q_2, \epsilon)\}.
\]
While in the state $q_2$, we compare the input with the content of the stack:
\[
\delta(q_2, 0, A) = (q_2, \epsilon) \quad \delta(q_2, 1, B) = (q_2, \epsilon)
\]
Finally, we have the possibility to empty the stack spontaneously if its top symbol is $Z_0$,
because this can happen only in the case if the input word was a palindrome (including the empty input):
\[
\delta(q_1, \epsilon, Z_0) = (q_2, \epsilon) \quad \delta(q_2, \epsilon, Z_0) = (q_2, \epsilon)
\]
\end{exl}

\end{document}