\documentclass[preview, multi=page, margin=5mm, class=report]{standalone}
\usepackage[utf8]{inputenc}

\usepackage{amsmath,amssymb,amsthm}
\usepackage{graphicx,color}
\usepackage{hyperref,url}
\graphicspath{{Fig/}}

\usepackage{mathtools}
\usepackage{bussproofs}
\usepackage{stackengine}
\def\ruleoffset{1pt}
\newcommand\specialvdash[2]{\mathrel{\ensurestackMath{
  \mkern2mu\rule[-\dp\strutbox]{.4pt}{\baselineskip}\stackon[\ruleoffset]{
    \stackunder[\dimexpr\ruleoffset-.5\ht\strutbox+.5\dp\strutbox]{
      \rule[\dimexpr.5\ht\strutbox-.5\dp\strutbox]{2.5ex}{.4pt}}{
        \scriptstyle #1}}{\scriptstyle#2}\mkern2mu}}
}

\usepackage[table]{xcolor}

\renewcommand\thesection{\arabic{section}}
\renewcommand\thefigure{\arabic{figure}}
\renewcommand\theequation{\arabic{equation}}

\newtheorem{dfn}{Definition}[section]
\newtheorem{thm}[dfn]{Theorem}
\newtheorem{lem}[dfn]{Lemma}
\newtheorem{cor}[dfn]{Corollary}


\theoremstyle{definition}
\newtheorem{exl}[dfn]{Example}
\newtheorem{rem}[dfn]{Remark}
\newtheorem{exc}{Exercise}[section]

\def\R{\mathbb{R}}
\def\N{\mathbb{N}}
\def\Z{\mathbb{Z}}
\def\C{\mathbb{C}}
\def\cP{\mathcal{P}}
\def\cV{\mathcal{V}}
\def\cF{\mathcal{F}}
\def\Th{\mathrm{Th}}

\renewcommand{\emptyset}{\varnothing}
\renewcommand{\phi}{\varphi}
\renewcommand{\epsilon}{\varepsilon}
\def\gcd{\operatorname{gcd}}

\def\Prop{\mathrm{PROP}}
\begin{document}
\setcounter{section}{1}
\setcounter{subsection}{4}
\setcounter{dfn}{10}

\begin{proof}
One proves that $\hat{v}((A \to B) \wedge (B \to A)) = 1$ if and only if $\hat{v}(A) = \hat{v}(B)$ by a case distinction,
that is by considering all four possible combinations of values $\hat{v}(A)$ and $\hat{v}(B)$.
Thus if $\hat{v}(A) = \hat{v}(B)$ for all $v$, then $(A \to B) \wedge (B \to A)$ is a tautology, and vice versa.
\end{proof}


There are several simple and useful equivalences between propositional formulas.
\begin{itemize}
\item
Associativity laws
\[
(A \wedge B) \wedge C \simeq A \wedge (B \wedge C) \qquad (A \vee B) \vee C \simeq A \vee (B \vee C)
\]
\item
Commutativity laws
\[
A \wedge B \simeq B \wedge A \qquad A \vee B \simeq B \vee A
\]
\item
Distributivity laws
\[
A \wedge (B \vee C) \simeq (A \wedge B) \vee (A \wedge C) \qquad A \vee (B \wedge C) \simeq (A \vee B) \wedge (A \vee C)
\]
\item
De Morgan's laws
\[
\neg(A \wedge B) \simeq \neg A \vee \neg B \qquad \neg(A \vee B) \simeq \neg A \wedge \neg B
\]
\item
Idempotency laws
\[
A \wedge A \simeq A \qquad A \vee A \simeq A
\]
\item
Double negation law
\[
\neg\neg A \simeq A
\]
\end{itemize}

The associativity laws allow us to omit parentheses in conjunctions or disjunctions.
Due to it we can allow abuse of notation and write strings like this one:
\[
p \wedge q \wedge r \wedge s.
\]
This is against the syntax rules.
In order to conform the rules, we must put some parentheses.
This leads to several different propositional formulas, like $((p \wedge q) \wedge (r \wedge s))$ or $((p \wedge (q \wedge r)) \wedge s)$.
Although these formulas are different, they are logically equivalent.
Thus, the string $p \wedge q \wedge r \wedge s$ stands for a class of equivalent formulas.

The distributivity laws allow us to operate with brackets as if $\wedge$ is the multiplication and $\vee$ is the addition
or vice versa.
For example,
\[
\neg p \vee (p \wedge q) \simeq (\neg p \vee p) \wedge (\neg p \vee q).
\]

Let us extend the language $\Prop$ by adding two new symbols $\top$ and $\perp$.
We define a language $\widetilde{\Prop}$ by adding to the point 1. in Definition \ref{dfn:PropForm} that $\top$ and $\perp$ belong to $\widetilde{\Prop}$
and leaving the other conditions as they are.
The symbols $\top$ and $\perp$ are \emph{logical constants}: when computing the truth value of a proposition $A$
we replace each $\top$ by $1$ and each $\perp$ by $0$.
In $\widetilde{\Prop}$ there are the following logical equivalences.
\begin{itemize}
\item
Laws of zero and one.
\begin{gather*}
(A \wedge \perp) \simeq \perp \qquad (A \vee \perp) \simeq A\\
(A \wedge \top) \simeq A \qquad (A \vee \top) \simeq \top\\
(A \wedge \neg A) \simeq \perp \qquad (A \vee \neg A) \simeq \top
\end{gather*}
\end{itemize}

These equivalences can be used in order to simplify certain propositions from $\Prop$.
As an example, let us simplify the formula we just obtained.
\[
(\neg p \vee p) \wedge (\neg p \vee q) \simeq \top \wedge (\neg p \vee q) \simeq \neg p \vee q.
\]




\end{document}