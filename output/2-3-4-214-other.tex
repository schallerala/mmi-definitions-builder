\documentclass[preview, multi=page, margin=5mm, class=report]{standalone}
\usepackage[utf8]{inputenc}

\usepackage{amsmath,amssymb,amsthm}
\usepackage{graphicx,color}
\usepackage{hyperref,url}
\graphicspath{{Fig/}}

\usepackage{mathtools}
\usepackage{bussproofs}
\usepackage{stackengine}
\def\ruleoffset{1pt}
\newcommand\specialvdash[2]{\mathrel{\ensurestackMath{
  \mkern2mu\rule[-\dp\strutbox]{.4pt}{\baselineskip}\stackon[\ruleoffset]{
    \stackunder[\dimexpr\ruleoffset-.5\ht\strutbox+.5\dp\strutbox]{
      \rule[\dimexpr.5\ht\strutbox-.5\dp\strutbox]{2.5ex}{.4pt}}{
        \scriptstyle #1}}{\scriptstyle#2}\mkern2mu}}
}

\usepackage[table]{xcolor}

\renewcommand\thesection{\arabic{section}}
\renewcommand\thefigure{\arabic{figure}}
\renewcommand\theequation{\arabic{equation}}

\newtheorem{dfn}{Definition}[section]
\newtheorem{thm}[dfn]{Theorem}
\newtheorem{lem}[dfn]{Lemma}
\newtheorem{cor}[dfn]{Corollary}


\theoremstyle{definition}
\newtheorem{exl}[dfn]{Example}
\newtheorem{rem}[dfn]{Remark}
\newtheorem{exc}{Exercise}[section]

\def\R{\mathbb{R}}
\def\N{\mathbb{N}}
\def\Z{\mathbb{Z}}
\def\C{\mathbb{C}}
\def\cP{\mathcal{P}}
\def\cV{\mathcal{V}}
\def\cF{\mathcal{F}}
\def\Th{\mathrm{Th}}

\renewcommand{\emptyset}{\varnothing}
\renewcommand{\phi}{\varphi}
\renewcommand{\epsilon}{\varepsilon}
\def\gcd{\operatorname{gcd}}

\def\Prop{\mathrm{PROP}}
\begin{document}
\setcounter{section}{3}
\setcounter{subsection}{4}
\setcounter{dfn}{13}


The faces of the dual graph $G^*$ correspond to the vertices of $G$:
the duals of the edges incident to $v \in V(G)$ form a cycle around $v$.
Thus we have bijections
\[
V(G) \mapsto F(G^*), \quad E(G) \mapsto E(G^*), \quad F(G) \mapsto V(G^*).
\]

Observe that Theorem \ref{thm:DualHandshake} is equivalent to the handshake lemma for the dual graph $G^*$.

Finally, note that the dual of $G^*$ is $G$ again:
\[
(G^*)^* = G.
\]

\bigskip

For more information on planar graphs, see \cite[Chapter 10]{BM}.





\end{document}
