\documentclass[preview, multi=page, margin=5mm, class=report]{standalone}
\usepackage[utf8]{inputenc}

\usepackage{amsmath,amssymb,amsthm}
\usepackage{graphicx,color}
\usepackage{hyperref,url}
\graphicspath{{Fig/}}

\usepackage{mathtools}
\usepackage{bussproofs}
\usepackage{stackengine}
\def\ruleoffset{1pt}
\newcommand\specialvdash[2]{\mathrel{\ensurestackMath{
  \mkern2mu\rule[-\dp\strutbox]{.4pt}{\baselineskip}\stackon[\ruleoffset]{
    \stackunder[\dimexpr\ruleoffset-.5\ht\strutbox+.5\dp\strutbox]{
      \rule[\dimexpr.5\ht\strutbox-.5\dp\strutbox]{2.5ex}{.4pt}}{
        \scriptstyle #1}}{\scriptstyle#2}\mkern2mu}}
}

\usepackage[table]{xcolor}

\renewcommand\thesection{\arabic{section}}
\renewcommand\thefigure{\arabic{figure}}
\renewcommand\theequation{\arabic{equation}}

\newtheorem{dfn}{Definition}[section]
\newtheorem{thm}[dfn]{Theorem}
\newtheorem{lem}[dfn]{Lemma}
\newtheorem{cor}[dfn]{Corollary}


\theoremstyle{definition}
\newtheorem{exl}[dfn]{Example}
\newtheorem{rem}[dfn]{Remark}
\newtheorem{exc}{Exercise}[section]

\def\R{\mathbb{R}}
\def\N{\mathbb{N}}
\def\Z{\mathbb{Z}}
\def\C{\mathbb{C}}
\def\cP{\mathcal{P}}
\def\cV{\mathcal{V}}
\def\cF{\mathcal{F}}
\def\Th{\mathrm{Th}}

\renewcommand{\emptyset}{\varnothing}
\renewcommand{\phi}{\varphi}
\renewcommand{\epsilon}{\varepsilon}
\def\gcd{\operatorname{gcd}}

\def\Prop{\mathrm{PROP}}
\begin{document}
\setcounter{section}{2}
\setcounter{subsection}{4}
\setcounter{dfn}{5}

\subsection{Gentzen's sequent calculus: the idea}
Given a proposition $A$, we want to determine whether it is a tautology.
Recall that $A$ is \emph{not} a tautology if and only if there is a valuation $v$ that falsifies $A$.
(We will also call $v$ a \emph{counterexample} to $A$.)
So let us try to falsify $A$ or to show that this is impossible.
As an example, take
\[
A = (p \to q) \to (p \vee q).
\]

In order to falsify a proposition of the form $B \to C$ one has to satisfy $B$ and falsify $C$ at the same time.
This reduces our problem to a combination of two simpler ones.
Let us write our new task on the top of the old one:
\begin{prooftree}
\AxiomC{$p \to q \vdash p \vee q$}
\UnaryInfC{$\vdash (p \to q) \to (p \vee q)$}
\end{prooftree}
The turnstile symbol is used as a separator: on the left we write the things we want to satisfy, on the right the things we want to falsify.
(One can use any other separator.
The choice of $\vdash$ looks awkward because the symbol was used earlier to denote provability.
This choice is partially motivated by the subsequent sections.)

So, now we want to satisfy $p \to q$ and falsify $p \vee q$.
In order to falsify $p \vee q$ we have to falsify both of them at the same time.
We express this by $p \to q \vdash p, q$.
Now on the right hand side we have a list of formulas that we want to falsify simultaneously.
Our diagram takes the following form:
\begin{prooftree}
\AxiomC{$p \to q \vdash p, q$}
\UnaryInfC{$p \to q \vdash p \vee q$}
\UnaryInfC{$\vdash (p \to q) \to (p \vee q)$}
\end{prooftree}

There are two ways to satisfy $p \to q$: one has either to satisfy $q$ or to falsify $p$.
This introduces branching in the diagram:
\begin{center}
\AxiomC{$q \vdash p, q$}
\AxiomC{$\vdash p, p, q$}
\BinaryInfC{$p \to q \vdash p, q$}
\UnaryInfC{$p \to q \vdash p \vee q$}
\UnaryInfC{$\vdash (p \to q) \to (p \vee q)$}
\DisplayProof
\end{center}

Now, any valuation that solves one of the problems on the top of the diagram also solves the problem in the bottom.
The first problem sounds ``satisfy $p$ and falsify $p$ and $q$''.
This is, of course, impossible.
But the second problem says ``falsify $p$ and $q$''.
This is solved by setting $v(p) = 0$, $v(q) = 0$.
From the construction principle of the diagram it follows that this valuation falsifies the initial proposition $A = (p \to q) \to (p \vee q)$.
Thus we have shown that $A$ is not a tautology.

A couple of remarks are in order.
First, there is no need to repeat two equal propositions on the same side of $\vdash$ as we did with $p$ on the top right.
(We did it just to show that trying to satisfy $p \to q$ brings $q$ to the left or $p$ to the right.)
Second, sometimes we have a choice which connective to eliminate.
Here we had it at the second step.
If we choose to eliminate $p \to q$ before $p \vee q$, then the branching happens earlier, and the final diagram looks as follows.
\begin{prooftree}
\AxiomC{$q \vdash p,q$}
\UnaryInfC{$q \vdash p \vee q$}
\AxiomC{$\vdash p,q$}
\UnaryInfC{$\vdash p, p \vee q$}
\BinaryInfC{$p \to q \vdash p \vee q$}
\UnaryInfC{$\vdash (p \to q) \to (p \vee q)$}
\end{prooftree}




\end{document}