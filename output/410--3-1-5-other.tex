\documentclass[preview, multi=page, margin=5mm, class=report]{standalone}
\usepackage[utf8]{inputenc}

\usepackage{amsmath,amssymb,amsthm}
\usepackage{graphicx,color}
\usepackage{hyperref,url}
\graphicspath{{Fig/}}

\usepackage{mathtools}
\usepackage{bussproofs}
\usepackage{stackengine}
\def\ruleoffset{1pt}
\newcommand\specialvdash[2]{\mathrel{\ensurestackMath{
  \mkern2mu\rule[-\dp\strutbox]{.4pt}{\baselineskip}\stackon[\ruleoffset]{
    \stackunder[\dimexpr\ruleoffset-.5\ht\strutbox+.5\dp\strutbox]{
      \rule[\dimexpr.5\ht\strutbox-.5\dp\strutbox]{2.5ex}{.4pt}}{
        \scriptstyle #1}}{\scriptstyle#2}\mkern2mu}}
}

\usepackage[table]{xcolor}

\renewcommand\thesection{\arabic{section}}
\renewcommand\thefigure{\arabic{figure}}
\renewcommand\theequation{\arabic{equation}}

\newtheorem{dfn}{Definition}[section]
\newtheorem{thm}[dfn]{Theorem}
\newtheorem{lem}[dfn]{Lemma}
\newtheorem{cor}[dfn]{Corollary}


\theoremstyle{definition}
\newtheorem{exl}[dfn]{Example}
\newtheorem{rem}[dfn]{Remark}
\newtheorem{exc}{Exercise}[section]

\def\R{\mathbb{R}}
\def\N{\mathbb{N}}
\def\Z{\mathbb{Z}}
\def\C{\mathbb{C}}
\def\cP{\mathcal{P}}
\def\cV{\mathcal{V}}
\def\cF{\mathcal{F}}
\def\Th{\mathrm{Th}}

\renewcommand{\emptyset}{\varnothing}
\renewcommand{\phi}{\varphi}
\renewcommand{\epsilon}{\varepsilon}
\def\gcd{\operatorname{gcd}}

\def\Prop{\mathrm{PROP}}
\begin{document}
\setcounter{section}{1}
\setcounter{subsection}{5}
\setcounter{dfn}{19}

\begin{proof}
Let $A$ be a propositional formula.
By Corollary \ref{cor:DNF} the negation of $A$ is equivalent to some formula in DNF:
\[
\neg A \simeq \bigvee_{\alpha = 1}^N B_\alpha, \quad B_\alpha = C_{\alpha,1} \wedge \cdots \wedge C_{\alpha,k_{\alpha}}, \quad C_{\alpha,i} \text{ literals}.
\]
By De Morgan's law we have
\[
A \simeq \neg\neg A \simeq \neg \bigvee_{\alpha = 1}^N B_\alpha \simeq \bigwedge_{\alpha = 1}^N \neg B_\alpha
\simeq \bigwedge_{\alpha = 1}^N (\neg C_{\alpha,1} \vee \cdots \vee \neg C_{\alpha,k_{\alpha}}).
\]
The negation of a literal is a negated or a doubly negated symbol.
Double negations can be removed, and we obtain a formula in CNF.
\end{proof}

An unsatisfiable formula is equivalent to $p \wedge \neg p$.
This formula is at the same time in DNF (it is a single conjunctive clause) and in CNF (it is a conjunction of two disjunctive clauses).
A tautology is equivalent to $p \vee \neg p$, which is also in DNF and in CNF.


\end{document}