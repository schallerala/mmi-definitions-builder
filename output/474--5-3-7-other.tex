

Using relation \eqref{eqn:RecNK}, one can compute the entries of the table $p(n, \le k)$ recursively.
First, one fills the row $k=1$ and the column $n=0$ with ones.
Then, one can fill the table row after row or column after column.
One should observe that for $k > n$ one has $p(n, \le k) = p(n, \le n)$.
If one goes column after column, then in order to fill the column for $n = i$
one marks the diagonal $n+k = i$ and computes the $(n,k)$-entry as the sum of the entry immediately above it
and the entry on the intersection of the current row and the marked diagonal.

\begin{center}
\begin{tabular}{r|rrrrrrr}
$k^{\scalebox{1}{$n$}}$\hspace{-.2cm} & 0 & 1 & 2 & 3 & 4 & 5 & 6\\
\hline
1 & 1 & 1 & 1 & 1 & 1 & 1 & 1\\
2 & 1 & 1 & 2 & 2 & 3 & 3 & 4\\
3 & 1 & 1 & 2 & 3 & 4 & 5 & 7\\
4 & 1 & 1 & 2 & 3 & 5 & 6 & 9\\
5 & 1 & 1 & 2 & 3 & 5 & 7 & 10\\
6 & 1 & 1 & 2 & 3 & 5 & 7 & 11
\end{tabular}
\end{center}

There is a recurrence which allows a much faster computation of the number of partitions.
