\documentclass[preview, multi=page, margin=5mm, class=report]{standalone}
\usepackage[utf8]{inputenc}

\usepackage{amsmath,amssymb,amsthm}
\usepackage{graphicx,color}
\usepackage{hyperref,url}
\graphicspath{{Fig/}}

\usepackage{mathtools}
\usepackage{bussproofs}
\usepackage{stackengine}
\def\ruleoffset{1pt}
\newcommand\specialvdash[2]{\mathrel{\ensurestackMath{
  \mkern2mu\rule[-\dp\strutbox]{.4pt}{\baselineskip}\stackon[\ruleoffset]{
    \stackunder[\dimexpr\ruleoffset-.5\ht\strutbox+.5\dp\strutbox]{
      \rule[\dimexpr.5\ht\strutbox-.5\dp\strutbox]{2.5ex}{.4pt}}{
        \scriptstyle #1}}{\scriptstyle#2}\mkern2mu}}
}

\usepackage[table]{xcolor}

\renewcommand\thesection{\arabic{section}}
\renewcommand\thefigure{\arabic{figure}}
\renewcommand\theequation{\arabic{equation}}

\newtheorem{dfn}{Definition}[section]
\newtheorem{thm}[dfn]{Theorem}
\newtheorem{lem}[dfn]{Lemma}
\newtheorem{cor}[dfn]{Corollary}


\theoremstyle{definition}
\newtheorem{exl}[dfn]{Example}
\newtheorem{rem}[dfn]{Remark}
\newtheorem{exc}{Exercise}[section]

\def\R{\mathbb{R}}
\def\N{\mathbb{N}}
\def\Z{\mathbb{Z}}
\def\C{\mathbb{C}}
\def\cP{\mathcal{P}}
\def\cV{\mathcal{V}}
\def\cF{\mathcal{F}}
\def\Th{\mathrm{Th}}

\renewcommand{\emptyset}{\varnothing}
\renewcommand{\phi}{\varphi}
\renewcommand{\epsilon}{\varepsilon}
\def\gcd{\operatorname{gcd}}

\def\Prop{\mathrm{PROP}}
\begin{document}
\setcounter{section}{2}
\setcounter{subsection}{5}
\setcounter{dfn}{9}

\begin{proof}
The proof is done by looking at the truth tables of the logical connectives.
Let us prove, for example, the ($\wedge$ right) rule:
\begin{prooftree}
\AxiomC{$\Gamma \vdash A, \Delta$}
\AxiomC{$\Gamma \vdash B, \Delta$}
\BinaryInfC{$\Gamma \vdash A \wedge B, \Delta$}
\end{prooftree}
A valuation $v$ falsifies the conclusion if and only if
it satisfies all propositions in $\Gamma$, falsifies $A \wedge B$, and falsifies all propositions in $\Delta$.
But $v$ falsifies $A \wedge B$ if and only if it either falsifies $A$ or falsifies $B$.
Thus $v$ falsifies the conclusion if and only if it
\begin{itemize}
\item
satisfies $\Gamma$ and falsifies $A$ and $\Delta$, or
\item
satisfies $\Gamma$ and falsifies $B$ and $\Delta$.
\end{itemize}
These conditions are the premises of the ($\wedge$ right) rule, therefore the statement of the lemma holds for this rule.
Similar arguments work for all of the other rules.
\end{proof}

For any proposition $A$ one can form a sequent $\vdash A$, which is falsifiable, respectively valid, if and only if
$A$ is falsifiable, respectively valid (a tautology).
Thus if we learn to prove all valid sequents, then we will be able to prove all tautologies.




\end{document}