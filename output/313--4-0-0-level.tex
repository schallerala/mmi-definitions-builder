\chapter{Predicate logic}
\section*{Introduction}
Predicate logic is a more complicated and powerful system than the propositional logic.

Before proceeding to formal definitions, let us have a glimpse at how it works.
Consider the statement
\begin{quote}
For every number $x$ one has $x < x+1$.
\end{quote}
It can be expressed by a predicate formula
\begin{equation}
\label{eqn:PredForm}
\forall x P(x, f(x)),
\end{equation}
where $f(x) = x+1$, and $P(x,y)$ means $x < y$.
Functions of one or several arguments that take truth values (such as $x < y$ or ``$x$ is blue'') are called \emph{predicates}.
Formula \eqref{eqn:PredForm} contains all the main building blocks of the predicate logic: a variable $x$, a function $f$, and a predicate $P$.

Let us now forget the origin of formula \eqref{eqn:PredForm}.
In order to make sense of it, its elements must be interpreted:
what kind of objects are represented by the variable $x$, how is the function $f$ defined, and what does $P(x,y)$ mean.
This interpretation can be as above, but can also be different.
For example, the same formula can be interpreted as
\begin{quote}
It gets colder every day.
\end{quote}
Now $x$ is a day, $f(x)$ is the day after day $x$, and $P(x,y)$ means ``$y$ is colder than $x$''.

Our first interpretation evaluates the formula \eqref{eqn:PredForm} to true, while the second evaluates it to false.
A formula of the predicate logic is \emph{valid} if it evaluates to true in all interpretations.
Similarly to the propositional logic, one aims at finding a method (a proof theory) to establish the validity of a formula.


